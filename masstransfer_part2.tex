\documentclass[12pt]{article}
\usepackage[paperwidth=8.5in, paperheight=11in, margin=1.0in, headheight=15pt]{geometry}
\usepackage{amsmath,amssymb,amsthm}
\usepackage[most]{tcolorbox}
\usepackage{enumitem}
\usepackage{xcolor}
\usepackage{hyperref}
\usepackage{fancyhdr}
\usepackage{titlesec}
\usepackage{graphicx}
% Define custom colors for chemical engineering theme
\definecolor{conceptcolor}{RGB}{52, 73, 94}      % Dark blue-gray
\definecolor{formulacolor}{RGB}{231, 76, 60}     % Red for formulas
\definecolor{examplecolor}{RGB}{39, 174, 96}     % Green for examples
\definecolor{stepcolor}{RGB}{142, 68, 173}       % Purple for solution steps
\definecolor{keycolor}{RGB}{243, 156, 18}        % Orange for key points
% Configure fancy headers
\pagestyle{fancy}
\fancyhf{}
\fancyhead[L]{PE Study Guide}
\fancyhead[R]{Process Fundamentals}
\fancyfoot[C]{\thepage}
\renewcommand{\baselinestretch}{1.1}
\setlength{\parindent}{0.25in}
\setlength{\parskip}{3pt}
% Configure section formatting
\titleformat{\section}
  {\normalfont\LARGE\bfseries\color{conceptcolor}}
  {\thesection}{1em}{}
\titleformat{\subsection}
  {\normalfont\Large\bfseries\color{conceptcolor}}
  {\thesubsection}{1em}{}
% Define custom environments
\newtcolorbox{conceptbox}[1][]{
  enhanced,
  colback=conceptcolor!10,
  colframe=conceptcolor,
  arc=3mm,
  title=Key Concept,
  fonttitle=\bfseries\sffamily\normalsize,
  fontupper=\small,
  #1
}
\newtcolorbox{formulabox}[1][]{
  enhanced,
  colback=formulacolor!10,
  colframe=formulacolor,
  arc=2mm,
  title=Important Formula,
  fonttitle=\bfseries\sffamily\normalsize,
  fontupper=\small,
  #1
}
\newtcolorbox{examplebox}[2][]{
  enhanced,
  colback=examplecolor!10,
  colframe=examplecolor,
  arc=3mm,
  title=Example Problem: #2,
  fonttitle=\bfseries\sffamily\normalsize,
  fontupper=\small,
  #1
}
\newtcolorbox{stepbox}[1][]{
  enhanced,
  colback=stepcolor!10,
  colframe=stepcolor,
  arc=2mm,
  title=Solution Steps,
  fonttitle=\bfseries\sffamily\normalsize,
  fontupper=\small,
  #1
}
\newtcolorbox{keybox}[1][]{
  enhanced,
  colback=keycolor!10,
  colframe=keycolor,
  arc=2mm,
  title=Key Variables \& Definitions,
  fonttitle=\bfseries\sffamily\normalsize,
  fontupper=\small,
  #1
}

\section*{Batch Column: Variable Reflux}
An alternative operating mode for batch distillation is to maintain a \textbf{constant distillate composition}, $x_D$. To achieve this, the \textbf{reflux ratio, R}, must be continuously increased throughout the run. As the liquid in the reboiler becomes leaner in the more volatile component, more reflux (a higher $R$) is needed to achieve the required separation and keep the distillate product on-spec.

\subsection*{Material Balances for Constant Distillate Composition}
\begin{conceptbox}
When the distillate composition ($x_D$) is held constant, the complex Rayleigh integral is not needed. The process can be analyzed with simple overall and component mole balances, treating the entire process as a single system with initial and final states.
\end{conceptbox}

\begin{formulabox}
The governing equations are the overall material balances:
$$ \text{Total Balance: } W_i = D + W_f \quad \text{(Equation 4)} $$
$$ \text{Component Balance: } W_i x_{W,i} = D x_D + W_f x_{W,f} \quad \text{(Equation 5)} $$
These two equations can be solved simultaneously for two unknowns (e.g., $D$ and $W_f$).
\end{formulabox}

\begin{keybox}
\begin{itemize}[itemsep=0pt]
    \item $W_i$, $W_f$: Initial and final moles of liquid in the reboiler.
    \item $D$: Total moles of distillate collected.
    \item $x_{W,i}$, $x_{W,f}$: Initial and final mole fraction in the reboiler.
    \item $x_D$: Constant mole fraction of the distillate.
\end{itemize}
\end{keybox}

\subsection*{Example Problem: Variable Reflux Distillation}
\begin{examplebox}{Variable Reflux Distillation}
A batch still is charged with 1000 moles of an ethanol-water mixture containing 30 mole \% ethanol ($x_{W,i}=0.3$). The column has 2 equilibrium stages (including the partial reboiler). The distillation is run with a constant distillate composition of $x_D=0.6$ until the reboiler composition drops to $x_{W,f}=0.09$. How much distillate is produced, and what is the range of reflux ratios used?
\end{examplebox}

\begin{stepbox}[title=Part 1: Calculate Amount of Distillate]
\begin{enumerate}[label=\textbf{Step \arabic*:}, wide=0pt, leftmargin=*, itemsep=2pt]
    \item \textbf{Set up Balances:} 
    
    Use the known values in Equations 4 and 5.
    
    Given: $W_i = 1000$ mol, $x_{W,i} = 0.30$, $x_D = 0.60$ (constant), $x_{W,f} = 0.09$.
    
    Total material balance:
    $$ 1000 = D + W_f \implies W_f = 1000 - D $$
    
    Component balance:
    $$ (1000)(0.30) = D(0.60) + W_f(0.09) $$
    
    \item \textbf{Solve for Distillate (D):} Substitute the expression for $W_f$ into the component balance.
    $$ 300 = 0.60 D + (1000 - D)(0.09) $$
    $$ 300 = 0.60 D + 90 - 0.09 D $$
    $$ 210 = 0.51 D $$
    $$ D = \frac{210}{0.51} \approx 411.8 \, \text{moles} $$
    
    \item \textbf{Final Answer (Part 1):} 
    
    The total amount of distillate produced is approximately \textbf{412 moles}.
\end{enumerate}
\end{stepbox}
\newpage
\begin{stepbox}[title=Part 2: Determine the Range of Reflux Ratios]
\begin{enumerate}[label=\textbf{Step \arabic*:}, wide=0pt, leftmargin=*, itemsep=2pt]
    \item \textbf{Strategy:} The reflux ratio must be adjusted continuously. We find the required range by determining the reflux ratio needed at the start ($R_i$) and at the end ($R_f$) of the process using the McCabe-Thiele method. The slope of the operating line is $m = R/(R+1)$.
    
    \item \textbf{Calculate Initial Reflux Ratio ($R_i$):} At the start, $x_D = 0.6$ and $x_{W,i} = 0.3$. We find the operating line that connects the point ($0.6, 0.6$) on the $y=x$ line and allows for a bottoms composition of $0.3$ in exactly 2 stages. This requires a graphical trial-and-error process, which shows this corresponds to an operating line with a slope of 0.125.
    $$ \text{slope} = \frac{R_i}{R_i+1} = 0.125 $$
    $$ R_i = 0.125(R_i+1) \implies R_i = 0.125 R_i + 0.125 $$
    $$ 0.875 R_i = 0.125 \implies R_i = \frac{0.125}{0.875} \approx 0.143 $$
    
    \item \textbf{Calculate Final Reflux Ratio ($R_f$):} At the end, $x_D=0.6$ and $x_{W,f}=0.09$. We repeat the trial-and-error process. The operating line that connects ($0.6, 0.6$) to a bottoms composition of $0.09$ in 2 stages corresponds to a slope of 0.75.
    $$ \text{slope} = \frac{R_f}{R_f+1} = 0.75 $$
    $$ R_f = 0.75(R_f+1) \implies R_f = 0.75 R_f + 0.75 $$
    $$ 0.25 R_f = 0.75 \implies R_f = \frac{0.75}{0.25} = 3.0 $$
    
    \item \textbf{Final Answer (Part 2):} The required reflux ratio increases from an initial value of \textbf{R = 0.14} to a final value of \textbf{R = 3.0}.
\end{enumerate}
\end{stepbox}

\newpage
\section*{Distillation – Side Stream Feed}
Standard distillation columns have a single feed, creating a rectifying section (above the feed) and a stripping section (below). When a column has multiple feeds, such as a main feed and a secondary steam feed, additional operating sections are created. A column with two feeds will have three sections: top, middle, and bottom, each with a unique operating line on a McCabe-Thiele diagram.

\subsection*{Constructing the Middle Operating Line}
\begin{conceptbox}
For a column with two feeds, a \textbf{Middle Operating Line (MOL)} must be constructed. This line represents the material balance in the section between the two feed points. The MOL is the straight line that connects two specific intersection points on the McCabe-Thiele diagram:
\begin{enumerate}[itemsep=0pt]
    \item The intersection of the \textbf{Top Operating Line (TOL)} with the \textbf{feed line for the upper feed}.
    \item The intersection of the \textbf{Bottom Operating Line (BOL)} with the \textbf{feed line for the lower feed}.
\end{enumerate}
\end{conceptbox}

\subsection*{Example Problem: Column with Steam Injection}
\begin{examplebox}{Column with Steam Injection}
An ethanol-water distillation column has a total condenser and a partial reboiler. It is fed with a main subcooled liquid feed and a secondary feed of pure saturated steam, which enters on the first stage above the reboiler. The Top Operating Line (TOL), Bottom Operating Line (BOL), and main feed line are plotted. Determine the optimal feed plate location for the main feed and the total number of equilibrium stages required.
\end{examplebox}

\begin{stepbox}
\begin{enumerate}[label=\textbf{Step \arabic*:}, wide=0pt, leftmargin=*, itemsep=2pt]
    \item \textbf{Strategy:} The solution requires correctly constructing the Middle Operating Line (MOL). Once all three operating lines (TOL, MOL, BOL) are drawn, we can step off the stages from the bottom ($x_B$) to the top ($x_D$), switching operating lines as we cross the feed locations.

    \item \textbf{Construct the Steam Feed Line:} A feed line's properties are set by its composition ($z$) and thermal quality ($q$).
    \begin{itemize}[itemsep=0pt]
        \item \textbf{Composition ($z_s$):} The feed is pure steam (pure water), so its mole fraction of ethanol is zero. The feed line originates at ($z_s, z_s$) = (0, 0).
        \item \textbf{Quality ($q$):} The feed is a saturated vapor, for which $q=0$. The slope of the feed line is $\frac{q}{q-1} = \frac{0}{0-1} = 0$.
    \end{itemize}
    Therefore, the steam feed line is a \textbf{horizontal line at y=0} (the x-axis).

    \item \textbf{Construct the Middle Operating Line (MOL):} Following the rule from the Key Concept, we draw a straight line connecting the two intersection points:
    \begin{itemize}[itemsep=0pt]
        \item \textbf{Point 1:} Intersection of the given TOL and the main (upper) feed line.
        \item \textbf{Point 2:} Intersection of the given BOL and the steam (lower) feed line. This point lies on the x-axis.
    \end{itemize}

    \item \textbf{Step Off the Stages:} We count stages from the bottom up.
    \begin{itemize}[itemsep=0pt]
        \item \textbf{Partial Reboiler:} The reboiler counts as Stage 0. Start at ($x_B, x_B$) on the y=x line. The first step goes from the equilibrium curve down to the \textbf{BOL}.
        \item \textbf{Stage 1 (Steam Feed):} The steam enters on Stage 1. This is the stage immediately above the reboiler. After stepping up from the reboiler, we have crossed the lower feed location. Therefore, all subsequent steps must use the next operating line up, which is the \textbf{MOL}.
        \item \textbf{Switch to TOL:} Continue stepping off stages using the MOL. When a stage step crosses the main feed line, switch to the \textbf{TOL} for all remaining stages until $x_D$ is reached.
    \end{itemize}

    \item \textbf{Determine Optimal Feed Location:} The optimal stage for a feed is where the stage composition is closest to the feed composition. Graphically, this is the stage where we switch operating lines. In this case, the switch from the MOL to the TOL occurs at \textbf{Stage 2}. Therefore, the main feed should be introduced on Stage 2.

    \item \textbf{Final Answer:} Following the procedure yields a \textbf{partial reboiler and 5 ideal stages}, for a total of 6 equilibrium contacts. The optimal location for the main feed is \textbf{Stage 2}.
\end{enumerate}
\end{stepbox}

\newpage

\section*{Single-stage Batch Distillation}
Single-stage batch distillation, also known as simple or differential distillation, is the most basic method of separating a finite batch of a liquid mixture. In this process, a liquid charge is placed in a heated vessel (the "still"), and the vapor generated is immediately removed and condensed.

\begin{conceptbox}[title=The Principle of Simple Distillation]
Simple distillation works because the vapor generated from a boiling liquid mixture is typically richer in the more volatile component (MVC) than the liquid from which it came. As this enriched vapor is continuously removed, the liquid remaining in the still becomes progressively leaner in the MVC. This means the composition of the liquid in the still, the composition of the vapor being produced, and the boiling temperature all change continuously throughout the process.
\end{conceptbox}

\subsection*{Important Equations}
The process is inherently unsteady-state, so its analysis relies on differential mass balances that are integrated over the course of the distillation.

\begin{formulabox}[title=Key Equations for Simple Distillation]
\textbf{Differential Mass Balances (on the still):}
\begin{itemize}[itemsep=2pt]
    \item Overall Balance: $\frac{dW}{dt} = -\dot{D}$
    \item Component Balance (MVC): $\frac{d(W x_W)}{dt} = -\dot{D} y_D$
\end{itemize}

\textbf{The Rayleigh Equation (Primary Design Equation):}
This equation relates the change in liquid composition to the change in the total amount of liquid in the still. Note that the relationship between $y_D$ and $x_W$ comes from vapor-liquid equilibrium (VLE) data.
$$ \int_{x_{W0}}^{x_W} \frac{dx_W}{y_D - x_W} = \ln\left(\frac{W}{W_0}\right) $$

\textbf{Overall Material Balances (for the entire process):}
These are used to relate the initial state to the final state and the total distillate collected.
\begin{itemize}[itemsep=2pt]
    \item Total Moles: $W_0 = W + D$
    \item Component Moles: $W_0 x_{W0} = W x_W + D x_{D,avg}$
\end{itemize}
\end{formulabox}

\newpage
\subsection*{Derivation of the Rayleigh Equation}
The Rayleigh equation is derived by combining the differential balances to eliminate time, providing a direct relationship between the amount of liquid remaining and its composition.

\begin{stepbox}[title=Derivation of the Rayleigh Equation]
\begin{enumerate}[label=\textbf{Step \arabic*:}, wide=0pt, leftmargin=*, itemsep=2pt]
    \item \textbf{Start with Differential Balances:} We begin with the fundamental unsteady-state balances on the contents of the still.
    $$ \frac{dW}{dt} = -\dot{D} \quad \text{(Overall)} $$
    $$ \frac{d(W x_W)}{dt} = -\dot{D} y_D \quad \text{(Component)} $$

    \item \textbf{Expand the Component Balance:} Apply the product rule for differentiation to the left side of the component balance.
    $$ W\frac{dx_W}{dt} + x_W\frac{dW}{dt} = -\dot{D} y_D $$

    \item \textbf{Combine the Balances:} To eliminate the distillate rate $\dot{D}$, substitute the overall balance ($\frac{dW}{dt} = -\dot{D}$) into the expanded component balance.
    $$ W\frac{dx_W}{dt} + x_W\frac{dW}{dt} = \left(\frac{dW}{dt}\right) y_D $$

    \item \textbf{Separate the Variables:} Rearrange the equation to group all terms involving composition ($x_W, y_D$) on one side and all terms involving the total amount ($W$) on the other.
    $$ W\frac{dx_W}{dt} = y_D\frac{dW}{dt} - x_W\frac{dW}{dt} = (y_D - x_W)\frac{dW}{dt} $$
    Dividing both sides by $W$ and $(y_D - x_W)$, and cancelling the $dt$ term, yields:
    $$ \frac{dx_W}{y_D - x_W} = \frac{dW}{W} $$

    \item \textbf{Integrate the Result:} Integrate both sides from the initial state ($W_0, x_{W0}$) to a final state ($W, x_W$).
    $$ \int_{x_{W0}}^{x_W} \frac{dx_W}{y_D - x_W} = \int_{W_0}^{W} \frac{dW}{W} $$
    Solving the right-side integral gives the final form of the Rayleigh Equation:
    $$ \int_{x_{W0}}^{x_W} \frac{dx_W}{y_D - x_W} = \ln\left(\frac{W}{W_0}\right) $$
\end{enumerate}
\end{stepbox}

\newpage

\subsection*{Conceptual Example: Distillation of a System with an Azeotrope}
\begin{conceptbox}[title=Azeotropes in Distillation]
An \textbf{azeotrope} is a liquid mixture which, at its boiling point, produces a vapor of the exact same composition as the liquid ($y=x$). Such mixtures represent a limit to separation by conventional distillation. A \textbf{minimum-boiling azeotrope} occurs when the boiling point of the azeotropic mixture is lower than that of either pure component. This azeotrope acts as an "attractor" for the vapor composition during distillation of mixtures on one side of it, and a "repeller" on the other.
\end{conceptbox}

\begin{examplebox}{Distillation of a Minimum-Boiling Azeotrope}
A batch still is charged with a liquid mixture with a mole fraction of 0.2 of component 1. The T-x-y diagram shows that this system has a minimum-boiling azeotrope at a mole fraction of 0.5. With constant heating, describe how the temperature of the still and the composition of the vapor product change over time.
\end{examplebox}

\begin{stepbox}[title=Analysis using the T-x-y Diagram]

\begin{enumerate}[label=\textbf{Step \arabic*:}, wide=0pt, leftmargin=*, itemsep=2pt]
    \item \textbf{Initial Vaporization:}
    The initial liquid at $x_W = 0.2$ is heated to its boiling point. At this temperature, the vapor in equilibrium with it (found by moving horizontally across the tie-line to the dew-point line) has a composition of approximately $y_D = 0.05$.

    \item \textbf{The Distillation Path:}
    Since the vapor being removed ($y_D \approx 0.05$) is \textit{leaner} in component 1 than the liquid it came from ($x_W = 0.2$), component 2 is being removed preferentially. This causes the liquid remaining in the still to become progressively \textbf{enriched in component 1}. As the liquid composition $x_W$ increases from 0.2 towards 0.5, we move up along the bubble-point (liquid) curve. This has two effects:
    \begin{itemize}
        \item The \textbf{boiling temperature of the liquid in the still increases}.
        \item The composition of the vapor being produced, $y_D$, also becomes richer in component 1.
    \end{itemize}

    \item \textbf{Approaching the Azeotrope:}
    This enrichment process continues, with the liquid in the still becoming more and more concentrated in component 1. As $x_W$ approaches the azeotropic composition of 0.5, the boiling temperature of the mixture continues to rise, approaching the minimum boiling point of the azeotrope. The vapor composition $y_D$ also approaches 0.5.

    \item \textbf{Azeotropic Boiling:}
    Eventually, the liquid in the still will reach the azeotropic composition, $x_W = 0.5$. At this point, the liquid and vapor compositions are identical ($x_W = y_D = 0.5$). Any further vaporization produces vapor of that same azeotropic composition. Since the vapor being removed has the same composition as the liquid, the composition of the liquid in the still \textbf{no longer changes}. Consequently, the boiling temperature also becomes \textbf{constant} and remains at the azeotropic boiling point until all the remaining liquid is boiled away. This is the limit of separation for this starting mixture.
\end{enumerate}
\end{stepbox}

\newpage

\subsection*{Example Problem: Distillation using T-x-y Data}
\begin{examplebox}{Distillation using T-x-y Data}
A simple pot still is charged with 50 moles of an ethanol-water mixture. The distillation is started, and the initial vapor temperature (head temperature) is 84$^\circ$C. The process is stopped when the head temperature rises to 89$^\circ$C. Using the provided T-x-y diagram for the ethanol-water system, determine:
\begin{enumerate}[itemsep=0pt]
    \item The total amount of distillate collected (D).
    \item The final composition of the liquid remaining in the still (the waste).
    \item The average composition of the total collected distillate.
\end{enumerate}
\end{examplebox}

\begin{stepbox}[title=Steps 1-2: Strategy and Approach]
\begin{enumerate}[label=\textbf{Step \arabic*:}, wide=0pt, leftmargin=*, itemsep=2pt]
    \item \textbf{Strategy: The Rayleigh Equation with Numerical Integration}
    The governing equation for this process is the Rayleigh equation. Since this is a single-stage still, the vapor leaving ($y_D$) is in direct equilibrium with the liquid in the still ($x_W$).
    $$ \int_{x_{W0}}^{x_{Wf}} \frac{dx_W}{y - x_W} = \ln\left(\frac{W_f}{W_0}\right) $$
    Our plan is to: (a) Use the given temperatures and the T-x-y diagram to find the initial ($x_{W0}$) and final ($x_{Wf}$) liquid compositions. (b) Use the x-y data from the diagram to numerically evaluate the integral on the left-hand side. (c) Solve for the final moles in the still, $W_f$. (d) Use overall material balances to find the amount and average composition of the distillate.

    \item \textbf{Determine Compositions from T-x-y Data}
    The head temperature is the boiling point of the liquid currently in the still. We use this to find the liquid composition from the bubble-point line on the diagram.
    
    Initial State ($T=84^\circ$C): Following the 84°C line to the liquid curve gives the initial still composition: $\mathbf{x_{W0} \approx 0.17}$. Final State ($T=89^\circ$C): Following the 89°C line to the liquid curve gives the final still composition: $\mathbf{x_{Wf} \approx 0.07}$. The distillation proceeds from a liquid concentration of 17\% ethanol down to 7\% ethanol.
\end{enumerate}
\end{stepbox}
\newpage
\begin{stepbox}[title=Steps 3-5: Analysis and Calculations]
\begin{enumerate}[label=\textbf{Step \arabic*:}, wide=0pt, leftmargin=*, itemsep=2pt, start=3]
    
    \item \textbf{Perform Numerical Integration}
    We must evaluate the integral $\int_{0.17}^{0.07} \frac{dx_W}{y - x_W}$. This is done by taking several points from the VLE diagram between $x_W = 0.07$ and $x_W = 0.17$, calculating the value of $1/(y-x_W)$ at each point, and finding the area under the curve. Using a numerical method like the trapezoidal rule, the value is found to be:
    $$ \int_{0.17}^{0.07} \frac{dx_W}{y - x_W} \approx -0.27 $$
    The value is negative because we are integrating from a higher concentration to a lower one ($x_{Wf} < x_{W0}$).
    
    \item \textbf{Solve for Final Moles in Still ($W_f$)}
    Substitute the result of the integral into the Rayleigh equation. We are given $W_0 = 50$ moles.
    $$ \ln\left(\frac{W_f}{W_0}\right) = -0.27 $$
    $$ \frac{W_f}{50} = e^{-0.27} \approx 0.763 \implies W_f = 50 \times 0.763 = 38.15 \, \text{moles} $$
    
    \item \textbf{Calculate Final Answers}
    With $W_f$ known, we can find the remaining quantities using simple material balances.
    
    Total Distillate Collected (D): $D = W_0 - W_f = 50 - 38.15 = 11.85$ moles.
    
    Average Distillate Composition ($x_{D,avg}$):
    $$ W_0 x_{W0} = W_f x_{Wf} + D x_{D,avg} $$
    $$ (50)(0.17) = (38.15)(0.07) + (11.85) x_{D,avg} $$
    $$ 8.5 = 2.67 + 11.85 x_{D,avg} \implies 11.85 x_{D,avg} = 5.83 $$
    $$ x_{D,avg} = \frac{5.83}{11.85} \approx 0.492 $$
\end{enumerate}
\begin{keybox}[title=Final Answer Summary for Example 1]
Total Distillate Collected (D): $\approx 12$ moles. Final Still Composition ($x_{Wf}$): 7\% ethanol. Average Distillate Composition ($x_{D,avg}$): $\approx 49\%$ ethanol.

A sanity check confirms this is reasonable: the instantaneous vapor was richer than 49\% at the start and leaner at the end.
\end{keybox}
\end{stepbox}

\newpage

\subsection*{Example Problem: Distillation using Raoult's Law}
\begin{examplebox}{Distillation with Raoult's Law VLE}
A single-stage batch still is charged with 75 mol of an 82 mol\% methanol / 18 mol\% water mixture. The distillation is run until the \textbf{average} distillate concentration collected is 90.0 mol\% methanol. Assuming the mixture is an ideal solution (follows Raoult's Law) and operates at 760 mmHg:
\begin{enumerate}[itemsep=0pt]
    \item How much total distillate will be collected?
    \item How many moles will remain in the still, and what is its final methanol concentration?
\end{enumerate}
The Antoine equations ($P^{\text{sat}}$ in mmHg, T in $^\circ$C) are:
\begin{itemize}[itemsep=0pt]
    \item Methanol: $\log_{10}(P_1^{\text{sat}}) = 8.081 - \frac{1582}{T + 239.7}$
    \item Water: $\log_{10}(P_2^{\text{sat}}) = 8.071 - \frac{1731}{T + 233.4}$
\end{itemize}
\end{examplebox}

\begin{stepbox}
\begin{enumerate}[label=\textbf{Step \arabic*:}, wide=0pt, leftmargin=*, itemsep=2pt]
    \item \textbf{Strategy: Iterative Numerical Solution}
    This problem is more complex for two reasons: 1) The VLE data is not given directly and must be calculated using Raoult's Law. 2) The stopping condition is based on the \textit{average} distillate composition, which depends on the entire path of the distillation. This means we cannot solve directly and must use an iterative, "marching" solution.
    \item \textbf{Set up the VLE Calculation Block}
    At any given temperature $T$, we can find the  equilibrium liquid ($x_W$) and vapor ($y_D$) compositions at $P_{total} = 760$ mmHg.
    \begin{conceptbox}[title=VLE Calculation using Raoult's Law]
    For any temperature T:
    \begin{enumerate}[label=\alph*)]
        \item Calculate the vapor pressures of methanol ($P_1^{\text{sat}}$) and water ($P_2^{\text{sat}}$) using their Antoine equations.
        \item Find the liquid mole fraction ($x_W$) that boils at this temperature and total pressure:
        $$ x_W = \frac{P_{total} - P_2^{\text{sat}}}{P_1^{\text{sat}} - P_2^{\text{sat}}} $$
        \item Find the corresponding equilibrium vapor composition ($y_D$):
        $$ y_D = \frac{x_W P_1^{\text{sat}}}{P_{total}} $$
    \end{enumerate}
    This block of calculations forms the core of our iterative solution.
    \end{conceptbox}
\end{enumerate}
\end{stepbox}

\begin{stepbox}
\begin{enumerate}[label=\textbf{Step \arabic*:}, wide=0pt, leftmargin=*, itemsep=2pt]


    \item \textbf{Describe the Iterative "Marching" Procedure}
    The spreadsheet calculation proceeds by taking small steps (e.g., small increments of T or $x_W$) and updating all system properties until the target condition ($x_{D,avg}=0.90$) is met.
    \begin{enumerate}[label=\alph*), itemsep=2pt]
        \item \textbf{Initialize:} Start with the initial conditions: $W_0=75$ mol, $x_{W0}=0.82$.
        \item \textbf{Take a Step:} Choose a new, slightly lower liquid composition, $x_{W, new}$.
        \item \textbf{Calculate VLE:} At both the old and new $x_W$, calculate the corresponding temperatures and vapor compositions ($y_D$) using the VLE block from Step 2.
        \item \textbf{Update the Rayleigh Integral:} Approximate the integral over this small step using the trapezoidal rule:
        $$ \Delta(\text{Integral}) = \frac{1}{2} \left( \left[\frac{1}{y_D-x_W}\right]_{\text{old}} + \left[\frac{1}{y_D-x_W}\right]_{\text{new}} \right) \cdot (x_{W,\text{new}} - x_{W,\text{old}}) $$
        The total value of the integral is the running sum of these small step changes.
        \item \textbf{Update Moles in Still (W):} Use the new total value of the integral in the Rayleigh equation: $W = W_0 \exp(\text{Integral})$.
        \item \textbf{Update Total Distillate (D):} Use the overall balance: $D = W_0 - W$.
        \item \textbf{Update Average Composition ($x_{D,avg}$):} Use the overall component balance, solved for $x_{D,avg}$: $x_{D,avg} = (W_0 x_{W0} - W x_W) / D$.
        \item \textbf{Check and Repeat:} Compare the calculated $x_{D,avg}$ to the target of 0.90. If they don't match, repeat from step (b) with another small step in $x_W$. The process is complete when $x_{D,avg} = 0.90$.
    \end{enumerate}
\end{enumerate}
\begin{keybox}[title=Final Answer Summary for Example 2]
The iterative calculation is continued until the target is met. The results obtained from this procedure are:
\begin{itemize}[itemsep=2pt]
    \item \textbf{Total Distillate Collected (D):} $\approx 62.5$ moles.
    \item \textbf{Moles Remaining in Still (W):} $\approx 12.5$ moles.
    \item \textbf{Final Still Composition ($x_{Wf}$):} $\approx 42\%$ methanol.
\end{itemize}
\end{keybox}
\end{stepbox}

\newpage

\section*{Multi-Stage Batch Distillation}
While single-stage (or simple) distillation can achieve a basic level of separation, it is often insufficient for producing high-purity products. \textbf{Multi-stage batch distillation} significantly enhances separation by using a column containing trays or packing. This setup allows for multiple successive vaporization and condensation cycles, resulting in a much purer distillate product than is possible from a single stage.

\subsection*{Important Equations}
The analysis of multi-stage batch distillation combines overall material balances with the McCabe-Thiele method's operating line and the integrated Rayleigh equation.

\begin{formulabox}[title=Key Equations for Multi-Stage Batch Distillation]
\textbf{Overall Material Balances (Entire Process):}
\begin{itemize}[itemsep=2pt]
    \item Total Moles: $F = W_{\text{final}} + D_{\text{total}}$
    \item Component Moles: $F x_F = W_{\text{final}} x_{W,\text{final}} + D_{\text{total}} x_{D,\text{avg}}$
\end{itemize}

\textbf{Operating Line (Rectifying Section):}
This is a material balance that relates the vapor ($y_n$) rising from a stage to the liquid ($x_{n+1}$) entering it from above.
$$ y_n = \left(\frac{R}{R+1}\right)x_{n+1} + \left(\frac{1}{R+1}\right)x_D $$

\textbf{The Rayleigh Equation:}
This equation relates the change in the still's composition to the total amount of material boiled off.
$$ \ln\left(\frac{W_{\text{final}}}{F}\right) = \int_{x_F}^{x_{W,\text{final}}} \frac{dx_W}{x_D - x_W} $$
\end{formulabox}

\begin{keybox}
\begin{itemize}[itemsep=2pt]
    \item $F$: Initial moles of feed charged to the still (reboiler).
    \item $W_{\text{final}}$: Final moles of liquid remaining in the still.
    \item $D_{\text{total}}$: Total moles of distillate product collected.
    \item $x_F, x_{W,\text{final}}$: Mole fractions of the MVC in the initial feed and final still liquid.
    \item $x_D$: \textbf{Instantaneous} mole fraction of the distillate product.
    \item $x_{D,avg}$: \textbf{Average} mole fraction of the total collected distillate.
    \item $R$: The reflux ratio ($L/D$), the ratio of liquid returned to the column to distillate product removed.
    \item $y_n, x_{n+1}$: Vapor and liquid mole fractions on adjacent stages $n$ and $n+1$.
\end{itemize}
\end{keybox}

\subsection*{Conceptual Operation and The Dynamic Nature of the Process}
A multi-stage batch still consists of a reboiler, a column with trays, and an overhead condenser with a reflux splitter. The trays provide surface area for the rising vapor and descending liquid (reflux) to contact and exchange components. This repeated contact is what drives the enhanced separation.

\begin{conceptbox}[title=The Unsteady-State Process]
Unlike continuous distillation, batch distillation is an inherently \textbf{unsteady-state} process. As the distillation proceeds and the more volatile component is removed, the entire system is in flux:
\begin{enumerate}[itemsep=2pt]
    \item The liquid in the still ($x_W$) becomes progressively leaner in the more volatile component.
    \item This change in the still's composition forces the entire concentration profile throughout the column to shift over time.
    \item As a result, the composition of the distillate product ($x_D$) also changes (typically becoming leaner) as the run progresses (if operating at a constant reflux ratio).
    \item Because the still liquid becomes richer in the less volatile (higher boiling point) component, the temperature in the still ($T_W$) will continuously \textbf{increase} throughout the distillation.
\end{enumerate}
\end{conceptbox}

\subsection*{Analysis using McCabe-Thiele Diagrams}
The McCabe-Thiele diagram is a powerful tool, but it can only analyze the state of the column at a \textbf{single instant in time}. To analyze the entire batch process, one must consider how the diagram changes over time.

\begin{conceptbox}[title=The Operating Line at a Moment in Time]
For a column with a reboiler, rectifying section, and total condenser (a common setup), the operating line is a straight line on the x-y diagram that:
\begin{itemize}[itemsep=2pt]
    \item Passes through the point $(x_D, x_D)$ on the $y=x$ line, where $x_D$ is the \textbf{instantaneous} distillate composition.
    \item Has a slope of $\frac{R}{R+1}$.
\end{itemize}
By stepping off stages between this operating line and the equilibrium curve, we can establish the relationship between the instantaneous still composition ($x_W$) and the instantaneous distillate composition ($x_D$) for a given number of stages ($N$) and reflux ratio ($R$).
\end{conceptbox}

As the still composition $x_W$ decreases, the entire operating line must shift down and to the left on the diagram to maintain the same number of stages. This causes the distillate composition $x_D$ to decrease as well, as illustrated below.

\newpage
\subsection*{Operating Strategies}
There are two primary ways to operate a multi-stage batch still, each with different outcomes and objectives.

\begin{keybox}[title=Operating Strategies for Multi-Stage Batch Distillation]
\begin{enumerate}[label=\textbf{\arabic*.}, wide=0pt, leftmargin=*, itemsep=4pt]
    \item \textbf{Constant Reflux Ratio:} This is the simpler method to implement. The operator sets a fixed reflux ratio, $R$, and leaves it constant for the entire run.
    \begin{itemize}
        \item \textbf{Outcome:} The purity of the distillate product, $x_D$, will gradually \textbf{decrease} over time as the still becomes depleted of the more volatile component.
        \item \textbf{Product:} All the collected distillate is mixed together in a single tank, resulting in a final product with an overall \textit{average} composition, $x_{D,avg}$.
    \end{itemize}
    \item \textbf{Variable Reflux Ratio:} This strategy is used when the goal is to produce a distillate of \textbf{constant composition}, $x_D$.
    \begin{itemize}
        \item \textbf{Procedure:} To achieve this, the operator must continuously \textbf{increase} the reflux ratio ($R$) as the distillation proceeds. Increasing the reflux provides more separation power (the operating line moves up), which compensates for the decreasing concentration in the still.
        \item \textbf{Outcome:} A consistent, on-spec product is produced throughout the run. This method is less efficient, requiring more time and energy per mole of product, and the run must be stopped when the reflux ratio reaches infinity (total reflux).
    \end{itemize}
\end{enumerate}
\end{keybox}

\newpage

\subsection*{Example Problem: Single-Stage Distillation using T-x-y Data}
\begin{examplebox}{Single-Stage Distillation using T-x-y Data}
A simple pot still is charged with 50 moles of an ethanol-water mixture. The distillation is started, and the initial vapor temperature (head temperature) is 84$^\circ$C. The process is stopped when the head temperature rises to 89$^\circ$C. Using the provided T-x-y diagram for the ethanol-water system, determine:
\begin{enumerate}[itemsep=0pt]
    \item The total amount of distillate collected (D).
    \item The final composition of the liquid remaining in the still (the waste).
    \item The average composition of the total collected distillate.
\end{enumerate}
\end{examplebox}

\begin{stepbox}
\begin{enumerate}[label=\textbf{Step \arabic*:}, wide=0pt, leftmargin=*, itemsep=2pt]
    \item \textbf{Strategy: The Rayleigh Equation with Numerical Integration}
    The governing equation for this process is the Rayleigh equation. Since this is a single-stage still, the vapor leaving ($y$) is in direct equilibrium with the liquid in the still ($x_W$).
    $$ \int_{x_{W0}}^{x_{Wf}} \frac{dx_W}{y - x_W} = \ln\left(\frac{W_f}{W_0}\right) $$
    Our plan is to:
    \begin{enumerate}[label=\alph*), itemsep=0pt]
        \item Use the given temperatures and the T-x-y diagram to find the initial ($x_{W0}$) and final ($x_{Wf}$) liquid compositions.
        \item Use the x-y data from the diagram to numerically evaluate the integral on the left-hand side.
        \item Solve for the final moles in the still, $W_f$.
        \item Use overall material balances to find the amount and average composition of the distillate.
    \end{enumerate}

    \item \textbf{Determine Compositions from T-x-y Data}
    The head temperature is the boiling point of the liquid currently in the still. We use this to find the liquid composition from the \textbf{bubble-point line} on the diagram.
    \begin{itemize}[itemsep=2pt]
        \item \textbf{Initial State ($T=84^\circ$C):} Following the 84$^\circ$C line to the liquid curve (bubble-point line) gives the initial still composition: $\mathbf{x_{W0} \approx 0.17}$.
        \item \textbf{Final State ($T=89^\circ$C):} Following the 89$^\circ$C line to the liquid curve gives the final still composition: $\mathbf{x_{Wf} \approx 0.07}$.
    \end{itemize}
\end{enumerate}
\end{stepbox}

\begin{stepbox}
\begin{enumerate}[label=\textbf{Step \arabic*:}, wide=0pt, leftmargin=*, itemsep=2pt, start = 3]

    \item \textbf{Perform Numerical Integration}
    We must evaluate the integral $\int_{0.17}^{0.07} \frac{dx_W}{y - x_W}$. This is done by taking several points from the VLE diagram between $x_W = 0.07$ and $x_W = 0.17$, calculating the value of $1/(y-x_W)$ at each point, and finding the area under the curve. Using a numerical method like the trapezoidal rule, the value is found to be:
    $$ \int_{0.17}^{0.07} \frac{dx_W}{y - x_W} \approx -0.27 $$
    The value is negative because we are integrating from a higher concentration to a lower one ($x_{Wf} < x_{W0}$).
    
    \item \textbf{Solve for Final Moles in Still ($W_f$)}
    Substitute the result of the integral into the Rayleigh equation. We are given $W_0 = 50$ moles.
    $$ \ln\left(\frac{W_f}{W_0}\right) = -0.27 $$
    $$ \frac{W_f}{50} = e^{-0.27} \approx 0.763 \implies W_f = 50 \times 0.763 = 38.15 \, \text{moles} $$
    
    \item \textbf{Calculate Final Answers}
    With $W_f$ known, we can find the remaining quantities using simple material balances.
    \begin{itemize}[itemsep=2pt]
        \item \textbf{Total Distillate Collected (D):} $D = W_0 - W_f = 50 - 38.15 = 11.85 \, \text{moles}$.
        \item \textbf{Average Distillate Composition ($x_{D,avg}$):}
        $$ W_0 x_{W0} = W_f x_{Wf} + D x_{D,avg} $$
        $$ (50)(0.17) = (38.15)(0.07) + (11.85) x_{D,avg} $$
        $$ 8.5 = 2.67 + 11.85 x_{D,avg} \implies 11.85 x_{D,avg} = 5.83 $$
        $$ x_{D,avg} = \frac{5.83}{11.85} \approx 0.492 $$
    \end{itemize}
\end{enumerate}
\begin{keybox}[title=Final Answer Summary for Example 1]
\begin{itemize}[itemsep=2pt]
    \item \textbf{Total Distillate Collected (D):} $\approx 12$ moles.
    \item \textbf{Final Still Composition ($x_{Wf}$):} 7\% ethanol.
    \item \textbf{Average Distillate Composition ($x_{D,avg}$):} $\approx 49\%$ ethanol.
\end{itemize}
\end{keybox}
\end{stepbox}

\newpage

\subsection*{Example Problem: Multi-Stage Distillation with Constant Reflux}
\begin{examplebox}{Multi-Stage Distillation with Constant Reflux}
An ethanol-water batch distillation unit consists of two total equilibrium stages (e.g., a reboiler and one tray). The unit is run with a constant reflux ratio of $R=3$. At the moment the instantaneous distillate composition is 60 mol\% ethanol, there are 750 mol of liquid in the reboiler. How much liquid is left in the reboiler when the instantaneous distillate composition has dropped to 50 mol\%?
\end{examplebox}

\begin{stepbox}
\begin{enumerate}[label=\textbf{Step \arabic*:}, wide=0pt, leftmargin=*, itemsep=2pt]
    \item \textbf{Strategy: Rayleigh Equation with McCabe-Thiele Data}
    This problem involves a multi-stage system, so the relationship between the distillate composition ($x_D$) and the reboiler composition ($x_W$) is not direct equilibrium. We must first use the McCabe-Thiele method to establish this relationship at several points. Then, we can use that data to numerically solve the Rayleigh equation.
    
    \item \textbf{Relate $x_D$ and $x_W$ via McCabe-Thiele}
    The operating line slope is constant: $m = R/(R+1) = 3/4 = 0.75$. For any given $x_D$, we draw the operating line and step down 2 stages to find the corresponding $x_W$. We need data for the start, middle, and end of the process to use Simpson's rule.
    \begin{itemize}[itemsep=2pt]
        \item When $x_D = 0.60$ (start), stepping down 2 stages gives $x_W = 0.09$. This is our initial reboiler composition, $x_{W,i}$.
        \item When $x_D = 0.50$ (end), stepping down 2 stages gives $x_W = 0.04$. This is our final reboiler composition, $x_{W,f}$.
        \item When $x_D = 0.55$ (midpoint), stepping down 2 stages gives $x_W = 0.06$.
    \end{itemize}
\end{enumerate}
\end{stepbox}


\begin{stepbox}
\begin{enumerate}[label=\textbf{Step \arabic*:}, wide=0pt, leftmargin=*, itemsep=2pt]

    \item \textbf{Numerical Integration with Simpson's Rule}
    We evaluate the integral $I = \int_{0.09}^{0.04} \frac{dx_W}{x_D - x_W}$ using our three data points. The function is $f(x_W) = \frac{1}{x_D - x_W}$.
    \begin{itemize}[itemsep=2pt]
        \item Initial ($x_{W,i} = 0.09$): $f_i = \frac{1}{0.60 - 0.09} = 1.961$
        \item Midpoint ($x_{W,m} = 0.06$): $f_m = \frac{1}{0.55 - 0.06} = 2.041$
        \item Final ($x_{W,f} = 0.04$): $f_f = \frac{1}{0.50 - 0.04} = 2.174$
    \end{itemize}
    Simpson's 1/3 rule for three points is $I \approx \frac{h}{3}[f_i + 4f_m + f_f]$, where $h$ is the step size between points. Here, the total interval is $x_{W,f} - x_{W,i} = 0.04 - 0.09 = -0.05$. This corresponds to two steps of size $h=-0.025$. A more general form is:
    $$ I \approx \frac{(x_{W,f} - x_{W,i})}{6} [f_i + 4f_m + f_f] $$
    $$ I \approx \frac{-0.05}{6} [1.961 + 4(2.041) + 2.174] = \frac{-0.05}{6} [12.299] \approx -0.1025 $$

    \item \textbf{Solve the Rayleigh Equation}
    We set the integral result equal to the log term. The initial amount in the still, $W_i$, is given as 750 moles.
    $$ \ln\left(\frac{W_f}{W_i}\right) = -0.1025 $$
    $$ \frac{W_f}{750} = e^{-0.1025} \approx 0.9026 $$
    $$ W_f = 750 \times 0.9026 = 676.95 \, \text{moles} $$
\end{enumerate}
\begin{keybox}[title=Final Answer Summary for Example 2]
The amount of liquid left in the reboiler is approximately \textbf{677 moles}.
\end{keybox}
\end{stepbox}

\newpage

nd{document}
