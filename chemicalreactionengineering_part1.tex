\documentclass[12pt]{article}
\usepackage[paperwidth=8.5in, paperheight=11in, margin=1.0in, headheight=15pt]{geometry}
\usepackage{amsmath,amssymb,amsthm}
\usepackage[most]{tcolorbox}
\usepackage{enumitem}
\usepackage{xcolor}
\usepackage{hyperref}
\usepackage{fancyhdr}
\usepackage{titlesec}
\usepackage{graphicx}
% Define custom colors for chemical engineering theme
\definecolor{conceptcolor}{RGB}{52, 73, 94}      % Dark blue-gray
\definecolor{formulacolor}{RGB}{231, 76, 60}     % Red for formulas
\definecolor{examplecolor}{RGB}{39, 174, 96}     % Green for examples
\definecolor{stepcolor}{RGB}{142, 68, 173}       % Purple for solution steps
\definecolor{keycolor}{RGB}{243, 156, 18}        % Orange for key points
% Configure fancy headers
\pagestyle{fancy}
\fancyhf{}
\fancyhead[L]{PE Study Guide}
\fancyhead[R]{Process Fundamentals}
\fancyfoot[C]{\thepage}
\renewcommand{\baselinestretch}{1.1}
\setlength{\parindent}{0.25in}
\setlength{\parskip}{3pt}
% Configure section formatting
\titleformat{\section}
  {\normalfont\LARGE\bfseries\color{conceptcolor}}
  {\thesection}{1em}{}
\titleformat{\subsection}
  {\normalfont\Large\bfseries\color{conceptcolor}}
  {\thesubsection}{1em}{}
% Define custom environments
\newtcolorbox{conceptbox}[1][]{
  enhanced,
  colback=conceptcolor!10,
  colframe=conceptcolor,
  arc=3mm,
  title=Key Concept,
  fonttitle=\bfseries\sffamily\normalsize,
  fontupper=\small,
  #1
}
\newtcolorbox{formulabox}[1][]{
  enhanced,
  colback=formulacolor!10,
  colframe=formulacolor,
  arc=2mm,
  title=Important Formula,
  fonttitle=\bfseries\sffamily\normalsize,
  fontupper=\small,
  #1
}
\newtcolorbox{examplebox}[2][]{
  enhanced,
  colback=examplecolor!10,
  colframe=examplecolor,
  arc=3mm,
  title=Example Problem: #2,
  fonttitle=\bfseries\sffamily\normalsize,
  fontupper=\small,
  #1
}
\newtcolorbox{stepbox}[1][]{
  enhanced,
  colback=stepcolor!10,
  colframe=stepcolor,
  arc=2mm,
  title=Solution Steps,
  fonttitle=\bfseries\sffamily\normalsize,
  fontupper=\small,
  #1
}
\newtcolorbox{keybox}[1][]{
  enhanced,
  colback=keycolor!10,
  colframe=keycolor,
  arc=2mm,
  title=Key Variables \& Definitions,
  fonttitle=\bfseries\sffamily\normalsize,
  fontupper=\small,
  #1
}

\begin{document}

\begin{center}
    \Huge\textbf{\color{conceptcolor}Chemical Reaction Engineering}
\end{center}
\hrule

\section*{First-Order Reaction Kinetics}

First-order reactions are chemical reactions where the rate of reaction is directly proportional to the concentration of only one of the reactants. Understanding their behavior is crucial for reactor design and analysis.

\begin{conceptbox}
The \textbf{half-life} ($t_{1/2}$) of a reaction is the time it takes for the concentration of a reactant to be reduced to half of its initial value. For a first-order reaction, the half-life is constant and does not depend on the initial concentration. This means that in every half-life interval, the amount of reactant will decrease by 50\%. This property is unique to first-order processes and provides a powerful conceptual tool for quick estimations.
\end{conceptbox}

\begin{examplebox}{Visualizing Half-Life in a First-Order Reaction}
\textbf{Question:} A first-order reaction, $A \rightarrow B$, starts with 16 molecules of reactant A. After 100 seconds, 8 molecules of A remain and 8 molecules of B have been formed. What will the mixture look like at 300 seconds?
\end{examplebox}

\begin{stepbox}
\begin{enumerate}[label=\textbf{Step \arabic*:}, wide=0pt, leftmargin=*, itemsep=2pt]
    \item \textbf{Analyze the Initial Data and Define Knowns:}
    
    Initial molecules of A at $t=0$ s: $N_{A0} = 16$. Molecules of A at $t=100$ s: $N_A = 8$.
    
    We observe that exactly half of the initial reactant A has been consumed in the first 100 seconds.
    
    \item \textbf{Identify the Half-Life ($t_{1/2}$):}
    
    Since the reactant amount dropped by exactly 50\% in 100 seconds: $t_{1/2} = 100 \, \text{s} $
    
    \item \textbf{Calculate Reactant Population at Each Half-Life Interval:}
    
    Track the population of A by repeatedly halving it for each 100 s interval:
    
    At $t=0$ s: $N_A = 16$. At $t=100$ s (1 half-life): $N_A = 8$. At $t=200$ s (2 half-lives): $N_A = 4$. At $t=300$ s (3 half-lives): $N_A = 2$.
    
    \item \textbf{Determine Final Composition:}
    
    The total number of molecules is conserved: $N_A + N_B = 16$. Molecules of A remaining at 300 s: $N_A = 2$. Molecules of B formed at 300 s: $N_B = 16 - 2 = 14$.
    
\end{enumerate}
\end{stepbox}

\newpage
\section*{Quantitative Analysis of First-Order Reactions}

While the half-life concept is useful for estimations, precise calculations require the integrated first-order rate law. This section covers the key equations and their application.

\begin{keybox}
\begin{itemize}[itemsep=0pt]
    \item $[A]$: Concentration of reactant A at time $t$ ($\text{mol/L}$).
    \item $[A]_0$: Initial concentration of reactant A at $t=0$ ($\text{mol/L}$).
    \item $k$: First-order rate constant, temperature-dependent ($\text{s}^{-1}$, $\text{min}^{-1}$, etc.).
    \item $t$: Elapsed time.
\end{itemize}
\end{keybox}

\begin{formulabox}[title=First-Order Integrated Rate Law]
The relationship between concentration and time for a first-order process is given by:
$$ [A] = [A]_0 e^{-kt} \quad \text{(Equation 1)} $$
This equation can be rearranged into a linear form by taking the natural logarithm:
$$ \ln\left(\frac{[A]}{[A]_0}\right) = -kt $$
\end{formulabox}

\subsection*{Relating Half-Life and the Rate Constant}
A crucial relationship can be derived by substituting the definition of half-life ($t = t_{1/2}$, $[A] = 0.5[A]_0$) into the integrated rate law.

\begin{formulabox}[title=Half-Life and Rate Constant Relationship]
$$ 0.5[A]_0 = [A]_0 e^{-kt_{1/2}} $$
$$ 0.5 = e^{-kt_{1/2}} $$
$$ \ln(0.5) = -kt_{1/2} $$
Since $\ln(0.5) = -\ln(2)$, we arrive at the simple and important result:
$$ k = \frac{\ln(2)}{t_{1/2}} \quad \text{(Equation 2)}$$
Note that the initial concentration $[A]_0$ does not appear in the final equation.
\end{formulabox}

\begin{examplebox}{First-Order Kinetics Calculations}
\textbf{Question:} The decomposition of dimethyl ether is a first-order process with a half-life of 2165 seconds at $450^\circ\text{C}$. The reaction occurs in a constant volume container with an initial dimethyl ether concentration of 2.0 M.
\begin{enumerate}[label=(\alph*)]
    \item What is the concentration of dimethyl ether after one hour?
    \item How much time is required for the concentration to become 0.2 M?
\end{enumerate}
\end{examplebox}

\begin{stepbox}[title=Solution Steps for Part (a): Concentration after one hour]
\begin{enumerate}[label=\textbf{Step \arabic*:}, wide=0pt, leftmargin=*, itemsep=2pt]
    \item \textbf{Identify Knowns and Goal:}
    \begin{itemize}[itemsep=0pt]
        \item Initial Concentration, $[A]_0 = 2.0 \, \text{M}$
        \item Half-life, $t_{1/2} = 2165 \, \text{s}$
        \item Time, $t = 1 \, \text{hour} = 3600 \, \text{s}$
        \item \textbf{Goal:} Find final concentration, $[A]$, at $t=3600$ s.
    \end{itemize}

    \item \textbf{Calculate the Rate Constant (k):}
    Using Equation 2, we find the rate constant from the given half-life. It is critical to use consistent units.
    $$ k = \frac{\ln(2)}{t_{1/2}} = \frac{0.6931}{2165 \, \text{s}} = 0.0003201 \, \text{s}^{-1} $$

    \item \textbf{Apply the First-Order Rate Equation:}
    Using Equation 1, substitute the known values to find the concentration $[A]$.
    $$ [A] = [A]_0 e^{-kt} $$
    $$ [A] = (2.0 \, \text{M}) \exp\left(-(0.0003201 \, \text{s}^{-1})(3600 \, \text{s})\right) $$
    $$ [A] = (2.0 \, \text{M}) \exp(-1.15236) $$
    $$ [A] = (2.0 \, \text{M}) \times (0.3159) = 0.6318 \, \text{M} $$

    \item \textbf{Conclusion for Part (a):}
    After one hour, the concentration of dimethyl ether will be approximately \textbf{0.63 M}.
\end{enumerate}
\end{stepbox}

\newpage
\begin{stepbox}[title=Solution Steps for Part (b): Time to reach 0.2 M]
\begin{enumerate}[label=\textbf{Step \arabic*:}, wide=0pt, leftmargin=*, itemsep=2pt]
    \item \textbf{Identify Knowns and Goal:}
    \begin{itemize}[itemsep=0pt]
        \item Initial Concentration, $[A]_0 = 2.0 \, \text{M}$
        \item Final Concentration, $[A] = 0.2 \, \text{M}$
        \item Rate Constant, $k = 0.0003201 \, \text{s}^{-1}$ (from Part a)
        \item \textbf{Goal:} Find the time required, $t$.
    \end{itemize}
    
    \item \textbf{Apply and Rearrange the First-Order Rate Equation:}
    We start with Equation 1 again, but this time we solve for the unknown time, $t$.
    $$ [A] = [A]_0 e^{-kt} $$
    $$ \frac{[A]}{[A]_0} = e^{-kt} $$
    Take the natural logarithm of both sides to solve for the exponent:
    $$ \ln\left(\frac{[A]}{[A]_0}\right) = -kt $$
    $$ t = -\frac{1}{k} \ln\left(\frac{[A]}{[A]_0}\right) $$
    
    \item \textbf{Substitute Values and Calculate Time:}
    $$ t = -\frac{1}{0.0003201 \, \text{s}^{-1}} \ln\left(\frac{0.2 \, \text{M}}{2.0 \, \text{M}}\right) $$
    $$ t = -\frac{1}{0.0003201 \, \text{s}^{-1}} \ln(0.1) $$
    $$ t = -\frac{1}{0.0003201 \, \text{s}^{-1}} (-2.3026) $$
    $$ t = 7193.4 \, \text{s} $$
    
    \item \textbf{Conclusion for Part (b):}
    It will take approximately \textbf{7193 seconds} (or about 2.0 hours) for the concentration to drop to 0.2 M.
\end{enumerate}
\end{stepbox}

\newpage
\section*{Temperature Dependence and Reactor Design}

The rate of reaction is highly sensitive to temperature. The \textbf{Arrhenius equation} describes this relationship, which is fundamental for designing reactors that operate at different temperatures.

\begin{keybox}
\begin{itemize}[itemsep=0pt]
    \item $k$: The rate constant (temperature-dependent).
    \item $A$: The pre-exponential or frequency factor (same units as $k$). Assumed constant.
    \item $E_a$: The activation energy, the minimum energy barrier for reaction ($\text{J/mol}$ or $\text{kcal/mol}$).
    \item $R$: The ideal gas constant ($8.314 \, \frac{\text{J}}{\text{mol} \cdot \text{K}}$ or $1.987 \times 10^{-3} \, \frac{\text{kcal}}{\text{mol} \cdot \text{K}}$).
    \item $T$: Absolute temperature in Kelvin (K).
\end{itemize}
\end{keybox}

\begin{formulabox}[title=The Arrhenius Equation]
The temperature dependence of the rate constant is given by:
$$ k = A e^{-E_a / (RT)} \quad \text{(Equation 3)}$$
To determine $E_a$ from experimental data, a linearized form is used by taking the natural logarithm:
$$ \ln(k) = \ln(A) - \frac{E_a}{R} \left(\frac{1}{T}\right) $$
This equation is in the form of a straight line, $y = b + mx$, where a plot of $\ln(k)$ versus $1/T$ yields a slope of $m = -E_a/R$.
\end{formulabox}

\begin{examplebox}{Activation Energy and Temperature Effects}
\textbf{Question:} For a reaction $A \rightarrow \text{Products}$, a plot of $\ln(k)$ versus $1/T$ (where T is in Kelvin) yields a straight line with a slope of $-18000$ K.
\begin{enumerate}[label=(\alph*)]
    \item What is the activation energy ($E_a$) for this reaction? Use $R \approx 1.99 \times 10^{-3}$ kcal/(mol$\cdot$K).
    \item Given $k_1 = 2.15 \, \text{s}^{-1}$ at $T_1 = 600$ K, calculate the rate constant $k_2$ at $T_2 = 700$ K.
\end{enumerate}
\end{examplebox}

\begin{stepbox}[title=Solution Steps for Part (a): Activation Energy]
\begin{enumerate}[label=\textbf{Step \arabic*:}, wide=0pt, leftmargin=*, itemsep=2pt]
    \item \textbf{Relate Slope to Activation Energy:}
    From the linearized Arrhenius equation:
    $$ \text{Slope} = -\frac{E_a}{R} $$
    \item \textbf{Substitute Known Values and Solve for $E_a$:}
    $$ -18000 \, \text{K} = -\frac{E_a}{1.99 \times 10^{-3} \, \text{kcal/(mol}\cdot\text{K)}} $$
    $$ E_a = (18000 \, \text{K}) \times (1.99 \times 10^{-3} \, \text{kcal/(mol}\cdot\text{K)}) = 35.82 \, \text{kcal/mol} $$
    \item \textbf{Conclusion for Part (a):}
    The activation energy is approximately \textbf{36 kcal/mol}.
\end{enumerate}
\end{stepbox}

\begin{stepbox}[title=Solution Steps for Part (b): Rate Constant at New Temperature]
\begin{enumerate}[label=\textbf{Step \arabic*:}, wide=0pt, leftmargin=*, itemsep=2pt]
    \item \textbf{Select the Appropriate Equation Form:}
    
    Use the two-point form of the Arrhenius equation:
    $$ \ln\left(\frac{k_2}{k_1}\right) = \frac{E_a}{R}\left(\frac{1}{T_1} - \frac{1}{T_2}\right) $$
    \item \textbf{Substitute Known Values:}
    From Part (a): $\frac{E_a}{R} = 18000 \, \text{K}$. 
    
    Using $T_1 = 600$ K, $T_2 = 700$ K, $k_1 = 2.15 \, \text{s}^{-1}$:
    $$ \ln\left(\frac{k_2}{2.15 \, \text{s}^{-1}}\right) = (18000 \, \text{K}) \left(\frac{1}{600 \, \text{K}} - \frac{1}{700 \, \text{K}}\right) $$
    $$ \ln\left(\frac{k_2}{2.15}\right) = 18000 (0.0002381) = 4.2857 $$
    \item \textbf{Solve for the New Rate Constant, $k_2$:}
    $$ \frac{k_2}{2.15} = e^{4.2857} \approx 72.65 $$
    $$ k_2 = 72.65 \times 2.15 \, \text{s}^{-1} \approx 156.2 \, \text{s}^{-1} $$
    \item \textbf{Conclusion for Part (b):}
    
    The rate constant at 700 K is approximately \textbf{156 s$^{-1}$}.
\end{enumerate}
\end{stepbox}

\newpage
\section*{CSTR Design Application}
The principles of reaction kinetics are applied directly to design chemical reactors, such as the Continuous Stirred-Tank Reactor (CSTR).

\begin{conceptbox}[title=CSTR Steady-State Material Balance]
For a CSTR operating at steady-state, conditions inside the reactor are uniform and constant. The material balance simplifies to an algebraic equation. The general balance is: Input - Output + Generation = Accumulation. At steady state, Accumulation is zero.
$$ F_{A0} - F_A + r_A V = 0 $$
Where $r_A$ is the rate of \textbf{formation} of A. For a reactant, $r_A$ is negative. For a first-order reaction, the rate of \textbf{consumption} is $-r_A = k C_A$, so the rate of formation is $r_A = -k C_A$.
\end{conceptbox}

\begin{keybox}
\begin{itemize}[itemsep=0pt]
    \item $F_{A0}$: Molar flow rate of A \textbf{into} the reactor ($\text{mol/s}$).
    \item $F_A$: Molar flow rate of A \textbf{out of} the reactor ($\text{mol/s}$).
    \item $V$: Volume of the reactor ($\text{L}$).
    \item $v_0$: Volumetric flow rate of the feed ($\text{L/s}$).
    \item $C_A$: Concentration of A in the reactor and in the outlet stream ($\text{mol/L}$).
    \item $Da$: Damköhler number (dimensionless), $Da = k\tau = kV/v_0$. Represents the ratio of reaction rate to convection rate.
\end{itemize}
\end{keybox}

\begin{formulabox}[title=CSTR Design Equation for a First-Order Reaction]
Substituting $r_A = -k C_A$ and $F_A = C_A v_0$ into the general balance gives:
$$ F_{A0} - F_A - k \left(\frac{F_A}{v_0}\right) V = 0 $$
Solving for the outlet molar flow rate, $F_A$:
$$ F_A = \frac{F_{A0}}{1 + \frac{kV}{v_0}} = \frac{F_{A0}}{1 + Da} \quad \text{(Equation 4)} $$
\end{formulabox}

\begin{examplebox}{Evaluating CSTR Performance}
\textbf{Question:} An existing 50 L CSTR is used for the liquid-phase reaction A $\rightarrow$ Products. The reactor operates isothermally at 600 K, where $k = 2.15 \, \text{s}^{-1}$. The feed enters at $v_0 = 500$ L/s with $C_{A0} = 0.4$ mol/L. To be suitable, the molar flow rate of A exiting the reactor ($F_A$) must be less than 5 mol/s. Is the reactor suitable?
\end{examplebox}

\begin{stepbox}
\begin{enumerate}[label=\textbf{Step \arabic*:}, wide=0pt, leftmargin=*, itemsep=2pt]
    \item \textbf{List Knowns and Goal:}
    \begin{itemize}[itemsep=0pt]
        \item Reactor Volume, $V = 50$ L
        \item Rate Constant, $k = 2.15 \, \text{s}^{-1}$ (at 600 K)
        \item Volumetric Flow Rate, $v_0 = 500$ L/s
        \item Inlet Concentration, $C_{A0} = 0.4$ mol/L
        \item Performance Target: $F_A < 5$ mol/s
        \item \textbf{Goal:} Calculate outlet flow rate $F_A$ and compare it to the target.
    \end{itemize}

    \item \textbf{Calculate the Inlet Molar Flow Rate ($F_{A0}$):}
    $$ F_{A0} = C_{A0} \times v_0 = (0.4 \, \text{mol/L}) \times (500 \, \text{L/s}) = 200 \, \text{mol/s} $$

    \item \textbf{Calculate the Damköhler Number (Da):}
    The Damköhler number is a key dimensionless group for reactor analysis.
    $$ Da = \frac{kV}{v_0} = \frac{(2.15 \, \text{s}^{-1})(50 \, \text{L})}{500 \, \text{L/s}} = \frac{107.5}{500} = 0.215 $$
    A small Da number ($Da \ll 1$) suggests that the residence time is short compared to the reaction time, leading to low conversion.

    \item \textbf{Apply the CSTR Design Equation (Eq. 4) to Find $F_A$:}
    $$ F_A = \frac{F_{A0}}{1 + Da} = \frac{200 \, \text{mol/s}}{1 + 0.215} = \frac{200 \, \text{mol/s}}{1.215} \approx 164.6 \, \text{mol/s} $$
    
    \item \textbf{Compare Result to Requirement and Conclude:}
    The calculated outlet flow rate is $F_A \approx 165$ mol/s. The requirement is $F_A < 5$ mol/s.
    Since $165 \, \text{mol/s} \gg 5 \, \text{mol/s}$, the reactor is \textbf{not suitable}. The conversion is far too low for the given process requirements. To meet the goal, the Damköhler number would need to be significantly increased, for example by increasing reactor volume $V$ or decreasing volumetric flow rate $v_0$.
\end{enumerate}
\end{stepbox}

\newpage
\section*{Conversion with Levenspiel Plots}

This section explores reactor performance analysis using a \textbf{Levenspiel plot}, a powerful graphical method for sizing and comparing different reactor types and configurations for a specific reaction at constant temperature.

\begin{conceptbox}[title=The Levenspiel Plot]
A Levenspiel plot is a graphical tool used in chemical reaction engineering to determine reactor volume. It plots a function related to the inverse reaction rate against the fractional conversion of a reactant. The area on this plot has units of volume, allowing for graphical calculation of reactor sizes.
\end{conceptbox}

\begin{keybox}
\begin{itemize}[itemsep=2pt]
    \item $X$: The fractional conversion of the limiting reactant A, defined as $X = \frac{F_{A0} - F_A}{F_{A0}}$. It is a dimensionless quantity ranging from 0 to 1.
    \item $r_A$: The rate of formation of reactant A (e.g., in $\text{mol/(L}\cdot\text{s)}$). Since A is a reactant being consumed, this value is negative. The rate of reaction is therefore expressed as $-r_A$, which is a positive quantity.
    \item $F_{A0}$: The molar feed rate of reactant A into the reactor (e.g., in $\text{mol/s}$).
    \item $\frac{F_{A0}}{-r_A}$: The y-axis of a standard Levenspiel plot. This group has units of volume: $\frac{\text{mol/s}}{\text{mol/(L}\cdot\text{s)}} = \text{L}$.
\end{itemize}
\end{keybox}

\subsection*{Graphical Interpretation of Reactor Design Equations}
The design equations for CSTRs and PFRs correspond to distinct areas on a Levenspiel plot.

\begin{formulabox}[title=CSTR Design Equation and Graphical Area]
The algebraic design equation for a CSTR is rearranged to isolate the reactor volume, $V$:
$$ V = \frac{F_{A0} X}{-r_A} $$
This can be interpreted graphically as the area of a rectangle on the Levenspiel plot:
$$ V = X \times \left(\frac{F_{A0}}{-r_A}\right)_{\text{at exit conversion X}} $$
\begin{itemize}[itemsep=2pt]
    \item \textbf{Height}: The value of $\frac{F_{A0}}{-r_A}$ evaluated at the final (exit) conversion, $X$.
    \item \textbf{Width}: The total conversion achieved, $X$.
\end{itemize}
\end{formulabox}

\begin{formulabox}[title=PFR Design Equation and Graphical Area]
The differential design equation for a PFR is integrated to find the required volume, $V$:
$$ V = F_{A0} \int_{0}^{X} \frac{dX}{-r_A} $$
This corresponds to the \textbf{area under the curve} of the Levenspiel plot, integrated from an initial conversion (usually 0) to the final conversion, $X$.
$$ V = \int_{0}^{X} \left(\frac{F_{A0}}{-r_A}\right) dX $$
\end{formulabox}

\begin{examplebox}{Reactor Sizing with a Levenspiel Plot}
\textbf{Question:}
Given a Levenspiel plot for a reaction where the y-value, $\frac{F_{A0}}{-r_A}$, is 50 L for conversions between $X=0.1$ and $X=0.5$. We have three available reactors: a 20 L PFR, a 15 L CSTR, and a 25 L CSTR.
\begin{enumerate}[label=(\alph*)]
    \item What single reactor results in the largest conversion, and what is that conversion?
    \item If you use two reactors in series, which two and in what order maximizes overall conversion?
\end{enumerate}
\end{examplebox}

\begin{stepbox}[title=Solution (Part a): Single Reactor Performance]
\begin{enumerate}[label=\textbf{Step \arabic*:}, wide=0pt, leftmargin=*, itemsep=2pt]
    \item \textbf{Analyze the 15 L CSTR:}
    We find the conversion $X$ where the CSTR rectangle area equals 15 L. The height of the rectangle is the y-value at the exit conversion. Assume the y-value is 50 L in the relevant range.
    $$ V_{\text{CSTR}} = X \times \left(\frac{F_{A0}}{-r_A}\right)_{\text{exit}} $$
    $$ 15 \, \text{L} = X \times (50 \, \text{L}) \implies X = \frac{15}{50} = 0.3 $$
    The 15 L CSTR achieves a conversion of \textbf{30\%}.

    \item \textbf{Analyze the 25 L CSTR:}
    Following the same procedure for the 25 L CSTR:
    $$ V_{\text{CSTR}} = X \times \left(\frac{F_{A0}}{-r_A}\right)_{\text{exit}} $$
    $$ 25 \, \text{L} = X \times (50 \, \text{L}) \implies X = \frac{25}{50} = 0.5 $$
    The 25 L CSTR achieves a conversion of \textbf{50\%}.

    \item \textbf{Analyze the 20 L PFR:}
    The PFR volume is the area under the curve. The source problem provides a calculation for the area up to a certain conversion, implying a more complex shape than a simple rectangle. Using the given value from the problem source:
    $$ V_{\text{PFR}} = \text{Area under curve up to X} = 20 \, \text{L} \quad (\text{given to correspond to } X = 0.3) $$
    Therefore, the 20 L PFR achieves a conversion of \textbf{30\%}.

    \item \textbf{Conclusion for Part (a):}
    Comparing the maximum conversion from each available reactor:
    \begin{itemize}[itemsep=0pt]
        \item 15 L CSTR $\rightarrow$ 30\% conversion
        \item 20 L PFR $\rightarrow$ 30\% conversion
        \item \textbf{25 L CSTR $\rightarrow$ 50\% conversion}
    \end{itemize}
    The \textbf{25 L CSTR} provides the highest single-reactor conversion.
\end{enumerate}
\end{stepbox}

\newpage
\begin{stepbox}[title=Solution (Part b): Reactors in Series]
\begin{enumerate}[label=\textbf{Step \arabic*:}, wide=0pt, leftmargin=*, itemsep=2pt]
    \item \textbf{Recall the Reactor Selection Heuristic:}
    The optimal reactor choice depends on the shape of the Levenspiel plot, which reflects how the reaction rate changes with conversion.
    \begin{itemize}
        \item \textbf{PFR is better} where the curve is \textbf{rising} (rate is decreasing). A PFR minimizes the required volume in this region.
        \item \textbf{CSTR is better} where the curve is \textbf{falling} (rate is increasing, e.g., in autocatalytic or some adiabatic reactions). A CSTR minimizes the required volume in this region.
    \end{itemize}

    \item \textbf{Determine the Optimal Sequence:}
    The goal is to arrange two reactors to maximize the total area "captured" on the Levenspiel plot.
    \begin{itemize}
        \item \textbf{First Reactor:} The problem implies the reaction rate is higher at intermediate conversions (the curve dips), making a CSTR more efficient for the first stage of conversion. The 25 L CSTR is the best single reactor, taking us to an initial conversion of $X=0.5$.
        \item \textbf{Second Reactor:} After the CSTR, the feed to the second reactor is already at 50\% conversion. For conversions beyond this point, the Levenspiel curve is typically rising (rate is decreasing). In this region, a PFR is more efficient. Therefore, we should place the 20 L PFR after the CSTR.
    \end{itemize}
    
    \item \textbf{Conclusion for Part (b):}
    To maximize overall conversion, the optimal arrangement is the \textbf{25 L CSTR followed by the 20 L PFR}. This sequence uses each reactor type in the conversion range where it is most volumetrically efficient.
\end{enumerate}
\end{stepbox}

\newpage
\section*{Reactor Performance for Adiabatic Reactions}

This section addresses a common qualitative comparison between reactor types when there are significant heat effects and no heat is added or removed (adiabatic operation).

\begin{examplebox}{PFR vs. CSTR for an Adiabatic Endothermic Reaction}
\textbf{Question:} An \textbf{endothermic} reaction is carried out adiabatically in a 100 L reactor. If a CSTR of this volume achieves 40\% conversion, will the conversion in a PFR of the same volume be greater than, less than, or equal to 40\%?
\end{examplebox}

\begin{stepbox}
\begin{enumerate}[label=\textbf{Step \arabic*:}, wide=0pt, leftmargin=*, itemsep=2pt]
    \item \textbf{Analyze the System Thermodynamics:}
    
    The reaction is endothermic and consumes heat from the surrounding fluid. Since the operation is adiabatic ($Q=0$), as the reaction proceeds and conversion $X$ increases, the temperature $T$ of the fluid must decrease.
    
    \item \textbf{Analyze the System Kinetics:}
    
    The rate constant $k$ follows the Arrhenius equation: $k = A e^{-E_a/RT}$. As temperature $T$ decreases, the reaction rate $-r_A$ also decreases. For this endothermic system, the rate gets slower as conversion gets higher.
    
    \item \textbf{Compare Reactor Temperature Profiles:}
    
    The key difference lies in the temperature distribution within each reactor.
    
    CSTR: Due to perfect mixing, the entire reactor operates at a uniform temperature equal to the final exit temperature, which corresponds to the low temperature at $X=0.4$. The reaction rate is therefore low throughout the entire reactor volume.
    
    PFR: Exhibits a temperature profile. The fluid enters at the high initial feed temperature ($T_0$) and gradually cools as it flows through the reactor. The average temperature in the PFR is higher than the uniform temperature in the CSTR.
    
    \item \textbf{Determine the Final Answer:}
    
    Since the PFR operates at a higher average temperature than the CSTR, it maintains a higher average reaction rate over its entire volume. For the same reactor volume (100 L), a higher average reaction rate results in greater conversion.
    
    Therefore, the conversion in the 100 L PFR will be \textbf{greater than 40\%}.
\end{enumerate}
\end{stepbox}

\newpage

\section*{Multiple Reactions and Selectivity}

In industrial processes, side reactions are common. \textbf{Selectivity} measures how well a reactor configuration produces the desired product compared to undesired byproducts. It is often a more critical performance metric than conversion alone.

\begin{keybox}[title=Selectivity and Yield Definitions]
\begin{itemize}[itemsep=2pt]
    \item \textbf{Instantaneous Selectivity ($S_D$):} The ratio of the rate of formation of the desired product (D) to the rate of formation of the undesired product (U) at any point in the reactor. $S_D = \frac{r_D}{r_U}$.
    \item \textbf{Overall Selectivity ($\bar{S}_D$):} The ratio of the total moles of desired product formed to the total moles of undesired product formed. For flow reactors, $\bar{S}_D = \frac{F_D}{F_U}$.
    \item \textbf{Yield:} The ratio of moles of desired product formed to the moles of reactant fed or consumed.
\end{itemize}
\end{keybox}

\subsection*{Optimizing Selectivity in Parallel Reactions}
Consider a reactant A undergoing two parallel reactions:
$$ A \xrightarrow{k_D} D \quad (\text{Desired, rate } r_D = k_D C_A^{\alpha_D}) $$
$$ A \xrightarrow{k_U} U \quad (\text{Undesired, rate } r_U = k_U C_A^{\alpha_U}) $$

\begin{formulabox}[title=Selectivity for Parallel Reactions]
The instantaneous selectivity is the ratio of the reaction rates.
$$ S_D = \frac{r_D}{r_U} = \frac{k_D C_A^{\alpha_D}}{k_U C_A^{\alpha_U}} = \left(\frac{k_D}{k_U}\right) C_A^{\alpha_D - \alpha_U} $$
To maximize selectivity, we must manipulate concentration ($C_A$) and temperature (which affects $k_D/k_U$).
\end{formulabox}

\textbf{Rule 1: Concentration and Reactor Choice} (When $\alpha_D \neq \alpha_U$)
\begin{itemize}
    \item If $\alpha_D > \alpha_U$ (desired reaction is higher order), we want to keep $C_A$ \textbf{high}.
        \begin{itemize}
            \item \textbf{Best Reactor Choice:} PFR or Batch Reactor.
            \item \textbf{Rationale:} These reactors start at high $C_A$, maximizing selectivity initially.
        \end{itemize}
    \item If $\alpha_D < \alpha_U$ (desired reaction is lower order), we want to keep $C_A$ \textbf{low}.
        \begin{itemize}
            \item \textbf{Best Reactor Choice:} CSTR.
            \item \textbf{Rationale:} A CSTR operates at the low exit concentration, favoring the lower-order reaction. A high recycle ratio in a PFR system can also simulate CSTR behavior.
        \end{itemize}
\end{itemize}

\textbf{Rule 2: Temperature Control}
\begin{formulabox}[title=Temperature Effect on Selectivity]
The ratio of rate constants depends on temperature via the Arrhenius equation.
$$ \frac{k_D}{k_U} = \frac{A_D e^{-E_{a,D}/RT}}{A_U e^{-E_{a,U}/RT}} = \left(\frac{A_D}{A_U}\right) \exp\left(-\frac{E_{a,D}-E_{a,U}}{RT}\right) $$
\end{formulabox}
\begin{itemize}
    \item If $E_{a,D} > E_{a,U}$ (desired reaction has higher activation energy), we want to use a \textbf{high temperature}. This makes the desired reaction rate increase more steeply with temperature than the undesired one.
    \item If $E_{a,D} < E_{a,U}$ (desired reaction has lower activation energy), we want to use a \textbf{low temperature}. This minimizes the rate of the undesired reaction, which is more sensitive to temperature.
\end{itemize}

\subsection*{Optimizing Selectivity in Series Reactions}
Consider the series reaction where B is the desired intermediate product:
$$ A \xrightarrow{k_1} B \xrightarrow{k_2} C $$
\begin{itemize}
    \item The concentration of the desired product B, $C_B$, will rise to a maximum and then fall as it is converted to the undesired product C.
    \item The most important variable for controlling selectivity is \textbf{reaction time} (in a batch reactor) or \textbf{space time} $\tau = V/v_0$ (in a flow reactor).
    \item \textbf{PFR/Batch:} These reactors are generally preferred because they prevent the newly formed product B from immediately mixing with reactants that can further convert it, as would happen in a CSTR.
    \item \textbf{Optimization Goal:} The reactor must be sized to achieve the optimal residence time that corresponds to the peak concentration of B. A residence time that is too short results in low conversion of A; a residence time that is too long converts the valuable product B into waste product C.
\end{itemize}

\newpage

\subsection*{Example: Selectivity Dependent on Product Concentration}

\begin{examplebox}{Reactor Choice for an Autocatalytic-like Reaction}
\textbf{Question:} The desired reaction $A + B \rightarrow 2D$ is accompanied by an undesired side reaction $A + B \rightarrow U_1 + U_2$. The rate laws are given below. If the reaction is run to 70\% conversion, would you expect the selectivity toward the desired product D to be higher in a Plug Flow Reactor (PFR) or a Continuous Stirred-Tank Reactor (CSTR)?
\begin{align*}
    r_D &= \frac{k_1 P_A P_B}{1 + K_D P_D} \quad (\text{Desired}) \\
    r_{U_1} &= \frac{k_2 P_A P_B}{(1 + K_D P_D)^2} \quad (\text{Undesired})
\end{align*}
\end{examplebox}

\begin{stepbox}
\begin{enumerate}[label=\textbf{Step \arabic*:}, wide=0pt, leftmargin=*, itemsep=2pt]
    \item \textbf{Find the Selectivity Expression:}
    Calculate the ratio of desired to undesired reaction rates:
    $$ S_D = \frac{r_D}{r_{U_1}} = \frac{\frac{k_1 P_A P_B}{1 + K_D P_D}}{\frac{k_2 P_A P_B}{(1 + K_D P_D)^2}} $$
    Simplifying:
    $$ S_D = \left(\frac{k_1}{k_2}\right) (1 + K_D P_D) $$
    
    \item \textbf{Understand the Selectivity Behavior:}
    
    The expression shows that selectivity increases with product D concentration ($P_D$). Higher $P_D$ means better selectivity toward the desired product. This means we want to maximize $P_D$ throughout the reactor.
    
    \item \textbf{Compare Reactor Types:}
    
    CSTR: The entire reactor operates at exit conditions. For 70\% conversion, the whole reactor has high $P_D$, giving consistently high selectivity throughout.
    
    PFR: Starts with zero $P_D$ at the inlet and gradually builds up to high $P_D$ at the exit. Average $P_D$ is lower than in the CSTR.
    
    \item \textbf{Conclusion:}
    Since higher $P_D$ gives better selectivity, and the CSTR maintains high $P_D$ throughout its volume, a \textbf{CSTR} will achieve higher overall selectivity.
\end{enumerate}
\end{stepbox}

\newpage
\subsection*{Example: Selectivity in CSTRs in Series}

\begin{examplebox}{Effect of a Second CSTR on Selectivity}
\textbf{Question:} A CSTR is used for the liquid-phase reaction $2A + B \rightarrow D$. A large stoichiometric excess of A is fed to the reactor. A parallel reaction $A + 2B \rightarrow U$ also occurs. Both reactions follow elementary rate laws. The selectivity of D to U ($S_{D/U}$) at the CSTR outlet is 2.0. If a second CSTR of equal volume is added in series, will the selectivity at the exit of the second reactor be greater than, less than, or equal to 2.0?
\end{examplebox}

\begin{stepbox}
\begin{enumerate}[label=\textbf{Step \arabic*:}, wide=0pt, leftmargin=*, itemsep=2pt]
    \item \textbf{Determine the Selectivity Expression:}
    Write the elementary rate laws and find the instantaneous selectivity expression.
    
    Rate of desired reaction: $r_D = k_D C_A^2 C_B$. Rate of undesired reaction: $r_U = k_U C_A C_B^2$.
    \begin{formulabox}[title=Selectivity Expression]
    $$ S_D = \frac{r_D}{r_U} = \frac{k_D C_A^2 C_B}{k_U C_A C_B^2} = \left(\frac{k_D}{k_U}\right) \left(\frac{C_A}{C_B}\right) $$
    \end{formulabox}
    This shows selectivity is proportional to the concentration ratio $C_A/C_B$.
    
    \item \textbf{Analyze Concentration Changes in the Second CSTR:}
    Determine how the ratio $C_A/C_B$ changes from the first to second reactor exit. Since temperature is constant, $k_D/k_U$ remains constant.
    
    Concentration of A ($C_A$): Since A is in large stoichiometric excess, the amount consumed is negligible. Therefore, $C_A$ remains approximately constant through both reactors.
    
    Concentration of B ($C_B$): B is the limiting reactant and is consumed in both reactors. The second CSTR consumes additional B, so $C_{B, \text{exit 2}} < C_{B, \text{exit 1}}$.
    
    \item \textbf{Conclusion:}
    The selectivity at the exit of the second reactor:
    $$ S_{D,2} \propto \frac{C_{A, \text{exit 2}}}{C_{B, \text{exit 2}}} $$
    Since $C_A$ stays constant and $C_B$ decreases, the ratio $C_A/C_B$ increases. Because selectivity is directly proportional to this ratio, the selectivity at the exit of the second reactor will be \textbf{greater than 2.0}.
\end{enumerate}
\end{stepbox}

\newpage
\subsection*{Example: Isothermal vs. Adiabatic Operation for Selectivity}

\begin{examplebox}{Temperature Effects on Selectivity and Catalyst Requirement}
\textbf{Question:} The selective oxidation of ethylene (A) to ethylene oxide (D) has an activation energy of 15 kcal/mol. The complete combustion of ethylene to $CO_2$ (U) has an activation energy of 20 kcal/mol. Both reactions are exothermic. The feed is fixed, and the final ethylene conversion is specified to be 20\%.
\begin{enumerate}[label=(\alph*)]
    \item To achieve high selectivity to ethylene oxide, is it better to run the reaction isothermally or adiabatically?
    \item Which reactor type (isothermal or adiabatic) would require a larger amount of catalyst?
\end{enumerate}
\end{examplebox}

\begin{stepbox}[title=Solution (Part a): Isothermal vs. Adiabatic for Selectivity]
\begin{enumerate}[label=\textbf{Step \arabic*:}, wide=0pt, leftmargin=*, itemsep=2pt]
    \item \textbf{Analyze Selectivity's Dependence on Temperature:}
    Selectivity ($S_D$) is proportional to $k_D/k_U$. We must determine how temperature affects this ratio based on the activation energies.
    
    Activation energy for desired reaction: $E_{a,D} = 15$ kcal/mol. Activation energy for undesired reaction: $E_{a,U} = 20$ kcal/mol.
    \begin{conceptbox}[title=The Rule of Temperature and Selectivity]
    Higher temperature always favors the reaction with the higher activation energy.
    \end{conceptbox}
    Since the undesired combustion reaction has the higher activation energy ($E_{a,U} > E_{a,D}$), to maximize selectivity for the desired product, we must operate at the \textbf{lowest possible temperature}.
    
    \item \textbf{Compare Reactor Temperature Profiles:}
    Both reactions are exothermic and release heat, which will raise the fluid temperature unless removed.
    
    Isothermal Reactor: Actively cooled to maintain a constant, low temperature (typically the inlet temperature).
    
    Adiabatic Reactor: Insulated, so the heat generated by the exothermic reactions causes the fluid temperature to rise continuously, resulting in a much higher average operating temperature.
    
    \item \textbf{Conclusion for Selectivity:}
    Since low temperature favors the desired product, and an isothermal reactor operates at a lower temperature than an adiabatic one, \textbf{isothermal operation} is better for achieving high selectivity.
\end{enumerate}
\end{stepbox}

\begin{stepbox}[title=Solution (Part b): Catalyst Requirement]
\begin{enumerate}[label=\textbf{Step \arabic*:}, wide=0pt, leftmargin=*, itemsep=2pt]
    \item \textbf{Relate Catalyst Amount to Reaction Rate:}
    The amount of catalyst required to achieve a fixed conversion is inversely proportional to the overall reaction rate. A faster reaction requires less catalyst (and a smaller reactor volume).
    $$ W_{\text{catalyst}} \propto \frac{1}{\text{Reaction Rate}} $$

    \item \textbf{Compare Overall Reaction Rates:}
    While higher temperature hurts selectivity here, it increases the absolute rates of \textit{both} reactions according to the Arrhenius equation. The adiabatic reactor operates at a much higher average temperature than the isothermal reactor. Therefore, the overall rate of ethylene consumption is much faster in the adiabatic reactor.

    \item \textbf{Conclusion for Catalyst Amount:}
    Because reaction rates are significantly higher at the elevated temperatures of the adiabatic reactor, less catalyst is needed to reach the 20\% conversion target. The \textbf{isothermal reactor would require a larger amount of catalyst}.
\end{enumerate}
\end{stepbox}

\begin{keybox}[title=The Engineering Trade-Off]
This problem highlights a classic engineering decision:
\begin{itemize}[itemsep=2pt]
    \item \textbf{Isothermal Reactor:} Higher selectivity (more valuable product, lower separation costs) but higher capital cost (larger reactor, more catalyst, cooling equipment).
    \item \textbf{Adiabatic Reactor:} Lower selectivity (less valuable product, higher separation costs) but lower capital cost (smaller reactor, less catalyst, no cooling needed).
\end{itemize}
The final choice depends on a detailed economic analysis.
\end{keybox}

\newpage

nd{document}
