\documentclass[12pt]{article}
\usepackage[paperwidth=8.5in, paperheight=11in, margin=1.0in, headheight=12pt]{geometry}
\usepackage{amsmath,amssymb,amsthm}
\usepackage[most]{tcolorbox}
\usepackage{enumitem}
\usepackage{xcolor}
\usepackage{hyperref}
\usepackage{fancyhdr}
\usepackage{titlesec}
\usepackage{graphicx}
% Define custom colors for chemical engineering theme
\definecolor{conceptcolor}{RGB}{52, 73, 94}      % Dark blue-gray
\definecolor{formulacolor}{RGB}{231, 76, 60}     % Red for formulas
\definecolor{examplecolor}{RGB}{39, 174, 96}     % Green for examples
\definecolor{stepcolor}{RGB}{142, 68, 173}       % Purple for solution steps
\definecolor{keycolor}{RGB}{243, 156, 18}        % Orange for key points
% Configure fancy headers
\pagestyle{fancy}
\fancyhf{}
\fancyhead[L]{PE Study Guide}
\fancyhead[R]{Process Fundamentals}
\fancyfoot[C]{\thepage}
\renewcommand{\baselinestretch}{1.1}
\setlength{\parindent}{0.25in}
\setlength{\parskip}{3pt}
% Configure section formatting
\titleformat{\section}
  {\normalfont\LARGE\bfseries\color{conceptcolor}}
  {\thesection}{1em}{}
\titleformat{\subsection}
  {\normalfont\Large\bfseries\color{conceptcolor}}
  {\thesubsection}{1em}{}
% Define custom environments
\newtcolorbox{conceptbox}[1][]{
  enhanced,
  colback=conceptcolor!10,
  colframe=conceptcolor,
  arc=3mm,
  title=Key Concept,
  fonttitle=\bfseries\sffamily\normalsize,
  fontupper=\small,
  #1
}
\newtcolorbox{formulabox}[1][]{
  enhanced,
  colback=formulacolor!10,
  colframe=formulacolor,
  arc=2mm,
  title=Important Formula,
  fonttitle=\bfseries\sffamily\normalsize,
  fontupper=\small,
  #1
}
\newtcolorbox{examplebox}[2][]{
  enhanced,
  colback=examplecolor!10,
  colframe=examplecolor,
  arc=3mm,
  title=Example Problem: #2,
  fonttitle=\bfseries\sffamily\normalsize,
  fontupper=\small,
  #1
}
\newtcolorbox{stepbox}[1][]{
  enhanced,
  colback=stepcolor!10,
  colframe=stepcolor,
  arc=2mm,
  title=Solution Steps,
  fonttitle=\bfseries\sffamily\normalsize,
  fontupper=\small,
  #1
}
\newtcolorbox{keybox}[1][]{
  enhanced,
  colback=keycolor!10,
  colframe=keycolor,
  arc=2mm,
  title=Key Variables \& Definitions,
  fonttitle=\bfseries\sffamily\normalsize,
  fontupper=\small,
  #1
}
\section*{Advanced Material Balance Examples}

This section applies foundational material balance principles to complex systems involving recycle loops and multiple unit operations. These examples demonstrate how to break down complex processes into solvable parts by choosing appropriate system boundaries.

\begin{conceptbox}[title=Complex System Analysis Strategy]
For multi-unit processes:
\begin{itemize}[itemsep=0pt]
    \item Choose system boundaries strategically to isolate unknowns
    \item Start with overall system balances before subsystem analysis
    \item Use degrees of freedom analysis to verify solvability
    \item Internal recycle streams don't cross overall system boundaries
\end{itemize}
\end{conceptbox}

\subsection*{Example 1: Evaporative Crystallization with Recycle}

\begin{examplebox}{Evaporative Crystallization with Recycle}
\textbf{Problem:} A continuous process involves an evaporator and crystallizer. Fresh feed (400 kg/h, 15 wt\% KCl) mixes with recycle, enters an evaporator where water is removed. The concentrated solution (38 wt\% KCl) goes to a crystallizer/filter. Output is filter cake (solid KCl crystals + saturated solution at 28 wt\% KCl). Crystals are 84\% of filter cake mass. Remaining filtrate (28 wt\% KCl) is recycled.

Find: (a) KCl crystal flow rate, (b) water vapor flow rate, (c) recycle flow rate.
\end{examplebox}

\begin{keybox}
\textbf{Process Stream Definitions:}
\begin{itemize}[itemsep=0pt]
    \item $\dot{m}_1$: Fresh feed = 400 kg/h, $x_1 = 0.15$
    \item $\dot{m}_2$: Water vapor from evaporator, $x_2 = 0$
    \item $\dot{m}_3$: Concentrated solution to crystallizer, $x_3 = 0.38$
    \item $\dot{m}_4$: Solid KCl crystals, $x_4 = 1.0$
    \item $\dot{m}_5$: Saturated solution with crystals, $x_5 = 0.28$
    \item $\dot{m}_6$: Recycle stream, $x_6 = 0.28$
\end{itemize}

\textbf{Given Constraint:} $\frac{\dot{m}_4}{\dot{m}_4 + \dot{m}_5} = 0.84$
\end{keybox}

\begin{conceptbox}[title=Solution Strategy]
\textbf{Two-Step Approach:}
\begin{enumerate}[itemsep=0pt]
    \item Overall system balance (excludes internal recycle)
    \item Crystallizer/filter subsystem balance (includes recycle)
\end{enumerate}
\end{conceptbox}

\begin{stepbox}
\begin{enumerate}[label=\textbf{Step \arabic*:}, wide=0pt, leftmargin=*, itemsep=2pt]
    \item \textbf{Overall System Analysis}
    
    \textbf{DoF Check:} 3 unknowns ($\dot{m}_2, \dot{m}_4, \dot{m}_5$), 3 equations (overall mass, KCl balance, filter cake relation) → DoF = 0 
    
    \textbf{Filter Cake Relation:}
    $\dot{m}_4 = 0.84(\dot{m}_4 + \dot{m}_5) \Rightarrow \dot{m}_5 = 0.1905\dot{m}_4$
    
    \item \textbf{Overall KCl Balance}
    
    KCl In = KCl Out: $\dot{m}_1 x_1 = \dot{m}_4 x_4 + \dot{m}_5 x_5$
    
    $(400)(0.15) = \dot{m}_4(1.0) + \dot{m}_5(0.28)$
    
    $60 = \dot{m}_4 + 0.28(0.1905\dot{m}_4) = 1.05334\dot{m}_4$
    
    $\dot{m}_4 = 56.96$ kg/h, $\dot{m}_5 = 10.85$ kg/h
    
    \item \textbf{Overall Mass Balance}
    
    Total In = Total Out: $\dot{m}_1 = \dot{m}_2 + \dot{m}_4 + \dot{m}_5$
    
    $400 = \dot{m}_2 + 56.96 + 10.85 \Rightarrow \dot{m}_2 = 332.19$ kg/h
    
    \item \textbf{Crystallizer/Filter Subsystem}
    
    \textbf{DoF Check:} 2 unknowns ($\dot{m}_3, \dot{m}_6$), 2 equations → DoF = 0 
    
    \textbf{Mass Balance:} $\dot{m}_3 = (\dot{m}_4 + \dot{m}_5) + \dot{m}_6 = 67.81 + \dot{m}_6$
    
    \textbf{KCl Balance:} $\dot{m}_3(0.38) = 56.96(1.0) + 10.85(0.28) + \dot{m}_6(0.28)$
    
    $(67.81 + \dot{m}_6)(0.38) = 59.998 + 0.28\dot{m}_6$
    
    $25.77 + 0.38\dot{m}_6 = 59.998 + 0.28\dot{m}_6$
    
    $\dot{m}_6 = 342.28$ kg/h
    
    \item \textbf{Final Answers}
    
    (a) KCl crystals: \textbf{57.0 kg/h}
    
    (b) Water vapor: \textbf{332.2 kg/h}
    
    (c) Recycle flow: \textbf{342.3 kg/h}
\end{enumerate}
\end{stepbox}

\subsection*{Example 2: Distillation with Condenser and Reboiler}

\begin{examplebox}{Distillation Column Analysis}
\textbf{Problem:} Feed to distillation column is 45 mol\% n-pentane/55 mol\% n-hexane. Vapor leaving top (98 mol\% pentane) goes to total condenser. Half of condensate returns as reflux, half withdrawn as overhead product at 85 kmol/h. Overhead contains 95\% of feed pentane. Bottom liquid goes to reboiler.

\textbf{Part A:} Find feed flow rate and bottoms composition.

\textbf{Part B:} Find vapor temperature at 1 atm and volumetric flow rates.
\end{examplebox}

\begin{keybox}
\textbf{Given Information:}
\begin{itemize}[itemsep=0pt]
    \item Feed: 45 mol\% pentane, 55 mol\% hexane
    \item Vapor from column: 98 mol\% pentane
    \item Overhead product: 85 kmol/h (50\% of condensate)
    \item Overhead contains 95\% of feed pentane
    \item Operating pressure: 1 atm (760 mmHg)
\end{itemize}
\end{keybox}

\begin{stepbox}
\begin{enumerate}[label=\textbf{Step \arabic*:}, wide=0pt, leftmargin=*, itemsep=2pt]
    \item \textbf{Find Feed Flow Rate}
    
    Pentane recovery relationship: $(N_{overhead})(y_{p,overhead}) = 0.95 \times (N_{feed})(x_{p,feed})$
    
    $(85)(0.98) = 0.95 \times (N_{feed})(0.45)$
    
    $83.3 = N_{feed} \times 0.4275 \Rightarrow N_{feed} = 194.85$ kmol/h
    
    \item \textbf{Find Bottoms Flow Rate}
    
    Overall molar balance: $N_{feed} = N_{overhead} + N_{bottoms}$
    
    $194.85 = 85 + N_{bottoms} \Rightarrow N_{bottoms} = 109.85$ kmol/h
    
    \item \textbf{Find Bottoms Composition}
    
    Overall pentane balance: $N_{feed} \times 0.45 = (N_{overhead} \times 0.98) + (N_{bottoms} \times x_{p,bottoms})$
    
    $194.85 \times 0.45 = 85 \times 0.98 + 109.85 \times x_{p,bottoms}$
    
    $87.68 = 83.3 + 109.85 x_{p,bottoms} \Rightarrow x_{p,bottoms} = 0.040$
    
    \textbf{Bottoms:} 4.0\% pentane, 96.0\% hexane
\end{enumerate}
\end{stepbox}

\begin{formulabox}[title=VLE and Volumetric Calculations]
\textbf{Dew Point Calculation (Raoult's Law):} $$ \sum \frac{y_i P}{P_i^{sat}(T)} = 1 $$
\textbf{Antoine Equation:} $$ P_i^{sat} = 10^{(A - \frac{B}{T+C})} $$
\textbf{Liquid Volume:} $$ \dot{V}_i = \frac{\dot{n}_i \times MW_i}{\rho_i} $$
\end{formulabox}

\begin{keybox}
\textbf{Antoine Constants (T in \textdegree C, P in mmHg):} \\
n-Pentane: A = 6.85221, B = 1064.63, C = 232.0 \\
n-Hexane: A = 6.87776, B = 1171.17, C = 224.41

\vspace{0.5em} % Small vertical space for separation

\textbf{Physical Properties:} \\
Pentane: MW = 72.15 kg/kmol, $\rho$ = 621 kg/m\textsuperscript{3} \\
Hexane: MW = 86.18 kg/kmol, $\rho$ = 659 kg/m\textsuperscript{3}
\end{keybox}

\begin{stepbox}
\begin{enumerate}[label=\textbf{Step \arabic*:}, wide=0pt, leftmargin=*, itemsep=2pt]
    \item \textbf{Apply dew point equation with Antoine constants:}

    $$ \frac{0.98 \times 760}{10^{(6.85221 - \frac{1064.63}{T+232.0})}} + \frac{0.02 \times 760}{10^{(6.87776 - \frac{1171.17}{T+224.41})}} = 1 $$
    
    Solving iteratively: $T = 37.3 C$ (310.45 K)
    
    \item \textbf{Vapor Volumetric Flow Rate}
    
    Total vapor to condenser: $2 \times 85 = 170$ kmol/h (reflux + product)
    
    $\dot{V}_{vapor} = \frac{(170)(0.08206)(310.45)}{1} = 4325$ m³/h
    
    \item \textbf{Liquid Product Volumetric Flow Rate}
    
    \textbf{Pentane volume:} $\dot{V}_{p} = \frac{(85 \times 0.98)(72.15)}{621} = 9.68$ m³/h
    
    \textbf{Hexane volume:} $\dot{V}_{h} = \frac{(85 \times 0.02)(86.18)}{659} = 0.22$ m³/h
    
    \textbf{Total liquid:} $\dot{V}_{liquid} = 9.68 + 0.22 = 9.90$ m³/h
    
    \item \textbf{Part B Final Results}
    
    Vapor temperature: \textbf{37.3°C} | Vapor flow rate: \textbf{4325 m³/h} | Liquid product flow rate: \textbf{9.90 m³/h}
\end{enumerate}
\end{stepbox}

\newpage

\subsection*{Example 3: Multi-Unit Crystallization with Recycle and Drying}

\begin{examplebox}{Multi-Unit Crystallization with Recycle and Drying}
\textbf{Problem:} Potassium dichromate ($K_2Cr_2O_7$, KD) is recovered from a 25 wt\% aqueous solution. Fresh feed joins recycle and feeds crystallizer/centrifuge where water evaporates causing crystallization. Saturated solution contains 15 wt\% KD. Slurry split: 90\% solution recycled, 10\% leaves with crystals as filter cake (85 wt\% solid KD, 15 wt\% saturated solution). Filter cake goes to dryer where dry air evaporates remaining water, leaving pure KD. Air leaves dryer with 0.08 mole fraction water.

For 1000 kg/h pure KD production, find:
\begin{enumerate}[itemsep=0pt]
    \item Water evaporated in crystallizer/centrifuge
    \item Recycle stream mass flow rate  
    \item Molar flow rate of dry air to dryer
\end{enumerate}
\end{examplebox}

\begin{conceptbox}[title=Solution Strategy]
Analyze different system boundaries strategically. Key insight: balances on non-reactive species (like KD) across boundaries where only one stream contains that species can directly solve for unknown flow rates.
\end{conceptbox}

\begin{keybox}
\textbf{Process Stream Analysis:}
\begin{itemize}[itemsep=0pt]
    \item Fresh feed: 25 wt\% KD aqueous solution
    \item Saturated solution: 15 wt\% KD, 85 wt\% water
    \item Filter cake: 85 wt\% solid KD crystals, 15 wt\% saturated solution
    \item Split ratio: 90\% solution recycled, 10\% with filter cake
    \item Final product: 1000 kg/h pure KD crystals
    \item Air outlet: 0.08 mole fraction water
\end{itemize}
\end{keybox}

\begin{stepbox}
\begin{enumerate}[label=\textbf{Step \arabic*:}, wide=0pt, leftmargin=*, itemsep=2pt]
    \item \textbf{Analyze Dryer - Find Filter Cake Flow Rate}
    
    \textbf{Dryer KD Balance:} Only KD-containing streams are filter cake (in) and pure crystals (out)
    
    KD in filter cake = (crystals) + (dissolved in solution)
    
    $(\dot{m}_{cake} \times 0.85) + (\dot{m}_{cake} \times 0.15 \times 0.15) = 1000$ kg/h
    
    $\dot{m}_{cake}(0.85 + 0.0225) = 1000 \Rightarrow \dot{m}_{cake} = \frac{1000}{0.8725} = 1146.1$ kg/h
    
    \item \textbf{Overall System Analysis - Find Fresh Feed Rate}
    
    \textbf{Overall KD Balance:} Only KD enters in fresh feed, only leaves as final product
    
    $\dot{m}_{feed} \times 0.25 = 1000 \Rightarrow \dot{m}_{feed} = 4000$ kg/h
    
    \item \textbf{Find Evaporated Water (Answer 1)}
    
    \textbf{Mass Balance around Mixing Point + Crystallizer:}
    
    Boundary includes feed mixing and crystallizer (recycle is internal)
    
    Total Mass In = Total Mass Out: $\dot{m}_{feed} = \dot{m}_{evap} + \dot{m}_{cake}$
    
    $4000 = \dot{m}_{evap} + 1146.1 \Rightarrow \dot{m}_{evap} = 2853.9$ kg/h
    
    \textbf{Answer 1: 2854 kg/h}
    
    \item \textbf{Find Recycle Stream Rate (Answer 2)}
    
    \textbf{Solution mass in filter cake:} $\dot{m}_{soln,cake} = 1146.1 \times 0.15 = 171.9$ kg/h
    
    This is the 10\% portion. Recycle is the 90\% portion:
    
    $\frac{\dot{m}_{recycle}}{\dot{m}_{soln,cake}} = \frac{90\%}{10\%} = 9$
    
    $\dot{m}_{recycle} = 9 \times 171.9 = 1547.1$ kg/h
    
    \textbf{Answer 2: 1547 kg/h}
    
    \item \textbf{Find Air Flow Rate to Dryer (Answer 3)}
    
    \textbf{Water entering dryer:} $\dot{m}_{H_2O,in} = 171.9 \times 0.85 = 146.1$ kg/h
    
    \textbf{Convert to molar:} $\dot{n}_{H_2O,out} = \frac{146.1}{18.015} = 8.11$ kmol/h
    
    \textbf{Total outlet stream:} $\dot{n}_{total,out} = \frac{8.11}{0.08} = 101.38$ kmol/h
    
    \textbf{Dry air flow:} $\dot{n}_{air,in} = 101.38 - 8.11 = 93.27$ kmol/h
    
    \textbf{Answer 3: 93.3 kmol/h}
\end{enumerate}
\end{stepbox}

\newpage

\subsection*{Example 4: Leaching Process with Multiple Recycle Streams}

\begin{examplebox}{Coffee Decaffeination Process}
\textbf{Problem:} 100 kg coffee beans (1.5 kg caffeine, 98.5 kg inert solids) fed to mixing tank. Decaffeinating solvent (DCS) extracts 90\% of initial caffeine. Treated beans carry 20 kg DCS to dryer. Dryer recovers 90\% of this DCS (recycled to mixer), 10\% leaves with final beans.

Solvent-caffeine solution from mixer (88 wt\% DCS) enters settling unit, splits into waste stream (5 wt\% DCS) and cleaned solvent recycled to mixing point with fresh DCS. Combined solvent entering mixer contains 95 wt\% DCS.

Find: (a) fresh DCS mass required per 100 kg beans, (b) composition of recycle from settling unit.
\end{examplebox}

\begin{conceptbox}[title=Solution Strategy]
Degree of Freedom analysis reveals overall system is solvable first. Work inward, solving for internal streams using systematic boundary selection.
\end{conceptbox}

\begin{keybox}
\textbf{Process Stream Information:}
\begin{itemize}[itemsep=0pt]
    \item Coffee beans: 100 kg (1.5 kg caffeine, 98.5 kg solids)
    \item Caffeine removal: 90\% of initial (1.35 kg removed)
    \item DCS to dryer: 20 kg
    \item DCS recovery from dryer: 90\% (18 kg recycled)
    \item DCS in final beans: 10\% (2 kg)
    \item Mixer outlet: 88 wt\% DCS
    \item Waste stream: 5 wt\% DCS  
    \item Combined solvent to mixer: 95 wt\% DCS
\end{itemize}
\end{keybox}

\begin{stepbox}
\begin{enumerate}[label=\textbf{Step \arabic*:}, wide=0pt, leftmargin=*, itemsep=2pt]
    \item \textbf{Pre-calculations from Problem Statement}
    
    \textbf{Caffeine removed:} $1.5 \times 0.90 = 1.35$ kg
    
    \textbf{Caffeine in final beans:} $1.5 \times 0.10 = 0.15$ kg
    
    \textbf{DCS recycled from dryer:} $20 \times 0.90 = 18$ kg
    
    \textbf{DCS in final beans:} $20 \times 0.10 = 2$ kg
    
    \item \textbf{Overall System DoF Analysis}
    
    \textbf{Boundary:} Fresh beans + Fresh DCS → Final beans + Waste solution
    
    \textbf{Unknowns:} $\dot{m}_{fresh}$, $\dot{m}_{waste}$ (2 variables)
    
    \textbf{Equations:} Caffeine balance, DCS balance (2 equations)
    
    \textbf{DoF = 2 - 2 = 0} (solvable)
    
    \item \textbf{Overall Caffeine Balance}
    
    Caffeine in beans = Caffeine in final beans + Caffeine in waste
    
    Waste is 5\% DCS, so 95\% caffeine:
    
    $1.5 = 0.15 + (\dot{m}_{waste} \times 0.95)$
    
    $1.35 = 0.95\dot{m}_{waste} \Rightarrow \dot{m}_{waste} = 1.42$ kg
    
    \item \textbf{Overall DCS Balance (Answer a)}
    
    DCS in fresh feed = DCS in final beans + DCS in waste
    
    $\dot{m}_{fresh} = 2 + (1.42 \times 0.05) = 2 + 0.071 = 2.071$ kg
    
    \textbf{Answer (a): 2.07 kg fresh DCS per 100 kg beans}
\end{enumerate}
\end{stepbox}

\begin{stepbox}
\begin{enumerate}[label=\textbf{Step \arabic*:}, wide=0pt, leftmargin=*, itemsep=2pt]
    \setcounter{enumi}{4}
    \item \textbf{Mixing Tank Balance Analysis}
    
    \textbf{Boundary:} Around mixing tank only
    
    \textbf{Unknowns:} $\dot{m}_{in,tank}$ (combined solvent in), $\dot{m}_{out,tank}$ (solution out)
    
    \textbf{DoF:} 2 unknowns - 2 balances = 0
    
    \item \textbf{Caffeine Balance on Mixing Tank}
    
    Caffeine in beans + Caffeine in solvent = Caffeine in solution + Caffeine with beans
    
    $1.5 + (\dot{m}_{in,tank} \times 0.05) = (\dot{m}_{out,tank} \times 0.12) + 0.15$
    
    \item \textbf{DCS Balance on Mixing Tank}
    
    DCS in solvent + DCS from dryer = DCS in solution + DCS with beans
    
    $(\dot{m}_{in,tank} \times 0.95) + 18 = (\dot{m}_{out,tank} \times 0.88) + 20$
    
    \textbf{Solving simultaneously:} $\dot{m}_{in,tank} = 20.4$ kg, $\dot{m}_{out,tank} = 19.75$ kg
    
    \item \textbf{Mixing Point Balance (Answer b)}
    
    \textbf{Overall mass balance:} $\dot{m}_{fresh} + \dot{m}_{recycle} = \dot{m}_{in,tank}$
    
    $2.071 + \dot{m}_{recycle} = 20.4 \Rightarrow \dot{m}_{recycle} = 18.33$ kg
    
    \textbf{DCS balance:} $(\dot{m}_{fresh} \times 1.0) + (\dot{m}_{recycle} \times x_{DCS,recycle}) = (\dot{m}_{in,tank} \times 0.95)$
    
    $(2.071 \times 1.0) + (18.33 \times x_{DCS,recycle}) = (20.4 \times 0.95)$
    
    $2.071 + 18.33x_{DCS,recycle} = 19.38$
    
    $x_{DCS,recycle} = \frac{17.309}{18.33} = 0.944$
    
    \textbf{Answer (b): Recycle stream is 94.4\% DCS, 5.6\% caffeine}
\end{enumerate}
\end{stepbox}

\begin{formulabox}[title=Final Results]
\textbf{(a) Fresh DCS Required:} 2.07 kg per 100 kg coffee beans

\textbf{(b) Recycle Stream Composition:} 94.4 wt\% DCS, 5.6 wt\% caffeine
\end{formulabox}

\newpage

\section*{Combustion Reactor with Product Recycle}

\begin{examplebox}{Butane Combustion with Product Recycle}
\textbf{Problem:} Butane ($C_4H_{10}$) is burned with 25\% excess oxygen. Combined feed to reactor: 100 mol/h butane. Single-pass conversion: 45\%. Mole ratio $CO_2$:$CO$ in reactor outlet: 9:1. Stream from reactor goes to separator where 90\% of unreacted butane and oxygen are recycled. Remaining 10\% leaves as waste.

Find:
\begin{enumerate}[itemsep=0pt]
    \item Molar composition of product gas from reactor
    \item Molar flow rates of butane and oxygen in fresh feed
    \item Overall conversion of butane in process
\end{enumerate}
\end{examplebox}

\begin{conceptbox}[title=Solution Strategy]
Use reactor information to solve outlet stream first. Then analyze separator and mixing point for fresh feed and recycle rates. Since specific CO formation reaction isn't given, use atomic species balances.
\end{conceptbox}

\begin{keybox}
\textbf{Given Information:}
\begin{itemize}[itemsep=0pt]
    \item Combined feed to reactor: 100 mol/h butane
    \item Excess oxygen: 25\% above stoichiometric
    \item Single-pass conversion: 45\%
    \item $CO_2$:$CO$ mole ratio: 9:1
    \item Separator recovery: 90\% of unreacted gases recycled
    \item Complete combustion stoichiometry: $C_4H_{10} + 6.5O_2 \rightarrow 4CO_2 + 5H_2O$
\end{itemize}
\end{keybox}

\begin{stepbox}
\begin{enumerate}[label=\textbf{Step \arabic*:}, wide=0pt, leftmargin=*, itemsep=2pt]
    \item \textbf{Reactor Feed Analysis}
    
    \textbf{Basis:} 100 mol/h butane to reactor
    
    \textbf{Theoretical $O_2$:} $100 \times 6.5 = 650$ mol/h
    
    \textbf{Actual $O_2$ fed:} $650 \times 1.25 = 812.5$ mol/h
    
    \item \textbf{Reactor Outlet - Unreacted Butane}
    
    With 45\% conversion: $\dot{n}_{C_4H_{10},out} = 100 \times (1-0.45) = 55$ mol/h
    
    \item \textbf{Hydrogen Balance Around Reactor}
    
    H atoms in = H atoms out:
    
    $(100 \times 10) = (55 \times 10) + (\dot{n}_{H_2O} \times 2)$
    
    $1000 = 550 + 2\dot{n}_{H_2O} \Rightarrow \dot{n}_{H_2O} = 225$ mol/h
    
    \item \textbf{Carbon Balance with $CO$ and $CO_2$ Ratio}
    
    C atoms in = C atoms out:
    
    $(100 \times 4) = (55 \times 4) + \dot{n}_{CO_2} + \dot{n}_{CO}$
    
    Given $\dot{n}_{CO_2} = 9\dot{n}_{CO}$:
    
    $400 = 220 + 9\dot{n}_{CO} + \dot{n}_{CO} = 220 + 10\dot{n}_{CO}$
    
    $\dot{n}_{CO} = 18$ mol/h, $\dot{n}_{CO_2} = 162$ mol/h
    
    \item \textbf{Oxygen Balance Around Reactor}
    
    O atoms in = O atoms out:
    
    $(812.5 \times 2) = (\dot{n}_{O_2,out} \times 2) + (162 \times 2) + (18 \times 1) + (225 \times 1)$
    
    $1625 = 2\dot{n}_{O_2,out} + 567 \Rightarrow \dot{n}_{O_2,out} = 529$ mol/h
\end{enumerate}
\end{stepbox}

\begin{formulabox}[title=Answer 1: Reactor Outlet Composition]
Total moles out: $55 + 225 + 18 + 162 + 529 = 989$ mol/h

\textbf{Mole Fractions:}
\begin{itemize}[itemsep=0pt]
    \item $y_{C_4H_{10}} = 55/989 = 5.6\%$
    \item $y_{H_2O} = 225/989 = 22.7\%$
    \item $y_{CO} = 18/989 = 1.8\%$
    \item $y_{CO_2} = 162/989 = 16.4\%$
    \item $y_{O_2} = 529/989 = 53.5\%$
\end{itemize}
\end{formulabox}

\begin{stepbox}
\begin{enumerate}[label=\textbf{Step \arabic*:}, wide=0pt, leftmargin=*, itemsep=2pt]
    \setcounter{enumi}{5}
    \item \textbf{Separator Analysis}
    
    \textbf{Recycle streams (90\% of unreacted):}
    
    $\dot{n}_{C_4H_{10},recycle} = 55 \times 0.90 = 49.5$ mol/h
    
    $\dot{n}_{O_2,recycle} = 529 \times 0.90 = 476.1$ mol/h
    
    \textbf{Waste streams (10\% of unreacted):}
    
    $\dot{n}_{C_4H_{10},waste} = 55 \times 0.10 = 5.5$ mol/h
    
    \item \textbf{Mixing Point Balance (Answer 2)}
    
    Fresh feed + Recycle = Combined feed to reactor:
    
    $\dot{n}_{C_4H_{10},fresh} + 49.5 = 100 \Rightarrow \dot{n}_{C_4H_{10},fresh} = 50.5$ mol/h
    
    $\dot{n}_{O_2,fresh} + 476.1 = 812.5 \Rightarrow \dot{n}_{O_2,fresh} = 336.4$ mol/h
    
    \item \textbf{Overall Butane Conversion (Answer 3)}
    
    $\text{Overall Conv.} = \frac{\dot{n}_{C_4H_{10},fresh} - \dot{n}_{C_4H_{10},waste}}{\dot{n}_{C_4H_{10},fresh}}$
    
    $= \frac{50.5 - 5.5}{50.5} = \frac{45}{50.5} = 0.891 = 89.1\%$
\end{enumerate}
\end{stepbox}

\begin{conceptbox}[title=Final Results Summary]
\begin{enumerate}[itemsep=0pt]
    \item \textbf{Reactor outlet composition:} 5.6\% $C_4H_{10}$, 22.7\% $H_2O$, 1.8\% $CO$, 16.4\% $CO_2$, 53.5\% $O_2$
    \item \textbf{Fresh feed rates:} 50.5 mol/h butane, 336.4 mol/h oxygen
    \item \textbf{Overall butane conversion:} 89.1\%
\end{enumerate}
\end{conceptbox}

\newpage

\section*{Methanol Synthesis with Recycle and Purge Stream}

\begin{examplebox}{Methanol Production with Recycle and Purge}
\textbf{Problem:} Methanol ($CH_3OH$) produced by reacting CO and $H_2$ over catalyst. Fresh feed: 30 mol\% CO, 60 mol\% $H_2$, 10 mol\% $N_2$ (inert). Feed mixes with recycle and goes to reactor. Reactor effluent goes to condenser separating pure liquid methanol. Remaining gases split: portion purged (prevent inert buildup), rest recycled. Recycle ratio: 3 moles recycled per 1 mole fresh feed. Recycle stream: 25 mol\% $N_2$.

For 100 mol/h fresh feed, find:
\begin{enumerate}[itemsep=0pt]
    \item Rate of methanol production
    \item Rate and composition of purge gas
    \item Overall and single-pass CO conversions
\end{enumerate}
\end{examplebox}

\begin{conceptbox}[title=Solution Strategy]
DoF analysis shows overall system is solvable first. Inert species ($N_2$) provides key to unlocking system. Use extent of reaction method.
\end{conceptbox}

\begin{keybox}
\textbf{Process Information:}
\begin{itemize}[itemsep=0pt]
    \item \textbf{Reaction:} $CO + 2H_2 \rightarrow CH_3OH$
    \item \textbf{Fresh feed:} 100 mol/h (30 mol/h CO, 60 mol/h $H_2$, 10 mol/h $N_2$)
    \item \textbf{Recycle ratio:} 3:1 (recycle:fresh feed)
    \item \textbf{Recycle composition:} 25 mol\% $N_2$
    \item \textbf{Products:} Pure liquid methanol + purge gas
\end{itemize}
\end{keybox}

\begin{stepbox}
\begin{enumerate}[label=\textbf{Step \arabic*:}, wide=0pt, leftmargin=*, itemsep=2pt]
    \item \textbf{Basis and Overall System Boundary}
    
    \textbf{Basis:} 100 mol/h fresh feed
    
    Contains: 30 mol/h CO, 60 mol/h $H_2$, 10 mol/h $N_2$
    
    \textbf{Boundary:} Fresh feed in → Methanol out + Purge out
    
    \item \textbf{Overall Nitrogen Balance}
    
    $N_2$ is inert - only enters in fresh feed, only leaves in purge:
    
    $(\text{$N_2$ In}) = (\text{$N_2$ Out})$
    
    $10 = \dot{n}_{purge} \times y_{N_2,purge} = \dot{n}_{purge} \times 0.25$
    
    $\dot{n}_{purge} = 40$ mol/h
    
    \item \textbf{Extent of Reaction from Reactant Balances}
    
    For each reactant: In - Out = Reacted = $\nu_i \xi$
    
    \textbf{CO balance:} $30 - (\dot{n}_{purge} \times y_{CO,purge}) = (1)\xi$
    
    $30 - 40y_{CO,purge} = \xi$ ... (Eq. 1)
    
    \textbf{$H_2$ balance:} $60 - (\dot{n}_{purge} \times y_{H_2,purge}) = (2)\xi$
    
    $60 - 40y_{H_2,purge} = 2\xi$ ... (Eq. 2)
    
    \item \textbf{Solve for Purge Composition}
    
    \textbf{Mole fraction constraint:} $y_{CO} + y_{H_2} + y_{N_2} = 1$
    
    $y_{CO,purge} + y_{H_2,purge} + 0.25 = 1$ ... (Eq. 3)
    
    From Eq. 2: $y_{H_2,purge} = \frac{60-2\xi}{40}$
    
    From Eq. 1: $y_{CO,purge} = \frac{30-\xi}{40}$
    
    Substitute into Eq. 3: $\frac{30-\xi}{40} + \frac{60-2\xi}{40} + 0.25 = 1$
    
    $\frac{90-3\xi}{40} = 0.75 \Rightarrow 90-3\xi = 30 \Rightarrow \xi = 20$ mol/h
    
    \item \textbf{Calculate Final Compositions}
    
    $y_{CO,purge} = \frac{30-20}{40} = 0.25$
    
    $y_{H_2,purge} = \frac{60-40}{40} = 0.50$
    
    $y_{N_2,purge} = 0.25$
\end{enumerate}
\end{stepbox}

\begin{formulabox}[title=Answers 1 vs 2: Production and Purge]
\textbf{(1) Methanol Production Rate:} $\dot{n}_{methanol} = \xi = 20$ mol/h

\textbf{(2) Purge Gas:} 
\begin{itemize}[itemsep=0pt]
    \item Rate: 40 mol/h
    \item Composition: 25\% CO, 50\% $H_2$, 25\% $N_2$
\end{itemize}
\end{formulabox}

\begin{stepbox}
\begin{enumerate}[label=\textbf{Step \arabic*:}, wide=0pt, leftmargin=*, itemsep=2pt]
    \setcounter{enumi}{5}
    \item \textbf{Overall CO Conversion}
    
    $\text{Overall Conv.} = \frac{\text{CO in fresh} - \text{CO in purge}}{\text{CO in fresh}}$
    
    $= \frac{30 - (40 \times 0.25)}{30} = \frac{30-10}{30} = 0.667 = 66.7\%$
    
    \item \textbf{Single-Pass CO Conversion}
    
    \textbf{Recycle rate:} $3 \times 100 = 300$ mol/h
    
    \textbf{Total feed to reactor:} $100 + 300 = 400$ mol/h
    
    \textbf{CO to reactor:} Fresh CO + Recycle CO
    
    $= 30 + (300 \times y_{CO,recycle}) = 30 + (300 \times 0.25) = 105$ mol/h
    
    \textbf{Single-pass conversion:} $\frac{\text{CO reacted}}{\text{CO to reactor}} = \frac{20}{105} = 0.190 = 19.0\%$
\end{enumerate}
\end{stepbox}

\begin{conceptbox}[title=Answer 3: CO Conversions]
\textbf{Overall Conversion:} 66.7\%

\textbf{Single-Pass Conversion:} 19.0\%

This clearly demonstrates how recycle significantly increases overall process conversion above the low single-pass reactor conversion.
\end{conceptbox}

\begin{conceptbox}[title=Key Insights: Recycle with Purge Systems]
\textbf{Critical Design Considerations:}
\begin{itemize}[itemsep=0pt]
    \item Inert buildup necessitates purge stream to maintain steady operation
    \item Recycle dramatically improves overall conversion vs. single-pass
    \item Purge rate determined by inert material balance
    \item Higher recycle ratios increase overall conversion but also operating costs
    \item Optimal purge rate balances material recovery with inert removal
\end{itemize}
\end{conceptbox}

\newpage

\section*{Introduction to Reactor Systems with Recycle}

Reactor systems with recycle streams are common in chemical processing to improve overall conversion and economic efficiency. However, when unreactive species (inerts) are present, a \textbf{purge stream} becomes necessary to prevent accumulation and maintain steady-state operation.

\begin{conceptbox}
A \textbf{purge stream} is a portion of a recycle stream that is withdrawn to prevent the buildup of materials that remain entirely in the recycle stream. Without a purge, inert species would accumulate indefinitely, preventing the system from reaching steady state.
\end{conceptbox}

The typical reactor system consists of four main components:
\begin{itemize}[itemsep=0pt]
\item \textbf{Mixing Point}: Where fresh feed combines with recycle stream
\item \textbf{Reactor}: Where chemical reaction occurs with specified conversion
\item \textbf{Separator}: Where products are separated from unreacted species
\item \textbf{Splitting Point}: Where recycle stream is divided into purge and recycle portions
\end{itemize}

\newpage

\section*{Material Balance Equations}

\subsection*{Mixing Point Balances}

At the mixing point, fresh feed (stream 1) combines with recycle (stream 7) to form reactor feed (stream 2).

\begin{formulabox}
Total material balance:
$$\dot{m}_2 = \dot{m}_1 + \dot{m}_7$$

Component material balance (for each component i):
$$x_{i,2}\dot{m}_2 = x_{i,1}\dot{m}_1 + x_{i,7}\dot{m}_7$$
\end{formulabox}

\subsection*{Reactor Balances}

For a reactor with fractional conversion X and stoichiometric feed:

\begin{formulabox}
For reactants (component 1 is limiting reactant):
$$x_{i,3}\dot{m}_3 = x_{i,2}\dot{m}_2 \cdot (1 - X)$$

For products:
$$x_{i,3}\dot{m}_3 = x_{1,2}\dot{m}_2 \cdot X \cdot a$$

For unreactive species (inerts):
$$x_{i,2}\dot{m}_2 = x_{i,3}\dot{m}_3$$
\end{formulabox}

\subsection*{Separator Balances}

The separator may not remove all reaction products and may also remove some feed species.

\begin{formulabox}
Total material balance:
$$\dot{m}_3 = \dot{m}_4 + \dot{m}_5$$

Component material balance:
$$x_{i,3}\dot{m}_3 = x_{i,4}\dot{m}_4 + x_{i,5}\dot{m}_5$$
\end{formulabox}

\subsection*{Splitting Point Balances}

At the splitting point, stream 5 is divided into purge (stream 6) and recycle (stream 7).

\begin{formulabox}
Flow rate relationships:
$$\dot{m}_5 = \dot{m}_6 + \dot{m}_7$$
$$\dot{m}_6 = \text{purge} \cdot \dot{m}_5$$
$$\dot{m}_7 = (1 - \text{purge}) \cdot \dot{m}_5$$

Composition relationships:
$$x_{i,5} = x_{i,6} = x_{i,7} \text{ (for each component i)}$$
\end{formulabox}

\begin{keybox}
\textbf{Key Variables \& Notation:}
\begin{itemize}[itemsep=0pt]
\item $\dot{m}_j$ = total molar flow rate at location j $\text{(mol/s)}$
\item $x_{i,j}$ = mole fraction of component i at location j $\text{(dimensionless)}$
\item $X$ = fractional conversion of limiting reactant $\text{(dimensionless)}$
\item $a$ = stoichiometric coefficient of product $\text{(dimensionless)}$
\item $\text{purge}$ = fraction of stream 5 that is purged $\text{(dimensionless)}$
\end{itemize}
\end{keybox}

\subsection*{Overall System Analysis}

\begin{conceptbox}
\textbf{Degrees of Freedom Analysis:} For a system with n components and 4 units, the total number of unknowns equals the number of independent material balance equations. Additional specifications (conversion, purge fraction, separator efficiency) reduce the degrees of freedom to zero for a solvable system.
\end{conceptbox}

\newpage

\subsection*{Worked Examples}

\begin{examplebox}{Haber Process - Conceptual Analysis}
The Haber process reaction: $\text{N}_2 + 3\text{H}_2 \rightarrow 2\text{NH}_3$

A fresh feed containing stoichiometric amounts of reactants plus inert gas combines with recycle. The reactor achieves 15\% conversion of nitrogen. Products enter a separator that removes all ammonia. If reactor feed is 10 mol/s with 20\% inert and remaining N$_2$ and H$_2$ in stoichiometric proportions, find the necessity for a purge stream.
\end{examplebox}

\begin{stepbox}
\begin{enumerate}[label=\textbf{Step \arabic*:}, wide=0pt, leftmargin=*, itemsep=2pt]

\item \textbf{Define basis and analyze reactor feed composition:}
Reactor feed = 10 mol/s total\\
Inert = 20\% = 2 mol/s\\
Remaining 8 mol/s contains N$_2$ and H$_2$ in stoichiometric ratio (1:3)\\
Therefore: N$_2$ = 2 mol/s, H$_2$ = 6 mol/s

\item \textbf{Calculate reactor outlet composition:}
With 15\% conversion of N$_2$:\\
N$_2$ converted = $0.15 \times 2 = 0.3$ mol/s\\
N$_2$ remaining = $2 - 0.3 = 1.7$ mol/s\\
H$_2$ consumed = $3 \times 0.3 = 0.9$ mol/s\\
H$_2$ remaining = $6 - 0.9 = 5.1$ mol/s\\
NH$_3$ produced = $2 \times 0.3 = 0.6$ mol/s\\
Inert unchanged = 2 mol/s

\item \textbf{Analyze steady-state feasibility without purge:}
After separator: NH$_3$ removed completely\\
Recycle stream contains: 1.7 mol/s N$_2$, 5.1 mol/s H$_2$, 2 mol/s inert\\
For steady state, fresh feed must contain zero inert to balance the system\\
However, problem states fresh feed contains inert gas\\
\textbf{Conclusion: System cannot reach steady state without purge}

\item \textbf{Solution with purge stream:}
Adding a purge stream (e.g., 25\% of separator outlet) allows:\\
- Removal of excess inert from the system\\
- Fresh feed can contain inert gas\\
- System achieves steady-state operation
\end{enumerate}
\end{stepbox}

\newpage

\begin{examplebox}{Haber Process – NH\textsubscript{3} Production}
\begin{itemize}[itemsep=0pt, topsep=0pt]
  \item Fresh feed: 74.775 mol/s H\textsubscript{2}, 24.925 mol/s N\textsubscript{2}, 0.30 mol/s Ar
  \item 40\% of N\textsubscript{2} converted to NH\textsubscript{3}
  \item Separator removes all NH\textsubscript{3}; others exit unaltered
  \item 1\% purge before recycle
\end{itemize}
\textbf{Find:} NH\textsubscript{3} production rate
\end{examplebox}

\vspace{0.5em}

\begin{stepbox}
\begin{enumerate}[label=\textbf{Step \arabic*:}, wide=0pt, leftmargin=*, itemsep=2pt]
\item \textbf{Setup:} 17 unknowns (flow rates across 7 locations), 17 equations (mass balances + specs).\\
Given: $X = 0.40$, purge = 0.01

\item \textbf{Mixer (1 + 7 → 2):}
\[
\dot{m}_{i,1} + \dot{m}_{i,7} = \dot{m}_{i,2}, \quad i = \mathrm{H_2, N_2, Ar}
\]

\item \textbf{Reactor (2 → 3):}
\[
\begin{aligned}
\dot{m}_{NH_3,3} &= 2 \times 0.40 \times \dot{m}_{N_2,2} \\
\dot{m}_{N_2,3} &= 0.60 \times \dot{m}_{N_2,2} \\
\dot{m}_{H_2,3} &= \dot{m}_{H_2,2} - 3 \times 0.40 \times \dot{m}_{N_2,2} \\
\dot{m}_{Ar,3} &= \dot{m}_{Ar,2}
\end{aligned}
\]

\item \textbf{Separator + Split:}\\
Stream 3 → NH\textsubscript{3} to 4, others to 5\\
Stream 5: 1\% purge (6), 99\% recycle (7)

\item \textbf{Solution:}\\
Use Excel Solver or similar to solve 17 linear equations.

\item \textbf{Result:} NH\textsubscript{3} production = \textbf{19.94 mol/s}

\item \textbf{Check:} Atom balances: N, H (NH\textsubscript{3} + purge); Ar (purge only)
\end{enumerate}
\end{stepbox}


\newpage

\subsection*{Problem-Solving Strategy}

\begin{conceptbox}
\textbf{Systematic Approach for Reactor-Recycle-Purge Systems:}
\begin{enumerate}[itemsep=0pt]
\item Draw complete process flow diagram with all streams labeled
\item Identify all unknown flow rates and compositions
\item Count degrees of freedom (unknowns vs. independent equations)
\item Write material balances for each unit operation
\item Apply process specifications (conversion, separation efficiency, purge fraction)
\item Solve system of equations (often requires computational tools for complex systems)
\item Verify solution using overall atomic balances
\end{enumerate}
\end{conceptbox}

\begin{keybox}
\textbf{Common Mistakes to Avoid:}
\begin{itemize}[itemsep=0pt]
\item Forgetting that total material balances are not independent of component balances
\item Not accounting for stoichiometric relationships in reactor balances
\item Assuming equal compositions in recycle and purge streams at splitting points
\item Neglecting the necessity of purge streams for inert species
\item Using overall atomic balances as independent equations in the solution process
\end{itemize}
\end{keybox}

\newpage

nd{document}
