\documentclass[12pt]{article}
\usepackage[paperwidth=8.5in, paperheight=11in, margin=1.0in, headheight=15pt]{geometry}
\usepackage{amsmath,amssymb,amsthm}
\usepackage[most]{tcolorbox}
\usepackage{enumitem}
\usepackage{xcolor}
\usepackage{hyperref}
\usepackage{fancyhdr}
\usepackage{titlesec}
\usepackage{graphicx}
% Define custom colors for chemical engineering theme
\definecolor{conceptcolor}{RGB}{52, 73, 94}      % Dark blue-gray
\definecolor{formulacolor}{RGB}{231, 76, 60}     % Red for formulas
\definecolor{examplecolor}{RGB}{39, 174, 96}     % Green for examples
\definecolor{stepcolor}{RGB}{142, 68, 173}       % Purple for solution steps
\definecolor{keycolor}{RGB}{243, 156, 18}        % Orange for key points
% Configure fancy headers
\pagestyle{fancy}
\fancyhf{}
\fancyhead[L]{PE Study Guide}
\fancyhead[R]{Process Fundamentals}
\fancyfoot[C]{\thepage}
\renewcommand{\baselinestretch}{1.1}
\setlength{\parindent}{0.25in}
\setlength{\parskip}{3pt}
% Configure section formatting
\titleformat{\section}
  {\normalfont\LARGE\bfseries\color{conceptcolor}}
  {\thesection}{1em}{}
\titleformat{\subsection}
  {\normalfont\Large\bfseries\color{conceptcolor}}
  {\thesubsection}{1em}{}
% Define custom environments
\newtcolorbox{conceptbox}[1][]{
  enhanced,
  colback=conceptcolor!10,
  colframe=conceptcolor,
  arc=3mm,
  title=Key Concept,
  fonttitle=\bfseries\sffamily\normalsize,
  fontupper=\small,
  #1
}
\newtcolorbox{formulabox}[1][]{
  enhanced,
  colback=formulacolor!10,
  colframe=formulacolor,
  arc=2mm,
  title=Important Formula,
  fonttitle=\bfseries\sffamily\normalsize,
  fontupper=\small,
  #1
}
\newtcolorbox{examplebox}[2][]{
  enhanced,
  colback=examplecolor!10,
  colframe=examplecolor,
  arc=3mm,
  title=Example Problem: #2,
  fonttitle=\bfseries\sffamily\normalsize,
  fontupper=\small,
  #1
}
\newtcolorbox{stepbox}[1][]{
  enhanced,
  colback=stepcolor!10,
  colframe=stepcolor,
  arc=2mm,
  title=Solution Steps,
  fonttitle=\bfseries\sffamily\normalsize,
  fontupper=\small,
  #1
}
\newtcolorbox{keybox}[1][]{
  enhanced,
  colback=keycolor!10,
  colframe=keycolor,
  arc=2mm,
  title=Key Variables \& Definitions,
  fonttitle=\bfseries\sffamily\normalsize,
  fontupper=\small,
  #1
}
\begin{document}
\section*{Properties of Radiative Heat Transfer}
This guide introduces the fundamental properties and fluxes that govern the transfer of energy by electromagnetic radiation. Understanding these concepts is the first step in analyzing heat exchange between surfaces.

\subsection*{Introduction to Radiative Heat Fluxes}
\begin{conceptbox}
To analyze radiative heat transfer at a surface, we must distinguish between energy being emitted, energy arriving, and energy leaving. We define four distinct radiative heat fluxes (energy transfer rate per unit area) to describe these processes.
\end{conceptbox}

\begin{keybox}[title=The Four Radiative Heat Fluxes]
\begin{itemize}[itemsep=2pt]
    \item \textbf{Emissive Power ($E$)}: The rate at which radiation is \textit{emitted} from a surface per unit area [W/m$^2$]. This is governed by the surface's temperature and material properties.
    \item \textbf{Irradiation ($G$)}: The rate at which radiation is \textit{incident upon} (strikes) a surface from all of its surroundings, per unit area [W/m$^2$].
    \item \textbf{Radiosity ($J$)}: The total rate at which radiation \textit{leaves} a surface per unit area [W/m$^2$]. This is the sum of emitted and reflected radiation.
    \item \textbf{Net Radiative Heat Flux ($q''_{rad}$)}: The net rate of energy transfer away from the surface by radiation. It is the difference between what leaves (radiosity) and what arrives (irradiation).
\end{itemize}
\end{keybox}

\begin{formulabox}
The net radiative heat flux from a surface is given by:
$$ q''_{rad} = J - G $$
If $J > G$, there is a net heat loss from the surface. If $G > J$, there is a net heat gain.
\end{formulabox}

\subsection*{Emissive Power ($E$)}
\begin{conceptbox}[title=The Blackbody Idealization]
To define the emission from a real surface, we first consider an idealized object called a \textbf{blackbody}. A blackbody is a perfect emitter and absorber of radiation.
\begin{itemize}[itemsep=2pt]
    \item It absorbs all incident radiation ($\alpha=1$).
    \item At a given temperature, it emits the maximum possible thermal radiation.
    \item Its emission is diffuse (uniform in all directions).
\end{itemize}
The emissive power of a blackbody ($E_b$) is described by the Stefan-Boltzmann Law.
\end{conceptbox}
\begin{formulabox}[title=Stefan-Boltzmann Law]
$$ E_b = \sigma T^4 $$
\begin{itemize}[itemsep=0pt]
    \item $E_b$: Emissive power of a blackbody [W/m$^2$].
    \item $\sigma$: The Stefan-Boltzmann constant, $5.67 \times 10^{-8}$ W/m$^2\cdot$K$^4$.
    \item $T$: The absolute temperature of the surface, which \textit{must} be in Kelvin [K].
\end{itemize}
\end{formulabox}

\begin{conceptbox}[title=Real Surfaces and Emissivity ($\epsilon$)]
Real surfaces emit less energy than a blackbody at the same temperature. This imperfection is accounted for by the surface property called emissivity, $\epsilon$.
\end{conceptbox}
\begin{formulabox}[title=Emissive Power of a Real Surface]
$$ E = \epsilon E_b = \epsilon \sigma T^4 $$
\begin{itemize}[itemsep=0pt]
    \item $\epsilon$: Emissivity, a dimensionless property where $0 \le \epsilon \le 1$. It represents the ratio of the actual emission to the ideal blackbody emission.
\end{itemize}
\end{formulabox}

\subsection*{Irradiation ($G$) and Related Surface Properties}
\begin{conceptbox}
When irradiation ($G$) strikes a surface, a portion is absorbed, a portion is reflected, and a portion may be transmitted through the material.
\end{conceptbox}
\begin{keybox}[title=Radiative Surface Properties]
\begin{itemize}[itemsep=2pt]
    \item \textbf{Absorptivity ($\alpha$)}: Fraction of irradiation that is absorbed ($0 \le \alpha \le 1$).
    \item \textbf{Reflectivity ($\rho$)}: Fraction of irradiation that is reflected ($0 \le \rho \le 1$).
    \item \textbf{Transmissivity ($\tau$)}: Fraction of irradiation that passes through ($0 \le \tau \le 1$).
\end{itemize}
\end{keybox}
\begin{formulabox}
Conservation of energy requires that these three fractions sum to one:
$$ \alpha + \rho + \tau = 1 $$
For \textbf{opaque} surfaces, no radiation is transmitted ($\tau=0$), so the relationship simplifies:
$$ \alpha + \rho = 1 \quad (\text{Opaque Surface}) $$
\end{formulabox}

\subsection*{Radiosity ($J$)}
\begin{conceptbox}
Radiosity is the total radiative flux leaving a surface. It is the sum of the energy the surface emits on its own plus the portion of the incoming irradiation that it reflects.
\end{conceptbox}
\begin{formulabox}
$$ J = E + \rho G $$
For an opaque surface ($\rho = 1-\alpha$) with emissivity $\epsilon$:
$$ J = \epsilon \sigma T^4 + (1 - \alpha) G $$
\end{formulabox}

\begin{keybox}[title=Common Simplifying Assumptions for Surfaces]
\begin{itemize}[itemsep=2pt]
    \item \textbf{Opaque}: Assumed for most solids, means $\tau=0$ and $\alpha + \rho = 1$.
    \item \textbf{Diffuse}: Radiative properties do not depend on direction.
    \item \textbf{Gray}: Radiative properties do not depend on wavelength. For a gray surface, absorptivity is equal to emissivity: $\alpha = \epsilon$.
\end{itemize}
\end{keybox}

\newpage
\section*{Radiation Exchange Between Surfaces}
\subsection*{The Resistance Network Analogy}
\begin{conceptbox}
The exchange of radiation between multiple opaque, diffuse, gray surfaces in an enclosure can be modeled using an electrical analogy. This powerful method simplifies complex radiation problems into circuit analysis problems.
\begin{itemize}[itemsep=2pt]
    \item \textbf{Heat Flow ($q$)} is analogous to \textbf{Electric Current ($I$)}.
    \item \textbf{Potentials ($E_b, J$)} are analogous to \textbf{Voltage ($V$)}.
    \item \textbf{Resistances ($R_{surf}, R_{space}$)} are analogous to \textbf{Electrical Resistance ($R$)}.
\end{itemize}
\end{conceptbox}

\begin{conceptbox}[title=Surface Resistance]
The \textbf{surface resistance} represents the opposition to heat being radiated away from a surface due to its non-ideal emissivity (i.e., it not being a blackbody). It connects the potential of an ideal blackbody at that temperature ($E_b$) to the actual potential of the surface (its radiosity, $J$).
\end{conceptbox}
\begin{formulabox}
The net heat transfer rate leaving surface $i$ is:
$$ q_i = \frac{E_{bi} - J_i}{R_{surf, i}} $$
The surface resistance for surface $i$ is:
$$ R_{surf, i} = \frac{1 - \epsilon_i}{\epsilon_i A_i} $$
\end{formulabox}

\begin{conceptbox}[title=Space (Geometric) Resistance]
The \textbf{space resistance} represents the geometric opposition to radiative transfer between two surfaces. It depends on the areas and orientations of the surfaces, which is captured by the view factor. It connects the radiosity potentials of two different surfaces.
\end{conceptbox}
\begin{formulabox}
The net heat transfer rate \textit{between} surface $i$ and surface $j$ is:
$$ q_{ij} = \frac{J_i - J_j}{R_{space, ij}} $$
The space resistance between surfaces $i$ and $j$ is:
$$ R_{space, ij} = \frac{1}{A_i F_{ij}} = \frac{1}{A_j F_{ji}} $$
\end{formulabox}

\subsection*{Constructing the Radiation Network}
\begin{stepbox}[title=How to Draw a Radiation Resistance Network]
For an enclosure of $n$ opaque, diffuse, gray surfaces, the network is drawn as follows:
\begin{enumerate}[label=\textbf{Step \arabic*:}, wide=0pt, leftmargin=*, itemsep=2pt]
    \item \textbf{Create Potential Nodes}: For each surface $i$, draw a node representing its blackbody emissive power, $E_{bi} = \sigma T_i^4$. This node's potential is fixed if the surface temperature $T_i$ is known.

    \item \textbf{Create Radiosity Nodes}: For each surface $i$, draw a corresponding radiosity node, $J_i$.
    
    \item \textbf{Add Surface Resistances}: For each surface $i$, connect its $E_{bi}$ node to its $J_i$ node with its surface resistance, $R_{surf, i} = (1 - \epsilon_i)/(\epsilon_i A_i)$. If a surface is a blackbody ($\epsilon_i=1$), its surface resistance is zero, and $E_{bi} = J_i$.
    
    \item \textbf{Add Space Resistances}: Connect \textit{every} radiosity node $J_i$ to \textit{every other} radiosity node $J_j$ with a space resistance, $R_{space, ij} = 1/(A_i F_{ij})$.
\end{enumerate}
Once the network is drawn, standard circuit analysis techniques (like nodal analysis based on Kirchhoff's current law) can be used to solve for unknown heat transfer rates and surface temperatures.
\end{stepbox}

\begin{conceptbox}
Using this network, the total net heat leaving surface $i$ ($q_i$) is the "current" flowing out of the $E_{bi}$ node. This current is equal to the sum of all the "currents" flowing from the radiosity node $J_i$ to all other radiosity nodes $J_j$.
$$ q_i = \frac{E_{bi} - J_i}{R_{surf, i}} = \sum_{j=1}^{n} q_{ij} = \sum_{j=1}^{n} \frac{J_i - J_j}{R_{space, ij}} $$
\end{conceptbox}

\newpage

\section*{Estimating Heat of Vaporization using Antoine's Equation}
This guide details a common method for estimating a substance's heat of vaporization ($\Delta H_v$) by combining the empirical accuracy of the Antoine equation with the theoretical foundation of the Clausius-Clapeyron equation.

\subsection*{Objective}
\begin{conceptbox}
The goal is to estimate the heat of vaporization of benzene at a specific temperature ($60^\circ$C). The strategy involves using the Antoine equation to generate highly accurate vapor pressure data at temperatures bracketing our target temperature. These data points are then used to calculate the slope of the Clausius-Clapeyron line, from which the heat of vaporization can be determined.
\end{conceptbox}

\subsection*{Theoretical Framework}
\begin{conceptbox}[title=The Clausius-Clapeyron Equation]
This equation provides a fundamental relationship between a substance's saturation pressure ($P_{sat}$), its absolute temperature ($T$), and its heat of vaporization ($\Delta H_v$).
\end{conceptbox}
\begin{keybox}[title=Key Variables]
\begin{itemize}[itemsep=0pt]
    \item \textbf{$P_{sat}$}: Saturation vapor pressure.
    \item \textbf{$\Delta H_v$}: Molar heat of vaporization [J/mol].
    \item \textbf{$R$}: Universal gas constant, $8.314$ J/mol$\cdot$K.
    \item \textbf{$T$}: Absolute temperature [K].
\end{itemize}
\end{keybox}
\begin{formulabox}
Assuming $\Delta H_v$ is constant over a small temperature range, the integrated form of the equation is:
$$ \ln(P_{sat}) = -\frac{\Delta H_v}{R} \left(\frac{1}{T}\right) + C $$
This is an equation for a straight line ($y = mx + b$) where $y = \ln(P_{sat})$, $x = 1/T$, and the slope is $m = -\frac{\Delta H_v}{R}$.
\end{formulabox}

\begin{conceptbox}[title=The Antoine Equation]
The Antoine equation is a highly accurate semi-empirical correlation for saturation pressure data.
\end{conceptbox}
\begin{formulabox}
For benzene, the given equation is:
$$ \log_{10}(P_{sat}) = 6.906 - \frac{1211}{T + 220.8} $$
Note that for this specific set of constants, $T$ must be in degrees Celsius ($^\circ$C) and the resulting $P_{sat}$ is in Torr.
\end{formulabox}

\subsection*{Step-by-Step Calculation}

\begin{stepbox}[title=Step 1: Generate Data Points with the Antoine Equation]
We will calculate $P_{sat}$ at two temperatures bracketing $60^\circ$C: $T_1 = 55^\circ$C and $T_2 = 65^\circ$C.
\begin{itemize}[itemsep=2pt]
    \item \textbf{At $T_1 = 55^\circ$C}:
    $$ \log_{10}(P_{sat,1}) = 6.906 - \frac{1211}{55 + 220.8} = 6.906 - 4.391 = 2.515 $$
    $$ P_{sat,1} = 10^{2.515} = 327.3 \, \text{Torr} $$
    \item \textbf{At $T_2 = 65^\circ$C}:
    $$ \log_{10}(P_{sat,2}) = 6.906 - \frac{1211}{65 + 220.8} = 6.906 - 4.237 = 2.669 $$
    $$ P_{sat,2} = 10^{2.669} = 466.7 \, \text{Torr} $$
\end{itemize}
\end{stepbox}

\begin{stepbox}[title=Step 2: Prepare Data for the Clausius-Clapeyron Equation]
We must convert the generated data points to the absolute units required by the Clausius-Clapeyron equation: temperature in Kelvin [K] and pressure in natural log space [$\ln(P_{sat})$].
\begin{itemize}[itemsep=2pt]
    \item \textbf{For Point 1}:
    \begin{itemize}[itemsep=0pt]
        \item $T_1 = 55 + 273.15 = 328.15$ K $\implies 1/T_1 = 0.003047$ K$^{-1}$.
        \item $\ln(P_{sat,1}) = \ln(327.3) = 5.791$.
    \end{itemize}
    \item \textbf{For Point 2}:
    \begin{itemize}[itemsep=0pt]
        \item $T_2 = 65 + 273.15 = 338.15$ K $\implies 1/T_2 = 0.002957$ K$^{-1}$.
        \item $\ln(P_{sat,2}) = \ln(466.7) = 6.146$.
    \end{itemize}
\end{itemize}
\end{stepbox}

\begin{stepbox}[title=Step 3: Calculate the Slope and $\Delta H_v$]
We approximate the slope of the line using the two data points.
\begin{formulabox}
The slope, $m$, is approximated by the finite difference:
$$ m = \frac{\Delta y}{\Delta x} = \frac{\Delta(\ln P_{sat})}{\Delta(1/T)} = \frac{\ln(P_{sat,2}) - \ln(P_{sat,1})}{1/T_2 - 1/T_1} $$
$$ m = \frac{6.146 - 5.791}{0.002957 - 0.003047} = \frac{0.355}{-0.00009} = -3944 \, \text{K} $$
Now we use the relationship between the slope and the heat of vaporization:
$$ m = -\frac{\Delta H_v}{R} \implies \Delta H_v = -m \cdot R $$
$$ \Delta H_v = -(-3944 \, \text{K}) \times (8.314 \, \frac{\text{J}}{\text{mol}\cdot\text{K}}) = 32785 \, \frac{\text{J}}{\text{mol}} $$
\end{formulabox}
\begin{conceptbox}[title=Final Answer]
The estimated heat of vaporization for benzene at $60^\circ$C is approximately \textbf{32.8 kJ/mol}.
\end{conceptbox}
\end{stepbox}

\newpage
\section*{Using the Psychrometric Chart for Adiabatic Humidification}
\subsection*{Objective}
\begin{conceptbox}
This guide demonstrates how to use a psychrometric (or humidity) chart to solve problems involving adiabatic humidification. This process, also known as evaporative cooling, involves evaporating water into an air stream with no external heat supplied. The energy for evaporation is drawn from the air itself, causing the air's temperature to decrease.
\end{conceptbox}

\begin{examplebox}{Adiabatic Spray Chamber}
Humid air at $40^\circ$C and 20\% relative humidity (RH) enters an adiabatic spray chamber at a volumetric flow rate of 100 m$^3$/hr. The air is humidified by evaporating liquid water and leaves the chamber at 50\% RH. Determine the outlet temperature of the air and the rate of water evaporation.
\end{examplebox}

\begin{conceptbox}[title=Key Concept for Adiabatic Humidification]
On a psychrometric chart, an adiabatic humidification process occurs along a line of \textbf{constant wet-bulb temperature}. The process moves from the initial state up and to the left along this diagonal line.
\end{conceptbox}

\subsection*{Step-by-Step Chart Analysis}

\begin{stepbox}[title=Step 1: Find the Outlet Temperature]
\begin{enumerate}[label=\textbf{Part \arabic*:}, wide=0pt, leftmargin=*, itemsep=2pt]
    \item \textbf{Locate Inlet State (Point 1)}: Find the intersection of the vertical line for the inlet dry-bulb temperature ($T_{db,1} = 40^\circ$C) and the curved line for the inlet relative humidity (RH$_1$ = 20\%).
    
    \item \textbf{Follow the Process Line}: From Point 1, identify the diagonal line of constant wet-bulb temperature and follow it upwards and to the left.
    
    \item \textbf{Locate Outlet State (Point 2)}: The process stops when the constant wet-bulb temperature line intersects the curve for the outlet relative humidity (RH$_2$ = 50\%). Mark this as Point 2.
    
    \item \textbf{Read Outlet Temperature}: From Point 2, move vertically downwards to the x-axis (dry-bulb temperature) to read the outlet temperature.
    \begin{itemize}
        \item The result is $T_{db,2} \approx 30^\circ$C.
    \end{itemize}
\end{enumerate}
\end{stepbox}

\begin{stepbox}[title=Step 2: Find the Rate of Water Evaporation ($\dot{m}_{evap}$)]
\begin{formulabox}
The rate of evaporation is the mass flow rate of dry air multiplied by the change in the absolute humidity of the air.
$$ \dot{m}_{evap} = \dot{m}_{dry\,air} \times (\omega_2 - \omega_1) $$
\end{formulabox}
\begin{enumerate}[label=\textbf{Part \arabic*:}, wide=0pt, leftmargin=*, itemsep=2pt]
    \setcounter{enumi}{4}
    \item \textbf{Read Properties from the Chart}:
    \begin{itemize}[itemsep=2pt]
        \item \textbf{Absolute Humidities ($\omega$)}: Read from the vertical y-axis by moving horizontally from each point.
            \begin{itemize}[itemsep=0pt]
                \item From Point 1: $\omega_1 \approx 0.0092$ kg H$_2$O / kg dry air.
                \item From Point 2: $\omega_2 \approx 0.0132$ kg H$_2$O / kg dry air.
            \end{itemize}
        \item \textbf{Humid Volume ($v_h$)}: Read from the steeper diagonal lines to find the volume of humid air per kg of dry air at the inlet.
            \begin{itemize}[itemsep=0pt]
                \item At Point 1: $v_{h,1} \approx 0.9$ m$^3$/kg dry air.
            \end{itemize}
    \end{itemize}
\end{enumerate}
\end{stepbox}

\newpage
\begin{stepbox}
\begin{enumerate}[label=\textbf{Part \arabic*:}, wide=0pt, leftmargin=*, itemsep=2pt, start=6]
    \item \textbf{Perform Final Calculations}:
    \begin{itemize}[itemsep=2pt]
        \item First, calculate the mass flow rate of dry air using the inlet volumetric flow rate and humid volume.
        \begin{formulabox}
        $$ \dot{m}_{dry\,air} = \frac{\text{Volumetric Flow Rate}}{\text{Humid Volume}} = \frac{\dot{V}_1}{v_{h,1}} $$
        $$ \dot{m}_{dry\,air} = \frac{100 \, \text{m}^3/\text{hr}}{0.9 \, \text{m}^3/\text{kg dry air}} \approx 111.1 \, \text{kg dry air/hr} $$
        \end{formulabox}
        \item Now, calculate the rate of water evaporation using the mass flow rate and the change in absolute humidity.
        \begin{formulabox}
        $$ \dot{m}_{evap} = \dot{m}_{dry\,air} \times (\omega_2 - \omega_1) $$
        $$ \dot{m}_{evap} = (111.1 \, \text{kg dry air/hr}) \times (0.0132 - 0.0092) \, \frac{\text{kg H}_2\text{O}}{\text{kg dry air}} $$
        $$ \dot{m}_{evap} = (111.1) \times (0.004) \approx 0.444 \, \text{kg H}_2\text{O}/\text{hr} $$
        \end{formulabox}
    \end{itemize}
\end{enumerate}
\begin{conceptbox}[title=Final Answer]
The final state of the air is approximately \textbf{30$^\circ$C}. The rate of water evaporation required is approximately \textbf{0.44 kg/hr}.
\end{conceptbox}
\end{stepbox}

\newpage

\section*{Local vs. Average Heat Transfer Coefficients}
\subsection*{Fundamental Concept: Convective Heat Transfer}
\begin{conceptbox}
Convective heat transfer, described by Newton's Law of Cooling, is driven by the temperature difference between a surface and a moving fluid. The proportionality constant in this relationship is the convective heat transfer coefficient, $h$. The primary challenge in convection analysis is determining the correct value of $h$ for a given situation.
\end{conceptbox}
\begin{keybox}[title=Variables for Newton's Law of Cooling]
\begin{itemize}[itemsep=0pt]
    \item \textbf{$q_{conv}$}: Rate of heat transfer by convection [W].
    \item \textbf{$h$}: Convective heat transfer coefficient [W/m$^2\cdot$K].
    \item \textbf{$A$}: Surface area for heat transfer [m$^2$].
    \item \textbf{$T_s$}: Temperature of the solid surface.
    \item \textbf{$T_{\infty}$}: Temperature of the fluid far from the surface.
\end{itemize}
\end{keybox}
\begin{formulabox}
$$ q_{conv} = h A (T_s - T_{\infty}) $$
\end{formulabox}

\subsection*{Defining Local and Average Coefficients}
\begin{conceptbox}
The heat transfer coefficient, $h$, is generally not constant across a surface. The properties of the fluid's thermal boundary layer change with position, causing $h$ to vary. This necessitates a distinction between the local and average coefficients.
\begin{itemize}[itemsep=2pt]
    \item \textbf{Local Heat Transfer Coefficient ($h_x$)}: This is the value of the coefficient at a \textit{specific point $x$} on the surface. It is typically highest at the leading edge (where the boundary layer is thinnest) and decreases along the direction of flow.
    \item \textbf{Average Heat Transfer Coefficient ($\bar{h}_L$)}: This is the integrated average of the local coefficient, $h_x$, over the \textit{entire surface} of length $L$. This average value is used to calculate the total heat transfer rate from the entire surface.
\end{itemize}
\end{conceptbox}
\begin{formulabox}
The average heat transfer coefficient is mathematically defined as:
$$ \bar{h}_L = \frac{1}{L} \int_{0}^{L} h_x \,dx $$
\end{formulabox}

\begin{examplebox}{Calculating the Average Coefficient from the Local Coefficient}
The local heat transfer coefficient for flow over a 1-meter long flat plate is given by the relation $h_x = 5x^{-1/3}$ W/m$^2\cdot$K, where $x$ is the distance from the leading edge in meters. Find the average heat transfer coefficient, $\bar{h}_L$, for the entire plate.
\end{examplebox}
\begin{stepbox}
\begin{enumerate}[label=\textbf{Step \arabic*:}, wide=0pt, leftmargin=*, itemsep=2pt]
    \item \textbf{Set up the Integral}
    
    We apply the definition of the average heat transfer coefficient with the given function for $h_x$ and the plate length $L=1$ m.
    $$ \bar{h}_L = \frac{1}{L} \int_{0}^{L} (5x^{-1/3}) \,dx = \frac{1}{1} \int_{0}^{1} 5x^{-1/3} \,dx $$
    
    \item \textbf{Perform the Integration}
    
    First, we evaluate the indefinite integral of the function.
    $$ \int 5x^{-1/3} \,dx = 5 \int x^{-1/3} \,dx = 5 \left( \frac{x^{(-1/3 + 1)}}{(-1/3 + 1)} \right) = 5 \left( \frac{x^{2/3}}{2/3} \right) = 7.5 x^{2/3} $$
    Next, we evaluate the definite integral from the limits of 0 to 1.
    $$ \bar{h}_L = \left[ 7.5 x^{2/3} \right]_{0}^{1} = (7.5 \cdot 1^{2/3}) - (7.5 \cdot 0^{2/3}) = 7.5 - 0 = 7.5 $$
    
    \item \textbf{Final Answer and Comparison}
    
    The average heat transfer coefficient over the 1-meter plate is $\bar{h}_L = 7.5$ W/m$^2\cdot$K.
    
    \begin{itemize}[itemsep=0pt]
        \item As a comparison, the local coefficient at the end of the plate ($x=1$ m) is $h_x(1) = 5(1)^{-1/3} = 5$ W/m$^2\cdot$K. As expected, the average value is higher than the local value at the end because the local coefficient is much larger near the leading edge.
    \end{itemize}
\end{enumerate}
\end{stepbox}

\newpage
\section*{The Overall Heat Transfer Coefficient (U)}
\subsection*{Fundamental Concept: Thermal Resistance Networks}
\begin{conceptbox}
When heat is transferred through multiple layers or by multiple modes in series (e.g., convection, then conduction, then convection again), it is useful to model the process using a thermal resistance network. Each layer or process offers a resistance to heat flow. The overall heat transfer coefficient, $U$, is a single value that represents the total resistance of the entire system.
\end{conceptbox}
\begin{formulabox}
The total heat transfer rate ($q$) is given by:
$$ q = U A \Delta T_{\text{overall}} $$
The overall coefficient $U$ is related to the total thermal resistance ($R_{total}$) by:
$$ U = \frac{1}{A R_{total}} $$
It is often more convenient to work with resistances per unit area, $R'' = A R_{total}$. Then,
$$ U = \frac{1}{R''_{total}} = \frac{1}{\sum R''_{i}} $$
\end{formulabox}

\subsection*{Resistances in Series for a Composite Plane Wall}
\begin{conceptbox}
Consider a composite wall made of several layers of different materials, with fluids on both sides. The heat must flow through multiple thermal resistances in series. The total resistance is simply the sum of the individual resistances.
\end{conceptbox}
\begin{formulabox}[title=Individual Thermal Resistances per Unit Area ($R''$)]
\begin{itemize}[itemsep=2pt]
    \item \textbf{Conduction Resistance (Plane Wall)}:
    $$ R''_{cond} = \frac{L}{k} $$
    where $L$ is the layer thickness and $k$ is its thermal conductivity.
    \item \textbf{Convection Resistance}:
    $$ R''_{conv} = \frac{1}{h} $$
    where $h$ is the convective heat transfer coefficient at the fluid-solid interface.
\end{itemize}
\end{formulabox}
\begin{examplebox}{Overall U for a Composite Plane Wall}
A wall is composed of three different material layers ($L_1, k_1$; $L_2, k_2$; $L_3, k_3$). It is exposed to an inside fluid with a convection coefficient $h_1$ and an outside fluid with a convection coefficient $h_2$.
\end{examplebox}
\begin{stepbox}
The total resistance is the sum of five resistances in series: inside convection, three conduction layers, and outside convection.
\begin{formulabox}
The total thermal resistance per unit area is:
$$ R''_{total} = R''_{conv,1} + R''_{cond,1} + R''_{cond,2} + R''_{cond,3} + R''_{conv,2} $$
$$ R''_{total} = \frac{1}{h_1} + \frac{L_1}{k_1} + \frac{L_2}{k_2} + \frac{L_3}{k_3} + \frac{1}{h_2} $$
The overall heat transfer coefficient, $U$, is the inverse of this sum:
$$ U = \left( \frac{1}{h_1} + \frac{L_1}{k_1} + \frac{L_2}{k_2} + \frac{L_3}{k_3} + \frac{1}{h_2} \right)^{-1} $$
\end{formulabox}
\end{stepbox}

\newpage
nd{document}
