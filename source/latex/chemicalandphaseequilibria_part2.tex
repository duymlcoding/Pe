\documentclass[12pt]{article}
\usepackage[paperwidth=8.5in, paperheight=11in, margin=1.0in, headheight=15pt]{geometry}
\usepackage{amsmath,amssymb,amsthm}
\usepackage[most]{tcolorbox}
\usepackage{enumitem}
\usepackage{xcolor}
\usepackage{hyperref}
\usepackage{fancyhdr}
\usepackage{titlesec}
\usepackage{graphicx}
% Define custom colors for chemical engineering theme
\definecolor{conceptcolor}{RGB}{52, 73, 94}      % Dark blue-gray
\definecolor{formulacolor}{RGB}{231, 76, 60}     % Red for formulas
\definecolor{examplecolor}{RGB}{39, 174, 96}     % Green for examples
\definecolor{stepcolor}{RGB}{142, 68, 173}       % Purple for solution steps
\definecolor{keycolor}{RGB}{243, 156, 18}        % Orange for key points
% Configure fancy headers
\pagestyle{fancy}
\fancyhf{}
\fancyhead[L]{PE Study Guide}
\fancyhead[R]{Process Fundamentals}
\fancyfoot[C]{\thepage}
\renewcommand{\baselinestretch}{1.1}
\setlength{\parindent}{0.25in}
\setlength{\parskip}{3pt}
% Configure section formatting
\titleformat{\section}
  {\normalfont\LARGE\bfseries\color{conceptcolor}}
  {\thesection}{1em}{}
\titleformat{\subsection}
  {\normalfont\Large\bfseries\color{conceptcolor}}
  {\thesubsection}{1em}{}
% Define custom environments
\newtcolorbox{conceptbox}[1][]{
  enhanced,
  colback=conceptcolor!10,
  colframe=conceptcolor,
  arc=3mm,
  title=Key Concept,
  fonttitle=\bfseries\sffamily\normalsize,
  fontupper=\small,
  #1
}
\newtcolorbox{formulabox}[1][]{
  enhanced,
  colback=formulacolor!10,
  colframe=formulacolor,
  arc=2mm,
  title=Important Formula,
  fonttitle=\bfseries\sffamily\normalsize,
  fontupper=\small,
  #1
}
\newtcolorbox{examplebox}[2][]{
  enhanced,
  colback=examplecolor!10,
  colframe=examplecolor,
  arc=3mm,
  title=Example Problem: #2,
  fonttitle=\bfseries\sffamily\normalsize,
  fontupper=\small,
  #1
}
\newtcolorbox{stepbox}[1][]{
  enhanced,
  colback=stepcolor!10,
  colframe=stepcolor,
  arc=2mm,
  title=Solution Steps,
  fonttitle=\bfseries\sffamily\normalsize,
  fontupper=\small,
  #1
}
\newtcolorbox{keybox}[1][]{
  enhanced,
  colback=keycolor!10,
  colframe=keycolor,
  arc=2mm,
  title=Key Variables \& Definitions,
  fonttitle=\bfseries\sffamily\normalsize,
  fontupper=\small,
  #1
}

\section*{Phase Equilibria: Ideal Systems}
This guide covers vapor-liquid equilibrium (VLE) for ideal systems, where interactions between molecules are uniform. The behavior of these systems is described by Raoult's Law, which provides a simple and direct relationship between the liquid phase composition, vapor phase composition, and system pressure at a given temperature.

\subsection*{Saturation Pressure and the Antoine Equation}
\begin{conceptbox}
The \textit{saturation pressure} ($P_i^{sat}$) of a pure component is the pressure at which it boils at a given temperature. It is a fundamental property in VLE and is highly sensitive to temperature. The Antoine equation is a widely used semi-empirical correlation that accurately relates saturation pressure to temperature.
\end{conceptbox}

\begin{keybox}
The following variables are used in the Antoine equation:
\begin{itemize}[itemsep=0pt]
    \item \textbf{$P_i^{sat}$}: Saturation pressure of pure component $i$. Units can be kPa, bar, mmHg, etc.
    \item \textbf{$T$}: System temperature. The required units (K or $^\circ$C) depend on the constants.
    \item \textbf{$A_i, B_i, C_i$}: Component-specific Antoine constants obtained from experimental data.
\end{itemize}
\end{keybox}

\begin{formulabox}
A common form of the Antoine equation is:
$$ \log_{10}(P_i^{sat}) = A_i - \frac{B_i}{C_i + T} $$
Another form uses the natural logarithm, which requires a different set of constants:
$$ \ln(P_i^{sat}) = A_i - \frac{B_i}{C_i + T} $$
It is critical to use the correct temperature units and logarithm base associated with the published constants.
\end{formulabox}

\subsection*{Raoult's Law for Ideal Solutions}
\begin{conceptbox}
Raoult's Law is the cornerstone of ideal VLE. It states that the partial pressure of a component in the vapor phase is equal to its mole fraction in the liquid phase multiplied by its pure-component saturation pressure at the system temperature. This law assumes both the liquid and vapor phases behave ideally.
\end{conceptbox}

\begin{keybox}
The variables involved in Raoult's Law are:
\begin{itemize}[itemsep=0pt]
    \item \textbf{$y_i$}: Mole fraction of component $i$ in the vapor phase.
    \item \textbf{$x_i$}: Mole fraction of component $i$ in the liquid phase.
    \item \textbf{$P$}: Total pressure of the system.
    \item \textbf{$P_i$}: Partial pressure of component $i$ in the vapor phase ($P_i = y_i P$).
    \item \textbf{$P_i^{sat}$}: Saturation pressure of pure component $i$ at the system temperature $T$.
\end{itemize}
\end{keybox}

\begin{formulabox}
The mathematical statement of Raoult's Law is:
$$ y_i P = x_i P_i^{sat} \quad \text{(for each component } i \text{)} $$
\end{formulabox}

\newpage

\subsection*{Bubble and Dew Point Calculations}
\begin{conceptbox}
For a binary mixture, Raoult's Law allows us to define two critical types of VLE calculations:
\begin{itemize}[itemsep=0pt]
    \item \textbf{Bubble Point}: Given a liquid of known composition ($x_i$), the bubble point is the state (T or P) where the first bubble of vapor forms. At this point, the sum of the partial pressures equals the total pressure.
    \item \textbf{Dew Point}: Given a vapor of known composition ($y_i$), the dew point is the state (T or P) where the first drop of liquid forms. At this point, the liquid mole fractions calculated from the vapor must sum to one.
\end{itemize}
\end{conceptbox}
\begin{formulabox}
The governing equations for a binary system (components 1 and 2) are:
\begin{itemize}[itemsep=0pt]
    \item \textbf{Bubble Pressure Equation}: Used when liquid composition ($x_1, x_2$) is known.
    $$ P = x_1 P_1^{sat} + x_2 P_2^{sat} $$
    \item \textbf{Dew Pressure Equation}: Used when vapor composition ($y_1, y_2$) is known.
    $$ P = \frac{1}{\frac{y_1}{P_1^{sat}} + \frac{y_2}{P_2^{sat}}} $$
\end{itemize}
When temperature is the unknown, these equations must be solved iteratively.
\end{formulabox}

\newpage
\subsection*{Example Problems}

\begin{examplebox}{Bubble Temperature Calculation}
Calculate the bubble temperature at 85 kPa for an ideal binary liquid solution with a composition of $x_1 = 0.40$. The saturation pressures are given by the following Antoine equations, where T is in $^\circ$C and P is in kPa:
$$ \ln(P_1^{sat}) = 14.3 - \frac{2945}{T + 224} $$
$$ \ln(P_2^{sat}) = 14.2 - \frac{2943}{T + 209} $$
\end{examplebox}
\begin{stepbox}
\begin{enumerate}[label=\textbf{Step \arabic*:}, wide=0pt, leftmargin=*, itemsep=2pt]
    \item \textbf{Define the Goal and Governing Equation}
    
    This is a \textbf{bubble temperature} calculation. We are given the total pressure ($P=85$ kPa) and the liquid composition ($x_1=0.40$, $x_2=0.60$). We need to find the temperature $T$ that satisfies the bubble pressure equation:
    $$ P = x_1 P_1^{sat}(T) + x_2 P_2^{sat}(T) $$
    Since $T$ is inside the exponential functions for $P^{sat}$, this equation is non-linear and requires an iterative solution.

    \item \textbf{Set Up the Iterative Solution}
    
    We need to find the root of the following function $f(T) = 0$:
    $$ f(T) = x_1 P_1^{sat}(T) + x_2 P_2^{sat}(T) - P = 0 $$
    Substituting the known values and expressions:
    $$ 0.40 \exp\left(14.3 - \frac{2945}{T + 224}\right) + 0.60 \exp\left(14.2 - \frac{2943}{T + 209}\right) - 85 = 0 $$
    The solution process involves:
    \begin{itemize}[itemsep=0pt]
        \item Guess a temperature, $T_{guess}$.
        \item Calculate $P_1^{sat}$ and $P_2^{sat}$ at $T_{guess}$.
        \item Calculate the total pressure, $P_{calc} = x_1 P_1^{sat} + x_2 P_2^{sat}$.
        \item If $P_{calc} > P_{actual}$ (85 kPa), the guess is too high. If $P_{calc} < P_{actual}$, the guess is too low. Adjust $T_{guess}$ and repeat until $P_{calc} \approx P_{actual}$.
    \end{itemize}
    
    \item \textbf{Final Answer}
    
        \textbf{Bubble Temperature: $T = 84.4^\circ$C}

\end{enumerate}
\end{stepbox}

\newpage
\begin{examplebox}{Isothermal Flash Calculation}
A fixed-volume tank initially contains 1.0 mol of pure component A as a vapor at 1.9 bar. The saturation pressure of A at the tank temperature is $P_A^{sat} = 2.0$ bar. Then, 1.0 mol of liquid component B, with a saturation pressure of $P_B^{sat} = 1.0$ bar, is added to the tank. After the system reaches equilibrium at the same temperature, the final pressure is 1.44 bar. Assuming ideal behavior, determine the phases present, their compositions, and the number of moles of each phase.
\end{examplebox}
\begin{stepbox}
\begin{enumerate}[label=\textbf{Step \arabic*:}, wide=0pt, leftmargin=*, itemsep=2pt]
    \item \textbf{Analyze the System State}
    \begin{itemize}[itemsep=0pt]
        \item \textbf{Initial State}: Component A is at $P = 1.9$ bar, which is below its saturation pressure of $P_A^{sat} = 2.0$ bar. Therefore, it exists as a single-phase superheated vapor.
        \item \textbf{Final State}: The final mixture has a pressure of $P = 1.44$ bar. Since this pressure lies between the saturation pressures of the two pure components ($P_B^{sat} < P < P_A^{sat}$), the system at equilibrium must consist of a \textbf{vapor-liquid mixture}. If the final pressure were greater than $P_A^{sat}$, it would be a subcooled liquid. If it were less than $P_B^{sat}$, it would be a superheated vapor.
    \end{itemize}

    \item \textbf{Determine Equilibrium Compositions ($x_A$, $y_A$)}
    
    Since two phases are present at a known total pressure, we can use the bubble pressure form of Raoult's Law to find the liquid phase composition, $x_A$.
    $$ P = x_A P_A^{sat} + x_B P_B^{sat} $$
    Substitute $x_B = 1 - x_A$:
    $$ 1.44 = x_A(2.0) + (1 - x_A)(1.0) $$
    $$ 1.44 = 2.0 x_A + 1.0 - 1.0 x_A \implies 0.44 = x_A $$
    The liquid phase composition is \textbf{$x_A = 0.44$} and \textbf{$x_B = 0.56$}.
    
    Next, find the vapor phase composition, $y_A$, using Raoult's Law:
    $$ y_A = \frac{x_A P_A^{sat}}{P} = \frac{0.44 \times 2.0}{1.44} = 0.611 $$
    The vapor phase composition is \textbf{$y_A = 0.611$} and \textbf{$y_B = 0.389$}.
\end{enumerate}
\end{stepbox}
\newpage
\begin{stepbox}
\begin{enumerate}[label=\textbf{Step \arabic*:}, wide=0pt, leftmargin=*, itemsep=2pt, start=3]
    \item \textbf{Determine Moles of Liquid (L) and Vapor (V)}
    
    This calculation, often called a "flash," uses overall mole balances.
    \begin{itemize}[itemsep=0pt]
        \item Total moles in system, $N = N_A + N_B = 1.0 + 1.0 = 2.0$ mol.
        \item Overall mole fraction of A, $z_A = N_A / N = 1.0 / 2.0 = 0.5$.
    \end{itemize}
    We can write two balance equations:
    \begin{itemize}[itemsep=0pt]
        \item \textbf{Overall Mole Balance}: The total moles are split between the liquid and vapor phases.
        $$ L + V = N = 2.0 $$
        \item \textbf{Component A Mole Balance}: The total moles of A are distributed between the two phases.
        $$ x_A L + y_A V = z_A N $$
    \end{itemize}
    Now we solve this system of two linear equations. Substitute $V = 2.0 - L$ into the component balance:
    $$ (0.44)L + (0.611)(2.0 - L) = (0.5)(2.0) $$
    $$ 0.44L + 1.222 - 0.611L = 1.0 $$
    $$ -0.171L = 1.0 - 1.222 $$
    $$ -0.171L = -0.222 \implies L = \frac{0.222}{0.171} = 1.298 \, \text{mol} $$
    Then, find the moles of vapor:
    $$ V = 2.0 - L = 2.0 - 1.298 = 0.702 \, \text{mol} $$

    \item \textbf{Final Summary of Results}
    
    The final state of the system is a two-phase mixture with the following properties:
    \begin{itemize}[itemsep=0pt]
        \item \textbf{Phases Present}: Vapor and Liquid.
        \item \textbf{Liquid Phase}: Amount = \textbf{1.30 mol}, Composition = \textbf{$x_A=0.44, x_B=0.56$}.
        \item \textbf{Vapor Phase}: Amount = \textbf{0.70 mol}, Composition = \textbf{$y_A=0.61, y_B=0.39$}.
    \end{itemize}
\end{enumerate}
\end{stepbox}

\newpage
\section*{Absorption and Henry's Law}
This guide covers phase equilibrium for systems where a component is sparingly soluble in a liquid, forming a dilute solution. In these cases, the ideal solution model (Raoult's Law) is inaccurate for the solute. Instead, we use Henry's Law, an empirical relationship crucial for modeling gas absorption and stripping processes.

\subsection*{Henry's Law}
\begin{conceptbox}
Henry's Law provides a linear relationship for the equilibrium of a dilute species between a liquid and a gas phase. It states that the partial pressure of a sparingly soluble component in the vapor phase is directly proportional to its mole fraction in the liquid phase.
\begin{itemize}[itemsep=0pt]
    \item This law is most accurate at low concentrations (typically $x_i < 0.01$).
    \item The solvent (the component in high concentration) is still modeled using Raoult's Law.
\end{itemize}
\end{conceptbox}

\begin{keybox}
The variables used in the primary form of Henry's Law are:
\begin{itemize}[itemsep=0pt]
    \item \textbf{$H_i$}: Henry's constant for solute $i$ in a specific solvent. The units are pressure (e.g., bar, Pa, atm).
    \item \textbf{$x_i$}: Mole fraction of solute $i$ in the liquid phase.
    \item \textbf{$y_i$}: Mole fraction of solute $i$ in the gas phase.
    \item \textbf{$P$}: Total pressure of the system.
\end{itemize}
\end{keybox}

\begin{formulabox}
The form of Henry's Law analogous to Raoult's Law is:
$$ y_i P = x_i H_i $$
Here, $H_i$ can be viewed as a "pseudo-saturation-pressure" for a species that may not be able to exist as a pure liquid at the system's temperature and pressure.
\end{formulabox}

\subsection*{Temperature Dependence of Henry's Constant}
\begin{conceptbox}
The proportionality "constant" in Henry's Law, $H_i$, is not a true constant; it is highly dependent on temperature. Generally, the solubility of gases in liquids decreases as temperature increases, which corresponds to an increase in the value of $H_i$.
\end{conceptbox}

\begin{formulabox}
A common empirical equation to model the temperature dependence of Henry's constant is:
$$ H = \frac{1}{\exp(A + B/T + C \ln T + D T)} $$
Where $A$, $B$, $C$, and $D$ are constants specific to the solute-solvent pair, and $T$ is the absolute temperature (in Kelvin).
\end{formulabox}

\subsection*{Alternative Forms of Henry's Law}
\begin{conceptbox}
It is critical to be aware that Henry's Law is used in many different forms, distinguished by the units of the Henry's constant. Always check the definition and units of a given $H_i$ before use.
\end{conceptbox}

\begin{keybox}
Additional variables used in alternative forms:
\begin{itemize}[itemsep=0pt]
    \item \textbf{$C_i$}: Molar concentration of species $i$ in the liquid [mol/L].
    \item \textbf{$C_{i,gas}$}: Molar concentration of species $i$ in the gas [mol/L].
\end{itemize}
\end{keybox}

\begin{formulabox}
Some common alternative definitions of Henry's Law are:
\begin{itemize}[itemsep=0pt]
    \item \textbf{Volatility form}: $H_i = \frac{y_i P}{x_i}$ [Units of pressure]
    \item \textbf{Solubility form 1}: $H_i = \frac{C_i}{y_i P}$ [Units of mol/(Volume·Pressure)]
    \item \textbf{Solubility form 2}: $H_i = \frac{x_i}{y_i P}$ [Units of 1/Pressure]
    \item \textbf{Dimensionless form}: $H_i = \frac{C_i}{C_{i,gas}}$ [Dimensionless]
\end{itemize}
\end{formulabox}

\newpage
\subsection*{Example Problem}

\begin{examplebox}{Single-Stage Stripping Process}
A stream of dry air at 5 bar and 20$^\circ$C is used to strip a volatile organic compound (VOC) from a wastewater stream. The process occurs in a single equilibrium stage. The Henry's constant for the VOC in water at this temperature is 2.5 bar. The inlet water contains the VOC at a mole fraction of 0.0005. What flow rate of air (in moles per mole of water) is needed to remove 95\% of the VOC from the water?
\end{examplebox}

\begin{stepbox}
\begin{enumerate}[label=\textbf{Step \arabic*:}, wide=0pt, leftmargin=*, itemsep=2pt]
    \item \textbf{Define System, Basis, and Goal}

    \textbf{System:} A single-stage stripper where a VOC (component C) is transferred from a liquid water stream (W) to a gaseous air stream (A). The outlet streams are in equilibrium.

    \textbf{Basis:} 1.0 mole of water entering the stripper ($n_{W,in} = 1.0$ mol).

    \textbf{Goal:} Find the required molar flow rate of air, $n_A$.

    \textbf{Given Data:}  
    Total Pressure: $P = 5$ bar  
    Henry's Constant: $H_C = 2.5$ bar  
    Inlet VOC mole fraction in liquid: $x_{C,in} = 0.0005$  
    Removal efficiency: 95\%

    \item \textbf{State Simplifying Assumptions}

    Since the system is very dilute with respect to the VOC, we assume:

    \textbf{Assumption 1:} The total molar flow rate of the liquid stream is approximately constant. VOC content is negligible, so $n_{L,out} \approx n_{W,in} = 1.0$ mol.

    \textbf{Assumption 2:} The total molar flow rate of the vapor stream is approximately constant and equal to the inlet air flow. VOC and water vapor are negligible compared to air, so $n_{V,out} \approx n_{A,in}$.

    \item \textbf{Perform Contaminant Mole Balance}

    Track the VOC (C) moles through the system:

    Moles of VOC in (liquid):  
    \[
    n_{C,in} = x_{C,in} \times n_{L,in} = 0.0005 \times 1.0 = 5 \times 10^{-4}\ \text{mol}
    \]

    Moles of VOC out (liquid):  
    \[
    n_{C,out,L} = 0.05 \times n_{C,in} = 0.05 \times (5 \times 10^{-4}) = 2.5 \times 10^{-5}\ \text{mol}
    \]

    Moles of VOC out (vapor):  
    \[
    n_{C,out,V} = 0.95 \times n_{C,in} = 0.95 \times (5 \times 10^{-4}) = 4.75 \times 10^{-4}\ \text{mol}
    \]
\end{enumerate}
\end{stepbox}


\newpage
\begin{stepbox}
\begin{enumerate}[label=\textbf{Step \arabic*:}, wide=0pt, leftmargin=*, itemsep=2pt, start=4]
    \item \textbf{Apply Henry's Law at the Outlet}
    
    The outlet liquid and vapor streams are in equilibrium. We can relate their compositions using Henry's Law.
    $$ y_{C,out} P = x_{C,out} H_C $$
    First, we calculate the mole fraction of the VOC in the outlet liquid stream using Assumption 1 ($n_{L,out} \approx 1.0$ mol).
    $$ x_{C,out} = \frac{n_{C,out,L}}{n_{L,out}} \approx \frac{2.5 \times 10^{-5} \, \text{mol}}{1.0 \, \text{mol}} = 2.5 \times 10^{-5} $$
    Now, we can solve for the mole fraction of the VOC in the outlet vapor stream.
    $$ y_{C,out} = \frac{x_{C,out} H_C}{P} = \frac{(2.5 \times 10^{-5})(2.5 \, \text{bar})}{5 \, \text{bar}} = 1.25 \times 10^{-5} $$
    This is the maximum mole fraction of the VOC that the air can contain at equilibrium with the outlet water.

    \item \textbf{Calculate the Required Air Flow Rate}
    
    The moles of VOC in the outlet vapor can also be defined by its mole fraction and the total molar flow rate of the vapor stream.
    $$ n_{C,out,V} = y_{C,out} \times n_{V,out} $$
    Using our calculated values and Assumption 2 ($n_{V,out} \approx n_{A,in}$), we can solve for the required inlet air flow rate, $n_{A,in}$.
    $$ 4.75 \times 10^{-4} \, \text{mol} = (1.25 \times 10^{-5}) \times n_{A,in} $$
    $$ n_{A,in} = \frac{4.75 \times 10^{-4}}{1.25 \times 10^{-5}} = 38 \, \text{mol} $$
    The result indicates that to achieve the desired stripping, we need \textbf{38 moles of air for every 1 mole of water}.
\end{enumerate}
\end{stepbox}

\newpage
\begin{stepbox}
\begin{enumerate}[label=\textbf{Step \arabic*:}, wide=0pt, leftmargin=*, itemsep=2pt, start=6]
    \item \textbf{Evaluate the Simplifying Assumptions}
    
    A rigorous solution would account for the water that evaporates into the air stream. We can check if this amount was truly negligible. The solvent (water) is nearly pure, so we can use Raoult's Law for it.
    \begin{itemize}[itemsep=0pt]
        \item \textbf{Water Saturation Pressure}: At 20$^\circ$C, the saturation pressure of water is $P_W^{sat} \approx 0.0234$ bar.
        \item \textbf{Water Mole Fraction in Outlet Liquid}: The outlet liquid is almost pure water, so $x_{W,out} \approx 1$.
        \item \textbf{Calculate Water Mole Fraction in Outlet Vapor}: Using Raoult's Law for the water component:
        $$ y_{W,out} P = x_{W,out} P_W^{sat} $$
        $$ y_{W,out} = \frac{(1.0)(0.0234 \, \text{bar})}{5 \, \text{bar}} \approx 0.0047 $$
        \item \textbf{Calculate Moles of Evaporated Water}: The total moles of vapor leaving is the sum of the air, the stripped VOC, and the evaporated water ($n_{V,out} = n_A + n_{C,out,V} + n_{W,evap}$). The amount of VOC is tiny, so $n_{V,out} \approx n_A + n_{W,evap}$.
        $$ n_{W,evap} = y_{W,out} \times n_{V,out} \approx y_{W,out} \times (n_A + n_{W,evap}) $$
        Since $n_{W,evap}$ will be small compared to $n_A$, we can approximate $n_{V,out} \approx n_A = 38$ mol.
        $$ n_{W,evap} \approx 0.0047 \times 38 = 0.178 \, \text{mol} $$
    \end{itemize}
    \textbf{Conclusion on Assumptions}:
    \begin{itemize}[itemsep=0pt]
        \item The amount of evaporated water (0.18 mol) is only about 0.5\% of the total air flow (38 mol). Thus, the assumption that $n_{V,out} \approx n_{A,in}$ was excellent.
        \item The amount of evaporated water is about 18\% of the liquid water feed (1 mol). This is more significant, but since our primary goal was to find the air-to-water ratio, and the air flow is by far the largest stream, the effect on the final answer is minor. A fully rigorous iterative solution would yield a nearly identical result of ~38 mol air/mol water. Our simplifying assumptions were justified.
    \end{itemize}
\end{enumerate}
\end{stepbox}

\newpage

\section*{Phase Transitions: Clapeyron, \\ Clausius-Clapeyron, and Antoine Equations}
This guide covers the thermodynamic relationships that describe the equilibrium between different phases of a pure substance. We will derive the exact Clapeyron equation, explore its useful approximation (the Clausius-Clapeyron equation), and discuss the practical, empirical Antoine equation.

\subsection*{The Clapeyron Equation}
\begin{conceptbox}
The Clapeyron equation provides an exact thermodynamic relationship for the slope of a phase boundary line on a pressure-temperature (P-T) diagram (e.g., the boiling curve). It is derived from the fundamental condition that the molar Gibbs free energies of two phases must be equal for them to coexist in equilibrium.
\end{conceptbox}

\begin{formulabox}[title=Forms of the Clapeyron Equation]
The Clapeyron equation is general and applies to any phase transition. The pressure $P$ is the saturation pressure, $P^{sat}$, for transitions involving a vapor phase.
\begin{itemize}[itemsep=2pt]
    \item \textbf{Vapor-Liquid Equilibrium (Vaporization)}
    $$ \frac{dP^{sat}}{dT} = \frac{\Delta H^{vap}}{T(V^V - V^L)} $$
    \item \textbf{Vapor-Solid Equilibrium (Sublimation)}
    $$ \frac{dP^{sat}}{dT} = \frac{\Delta H^{sub}}{T(V^V - V^S)} $$
    \item \textbf{Liquid-Solid Equilibrium (Fusion/Melting)}
    $$ \frac{dP}{dT} = \frac{\Delta H^{fus}}{T(V^L - V^S)} $$
\end{itemize}
\end{formulabox}

\begin{formulabox}[title=Derivation of the Clapeyron Equation]
The derivation proceeds in the following steps:
\begin{itemize}[itemsep=2pt]
    \item Start with the fundamental relation for Gibbs free energy ($G$):
    $$ dG = VdP - SdT $$
    \item State the condition for phase equilibrium between two phases (e.g., Liquid and Vapor):
    $$ G^L = G^V $$
    \item If we move along the equilibrium line, the changes in Gibbs free energy for each phase must be equal to maintain equilibrium:
    $$ dG^L = dG^V $$
    \item Substitute the fundamental relation into the equality:
    $$ V^L dP - S^L dT = V^V dP - S^V dT $$
    \item Rearrange the terms to solve for the slope of the phase boundary, $\frac{dP}{dT}$:
    $$ (S^V - S^L) dT = (V^V - V^L) dP $$
    $$ \frac{dP}{dT} = \frac{S^V - S^L}{V^V - V^L} = \frac{\Delta S_{vap}}{\Delta V_{vap}} $$
    \item At equilibrium, the Gibbs free energy of vaporization is zero ($\Delta G_{vap} = 0$). From the definition $\Delta G = \Delta H - T\Delta S$, we find that the entropy of vaporization is $\Delta S_{vap} = \frac{\Delta H_{vap}}{T}$.
    \item Substitute this into the slope equation to arrive at the final form of the Clapeyron equation:
    $$ \frac{dP}{dT} = \frac{\Delta H_{vap}}{T \Delta V_{vap}} $$
\end{itemize}
\end{formulabox}


\subsection*{The Clausius-Clapeyron Equation}
\begin{conceptbox}
The Clausius-Clapeyron equation is a simplified, approximate form of the Clapeyron equation that is valid for vapor-liquid and vapor-solid equilibria under specific conditions. It is derived using two key assumptions:
\begin{itemize}[itemsep=2pt]
    \item The molar volume of the condensed phase (liquid or solid) is negligible compared to the molar volume of the vapor phase ($V^V \gg V^L$ or $V^V \gg V^S$). Therefore, $\Delta V \approx V^V$.
    \item The vapor phase behaves as an ideal gas, allowing the use of the ideal gas law: $V^V = \frac{RT}{P^{sat}}$.
\end{itemize}
\end{conceptbox}
\begin{formulabox}[title=Derivation and Forms of the Clausius-Clapeyron Equation]
\begin{itemize}[itemsep=2pt]
    \item Substitute the assumptions into the Clapeyron equation for vaporization:
    $$ \frac{dP^{sat}}{dT} = \frac{\Delta H^{vap}}{T(V^V)} \approx \frac{\Delta H^{vap}}{T(RT/P^{sat})} = \frac{P^{sat}\Delta H^{vap}}{RT^2} $$
    \item Rearrange to a common differential form:
    $$ \frac{1}{P^{sat}}\frac{dP^{sat}}{dT} = \frac{\Delta H^{vap}}{RT^2} \implies \frac{d(\ln P^{sat})}{dT} = \frac{\Delta H^{vap}}{RT^2} $$
    \item A second useful differential form shows that a plot of $\ln(P^{sat})$ versus $1/T$ is a straight line with a slope of $-\frac{\Delta H^{vap}}{R}$:
    $$ d(\ln P^{sat}) = -\frac{\Delta H^{vap}}{R} d\left(\frac{1}{T}\right) $$
\end{itemize}
\end{formulabox}
\begin{conceptbox}
To integrate the Clausius-Clapeyron equation, a third assumption is required:
\begin{itemize}[itemsep=2pt]
    \item The heat of vaporization, $\Delta H^{vap}$, is constant over the temperature range of integration.
\end{itemize}
\end{conceptbox}
\begin{formulabox}[title=Integrated Clausius-Clapeyron Equation]
Integrating the differential form between two states $(T_1, P^{sat}_1)$ and $(T_2, P^{sat}_2)$ yields:
$$ \ln\left(\frac{P^{sat}_2}{P^{sat}_1}\right) = -\frac{\Delta H^{vap}}{R}\left(\frac{1}{T_2} - \frac{1}{T_1}\right) $$
This equation is very useful for estimating vapor pressures or heats of vaporization from limited data.
\end{formulabox}

\subsection*{The Antoine Equation}
\begin{conceptbox}
While the Clausius-Clapeyron equation is derived from theory, the Antoine equation is a practical, semi-empirical correlation that provides a more accurate fit for experimental vapor pressure data over a wider temperature range.
\end{conceptbox}
\begin{formulabox}[title=Antoine Equation]
$$ \log_{10} P^{sat} = A - \frac{B}{T + C} $$
The constants $A, B,$ and $C$ are specific to each substance and depend on the units used for pressure ($P^{sat}$) and temperature ($T$).
\end{formulabox}

\newpage
\subsection*{Example Problems}

\begin{examplebox}{Melting Point of Ice under Pressure}
The molar heat of fusion of ice is 335 J/g. The densities of liquid water and ice at 0$^\circ$C are 1.00 g/cm$^3$ and 0.915 g/cm$^3$, respectively. Calculate the melting temperature of ice when the system pressure is 110 MPa.
\end{examplebox}
\begin{stepbox}
\begin{enumerate}[label=\textbf{Step \arabic*:}, wide=0pt, leftmargin=*, itemsep=2pt]
    \item \textbf{Identify Strategy and Convert to Molar SI Units}
    
    We must use the Clapeyron equation for liquid-solid equilibrium. It is an exact relation, which is necessary because the pressure change is large. First, we convert all given quantities to a consistent molar basis using the molecular weight of water (18.015 g/mol).
    
    \begin{itemize}[itemsep=2pt]
        \item \textbf{Molar Heat of Fusion ($\Delta H^{fus}$)}:
        $$ \Delta H^{fus} = 335 \frac{\text{J}}{\text{g}} \times 18.015 \frac{\text{g}}{\text{mol}} = 6035 \frac{\text{J}}{\text{mol}} $$
        
        \item \textbf{Molar Volume of Liquid ($V^L$)}:
        $$ \rho_L = 1.00 \frac{\text{g}}{\text{cm}^3} \times \frac{10^6 \text{ cm}^3}{1 \text{ m}^3} \times \frac{1 \text{ kg}}{1000 \text{ g}} = 1000 \frac{\text{kg}}{\text{m}^3} $$
        $$ V^L = \frac{MW}{\rho_L} = \frac{18.015 \times 10^{-3} \text{ kg/mol}}{1000 \text{ kg/m}^3} = 1.8015 \times 10^{-5} \frac{\text{m}^3}{\text{mol}} $$
        
        \item \textbf{Molar Volume of Solid ($V^S$)}:
        $$ \rho_S = 0.915 \frac{\text{g}}{\text{cm}^3} = 915 \frac{\text{kg}}{\text{m}^3} $$
        $$ V^S = \frac{MW}{\rho_S} = \frac{18.015 \times 10^{-3} \text{ kg/mol}}{915 \text{ kg/m}^3} = 1.9688 \times 10^{-5} \frac{\text{m}^3}{\text{mol}} $$
    \end{itemize}
\end{enumerate}
\end{stepbox}

\newpage
\begin{stepbox}
\begin{enumerate}[label=\textbf{Step \arabic*:}, wide=0pt, leftmargin=*, itemsep=2pt, start=2]
    \item \textbf{Apply the Clapeyron Equation}
    
    The governing equation for the solid-liquid boundary is:
    $$ \frac{dP}{dT} = \frac{\Delta H^{fus}}{T(V^L - V^S)} $$
    The term $\frac{\Delta H^{fus}}{V^L - V^S}$ can be treated as approximately constant. Let's calculate its value.
    $$ \Delta V^{fus} = V^L - V^S = (1.8015 - 1.9688) \times 10^{-5} = -1.673 \times 10^{-6} \frac{\text{m}^3}{\text{mol}} $$
    Note that for water, $\Delta V^{fus}$ is negative because ice is less dense than liquid water.
    $$ \frac{\Delta H^{fus}}{\Delta V^{fus}} = \frac{6035 \text{ J/mol}}{-1.673 \times 10^{-6} \text{ m}^3/\text{mol}} = -3.607 \times 10^9 \frac{\text{J}}{\text{m}^3} = -3.607 \times 10^9 \text{ Pa} $$
    The differential equation simplifies to:
    $$ \frac{dP}{dT} \approx \frac{-3.607 \times 10^9 \text{ Pa}}{T} $$
    
    \item \textbf{Integrate the Equation}
    
    We separate variables and integrate from a known reference point (State 1) to the desired final state (State 2).
    \begin{itemize}[itemsep=2pt]
        \item \textbf{State 1 (Normal Melting Point)}: $P_1 = 0.1013 \text{ MPa} = 1.013 \times 10^5 \text{ Pa}$, $T_1 = 273.15$ K
        \item \textbf{State 2}: $P_2 = 110 \text{ MPa} = 1.1 \times 10^8 \text{ Pa}$, $T_2 = ?$
    \end{itemize}
    $$ \int_{P_1}^{P_2} dP = (-3.607 \times 10^9 \text{ Pa}) \int_{T_1}^{T_2} \frac{dT}{T} $$
    $$ P_2 - P_1 = (-3.607 \times 10^9) \ln\left(\frac{T_2}{T_1}\right) $$
    $$ (1.1 \times 10^8 - 1.013 \times 10^5) = (-3.607 \times 10^9) \ln\left(\frac{T_2}{273.15}\right) $$
    $$ 1.099 \times 10^8 = (-3.607 \times 10^9) \ln\left(\frac{T_2}{273.15}\right) $$
    $$ \ln\left(\frac{T_2}{273.15}\right) = \frac{1.099 \times 10^8}{-3.607 \times 10^9} = -0.03047 $$
    $$ \frac{T_2}{273.15} = e^{-0.03047} = 0.96998 $$
    $$ T_2 = 0.96998 \times 273.15 = 264.95 \text{ K} $$

\end{enumerate}
\end{stepbox}

\newpage
\begin{examplebox}{Estimate $\Delta H_{vap}$ from the Antoine Equation}
Given the Antoine equation for benzene, $\log_{10}P^{sat} = 6.90 - 1211/(220.8 + T)$, where $P^{sat}$ is in torr and $T$ is in $^\circ$C, estimate the heat of vaporization of benzene at 60$^\circ$C.
\end{examplebox}
\begin{stepbox}
\begin{enumerate}[label=\textbf{Step \arabic*:}, wide=0pt, leftmargin=*, itemsep=2pt]
    \item \textbf{Identify Strategy}
    
    The Clausius-Clapeyron equation relates the heat of vaporization to the slope of a $\ln(P^{sat})$ vs. $1/T$ plot: $\frac{d(\ln P^{sat})}{d(1/T)} = -\frac{\Delta H_{vap}}{R}$. We can use the given Antoine equation to approximate this derivative at 60$^\circ$C using a finite difference method. We will calculate $P^{sat}$ at two temperatures bracketing 60$^\circ$C (e.g., 55$^\circ$C and 65$^\circ$C) and use the slope between these two points.
    
    \item \textbf{Generate Data Points from Antoine Equation}
    
    First, calculate $P^{sat}$ at $T_1 = 55^\circ$C and $T_2 = 65^\circ$C. Then convert all values to absolute units (K, ln(P)).
    \begin{itemize}[itemsep=2pt]
        \item \textbf{At T$_1$ = 55$^\circ$C (328.15 K)}:
        $$ \log_{10} P^{sat}_{1} = 6.90 - \frac{1211}{220.8 + 55} = 2.509 \implies P^{sat}_{1} = 10^{2.509} = 322.8 \text{ torr} $$
        $$ \ln(P^{sat}_{1}) = \ln(322.8) = 5.777 $$
        $$ 1/T_1 = 1/328.15 = 0.0030474 \text{ K}^{-1} $$
        
        \item \textbf{At T$_2$ = 65$^\circ$C (338.15 K)}:
        $$ \log_{10} P^{sat}_{2} = 6.90 - \frac{1211}{220.8 + 65} = 2.663 \implies P^{sat}_{2} = 10^{2.663} = 460.2 \text{ torr} $$
        $$ \ln(P^{sat}_{2}) = \ln(460.2) = 6.132 $$
        $$ 1/T_2 = 1/338.15 = 0.0029573 \text{ K}^{-1} $$
    \end{itemize}
    
    \item \textbf{Calculate the Slope and $\Delta H_{vap}$}
    
    Approximate the slope using the two-point finite difference formula:
    $$ -\frac{\Delta H_{vap}}{R} \approx \frac{\Delta(\ln P^{sat})}{\Delta(1/T)} = \frac{\ln(P^{sat}_{2}) - \ln(P^{sat}_{1})}{1/T_{2} - 1/T_{1}} $$
    $$ -\frac{\Delta H_{vap}}{R} = \frac{6.132 - 5.777}{0.0029573 - 0.0030474} = \frac{0.355}{-0.0000901} = -3940 \text{ K} $$
    Now, solve for $\Delta H_{vap}$ using $R = 8.314$ J/(mol·K):
    $$ \Delta H_{vap} = (3940 \text{ K}) \times R = 3940 \text{ K} \times 8.314 \frac{\text{J}}{\text{mol·K}} = 32755 \frac{\text{J}}{\text{mol}} $$

\end{enumerate}
\end{stepbox}

\newpage

\section*{Vapor-Liquid Equilibrium: Non-Ideal Solutions}
This guide addresses vapor-liquid equilibrium (VLE) in real mixtures where molecular interactions cause deviations from ideal behavior. We introduce the activity coefficient ($\gamma$) as a correction factor to Raoult's Law to accurately model these non-ideal liquid solutions.

\begin{conceptbox}
In non-ideal solutions, the intermolecular forces between unlike molecules (A-B) differ from those between like molecules (A-A and B-B). The liquid-phase activity coefficient, $\gamma_i$, quantifies this deviation for each component $i$.
\begin{itemize}[itemsep=0pt]
    \item \textbf{Positive Deviation ($\gamma_i > 1$)}: Occurs when A-B attractions are weaker than A-A and B-B attractions (molecules "repel" each other). This leads to a higher-than-ideal total vapor pressure.
    \item \textbf{Negative Deviation ($\gamma_i < 1$)}: Occurs when A-B attractions are stronger than A-A and B-B attractions (molecules "prefer" to stay in the liquid). This leads to a lower-than-ideal total vapor pressure.
    \item \textbf{Ideal Solution ($\gamma_i = 1$)}: The system follows Raoult's Law.
\end{itemize}
\end{conceptbox}

\subsection*{Modified Raoult's Law}
\begin{conceptbox}
Modified Raoult's Law is the fundamental equation for VLE in non-ideal systems. It incorporates the activity coefficient to correct the liquid-phase contribution to the partial pressure of a component.
\end{conceptbox}
\begin{keybox}
The key variables for non-ideal VLE are:
\begin{itemize}[itemsep=0pt]
    \item \textbf{$\gamma_i$}: The activity coefficient of component $i$ in the liquid phase (dimensionless).
    \item \textbf{$x_i, y_i$}: Mole fractions of component $i$ in the liquid and vapor phases.
    \item \textbf{$P, P_i^{sat}$}: Total system pressure and saturation pressure of pure component $i$.
\end{itemize}
\end{keybox}
\begin{formulabox}
The mathematical statement of Modified Raoult's Law is:
$$ y_i P = x_i \gamma_i P_i^{sat} $$
\end{formulabox}

\subsection*{Activity Coefficient Models}
\begin{conceptbox}
Activity coefficients are functions of composition and sometimes temperature. They must be calculated using a thermodynamic model derived from expressions for the excess Gibbs free energy ($G^E$). These models use empirical parameters fit to experimental VLE data.
\end{conceptbox}
\begin{formulabox}[title=Common Activity Coefficient Models]
\begin{itemize}[itemsep=2pt]
    \item \textbf{One-Parameter Margules Equation}: A simple symmetric model for binary systems. The parameter $A$ is constant.
    $$ \ln \gamma_1 = A x_2^2 \quad \text{and} \quad \ln \gamma_2 = A x_1^2 $$
    \item \textbf{Van Laar Equation}: A more flexible two-parameter model for binary systems. The parameters are $A_{12}$ and $A_{21}$.
    $$ \ln \gamma_1 = A_{12} \left( \frac{A_{21} x_2}{A_{12} x_1 + A_{21} x_2} \right)^2 $$
    $$ \ln \gamma_2 = A_{21} \left( \frac{A_{12} x_1}{A_{12} x_1 + A_{21} x_2} \right)^2 $$
\end{itemize}
\end{formulabox}

\subsection*{Non-Ideal VLE Calculation Equations}
\begin{formulabox}[title=Bubble and Dew Point Equations for Non-Ideal Systems]
\begin{itemize}[itemsep=2pt]
    \item \textbf{Bubble Pressure Equation (given $x_i$)}:
    $$ P = \sum_i x_i \gamma_i P_i^{sat} = x_1 \gamma_1 P_1^{sat} + x_2 \gamma_2 P_2^{sat} $$
    \item \textbf{Dew Pressure Equation (given $y_i$)}:
    $$ P = \frac{1}{\sum_i \frac{y_i}{\gamma_i P_i^{sat}}} = \frac{1}{\frac{y_1}{\gamma_1 P_1^{sat}} + \frac{y_2}{\gamma_2 P_2^{sat}}} $$
\end{itemize}
\end{formulabox}

\newpage
\subsection*{Example Problems}

\begin{examplebox}{Bubble Pressure using the Margules Equation}
A vapor-phase mixture containing 30 mol\% component 1 and 70 mol\% component 2 is compressed at a fixed temperature until it is completely liquefied. The saturation pressures are $P_1^{sat} = 0.82$ bar and $P_2^{sat} = 1.93$ bar. Experimental data shows that the bubble pressure of a 50:50 mixture of these components is 1.08 bar.
\begin{itemize}[itemsep=0pt]
    \item a) Assuming the one-parameter Margules equation applies, estimate the pressure required to completely liquefy the 30:70 mixture.
    \item b) Would the required pressure be higher or lower if the components formed an ideal solution? Explain.
\end{itemize}
\end{examplebox}
\begin{stepbox}
\begin{enumerate}[label=\textbf{Step \arabic*:}, wide=0pt, leftmargin=*, itemsep=2pt]
    \item \textbf{Find the Margules Parameter, A}
    
    The pressure to completely liquefy a vapor is its \textbf{bubble pressure}. We are given one data point: for a liquid with $x_1=0.5$ and $x_2=0.5$, the bubble pressure is $P=1.08$ bar. We use this point to find the parameter $A$.
    \begin{itemize}[itemsep=2pt]
        \item Start with the bubble pressure equation:
        $$ P = x_1 \gamma_1 P_1^{sat} + x_2 \gamma_2 P_2^{sat} $$
        \item Substitute the Margules expressions for $\gamma_1 = \exp(A x_2^2)$ and $\gamma_2 = \exp(A x_1^2)$:
        $$ 1.08 = (0.5) \exp(A (0.5)^2) (0.82) + (0.5) \exp(A (0.5)^2) (1.93) $$
        \item Since the composition is symmetric ($x_1=x_2=0.5$), the activity coefficients are equal: $\gamma_1 = \gamma_2 = \exp(0.25A)$. Factor this term out:
        $$ 1.08 = \exp(0.25A) [ (0.5)(0.82) + (0.5)(1.93) ] $$
        $$ 1.08 = \exp(0.25A) [ 0.41 + 0.965 ] = \exp(0.25A) [ 1.375 ] $$
        \item Solve for the exponential term:
        $$ \exp(0.25A) = \frac{1.08}{1.375} = 0.78545 $$
        \item Take the natural logarithm to solve for A:
        $$ 0.25A = \ln(0.78545) = -0.2415 $$
        $$ A = \frac{-0.2415}{0.25} = -0.966 $$
    \end{itemize}
\end{enumerate}
\end{stepbox}

\newpage
\begin{stepbox}
\begin{enumerate}[label=\textbf{Step \arabic*:}, wide=0pt, leftmargin=*, itemsep=2pt, start=2]
    \item \textbf{Calculate the Bubble Pressure of the 30:70 Mixture}
    
    Now we use the determined parameter $A=-0.966$ to find the bubble pressure for the target liquid composition: $x_1=0.30$ and $x_2=0.70$.
    \begin{itemize}[itemsep=2pt]
        \item First, calculate the activity coefficients at this new composition:
        $$ \ln \gamma_1 = A x_2^2 = (-0.966) (0.70)^2 = -0.4733 \implies \gamma_1 = e^{-0.4733} = 0.623 $$
        $$ \ln \gamma_2 = A x_1^2 = (-0.966) (0.30)^2 = -0.0869 \implies \gamma_2 = e^{-0.0869} = 0.917 $$
        \item Now, use these values in the bubble pressure equation:
        $$ P = x_1 \gamma_1 P_1^{sat} + x_2 \gamma_2 P_2^{sat} $$
        $$ P = (0.30)(0.623)(0.82) + (0.70)(0.917)(1.93) $$
        $$ P = 0.1533 + 1.2372 = 1.3905 \, \text{bar} $$
    \end{itemize}
    The pressure required to completely liquefy the mixture is \textbf{1.39 bar}.
    
    \item \textbf{Comparison with Ideal Solution}
    \begin{itemize}[itemsep=2pt]
        \item \textbf{Analysis of Non-Ideality}: We found that the Margules parameter $A$ is negative. This means that for all compositions, $\ln \gamma_i$ will be negative, and therefore the activity coefficients $\gamma_i$ will be less than 1. This signifies a \textbf{negative deviation} from Raoult's Law.
        \item \textbf{Molecular Interpretation}: A negative deviation implies that the attractive forces between unlike molecules (1-2) are stronger than the average attractive forces between like molecules (1-1 and 2-2). This strong attraction holds molecules in the liquid phase more effectively, resulting in a lower total vapor pressure compared to an ideal solution at the same composition.
        \item \textbf{Conclusion}: Since the real solution exhibits a lower vapor pressure, it requires less external pressure to be fully condensed. Therefore, the required pressure for an ideal solution would be \textbf{higher} than for the real solution.
        \item \textbf{Verification Calculation}: For an ideal solution, $\gamma_1=\gamma_2=1$.
        $$ P_{ideal} = x_1 P_1^{sat} + x_2 P_2^{sat} = (0.30)(0.82) + (0.70)(1.93) = 0.246 + 1.351 = 1.597 \, \text{bar} $$
        As predicted, the ideal pressure (1.60 bar) is significantly higher than the real pressure (1.39 bar).
    \end{itemize}
\end{enumerate}
\end{stepbox}

\newpage
\begin{examplebox}{Bubble Temperature using the Van Laar Equation}
Calculate the bubble temperature and the equilibrium vapor composition for a binary system with a liquid mole fraction of $x_1 = 0.40$ at a total pressure of 70 kPa. The system is non-ideal and is described by the two-parameter Van Laar model with $A_{12}=1.2$ and $A_{21}=0.8$. The saturation pressures are given by the Antoine equations (T in $^\circ$C, P in kPa):
$$ \ln(P_1^{sat}) = 14.3 - \frac{2940}{T + 224} \quad \text{and} \quad \ln(P_2^{sat}) = 14.2 - \frac{2975}{T + 209} $$
\end{examplebox}
\begin{stepbox}
\begin{enumerate}[label=\textbf{Step \arabic*:}, wide=0pt, leftmargin=*, itemsep=2pt]
    \item \textbf{Identify the Strategy}
    
    This is a bubble temperature calculation for a non-ideal system. We need to find the temperature $T$ and the vapor composition ($y_1, y_2$) that satisfy the VLE conditions simultaneously. This problem involves multiple non-linear equations and must be solved with a numerical tool (e.g., POLYMATH, Aspen Plus, MATLAB, or a Python script).
    
    \item \textbf{List the Unknowns and Governing Equations}
    
    For a numerical solver, we must define all variables and the equations that constrain them. The knowns are $P=70$ kPa, $x_1=0.40$, $x_2=0.60$, $A_{12}=1.2$, and $A_{21}=0.8$.
    
    \begin{itemize}[itemsep=2pt]
        \item \textbf{Unknowns (7)}: $T, y_1, y_2, P_1^{sat}, P_2^{sat}, \gamma_1, \gamma_2$
        \item \textbf{Equations (7)}:
            \begin{enumerate}[label=\textbf{Eq. \arabic*:}, wide=0pt, leftmargin=*, itemsep=2pt]
                \item $y_1 P = x_1 \gamma_1 P_1^{sat}$ \textit{(Modified Raoult's Law, Comp 1)}
                \item $y_2 P = x_2 \gamma_2 P_2^{sat}$ \textit{(Modified Raoult's Law, Comp 2)}
                \item $P_1^{sat} = \exp(14.3 - 2940 / (T + 224))$ \textit{(Antoine Equation, Comp 1)}
                \item $P_2^{sat} = \exp(14.2 - 2975 / (T + 209))$ \textit{(Antoine Equation, Comp 2)}
                \item $\ln \gamma_1 = A_{12} \left( \frac{A_{21} x_2}{A_{12} x_1 + A_{21} x_2} \right)^2$ \textit{(Van Laar Equation, Comp 1)}
                \item $\ln \gamma_2 = A_{21} \left( \frac{A_{12} x_1}{A_{12} x_1 + A_{21} x_2} \right)^2$ \textit{(Van Laar Equation, Comp 2)}
                \item $y_1 + y_2 = 1$ \textit{(Summation Constraint)}
            \end{enumerate}
    \end{itemize}
    
    \item \textbf{Solver Results and Interpretation}
    
    Inputting this system of equations into a numerical solver provides the solution.
    
    \begin{itemize}[itemsep=2pt]
        \item \textbf{Bubble Temperature}: $T \approx 84.1^\circ\text{C}$
        \item \textbf{Vapor Composition}: $y_1 \approx 0.69$, $y_2 \approx 0.31$
        \item \textbf{Activity Coefficients}: $\gamma_1 \approx 1.34$, $\gamma_2 \approx 1.15$
        \item \textbf{Saturation Pressures}: $P_1^{sat} \approx 89.9$ kPa, $P_2^{sat} \approx 47.9$ kPa
    \end{itemize}

\end{enumerate}
\end{stepbox}

\newpage

\section*{Fugacity of Mixtures}
nd{document}
