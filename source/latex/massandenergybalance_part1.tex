\documentclass[12pt]{article}
\usepackage[paperwidth=8.5in, paperheight=11in, margin=1.0in, headheight=12pt]{geometry}
\usepackage{amsmath,amssymb,amsthm}
\usepackage[most]{tcolorbox}
\usepackage{enumitem}
\usepackage{xcolor}
\usepackage{hyperref}
\usepackage{fancyhdr}
\usepackage{titlesec}
\usepackage{graphicx}
% Define custom colors for chemical engineering theme
\definecolor{conceptcolor}{RGB}{52, 73, 94}      % Dark blue-gray
\definecolor{formulacolor}{RGB}{231, 76, 60}     % Red for formulas
\definecolor{examplecolor}{RGB}{39, 174, 96}     % Green for examples
\definecolor{stepcolor}{RGB}{142, 68, 173}       % Purple for solution steps
\definecolor{keycolor}{RGB}{243, 156, 18}        % Orange for key points
% Configure fancy headers
\pagestyle{fancy}
\fancyhf{}
\fancyhead[L]{PE Study Guide}
\fancyhead[R]{Process Fundamentals}
\fancyfoot[C]{\thepage}
\renewcommand{\baselinestretch}{1.1}
\setlength{\parindent}{0.25in}
\setlength{\parskip}{3pt}
% Configure section formatting
\titleformat{\section}
  {\normalfont\LARGE\bfseries\color{conceptcolor}}
  {\thesection}{1em}{}
\titleformat{\subsection}
  {\normalfont\Large\bfseries\color{conceptcolor}}
  {\thesubsection}{1em}{}
% Define custom environments
\newtcolorbox{conceptbox}[1][]{
  enhanced,
  colback=conceptcolor!10,
  colframe=conceptcolor,
  arc=3mm,
  title=Key Concept,
  fonttitle=\bfseries\sffamily\normalsize,
  fontupper=\small,
  #1
}
\newtcolorbox{formulabox}[1][]{
  enhanced,
  colback=formulacolor!10,
  colframe=formulacolor,
  arc=2mm,
  title=Important Formula,
  fonttitle=\bfseries\sffamily\normalsize,
  fontupper=\small,
  #1
}
\newtcolorbox{examplebox}[2][]{
  enhanced,
  colback=examplecolor!10,
  colframe=examplecolor,
  arc=3mm,
  title=Example Problem: #2,
  fonttitle=\bfseries\sffamily\normalsize,
  fontupper=\small,
  #1
}
\newtcolorbox{stepbox}[1][]{
  enhanced,
  colback=stepcolor!10,
  colframe=stepcolor,
  arc=2mm,
  title=Solution Steps,
  fonttitle=\bfseries\sffamily\normalsize,
  fontupper=\small,
  #1
}
\newtcolorbox{keybox}[1][]{
  enhanced,
  colback=keycolor!10,
  colframe=keycolor,
  arc=2mm,
  title=Key Variables \& Definitions,
  fonttitle=\bfseries\sffamily\normalsize,
  fontupper=\small,
  #1
}
\begin{document}
\begin{center}
\Huge\textbf{\color{conceptcolor}Chemical Engineering PE Exam}\\
\LARGE\textbf{\color{conceptcolor}Process Fundamentals Study Guide}
\end{center}
\vspace{0.5cm}
\hrule
\vspace{0.5cm}
\section*{Introduction to Process Fundamentals}
Welcome to your chemical engineering refresher for the Professional Engineering (PE) exam. This guide is designed to rebuild your knowledge from the ground up, starting with the most fundamental concepts. The core of chemical process analysis rests on the laws of conservation, primarily the conservation of mass and energy. We will begin by focusing on the material balance, the foundation upon which most other process calculations are built.
\begin{conceptbox}
A \textbf{system} is a specific region of space we are interested in (e.g., a reactor, a distillation column, a pipe). The boundary separating the system from its surroundings is called the \textbf{control volume}. The analysis of what crosses this boundary is the key to solving process problems.
\end{conceptbox}
\section*{Material Balances}
The principle of mass conservation states that mass cannot be created or destroyed. When applied to a process system, this gives us the general material balance equation.
\subsection*{The General Balance Equation}
\begin{formulabox}
\textbf{Conceptual Balance Equation:}
$$ \text{Accumulation} = \text{In} - \text{Out} + \text{Generation} - \text{Consumption} $$
\end{formulabox}
\begin{keybox}
\textbf{Term Definitions:}
\begin{itemize}[leftmargin=*]
    \item \textbf{Accumulation}: The change in the amount of the quantity within the system over time
    \item \textbf{In}: The rate at which the quantity enters the system through its boundaries
    \item \textbf{Out}: The rate at which the quantity leaves the system through its boundaries
    \item \textbf{Generation}: The rate at which the quantity is produced within the system (e.g., by chemical reaction)
    \item \textbf{Consumption}: The rate at which the quantity is consumed within the system (e.g., by chemical reaction)
\end{itemize}
\end{keybox}
\begin{conceptbox}[title=Steady-State Systems]
A system is at \textbf{steady-state} if all process variables (temperature, pressure, flow rate, composition) are constant with respect to time. At steady-state, the \textbf{Accumulation} term is zero.
\end{conceptbox}
\subsection*{Overall Steady-State Balance (No Reaction)}
For a system at steady-state with no chemical reactions, the generation and consumption terms are also zero. The general balance equation simplifies dramatically.
\begin{conceptbox}[title=Fundamental Principle]
\textbf{What goes in must come out.}
\end{conceptbox}
\begin{keybox}
\textbf{Variable Definitions:}
\begin{itemize}[leftmargin=*]
    \item $\dot{m}_{in}$: The total mass flow rate into the system per unit time (e.g., kg/s, lb/hr). The dot over the 'm' signifies a rate
    \item $\dot{m}_{out}$: The total mass flow rate out of the system per unit time (e.g., kg/s, lb/hr)
\end{itemize}
\end{keybox}
\begin{formulabox}[title=Overall Mass Balance]
$$ \dot{m}_{in} = \dot{m}_{out} $$
This simple equation is the starting point for analyzing any non-reactive, steady-state process.
\end{formulabox}
\subsection*{Component Steady-State Balance (No Reaction)}
While the overall balance is useful, we often need to track individual chemical species (components) through a system. This is especially true for separation processes like distillation, absorption, or flash separation.
\begin{conceptbox}[title=Component Balance Principle]
The mass of a specific component entering the system must equal the mass of that same component leaving the system.
\end{conceptbox}
\begin{keybox}
\textbf{Standard Variable Definitions:}
\begin{itemize}[leftmargin=*]
    \item $F$: Total mass flow rate of the incoming \textbf{F}eed stream (kg/s)
    \item $V$: Total mass flow rate of the outgoing \textbf{V}apor stream (kg/s)
    \item $L$: Total mass flow rate of the outgoing \textbf{L}iquid stream (kg/s)
    \item $z_i$: The mass fraction of component $i$ in the feed stream
    \item $y_i$: The mass fraction of component $i$ in the vapor stream
    \item $x_i$: The mass fraction of component $i$ in the liquid stream
\end{itemize}
\end{keybox}
\begin{formulabox}[title=Component Balance Equation]
\textbf{Derivation:} Apply the "In = Out" principle to a single component, $i$:
$$ (\text{Mass of } i \text{ in}) = (\text{Mass of } i \text{ out}) $$
$$ F z_i = V y_i + L x_i $$
This equation can be written for each component in the mixture.
\end{formulabox}
\vspace{0.5cm}

\newpage

\begin{examplebox}{Flash Drum Separation}
\textbf{Problem Statement:} A feed stream of 1000 kg/hr containing 40\% ethanol and 60\% water by mass enters a flash drum. The vapor stream leaving the top is found to contain 70\% ethanol, and the liquid stream leaving the bottom contains 20\% ethanol. The system is at steady-state. Calculate the mass flow rates of the vapor ($V$) and liquid ($L$) streams.
\end{examplebox}

\begin{stepbox}
\begin{enumerate}[label=\textbf{Step \arabic*:}, wide=0pt, leftmargin=*, itemsep=2pt]
    \item \textbf{List Knowns and Define Basis}
    
    Basis: 1 hour of operation
    \begin{itemize}[itemsep=0pt]
        \item Feed Flow ($F$) = 1000 kg/hr
        \item Feed ethanol fraction ($z_e$) = 0.40
        \item Vapor ethanol fraction ($y_e$) = 0.70
        \item Liquid ethanol fraction ($x_e$) = 0.20
    \end{itemize}
    
    \item \textbf{Set up Material Balances}
    
    Two unknowns ($V$ and $L$) require two equations:
    
    \textbf{Overall Balance:} $F = V + L \Rightarrow 1000 = V + L$ \quad (Eq. 1)
    
    \textbf{Ethanol Balance:} $F z_e = V y_e + L x_e \Rightarrow 400 = 0.7V + 0.2L$ \quad (Eq. 2)
    
    \item \textbf{Solve System of Equations}
    
    From Eq. 1: $L = 1000 - V$. Substitute into Eq. 2:
    $$ 400 = 0.7V + 0.2(1000 - V) = 0.7V + 200 - 0.2V = 0.5V + 200 $$
    $$ V = \frac{400-200}{0.5} = 400 \text{ kg/hr} $$
    $$ L = 1000 - 400 = 600 \text{ kg/hr} $$
    
    \item \textbf{Verification}
    
    \textbf{Water balance check:} $z_w=0.6$, $y_w=0.3$, $x_w=0.8$
    
    In: $(1000)(0.6) = 600$ kg/hr \quad Out: $(400)(0.3) + (600)(0.8) = 600$ kg/hr 
\end{enumerate}
\end{stepbox}

\newpage

\section*{Material Balances with Chemical Reactions}

When a chemical reaction occurs, components are generated and consumed. This means the simple "In = Out" balance for components no longer holds. However, atoms are conserved.

\subsection*{Atom Balances}

\begin{conceptbox}[title=Atomic Conservation Principle]
For any atom (e.g., Carbon, Hydrogen, Oxygen), the total number of moles of that atom entering the system must equal the total number of moles leaving the system. This method is powerful because it does not require calculating generation or consumption terms.
\end{conceptbox}

\begin{formulabox}[title=Atom Balance Equation]
For atom 'A':
$$ (\text{moles of A})_{in} = (\text{moles of A})_{out} $$
\end{formulabox}

\begin{examplebox}{Methane Combustion}
\textbf{Problem:} 100 mol/hr of methane ($CH_4$) is burned completely with 25\% excess air. Calculate the molar flow rate and composition of the flue gas. Assume air is 21\% $O_2$ and 79\% $N_2$ by mole.
\end{examplebox}

\begin{stepbox}
\begin{enumerate}[label=\textbf{Step \arabic*:}, leftmargin=*]

\item \textbf{Basis and Reaction:} 100 mol/hr CH\textsubscript{4}; \quad CH\textsubscript{4} + 2O\textsubscript{2} $\rightarrow$ CO\textsubscript{2} + 2H\textsubscript{2}O

\item \textbf{Feed:} O\textsubscript{2} = 100 × 2 × 1.25 = 250 mol/hr; \quad N\textsubscript{2} = 250 × 79/21 = 940.5 mol/hr

\item \textbf{Products:} CO\textsubscript{2} = 100; \quad H\textsubscript{2}O = 200; \quad N\textsubscript{2} = 940.5; \quad O\textsubscript{2} (excess) = 50 mol/hr

\item \textbf{Total Flow:} 100 + 200 + 940.5 + 50 = 1290.5 mol/hr

\item \textbf{Mole \%:} CO\textsubscript{2} = 7.75\%, H\textsubscript{2}O = 15.50\%, N\textsubscript{2} = 72.88\%, O\textsubscript{2} = 3.87\%

\end{enumerate}
\end{stepbox}

\newpage

\section*{Transient (Unsteady-State) Material Balances}

So far, we have focused on steady-state systems where process variables do not change over time. Now, we will examine transient processes, where the amount of material inside the system changes.

\subsection*{Defining Process Types}

\begin{keybox}
\textbf{Process Classifications:}
\begin{itemize}[itemsep=0pt]
    \item \textbf{Steady-State}: Accumulation = 0, properties constant with time
    \item \textbf{Transient}: Accumulation $\neq$ 0, properties change with time
    \item \textbf{Batch}: No material enters or leaves during process
    \item \textbf{Continuous}: Material continuously flows in and out
    \item \textbf{Semi-Batch}: Combination of batch and continuous characteristics
\end{itemize}
\end{keybox}

\subsection*{The General Balance with Accumulation}

\begin{conceptbox}[title=Transient Balance Principle]
The rate at which mass builds up or depletes within the system (accumulation) equals the rate of mass flowing in minus the rate flowing out.
\end{conceptbox}

\begin{keybox}
\textbf{Variable Definitions:}
\begin{itemize}[itemsep=0pt]
    \item $M$: Total mass inside system (kg)
    \item $t$: Time (s)
    \item $\frac{dM}{dt}$: Rate of mass accumulation (kg/s)
    \item $\dot{m}_{in}$: Mass flow rate entering (kg/s)
    \item $\dot{m}_{out}$: Mass flow rate leaving (kg/s)
\end{itemize}
\end{keybox}

\begin{formulabox}[title=Transient Mass Balance]
For a non-reacting, transient system:
$$ \frac{dM}{dt} = \dot{m}_{in} - \dot{m}_{out} $$
\end{formulabox}

\begin{examplebox}{Filling a Tank}
\textbf{Problem:} A 2 m³ tank, initially half-full of water, is fed at 4 kg/s and withdrawn at 2 kg/s. How long to fill completely? ($\rho_{water} = 1000$ kg/m³)
\end{examplebox}

\begin{stepbox}
\begin{enumerate}[label=\textbf{Step \arabic*:}, wide=0pt, leftmargin=*, itemsep=2pt]
    \item \textbf{Define System and Knowns}
    
    Tank volume: 2 m³, Initial volume: 1 m³, Volume to fill: 1 m³
    
    $\dot{m}_{in} = 4$ kg/s, $\dot{m}_{out} = 2$ kg/s, $\rho = 1000$ kg/m³
    
    \item \textbf{Apply Material Balance}
    
    $\frac{dM}{dt} = 4 - 2 = 2$ kg/s (positive = accumulating)
    
    \item \textbf{Convert to Volumetric Rate}
    
    $\frac{dV}{dt} = \frac{dM/dt}{\rho} = \frac{2}{1000} = 0.002$ m³/s
    
    \item \textbf{Calculate Fill Time}
    
    $t = \frac{V_{fill}}{dV/dt} = \frac{1}{0.002} = 500$ seconds
\end{enumerate}
\end{stepbox}

\newpage

\section*{A Strategic Approach to Material Balances}

With the fundamental balance equations established, we now focus on a systematic strategy for translating word problems into solvable systems of equations.

\subsection*{Core Principles and Initial Analysis}

\begin{conceptbox}[title=The Golden Rule of Steady-State]
For any steady-state system, the total mass entering must equal the total mass leaving:
$$ \sum \dot{m}_{in} = \sum \dot{m}_{out} $$
This holds regardless of reactions, phases, or compositions.
\end{conceptbox}

\begin{keybox}
\textbf{Problem-Solving Checklist:}
\begin{itemize}[itemsep=0pt]
    \item \textbf{Identify the Goal}: What is the final answer needed?
    \item \textbf{Look for Keywords}: "overhead," "bottoms," "pure," "inert"
    \item \textbf{Check Assumptions}: If a substance enters, it must exit (unless consumed)
    \item \textbf{Mass Conservation}: Total mass in = total mass out (steady-state)
\end{itemize}
\end{keybox}

\subsection*{Drawing and Labeling Process Flow Diagrams}

\begin{conceptbox}[title=Variable Naming Convention]
A consistent notation prevents confusion and errors in complex problems.
\end{conceptbox}

\begin{keybox}
\textbf{Recommended Notation:}
\begin{itemize}[itemsep=0pt]
    \item \textbf{Flow Rates}: $\dot{m}$ (mass), $\dot{n}$ (molar), numbered by stream
    \item \textbf{Fractions}: $x$ (mass fraction), $y$ (mole fraction)
    \item \textbf{Combined}: $x_{B,1}$ = mass fraction of Benzene in stream 1
    \item \textbf{Pre-calculation}: Fill in simple unknowns immediately (fractions sum to 1)
\end{itemize}
\end{keybox}

\subsection*{Setting Up Independent Balance Equations}

\begin{formulabox}[title=Independent Equations Rule]
For a non-reactive system:
\begin{itemize}[itemsep=0pt]
    \item Number of independent equations = Number of chemical species
    \item Can use $N$ component balances OR $(N-1)$ component + 1 overall balance
    \item Cannot use all $N+1$ equations as independent set
\end{itemize}
\end{formulabox}

\begin{examplebox}{Setting Up Balances for Mixing Column}
\textbf{Problem:} Stream 1 (30\% A, 70\% B) and Stream 2 (40\% A, 60\% C) feed a column. Stream 3 is pure A. Stream 4 contains 20\% B. Set up the independent mass balance equations.
\end{examplebox}

\begin{stepbox}
\begin{enumerate}[label=\textbf{Step \arabic*:}, wide=0pt, leftmargin=*, itemsep=2pt]
    \item \textbf{Label Streams and Compositions}
    
    Stream 1: $x_{A,1}=0.30$, $x_{B,1}=0.70$, $x_{C,1}=0$
    
    Stream 2: $x_{A,2}=0.40$, $x_{B,2}=0$, $x_{C,2}=0.60$
    
    Stream 3: $x_{A,3}=1.0$, $x_{B,3}=0$, $x_{C,3}=0$
    
    Stream 4: $x_{B,4}=0.20$, $x_{A,4} + x_{C,4} = 0.80$
    
    \item \textbf{Write Component Balances}
    
    \textbf{Component A:} $0.30\dot{m}_1 + 0.40\dot{m}_2 = 1.0\dot{m}_3 + x_{A,4}\dot{m}_4$
    
    \textbf{Component B:} $0.70\dot{m}_1 = 0.20\dot{m}_4$
    
    \textbf{Component C:} $0.60\dot{m}_2 = (0.80 - x_{A,4})\dot{m}_4$
    
    \item \textbf{Verify Overall Balance}
    
    Sum of component balances gives: $\dot{m}_1 + \dot{m}_2 = \dot{m}_3 + \dot{m}_4$ 
\end{enumerate}
\end{stepbox}

\subsection*{Degrees of Freedom Analysis}

\begin{formulabox}[title=Degrees of Freedom Calculation]
$$ \text{DoF} = (\text{Number of Unknowns}) - (\text{Number of Independent Equations}) $$
\begin{itemize}[itemsep=0pt]
    \item DoF = 0: Exactly determined (solvable)
    \item DoF $>$ 0: Underspecified (need more information)
    \item DoF $<$ 0: Overspecified (contradictory information)
\end{itemize}
\end{formulabox}

\begin{conceptbox}[title=Example DoF Analysis]
In the mixing column example:
\begin{itemize}[itemsep=0pt]
    \item Unknowns: $\dot{m}_1, \dot{m}_2, \dot{m}_3, \dot{m}_4, x_{A,4}$ (5 total)
    \item Independent equations: 3 components = 3 equations
    \item DoF = 5 - 3 = 2 (underspecified)
\end{itemize}
Need 2 more pieces of information to solve!
\end{conceptbox}

\newpage

\section*{Material Balances with Multiple Reactions}

Chemical processes rarely involve a single, perfect reaction. Often, a primary reaction occurs alongside one or more side reactions. To analyze these systems, we can use two robust methods: balancing conserved atomic species or tracking the progress of each reaction individually using the extent of reaction.

\begin{conceptbox}[title=Multiple Reaction Systems]
When multiple reactions occur simultaneously, we cannot simply use component balances because species are both generated and consumed. However, atoms are always conserved, providing a powerful analysis tool.
\end{conceptbox}

\begin{examplebox}{Dehydrogenation of Ethane}
\textbf{Problem:} 100 moles of ethane ($C_2H_6$) are fed to a reactor. The primary reaction is dehydrogenation to form ethylene ($C_2H_4$) and hydrogen ($H_2$). A side reaction occurs where ethylene reacts with ethane to form propylene ($C_3H_6$) and methane ($CH_4$). Given: fractional conversion of ethane = 0.7, selectivity of ethylene to propylene = 5. Find the molar composition of the product gas.

\textbf{Reactions:}
\begin{enumerate}[itemsep=0pt]
    \item $C_2H_6 \longrightarrow C_2H_4 + H_2$
    \item $C_2H_6 + C_2H_4 \longrightarrow C_3H_6 + CH_4$
\end{enumerate}
\end{examplebox}

\begin{keybox}
\textbf{Process Flow Information:}
\begin{itemize}[itemsep=0pt]
    \item \textbf{Inlet}: 100 mol $C_2H_6$
    \item \textbf{Outlet}: Contains unreacted $C_2H_6$ and products $C_2H_4, H_2, C_3H_6, CH_4$
    \item \textbf{Unknowns}: 5 outlet molar flow rates
    \item \textbf{Available Info}: 2 atomic balances + conversion + selectivity + stoichiometric constraint
\end{itemize}
\end{keybox}

\subsection*{Method 1: Atomic Species Balances}

\begin{conceptbox}[title=Atomic Conservation Principle]
For any atom, the total moles entering the system must equal the total moles leaving the system. This method works regardless of reaction complexity.
\end{conceptbox}

\begin{formulabox}[title=Atomic Balance Equation]
For any atom X:
$$ (\text{Moles of atom X})_{in} = (\text{Moles of atom X})_{out} $$
\end{formulabox}

\begin{stepbox}
\begin{enumerate}[label=\textbf{Step \arabic*:}, wide=0pt, leftmargin=*, itemsep=2pt]
    \item \textbf{Apply Given Constraints}
    
    \textbf{Fractional Conversion:} $f = \frac{100 - \dot{n}_{C_2H_6}}{100} = 0.7 \Rightarrow \dot{n}_{C_2H_6} = 30$ mol
    
    \textbf{Selectivity:} $S = \frac{\dot{n}_{C_2H_4}}{\dot{n}_{C_3H_6}} = 5 \Rightarrow \dot{n}_{C_2H_4} = 5\dot{n}_{C_3H_6}$
    
    \textbf{Stoichiometry:} From Reaction 2: $\dot{n}_{C_3H_6} = \dot{n}_{CH_4}$
    
    \item \textbf{Carbon Balance}
    
    C atoms in: $100 \times 2 = 200$ mol C
    
    C atoms out: $(30 \times 2) + (5\dot{n}_{C_3H_6} \times 2) + (\dot{n}_{C_3H_6} \times 3) + (\dot{n}_{C_3H_6} \times 1)$
    
    $200 = 60 + 14\dot{n}_{C_3H_6} \Rightarrow \dot{n}_{C_3H_6} = 10$ mol
    
    \item \textbf{Calculate Other Species}
    
    $\dot{n}_{CH_4} = \dot{n}_{C_3H_6} = 10$ mol
    
    $\dot{n}_{C_2H_4} = 5 \times 10 = 50$ mol
    
    \item \textbf{Hydrogen Balance}
    
    H atoms in: $100 \times 6 = 600$ mol H
    
    H atoms out: $(30 \times 6) + (50 \times 4) + (2\dot{n}_{H_2}) + (10 \times 6) + (10 \times 4)$
    
    $600 = 480 + 2\dot{n}_{H_2} \Rightarrow \dot{n}_{H_2} = 60$ mol
    
    \item \textbf{Final Composition}
    
    Total: $30 + 50 + 60 + 10 + 10 = 160$ mol
    
    Mole fractions: $C_2H_6$: 18.75\%, $C_2H_4$: 31.25\%, $H_2$: 37.50\%, $C_3H_6$: 6.25\%, $CH_4$: 6.25\%
\end{enumerate}
\end{stepbox}

\subsection*{Method 2: Extent of Reaction}

\begin{conceptbox}[title=Extent of Reaction Concept]
The extent of reaction ($\xi$) tracks the progress of each individual reaction. It has units of moles and represents the "moles of reaction" that have occurred.
\end{conceptbox}

\begin{formulabox}[title=Species Balance with Extents]
For species $i$ involved in multiple reactions $j$:
$$ \dot{n}_i = \dot{n}_{i,in} + \sum_j \nu_{ij}\xi_j $$
where $\nu_{ij}$ is the stoichiometric coefficient (negative for reactants, positive for products).
\end{formulabox}

\begin{keybox}
\textbf{Species Balance Equations:}
\begin{itemize}[itemsep=0pt]
    \item $\dot{n}_{C_2H_6} = 100 - \xi_1 - \xi_2$ (reactant in both reactions)
    \item $\dot{n}_{C_2H_4} = \xi_1 - \xi_2$ (product in Rxn 1, reactant in Rxn 2)
    \item $\dot{n}_{H_2} = \xi_1$ (product in Rxn 1 only)
    \item $\dot{n}_{C_3H_6} = \xi_2$ (product in Rxn 2 only)
    \item $\dot{n}_{CH_4} = \xi_2$ (product in Rxn 2 only)
\end{itemize}
\end{keybox}

\begin{stepbox}
\begin{enumerate}[label=\textbf{Step \arabic*:}, wide=0pt, leftmargin=*, itemsep=2pt]
    \item \textbf{Apply Conversion Constraint}
    
    From $\dot{n}_{C_2H_6} = 30$: $30 = 100 - \xi_1 - \xi_2$
    
    Therefore: $\xi_1 + \xi_2 = 70$ (Equation 1)
    
    \item \textbf{Apply Selectivity Constraint}
    
    $S = \frac{\dot{n}_{C_2H_4}}{\dot{n}_{C_3H_6}} = \frac{\xi_1 - \xi_2}{\xi_2} = 5$
    
    Therefore: $\xi_1 - \xi_2 = 5\xi_2 \Rightarrow \xi_1 = 6\xi_2$ (Equation 2)
    
    \item \textbf{Solve for Extents}
    
    Substitute Eq. 2 into Eq. 1: $(6\xi_2) + \xi_2 = 70$
    
    $7\xi_2 = 70 \Rightarrow \xi_2 = 10$ mol
    
    $\xi_1 = 6 \times 10 = 60$ mol
    
    \item \textbf{Calculate Outlet Flows}
    
    $\dot{n}_{C_2H_6} = 100 - 60 - 10 = 30$ mol
    
    $\dot{n}_{C_2H_4} = 60 - 10 = 50$ mol
    
    $\dot{n}_{H_2} = 60$ mol, $\dot{n}_{C_3H_6} = 10$ mol, $\dot{n}_{CH_4} = 10$ mol
    
    \item \textbf{Verify Results}
    
    Same composition as Method 1 
\end{enumerate}
\end{stepbox}

\subsection*{Comparison of Methods}

\begin{conceptbox}[title=Method Selection Guidelines]
\textbf{Atomic Species Balance Method:}
\begin{itemize}[itemsep=0pt]
    \item More direct when reaction stoichiometry is complex
    \item Requires fewer initial equations
    \item May lead to more complex algebraic systems
\end{itemize}

\textbf{Extent of Reaction Method:}
\begin{itemize}[itemsep=0pt]
    \item Requires more setup (equation for every species)
    \item Often leads to simpler systems of equations
    \item Provides direct insight into individual reaction progress
    \item Preferred when tracking multiple reactions separately
\end{itemize}

Both methods produce identical results. Choose based on problem complexity and personal preference.
\end{conceptbox}

\newpage

\section*{Combustion Reactions}

Combustion is a rapid reaction between a substance with an oxidant, usually oxygen, to produce heat and light. In chemical engineering, we are primarily concerned with the combustion of hydrocarbon fuels for analyzing furnaces, boilers, and reactors.

\subsection*{Complete vs. Incomplete Combustion}

\begin{conceptbox}[title=Types of Combustion]
\textbf{Complete Combustion}: Sufficient oxygen for all carbon → $CO_2$ and all hydrogen → $H_2O$

\textbf{Incomplete Combustion}: Insufficient oxygen supply, carbon → $CO$ instead of $CO_2$
\end{conceptbox}

\begin{formulabox}[title=Combustion Reaction Examples]
\textbf{Complete Methane Combustion:}
$$ CH_4 + 2O_2 \longrightarrow CO_2 + 2H_2O $$

\textbf{Incomplete Methane Combustion:}
$$ CH_4 + \frac{3}{2}O_2 \longrightarrow CO + 2H_2O $$
\end{formulabox}

\subsection*{Key Terminology for Combustion}

\begin{keybox}
\textbf{Critical Combustion Terms:}
\begin{itemize}[itemsep=0pt]
    \item \textbf{Theoretical Oxygen}: Molar amount of $O_2$ required for complete combustion (from stoichiometry)
    \item \textbf{Theoretical Air}: Quantity of air containing the theoretical oxygen amount
    \item \textbf{Excess Air}: Air fed above theoretical requirement (prevents incomplete combustion)
\end{itemize}
\end{keybox}

\begin{formulabox}[title=Percent Excess Air]
$$ \% \text{ Excess Air} = \frac{(\text{moles of air})_{\text{fed}} - (\text{moles of air})_{\text{theoretical}}}{(\text{moles of air})_{\text{theoretical}}} \times 100\% $$
\end{formulabox}

\begin{examplebox}{Complete Combustion of Methane}
\textbf{Problem:} 100 moles of methane ($CH_4$) are burned with 25\% excess air. The reaction proceeds to completion with no partial combustion. Find the moles of air fed and the composition of the stack gas. Assume air is 21\% $O_2$ and 79\% $N_2$ by mole.
\end{examplebox}

\begin{stepbox}
\begin{enumerate}[label=\textbf{Step \arabic*:}, wide=0pt, leftmargin=*, itemsep=2pt]
    \item \textbf{Basis and Stoichiometry}
    
    Basis: 100 moles $CH_4$ feed
    
    Reaction: $CH_4 + 2O_2 \longrightarrow CO_2 + 2H_2O$
    
    \item \textbf{Calculate Theoretical Air}
    
    Theoretical $O_2$: $100 \times 2 = 200$ mol $O_2$
    
    Theoretical air: $\frac{200 \text{ mol } O_2}{0.21} = 952.4$ mol air
    
    \item \textbf{Calculate Fed Air}
    
    With 25\% excess: $(\text{moles air})_{\text{fed}} = 952.4 \times 1.25 = 1190.5$ mol air
    
    \item \textbf{Apply Extent of Reaction}
    
    Complete conversion: $\xi = 100$ mol (all methane consumed)
    
    Fed $O_2$: $1190.5 \times 0.21 = 250$ mol
    
    Fed $N_2$: $1190.5 \times 0.79 = 940.5$ mol
    
    \item \textbf{Calculate Product Composition}
    
    $\dot{n}_{CO_2} = 0 + (1)\xi = 100$ mol
    
    $\dot{n}_{H_2O} = 0 + (2)\xi = 200$ mol
    
    $\dot{n}_{O_2} = 250 + (-2)\xi = 50$ mol (excess)
    
    $\dot{n}_{N_2} = 940.5$ mol (inert)
    
    \item \textbf{Final Results}
    
    Air fed: \textbf{1190.5 mol}
    
    Stack gas: $CO_2$: 100 mol, $H_2O$: 200 mol, $O_2$: 50 mol, $N_2$: 940.5 mol
\end{enumerate}
\end{stepbox}

\newpage

\section*{Material Balances with Recycle}

In many chemical processes, a portion of the product stream is separated and fed back to the reactor inlet. This recycle stream increases overall conversion, recovers catalyst, or dilutes feed streams.

\subsection*{Key Definitions for Recycle Systems}

\begin{keybox}
\textbf{Recycle System Components:}
\begin{itemize}[itemsep=0pt]
    \item \textbf{Fresh Feed}: Material entering the process from external source
    \item \textbf{Recycle Stream}: Stream returned from downstream separation to mix with fresh feed
    \item \textbf{Combined Feed}: Stream entering reactor (fresh feed + recycle mixture)
\end{itemize}
\end{keybox}

\begin{formulabox}[title=Recycle Ratio]
$$ R_R = \frac{F_{T,R}}{F_f} $$
where $F_{T,R}$ = total flow rate of recycle stream, $F_f$ = total flow rate of fresh feed
\end{formulabox}

\subsection*{Overall Conversion vs. Single-Pass Conversion}

\begin{conceptbox}[title=Critical Distinction]
Understanding the difference between overall and single-pass conversion is essential for recycle system analysis and design optimization.
\end{conceptbox}

\begin{formulabox}[title=Conversion Definitions]
\textbf{Overall Conversion} (measures entire process performance):
$$ \text{Overall Conversion} = \frac{F_{A,0} - F_{A,f}}{F_{A,0}} $$

\textbf{Single-Pass Conversion} (measures reactor performance only):
$$ \text{Single-Pass Conversion} = \frac{F_{A,m} - F_{A,outlet}}{F_{A,m}} $$

where:
\begin{itemize}[itemsep=0pt]
    \item $F_{A,0}$ = reactant A flow in fresh feed
    \item $F_{A,f}$ = reactant A flow in final outlet stream
    \item $F_{A,m}$ = reactant A flow in combined feed to reactor
    \item $F_{A,outlet}$ = reactant A flow leaving reactor
\end{itemize}
\end{formulabox}

\begin{conceptbox}[title=Recycle Strategy]
The goal of a recycle loop is to achieve high overall conversion even when single-pass conversion is relatively low. This allows:
\begin{itemize}[itemsep=0pt]
    \item Smaller reactor volumes
    \item Better temperature control
    \item Higher overall process efficiency
    \item Economic optimization of conversion vs. separation costs
\end{itemize}
\end{conceptbox}

\newpage

nd{document}
