\documentclass[12pt]{article}
\usepackage[paperwidth=8.5in, paperheight=11in, margin=1.0in, headheight=15pt]{geometry}
\usepackage{amsmath,amssymb,amsthm}
\usepackage[most]{tcolorbox}
\usepackage{enumitem}
\usepackage{xcolor}
\usepackage{hyperref}
\usepackage{fancyhdr}
\usepackage{titlesec}
\usepackage{graphicx}
% Define custom colors for chemical engineering theme
\definecolor{conceptcolor}{RGB}{52, 73, 94}      % Dark blue-gray
\definecolor{formulacolor}{RGB}{231, 76, 60}     % Red for formulas
\definecolor{examplecolor}{RGB}{39, 174, 96}     % Green for examples
\definecolor{stepcolor}{RGB}{142, 68, 173}       % Purple for solution steps
\definecolor{keycolor}{RGB}{243, 156, 18}        % Orange for key points
% Configure fancy headers
\pagestyle{fancy}
\fancyhf{}
\fancyhead[L]{PE Study Guide}
\fancyhead[R]{Process Fundamentals}
\fancyfoot[C]{\thepage}
\renewcommand{\baselinestretch}{1.1}
\setlength{\parindent}{0.25in}
\setlength{\parskip}{3pt}
% Configure section formatting
\titleformat{\section}
  {\normalfont\LARGE\bfseries\color{conceptcolor}}
  {\thesection}{1em}{}
\titleformat{\subsection}
  {\normalfont\Large\bfseries\color{conceptcolor}}
  {\thesubsection}{1em}{}
% Define custom environments
\newtcolorbox{conceptbox}[1][]{
  enhanced,
  colback=conceptcolor!10,
  colframe=conceptcolor,
  arc=3mm,
  title=Key Concept,
  fonttitle=\bfseries\sffamily\normalsize,
  fontupper=\small,
  #1
}
\newtcolorbox{formulabox}[1][]{
  enhanced,
  colback=formulacolor!10,
  colframe=formulacolor,
  arc=2mm,
  title=Important Formula,
  fonttitle=\bfseries\sffamily\normalsize,
  fontupper=\small,
  #1
}
\newtcolorbox{examplebox}[2][]{
  enhanced,
  colback=examplecolor!10,
  colframe=examplecolor,
  arc=3mm,
  title=Example Problem: #2,
  fonttitle=\bfseries\sffamily\normalsize,
  fontupper=\small,
  #1
}
\newtcolorbox{stepbox}[1][]{
  enhanced,
  colback=stepcolor!10,
  colframe=stepcolor,
  arc=2mm,
  title=Solution Steps,
  fonttitle=\bfseries\sffamily\normalsize,
  fontupper=\small,
  #1
}
\newtcolorbox{keybox}[1][]{
  enhanced,
  colback=keycolor!10,
  colframe=keycolor,
  arc=2mm,
  title=Key Variables \& Definitions,
  fonttitle=\bfseries\sffamily\normalsize,
  fontupper=\small,
  #1
}

\section*{First Law of Thermodynamics - Open Systems}

\begin{conceptbox}
\textbf{Open System Definition:} An open system (also called a flow system or control volume) is one where \textbf{mass can cross the system boundaries}. This is typical for most industrial chemical processes involving pumps, turbines, heat exchangers, and reactors.

\textbf{Key Difference from Closed Systems:} The First Law for open systems must account for energy transported into and out of the system by flowing mass. This energy transport occurs as enthalpy ($H$) rather than internal energy ($U$) because flowing streams carry both internal energy and flow work ($PV$).
\end{conceptbox}

\begin{keybox}
\textbf{Key Variables \& Definitions:}
\begin{itemize}[itemsep=0pt]
    \item $U_{total\_sys}$ = Total internal energy of all mass within system \text{(kJ)}
    \item $\dot{m}_{in}, \dot{m}_{out}$ = Mass flow rates entering and leaving system \text{(kg/s)}
    \item $H_{in}, H_{out}$ = Specific enthalpy of inlet and outlet streams \text{(kJ/kg)}
    \item $\dot{Q}$ = Rate of heat transfer into system \text{(kJ/s, kW)}
    \item $\dot{W}$ = Rate of work done on system \text{(kJ/s, kW)}
    \item $m_i, m_f$ = Initial and final mass in system \text{(kg)}
    \item $U_i, U_f$ = Initial and final specific internal energy \text{(kJ/kg)}
    \item $Q^t, W^t$ = Total heat and work over time period $t$ \text{(kJ)}
\end{itemize}
\end{keybox}

\begin{formulabox}
\textbf{General Unsteady-State Energy Balance:}
$$\frac{d(U_{total\_sys})}{dt} = \dot{m}_{in} H_{in} - \dot{m}_{out} H_{out} + \dot{Q} + \dot{W} \quad \text{(Equation 1)}$$

\textbf{Integrated Form for Filling Tank:}
$$m_f U_f - m_i U_i = m_{in} H_{in} + Q^t + W^t \quad \text{(Equation 2)}$$
\end{formulabox}

\begin{formulabox}
\textbf{Steady-State Adiabatic Compressor/Turbine:}
At steady state: $\frac{dU}{dt} = 0$ and $\dot{m}_{in} = \dot{m}_{out} = \dot{m}$
$$\dot{m} H_{in} + \dot{W}_s = \dot{m} H_{out} \quad \text{(Equation 3)}$$

\textbf{Throttle Process (Isenthalpic):}
Steady-state, adiabatic, no work:
$$H_{in} = H_{out} \quad \text{(Equation 4)}$$
\end{formulabox}

\subsection*{Worked Examples}

\begin{examplebox}{Filling an Empty Tank with Steam}
An empty, adiabatic tank is filled by opening a valve and allowing steam at $450^\circ\text{C}$ and 10 MPa to flow into it until the pressure equalizes at 10 MPa. What is the final temperature in the tank?
\end{examplebox}

\begin{stepbox}
\begin{enumerate}[label=\textbf{Step \arabic*:}, wide=0pt, leftmargin=*, itemsep=2pt]
    \item \textbf{Define System and Process}
    System: Tank interior (control volume)\\
    Process: Unsteady-state filling, adiabatic, no shaft work
    
    \item \textbf{Apply Integrated Energy Balance}
    Starting with Equation 2:
    $$m_f U_f - m_i U_i = m_{in} H_{in} + Q^t + W^t$$
    
    \item \textbf{Apply Process Conditions}
    \begin{itemize}[itemsep=0pt]
        \item Initially empty: $m_i = 0 \rightarrow m_i U_i = 0$
        \item Adiabatic: $Q^t = 0$
        \item No shaft work: $W^t = 0$
        \item Mass conservation: $m_{in} = m_f$ (all entering mass stays)
    \end{itemize}
    
    \item \textbf{Simplify Energy Balance}
    $$m_f U_f = m_f H_{in}$$
    
    Dividing by $m_f$:
    $$U_f = H_{in}$$
    
    \textbf{Key Result:} Final specific internal energy equals inlet specific enthalpy.
\end{enumerate}
\end{stepbox}

\newpage

\begin{stepbox}
\begin{enumerate}[label=\textbf{Step \arabic*:}, wide=0pt, leftmargin=*, itemsep=2pt, resume]
    \setcounter{enumi}{4}
    \item \textbf{Find Inlet Enthalpy from Steam Tables}
    At $P_{in} = 10$ MPa and $T_{in} = 450^\circ\text{C}$:
    $$H_{in} \approx 3242 \text{ kJ/kg}$$
    
    \item \textbf{Determine Final State Properties}
    Known final state properties:
    \begin{itemize}[itemsep=0pt]
        \item $P_f = 10$ MPa (pressure equalizes)
        \item $U_f = 3242$ kJ/kg (from energy balance)
    \end{itemize}
    
    \item \textbf{Find Final Temperature}
    Using steam tables at 10 MPa, find temperature where $U = 3242$ kJ/kg:
    $$T_f \approx \textbf{600}^{\circ}\text{C}$$
    
    \item \textbf{Physical Interpretation}
    The temperature increases from $450^\circ\text{C}$ to $600^\circ\text{C}$ because the flow work done by upstream fluid to push steam into the tank converts to internal energy, raising the temperature significantly.
\end{enumerate}
\end{stepbox}

\newpage

\begin{examplebox}{Adiabatic Semi-Batch Reactor}
An adiabatic reactor initially contains 100 mol of product B at $50^\circ\text{C}$. Reactant A is fed at $90^\circ\text{C}$ at 15 mol/hr. The exothermic reaction A $\rightarrow$ B is very fast. Find reactor temperature after 6 hours.

\textbf{Given Data:} $\Delta H_{rxn} = -20$ kJ/mol, $C_{P,A} = 60$ J/(mol$\cdot$K), $C_{P,B} = 70$ J/(mol$\cdot$K)
\end{examplebox}

\begin{stepbox}
\begin{enumerate}[label=\textbf{Step \arabic*:}, wide=0pt, leftmargin=*, itemsep=2pt]
    \item \textbf{Set Up Molar Energy Balance}
    Since reaction is very fast, A instantly converts to B upon entering.
    System contains only B at any time:
    $$\frac{d(N_B H_B)}{dt} = \dot{N}_{A,in} H_{A,in}$$
    
    \item \textbf{Integrate Over 6-Hour Period}
    With constant inlet conditions:
    $$\int_{initial}^{final} d(N_B H_B) = \int_{0}^{6} \dot{N}_{A,in} H_{A,in} dt$$
    $$(N_B H_B)_{final} - (N_B H_B)_{initial} = (\dot{N}_{A,in} \cdot t) \cdot H_{A,in}$$
    
    \item \textbf{Calculate Mole Quantities}
    \begin{itemize}[itemsep=0pt]
        \item Initial moles B: $N_{B,i} = 100$ mol
        \item Moles A added: $N_{A,added} = 15 \text{ mol/hr} \times 6 \text{ hr} = 90$ mol
        \item Final moles B: $N_{B,f} = 100 + 90 = 190$ mol
    \end{itemize}

    \item \textbf{Expand with Enthalpy Expressions}
    Using reference state at $25^\circ\text{C}$:
    $$H(T) = H_f^\circ + C_P(T - 25)$$
    
    Energy balance becomes:
    $$N_{B,f}[H_{f,B}^\circ + C_{P,B}(T_f - 25)] - N_{B,i}[H_{f,B}^\circ + C_{P,B}(50 - 25)]$$
    $$= N_{A,added}[H_{f,A}^\circ + C_{P,A}(90 - 25)]$$
    
    \item \textbf{Rearrange to Isolate Heat of Reaction}
    $$(N_{B,f} - N_{B,i}) H_{f,B}^\circ + N_{B,f}C_{P,B}(T_f - 25) - N_{B,i}C_{P,B}(25)$$
    $$= N_{A,added}H_{f,A}^\circ + N_{A,added}C_{P,A}(65)$$
    
    Since $(N_{B,f} - N_{B,i}) = N_{A,added} = 90$ mol:
    $$N_{B,f}C_{P,B}(T_f - 25) = N_{A,added}(H_{f,A}^\circ - H_{f,B}^\circ) + N_{B,i}C_{P,B}(25) + N_{A,added}C_{P,A}(65)$$
\end{enumerate}
\end{stepbox}

\newpage

\begin{stepbox}
\begin{enumerate}[label=\textbf{Step \arabic*:}, wide=0pt, leftmargin=*, itemsep=2pt, resume]
    \setcounter{enumi}{5}

    \item \textbf{Substitute Heat of Reaction}
    Since $H_{f,A}^\circ - H_{f,B}^\circ = -\Delta H_{rxn} = -(-20000) = 20000$ J/mol:
    $$N_{B,f}C_{P,B}(T_f - 25) = N_{A,added}(20000) + N_{B,i}C_{P,B}(25) + N_{A,added}C_{P,A}(65)$$

    \item \textbf{Substitute Numerical Values}
    $$(190)(70)(T_f - 25) = (90)(20000) + (100)(70)(25) + (90)(60)(65)$$
    $$13300(T_f - 25) = 1,800,000 + 175,000 + 351,000$$
    $$13300(T_f - 25) = 2,326,000$$
    
    \item \textbf{Solve for Final Temperature}
    $$T_f - 25 = \frac{2,326,000}{13300} = 174.9$$
    $$T_f = 174.9 + 25 = \textbf{199.9}^{\circ}\text{C}$$

    \item \textbf{Final Answer and Analysis}
    The reactor temperature increases from $50^{\circ}\text{C}$ to approximately $\textbf{200}^{\circ}\text{C}$ due to:
    \begin{itemize}[itemsep=0pt]
        \item Exothermic reaction heat release ($-20$ kJ/mol)
        \item Addition of hot reactant ($90^{\circ}\text{C}$ vs. initial $50^{\circ}\text{C}$)
        \item Accumulation of thermal energy in semi-batch operation
    \end{itemize}
\end{enumerate}
\end{stepbox}

\begin{conceptbox}
\textbf{Key Open System Applications:}
\begin{itemize}[itemsep=0pt]
    \item Tank filling/emptying operations use integrated energy balances
    \item Semi-batch reactors require unsteady-state analysis with reaction terms
    \item Steady-state equipment (turbines, compressors) use simplified balances
    \item Enthalpy appears in flow terms due to flow work ($PV$) contribution
    \item Sign conventions remain critical for heat and work terms
    \item Steam tables are essential for property evaluation in phase-change processes
\end{itemize}
\end{conceptbox}

\newpage

\section*{Enthalpy}

\begin{conceptbox}
\textbf{Enthalpy Definition:} Enthalpy is a convenient composite thermodynamic property that arises frequently in energy balance calculations. It represents the total energy of a system including both internal energy and the energy required to make room for the system in its environment.

\textbf{Physical Significance:} The $PV$ term in enthalpy represents "flow work" - the energy required to displace surroundings and make room for the system. This makes enthalpy the natural energy property for flowing streams.
\end{conceptbox}

\begin{keybox}
\textbf{Key Variables \& Definitions:}
\begin{itemize}[itemsep=0pt]
    \item $H$ = Enthalpy, total energy of thermodynamic system \text{(J/mol, kJ/kg)}
    \item $U$ = Internal energy of system \text{(J/mol, kJ/kg)}
    \item $P$ = Absolute pressure \text{(Pa, N/m}$^2$\text{, atm)}
    \item $V$ = Molar or specific volume \text{(m}$^3$\text{/mol, m}$^3$\text{/kg)}
    \item $C_P$ = Heat capacity at constant pressure \text{(J/(mol·K))}
    \item $C_V$ = Heat capacity at constant volume \text{(J/(mol·K))}
    \item $R$ = Ideal gas constant = 8.314 \text{ J/(mol·K)}
\end{itemize}
\end{keybox}

\begin{formulabox}
\textbf{Fundamental Enthalpy Definition:}
$$H = U + PV \quad \text{(Equation 1)}$$

\textbf{Enthalpy Change (State Function):}
$$\Delta H = H_{final} - H_{initial} \quad \text{(Equation 2)}$$

\textbf{Unit Verification for PV term:}
$$\text{Units of } PV = \left(\frac{\text{N}}{\text{m}^2}\right) \cdot \left(\frac{\text{m}^3}{\text{mol}}\right) = \frac{\text{N·m}}{\text{mol}} = \frac{\text{J}}{\text{mol}}$$
\end{formulabox}

\begin{conceptbox}
\textbf{Enthalpy as State Function:} Enthalpy is composed entirely of state functions (U, P, V), making it a state function. Therefore, $\Delta H$ depends only on initial and final states, not on the process path taken between them.
\end{conceptbox}

\subsection*{Enthalpy Properties and Derivations}

\begin{examplebox}{Derivation: Enthalpy and Heat Transfer at Constant Pressure}
Show that for a closed, reversible system at constant pressure, the change in enthalpy equals the heat transferred: $\Delta H = Q_P$.
\end{examplebox}

\begin{stepbox}
\begin{enumerate}[label=\textbf{Step \arabic*:}, wide=0pt, leftmargin=*, itemsep=2pt]
    \item \textbf{Start with First Law (Differential Form)}
    For a reversible process:
    $$dU = dQ_{rev} - P dV$$
    
    \item \textbf{Rearrange for Heat Transfer}
    $$dQ_{rev} = dU + P dV$$
    
    \item \textbf{Apply Constant Pressure Condition}
    At constant pressure, $P dV = d(PV)$ since P is constant:
    $$dQ_{rev} = dU + d(PV) = d(U + PV)$$
    
    \item \textbf{Substitute Enthalpy Definition}
    Since $H = U + PV$:
    $$dQ_{rev} = dH$$
    
    \item \textbf{Final Result}
    Integrating over the entire process:
    $$Q_P = \Delta H$$
    
    This fundamental relationship makes enthalpy extremely useful for analyzing constant pressure processes.
\end{enumerate}
\end{stepbox}

\newpage

\begin{examplebox}{Derivation: Heat Capacity Relationship for Ideal Gas}
Derive the relationship $C_P = C_V + R$ for an ideal gas.
\end{examplebox}

\begin{stepbox}
\begin{enumerate}[label=\textbf{Step \arabic*:}, wide=0pt, leftmargin=*, itemsep=2pt]
    \item \textbf{Start with Differential Enthalpy Definition}
    $$dH = dU + d(PV)$$
    
    \item \textbf{Apply Ideal Gas Relations}
    For one mole of ideal gas:
    \begin{itemize}[itemsep=0pt]
        \item $dU = C_V dT$ (internal energy depends only on temperature)
        \item $PV = RT$ (ideal gas law)
        \item $d(RT) = R dT$ (R is constant)
    \end{itemize}
    
    \item \textbf{Substitute into Enthalpy Expression}
    $$dH = C_V dT + R dT = (C_V + R) dT$$
    
    \item \textbf{Compare with Heat Capacity Definition}
    By definition: $dH = C_P dT$ at constant pressure
    
    \item \textbf{Equate Expressions}
    $$C_P dT = (C_V + R) dT$$
    
    Therefore: $$C_P = C_V + R$$
    
    This fundamental relationship applies to all ideal gases.
\end{enumerate}
\end{stepbox}

\begin{formulabox}
\textbf{Key Enthalpy Relationships for Ideal Gas:}
$$C_P = C_V + R \quad \text{(Equation 3)}$$
$$\Delta H = C_P \Delta T \quad \text{(Equation 4)}$$
$$\Delta U = C_V \Delta T \quad \text{(Equation 5)}$$

\textbf{Constant Pressure Heat Transfer:}
$$Q_P = \Delta H \quad \text{(Equation 6)}$$
\end{formulabox}

\begin{conceptbox}
\textbf{Enthalpy in Open Systems:} Enthalpy is particularly important for open (flow) systems because the total energy carried by flowing fluid includes:
\begin{itemize}[itemsep=0pt]
    \item Internal energy ($U$) - thermal energy content
    \item Flow work ($PV$) - energy to push fluid into and out of system
    \item Combined: $H = U + PV$ - exactly the definition of enthalpy
\end{itemize}
This makes enthalpy the natural energy term for streams crossing system boundaries.
\end{conceptbox}

\newpage

\section*{Entropy and the Second Law}

\begin{conceptbox}
\textbf{Second Law of Thermodynamics:} The Second Law introduces entropy ($S$) and establishes the direction for spontaneous processes. The fundamental principle is that for any real (irreversible) process, the total entropy of the universe (system + surroundings) must increase.

\textbf{Process Classification:}
\begin{itemize}[itemsep=0pt]
    \item \textbf{Reversible process:} $\Delta S_{total} = 0$ (theoretical limit)
    \item \textbf{Irreversible process:} $\Delta S_{total}$ $>$ $0$ (all real processes)
    \item \textbf{Impossible process:} $\Delta S_{total}$ $<$ $0$ (violates Second Law)
\end{itemize}
\end{conceptbox}

\begin{keybox}
\textbf{Entropy Variables \& Definitions:}
\begin{itemize}[itemsep=0pt]
    \item $S$ = Entropy, measure of molecular disorder \text{(J/(mol·K))}
    \item $\Delta S_{total}$ = Total entropy change for process \text{(J/(mol·K))}
    \item $\Delta S_{system}$ = Entropy change of system being studied \text{(J/(mol·K))}
    \item $\Delta S_{surroundings}$ = Entropy change of surroundings \text{(J/(mol·K))}
    \item $dQ_{rev}$ = Differential heat transfer in reversible process \text{(J)}
    \item $T$ = Absolute temperature \text{(K)}
    \item $y_i$ = Mole fraction of component $i$ \text{(dimensionless)}
\end{itemize}
\end{keybox}

\begin{formulabox}
\textbf{Second Law Statement:}
$$\Delta S_{total} = \Delta S_{system} + \Delta S_{surroundings} \geq 0 \quad \text{(Equation 7)}$$

\textbf{Fundamental Entropy Definition:}
$$\Delta S = \int \frac{dQ_{rev}}{T} \quad \text{(Equation 8)}$$

\textbf{Phase Change Entropy:}
$$\Delta S = \frac{\Delta H}{T} \quad \text{(Equation 9)}$$
\end{formulabox}

\newpage

\begin{formulabox}
\textbf{Entropy Change for Ideal Gas:}
$$\Delta S = C_P \ln\left(\frac{T_2}{T_1}\right) - R \ln\left(\frac{P_2}{P_1}\right) \quad \text{(Equation 10)}$$
$$\Delta S = C_V \ln\left(\frac{T_2}{T_1}\right) + R \ln\left(\frac{V_2}{V_1}\right) \quad \text{(Equation 11)}$$

\textbf{Entropy of Mixing (Ideal Gases):}
$$\Delta S = -R \sum_{i} (y_i \ln y_i) \quad \text{(Equation 12)}$$

\textbf{Incompressible Liquids/Solids:}
$$\Delta S = C_P \ln\left(\frac{T_2}{T_1}\right) \quad \text{(Equation 13)}$$
\end{formulabox}

\subsection*{Entropy Change Derivations}

\begin{examplebox}{Derivation: Entropy Change for Ideal Gas (T and V form)}
Derive the entropy change equation in terms of temperature and volume for an ideal gas.
\end{examplebox}

\begin{stepbox}
\begin{enumerate}[label=\textbf{Step \arabic*:}, wide=0pt, leftmargin=*, itemsep=2pt]
    \item \textbf{Start with Fundamental Definition}
    $$dS = \frac{dQ_{rev}}{T}$$
    
    \item \textbf{Apply First Law for Reversible Process}
    $$dQ_{rev} = dU + P dV$$
    
    \item \textbf{Substitute Ideal Gas Relations}
    For one mole: $dU = C_V dT$
    $$dQ_{rev} = C_V dT + P dV$$

    \item \textbf{Express in Terms of Entropy}
    $$dS = \frac{C_V dT + P dV}{T} = \frac{C_V}{T} dT + \frac{P}{T} dV$$

\end{enumerate}
\end{stepbox}

\newpage

\begin{stepbox}
\begin{enumerate}[label=\textbf{Step \arabic*:}, wide=0pt, leftmargin=*, itemsep=2pt, resume]
    \setcounter{enumi}{4}
    
    \item \textbf{Use Ideal Gas Law}
    From $PV = RT$: $\frac{P}{T} = \frac{R}{V}$
    $$dS = \frac{C_V}{T} dT + \frac{R}{V} dV$$
    
    \item \textbf{Integrate from State 1 to State 2}
    Assuming constant heat capacities:
    $$\int_{S_1}^{S_2} dS = \int_{T_1}^{T_2} \frac{C_V}{T} dT + \int_{V_1}^{V_2} \frac{R}{V} dV$$
    
    \item \textbf{Evaluate Integrals}
    $$\Delta S = C_V \ln\left(\frac{T_2}{T_1}\right) + R \ln\left(\frac{V_2}{V_1}\right)$$
    
    This is the entropy change equation in terms of temperature and volume.
\end{enumerate}
\end{stepbox}

\begin{examplebox}{Derivation: Entropy Change for Ideal Gas (T and P form)}
Convert the (T,V) entropy equation to (T,P) form for an ideal gas.
\end{examplebox}

\begin{stepbox}
\begin{enumerate}[label=\textbf{Step \arabic*:}, wide=0pt, leftmargin=*, itemsep=2pt]
    \item \textbf{Start with (T,V) Form}
    $$\Delta S = C_V \ln\left(\frac{T_2}{T_1}\right) + R \ln\left(\frac{V_2}{V_1}\right)$$
    
    \item \textbf{Express Volume Ratio Using Ideal Gas Law}
    From $PV = RT$ at two states: $\frac{P_1V_1}{T_1} = \frac{P_2V_2}{T_2}$
    
    Therefore: $$\frac{V_2}{V_1} = \frac{T_2 P_1}{T_1 P_2}$$
    
    \item \textbf{Substitute Volume Ratio}
    $$\Delta S = C_V \ln\left(\frac{T_2}{T_1}\right) + R \ln\left(\frac{T_2 P_1}{T_1 P_2}\right)$$
    
    \item \textbf{Expand Logarithm}
    Using $\ln(ab) = \ln(a) + \ln(b)$:
    $$\Delta S = C_V \ln\left(\frac{T_2}{T_1}\right) + R \ln\left(\frac{T_2}{T_1}\right) + R \ln\left(\frac{P_1}{P_2}\right)$$
\end{enumerate}
\end{stepbox}

\newpage

\begin{stepbox}
\begin{enumerate}[label=\textbf{Step \arabic*:}, wide=0pt, leftmargin=*, itemsep=2pt, resume]
    \setcounter{enumi}{4}
    
    \item \textbf{Combine Temperature Terms}
    $$(C_V + R) \ln\left(\frac{T_2}{T_1}\right) + R \ln\left(\frac{P_1}{P_2}\right)$$
    
    \item \textbf{Apply Heat Capacity Relationship}
    Since $C_P = C_V + R$:
    $$\Delta S = C_P \ln\left(\frac{T_2}{T_1}\right) + R \ln\left(\frac{P_1}{P_2}\right)$$
    
    \item \textbf{Final Form}
    $$\Delta S = C_P \ln\left(\frac{T_2}{T_1}\right) - R \ln\left(\frac{P_2}{P_1}\right)$$
    
    This is the entropy change equation in terms of temperature and pressure.
\end{enumerate}
\end{stepbox}

\begin{conceptbox}
\textbf{Applications of Entropy in Chemical Engineering:}
\begin{itemize}[itemsep=0pt]
    \item Process feasibility analysis (Second Law compliance)
    \item Efficiency calculations for heat engines and refrigeration cycles
    \item Mixing processes and separation work requirements
    \item Phase equilibrium calculations
    \item Optimization of thermodynamic cycles
    \item Heat exchanger network synthesis
\end{itemize}

\textbf{Key Insight:} Both First Law (energy conservation) and Second Law (entropy increase) must be satisfied for any real process to be possible.
\end{conceptbox}

\newpage

\section*{Heat Capacities}

\begin{conceptbox}
\textbf{Heat Capacity Definition:} Heat capacity is a measure of the amount of heat energy required to raise the temperature of a substance by a specific amount. It is a fundamental property in thermodynamics and heat transfer calculations.

\textbf{Two Primary Types:}
\begin{itemize}[itemsep=0pt]
    \item \textbf{Constant Volume ($C_V$):} Heat capacity when volume is held constant
    \item \textbf{Constant Pressure ($C_P$):} Heat capacity when pressure is held constant
\end{itemize}

\textbf{Key Insight:} Even when volume or pressure changes during a process, these heat capacities can still be used to calculate $\Delta U$ and $\Delta H$ respectively.
\end{conceptbox}

\begin{keybox}
\textbf{Key Variables \& Definitions:}
\begin{itemize}[itemsep=0pt]
    \item $C_V$ = Heat capacity at constant volume \text{(J/(mol·K), J/(kg·K))}
    \item $C_P$ = Heat capacity at constant pressure \text{(J/(mol·K), J/(kg·K))}
    \item $U$ = Specific internal energy \text{(J/mol, J/kg)}
    \item $H$ = Specific enthalpy \text{(J/mol, J/kg)}
    \item $T$ = Absolute temperature \text{(K)}
    \item $V$ = Specific volume \text{(m}$^3$\text{/mol, m}$^3$\text{/kg)}
    \item $P$ = Pressure \text{(Pa, atm)}
    \item $R$ = Ideal gas constant = 8.314 \text{ J/(mol·K)}
\end{itemize}
\end{keybox}

\begin{formulabox}
\textbf{Fundamental Definitions:}

\textbf{Constant Volume Heat Capacity:}
$$C_V = \left(\frac{\partial U}{\partial T}\right)_V \quad \text{(Equation 1)}$$

\textbf{Constant Pressure Heat Capacity:}
$$C_P = \left(\frac{\partial H}{\partial T}\right)_P \quad \text{(Equation 2)}$$

\textbf{Internal Energy Change:}
$$\Delta U = \int_{T_1}^{T_2} C_V \,dT \quad \text{(Equation 3)}$$

\textbf{Enthalpy Change:}
$$\Delta H = \int_{T_1}^{T_2} C_P \,dT \quad \text{(Equation 4)}$$
\end{formulabox}

\begin{formulabox}
\textbf{Heat Capacity Relationships:}

\textbf{For Ideal Gas:}
$$C_P = C_V + R \quad \text{(Equation 5)}$$

\textbf{For Liquids and Solids (Nearly Incompressible):}
$$C_P \approx C_V \quad \text{(Equation 6)}$$

\textbf{Physical Explanation:} For incompressible substances, expansion work is negligible, so the energy required to heat at constant pressure is nearly the same as at constant volume.
\end{formulabox}

\newpage

\subsection*{Heat Transfer Applications}

\begin{examplebox}{Convective Heat Transfer from Solar Cell}
A solar cell (0.5 m long, 0.2 m wide) has a protective glass layer. Air at $25^\circ\text{C}$ blows over the solar cell at 5 m/s. Calculate the average convective heat transfer coefficient, $\bar{h}$, in W/(m$^2\cdot$K).
\end{examplebox}

\begin{stepbox}
\begin{enumerate}[label=\textbf{Step \arabic*:}, wide=0pt, leftmargin=*, itemsep=2pt]  
    \item \textbf{Estimate Film Temperature and Properties}
    Assume surface temperature leads to film temperature: $T_f = 320$ K
    
    Look up air properties at 320 K:
    \begin{itemize}[itemsep=0pt]
        \item Kinematic viscosity: $\nu = 1.7 \times 10^{-6}$ m$^2$/s (note: this appears to be an error in reference material)
        \item Thermal conductivity: $k_f = 0.0269$ W/(m$\cdot$K)
        \item Prandtl number: $Pr = 0.706$
    \end{itemize}
    
    \item \textbf{Calculate Reynolds Number}
    $$Re_L = \frac{U_{\infty} L}{\nu} = \frac{(5 \text{ m/s})(0.5 \text{ m})}{1.7 \times 10^{-6} \text{ m}^2\text{/s}} = 1.47 \times 10^6$$
    
    Since $Re_L$ $>$ $5 \times 10^5$, flow would typically be classified as turbulent.

    \item \textbf{Apply Nusselt Number Correlation}
    Following reference material (noting inconsistency), using laminar correlation:
    $$Nu_L = 0.664 Re_L^{1/2} Pr^{1/3}$$
    $$Nu_L = 0.664 (1.47 \times 10^6)^{1/2} (0.706)^{1/3} \approx 717$$
    
    \item \textbf{Calculate Heat Transfer Coefficient}
    \[
\bar{h} = \frac{Nu_L k_f}{L} = \frac{(717)(0.0269\, \text{W/(m$\cdot$K)})}{0.5\, \text{m}} = \textbf{38.6}\, \text{W/(m}^2\text{·K)}
\]

    \item \textbf{Note on Accuracy}
    
    Due to inconsistencies in reference material (incorrect kinematic viscosity and flow regime classification), a more realistic value would be closer to 12 W/(m$^2\cdot$K) using correct fluid properties and laminar flow correlation.
\end{enumerate}
\end{stepbox}


\newpage

\begin{examplebox}{Convective Heat Transfer from Motorcycle Fin}
A motorcycle cooling fin is 5 cm long and dissipates heat at $q' = 1200$ W per meter of width. The motorcycle travels at 40 km/h through ambient air at $27^\circ\text{C}$. Modeling the fin as a flat plate with uniform temperature, calculate its surface temperature $T_s$.
\end{examplebox}

\begin{stepbox}
\begin{enumerate}[label=\textbf{Step \arabic*:}, wide=0pt, leftmargin=*, itemsep=2pt]
    \item \textbf{Problem Analysis}    
    Heat transfer equation: $q' = 2\bar{h}L(T_s - T_{\infty})$. Factor of 2 accounts for heat transfer from both sides of fin.
    
    \item \textbf{Initial Guess and Film Temperature}
    Guess: $T_s = 250^\circ\text{C}$
    
    Film temperature: $T_f = \frac{T_s + T_{\infty}}{2} = \frac{250 + 27}{2} = 138.5^\circ\text{C} = 412$ K
    
    \item \textbf{Air Properties at 412 K}
    Look up properties:
    \begin{itemize}[itemsep=0pt]
        \item Thermal conductivity: $k_f = 0.0346$ W/(m$\cdot$K)
        \item Kinematic viscosity: $\nu = 27.85 \times 10^{-6}$ m$^2$/s
        \item Prandtl number: $Pr = 0.69$
    \end{itemize}

    \item \textbf{Calculate Reynolds Number}
    Convert velocity: $U_{\infty} = 40$ km/h $= \frac{40 \times 1000}{3600} = 11.11$ m/s
    
    Length: $L = 0.05$ m
    
    $$Re_L = \frac{U_{\infty} L}{\nu} = \frac{(11.11)(0.05)}{27.85 \times 10^{-6}} = 19,946$$
    
    Since $Re_L$ $<$ $5 \times 10^5$, flow is laminar.
    
    \item \textbf{Calculate Nusselt Number}
    For laminar flow over flat plate:
    $$Nu_L = 0.664 Re_L^{1/2} Pr^{1/3} = 0.664 (19,946)^{1/2} (0.69)^{1/3} = 82.9$$
    
    \item \textbf{Calculate Heat Transfer Coefficient}
    $$\bar{h} = \frac{Nu_L k_f}{L} = \frac{(82.9)(0.0346)}{0.05} = 57.4 \text{ W/(m}^2\text{·K)}$$
    
    \item \textbf{Solve for Surface Temperature}
    Rearranging heat transfer equation:
    $$T_s = T_{\infty} + \frac{q'}{2\bar{h}L} = 27 + \frac{1200}{2(57.4)(0.05)}$$
    
    \[
    T_s = 27 + \frac{1200}{5.74} = 27 + 209 = \mathbf{236^\circ\text{C}}
    \]
\end{enumerate}
\end{stepbox}


\begin{conceptbox}
\textbf{Key Heat Transfer Problem-Solving Strategy:}
\begin{itemize}[itemsep=0pt]
    \item For unknown surface temperatures, use iterative approach
    \item Film temperature determines fluid properties: $T_f = (T_s + T_{\infty})/2$
    \item Reynolds number determines flow regime: laminar if $Re_L$ $<$ $5 \times 10^5$
    \item Select appropriate Nusselt correlation based on geometry and flow regime
    \item Check convergence between guessed and calculated temperatures
    \item For fins and extended surfaces, account for heat transfer from all surfaces
\end{itemize}
\end{conceptbox}

\begin{formulabox}
\textbf{Key Heat Transfer Relationships:}

\textbf{Reynolds Number:}
$$Re_L = \frac{U_{\infty} L}{\nu}$$

\textbf{Nusselt Number (Laminar Flat Plate):}
$$Nu_L = 0.664 Re_L^{1/2} Pr^{1/3}$$

\textbf{Heat Transfer Coefficient:}
$$\bar{h} = \frac{Nu_L k_f}{L}$$

\textbf{Heat Transfer Rate:}
$$q' = \bar{h} A (T_s - T_{\infty})$$
\end{formulabox}

\newpage

nd{document}
