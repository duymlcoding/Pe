\documentclass[12pt]{article}
\usepackage[paperwidth=8.5in, paperheight=11in, margin=1.0in, headheight=15pt]{geometry}
\usepackage{amsmath,amssymb,amsthm}
\usepackage[most]{tcolorbox}
\usepackage{enumitem}
\usepackage{xcolor}
\usepackage{hyperref}
\usepackage{fancyhdr}
\usepackage{titlesec}
\usepackage{graphicx}
% Define custom colors for chemical engineering theme
\definecolor{conceptcolor}{RGB}{52, 73, 94}      % Dark blue-gray
\definecolor{formulacolor}{RGB}{231, 76, 60}     % Red for formulas
\definecolor{examplecolor}{RGB}{39, 174, 96}     % Green for examples
\definecolor{stepcolor}{RGB}{142, 68, 173}       % Purple for solution steps
\definecolor{keycolor}{RGB}{243, 156, 18}        % Orange for key points
% Configure fancy headers
\pagestyle{fancy}
\fancyhf{}
\fancyhead[L]{PE Study Guide}
\fancyhead[R]{Process Fundamentals}
\fancyfoot[C]{\thepage}
\renewcommand{\baselinestretch}{1.1}
\setlength{\parindent}{0.25in}
\setlength{\parskip}{3pt}
% Configure section formatting
\titleformat{\section}
  {\normalfont\LARGE\bfseries\color{conceptcolor}}
  {\thesection}{1em}{}
\titleformat{\subsection}
  {\normalfont\Large\bfseries\color{conceptcolor}}
  {\thesubsection}{1em}{}
% Define custom environments
\newtcolorbox{conceptbox}[1][]{
  enhanced,
  colback=conceptcolor!10,
  colframe=conceptcolor,
  arc=3mm,
  title=Key Concept,
  fonttitle=\bfseries\sffamily\normalsize,
  fontupper=\small,
  #1
}
\newtcolorbox{formulabox}[1][]{
  enhanced,
  colback=formulacolor!10,
  colframe=formulacolor,
  arc=2mm,
  title=Important Formula,
  fonttitle=\bfseries\sffamily\normalsize,
  fontupper=\small,
  #1
}
\newtcolorbox{examplebox}[2][]{
  enhanced,
  colback=examplecolor!10,
  colframe=examplecolor,
  arc=3mm,
  title=Example Problem: #2,
  fonttitle=\bfseries\sffamily\normalsize,
  fontupper=\small,
  #1
}
\newtcolorbox{stepbox}[1][]{
  enhanced,
  colback=stepcolor!10,
  colframe=stepcolor,
  arc=2mm,
  title=Solution Steps,
  fonttitle=\bfseries\sffamily\normalsize,
  fontupper=\small,
  #1
}
\newtcolorbox{keybox}[1][]{
  enhanced,
  colback=keycolor!10,
  colframe=keycolor,
  arc=2mm,
  title=Key Variables \& Definitions,
  fonttitle=\bfseries\sffamily\normalsize,
  fontupper=\small,
  #1
}

\section*{Pipe Flow: Determining Pumping Power}
A common and practical problem in fluid mechanics is determining the power required to pump a fluid through a piping system. This analysis requires a comprehensive application of the energy balance, accounting for changes in pressure, velocity, and elevation.

\subsection*{Example: Pumping Water for a Ski Resort}

\begin{examplebox}{Pumping Water for a Ski Resort}
\textbf{Question:} A ski resort needs to pump water at 10$^\circ$C from a reservoir at an elevation of 6500 ft to a snow-making machine at 7300 ft. The water flows through 1000 ft of 2-inch diameter commercial steel pipe at a rate of 0.25 ft$^3$/s. The pressure required at the inlet of the snow machine is 20 psi (gauge). The piping system includes a sharp-edged entrance, a fully open gate valve, and two standard 90$^\circ$ elbows. Determine the horsepower that must be delivered to the water by the pump.
\end{examplebox}

\begin{stepbox}[title=Solution (Part 1 of 6): Setting up the Energy Balance]
\begin{enumerate}[label=\textbf{Step \arabic*:}, wide=0pt, leftmargin=*, itemsep=2pt]
    \item \textbf{Strategy: The Extended Bernoulli Equation}
    \begin{conceptbox}[title=Why the Simple Bernoulli Equation Is Not Enough]
    The simple Bernoulli equation ($P + \frac{1}{2}\rho\nu^2 + \rho gz = \text{constant}$) is a statement of energy conservation for an \textit{ideal} fluid, where there is no friction and no external work being done. Our system is real, not ideal, because we are adding energy with a \textbf{pump} and losing energy to \textbf{friction} in the pipe and fittings. Therefore, we must use the \textbf{Extended Bernoulli Equation}, also known as the steady-state Mechanical Energy Balance.
    \end{conceptbox}
    \begin{formulabox}[title=The Extended Bernoulli Equation (in Head Form)]
    We write the energy balance between point 1 (the reservoir surface) and point 2 (the snow machine inlet):
    $$ \frac{P_1}{\gamma} + \frac{\nu_1^2}{2g} + z_1 + h_p = \frac{P_2}{\gamma} + \frac{\nu_2^2}{2g} + z_2 + h_L $$
    \end{formulabox}
    The equation contains several key terms: $\frac{P}{\gamma}$ represents the pressure head (energy stored due to pressure), $\frac{\nu^2}{2g}$ represents the velocity head (kinetic energy), $z$ represents the elevation head (potential energy due to height), $h_p$ represents the pump head (energy added by the pump), and $h_L$ represents the total head loss.
\end{enumerate}
\end{stepbox}
\newpage
\begin{stepbox}[title=Solution (Part 2 of 6): Simplifying the Energy Balance and Flow Analysis]
\begin{enumerate}[label=\textbf{Step \arabic*:}, wide=0pt, leftmargin=*, itemsep=2pt, start=2]
    \item \textbf{Simplify the Energy Balance:}
    Let's analyze the conditions at points 1 and 2 to eliminate terms and create a workable equation.
    
    At point 1 (reservoir surface): The reservoir surface is open to atmosphere, so its gauge pressure is $P_1 = 0$ psi. The reservoir is large compared to the pipe, so the velocity of the surface dropping is negligible: $\nu_1 \approx 0$ ft/s. We set the reservoir elevation as our reference datum, so $z_1 = 0$ ft.
    
    At point 2 (snow machine inlet): The elevation of the snow machine is $z_2 = 7300 \, \text{ft} - 6500 \, \text{ft} = 800$ ft above the reservoir. The pressure at point 2 is given as $P_2 = 20$ psi (gauge). The velocity at point 2, $\nu_2$, is the velocity inside the pipe, which we must calculate.
    
    Substituting these simplifications into the main equation:
    $$ \frac{0}{\gamma} + \frac{0^2}{2g} + 0 + h_p = \frac{P_2}{\gamma} + \frac{\nu_2^2}{2g} + z_2 + h_L $$
    $$ h_p = \frac{P_2}{\gamma} + \frac{\nu_2^2}{2g} + z_2 + h_L $$
    
    This is our working equation. To solve for $h_p$, we need to calculate each term on the right side.
    
    \item \textbf{Calculate Fluid Velocity and Flow Characteristics:}
    Before we can calculate head losses, we need to determine the fluid velocity and flow regime.
    
    First, convert the pipe diameter to consistent units: $D = 2 \, \text{in} \times \frac{1 \, \text{ft}}{12 \, \text{in}} = 0.1667$ ft.
    
    Calculate the pipe cross-sectional area: $A = \frac{\pi D^2}{4} = \frac{\pi (0.1667 \, \text{ft})^2}{4} = 0.02182$ ft$^2$.
    
    Determine the fluid velocity: $\nu_2 = \frac{\text{Flow Rate (Q)}}{A} = \frac{0.25 \, \text{ft}^3/\text{s}}{0.02182 \, \text{ft}^2} = 11.46$ ft/s.
    
    This velocity will be used in multiple calculations throughout our solution.
    
    Next, we need to determine if the flow is laminar or turbulent by calculating the Reynolds number. For water at 10°C (50°F), the kinematic viscosity is $\nu \approx 1.41 \times 10^{-5}$ ft$^2$/s.
    
    \begin{formulabox}[title=Reynolds Number Calculation]
    $$ Re = \frac{\nu_2 D}{\nu} = \frac{(11.46 \, \text{ft/s})(0.1667 \, \text{ft})}{1.41 \times 10^{-5} \, \text{ft}^2/\text{s}} = 1.35 \times 10^5 $$
    \end{formulabox}
    
    Since $Re = 135,000 > 4000$, the flow is definitively \textbf{turbulent}. This is crucial information for determining the friction factor in the next step.
\end{enumerate}
\end{stepbox}
\newpage
\begin{stepbox}[title=Solution (Part 3 of 6): Calculating Head Losses]
\begin{enumerate}[label=\textbf{Step \arabic*:}, wide=0pt, leftmargin=*, itemsep=2pt, start=4]
    \item \textbf{Calculate Major Head Loss (Pipe Friction):}
    \begin{conceptbox}[title=The Darcy-Weisbach Equation and the Moody Chart]
    Major head loss represents the energy lost due to friction as fluid flows along the length of a pipe. For turbulent flow in rough pipes, this loss depends on both the Reynolds number and the relative roughness of the pipe surface. We calculate it using the Darcy-Weisbach equation, where the key parameter is the \textbf{friction factor, $f$}, which we determine from the \textbf{Moody Chart}.
    \end{conceptbox}
    
    \begin{formulabox}[title=Darcy-Weisbach Equation]
    $$ h_{L,major} = f \frac{L}{D} \frac{\nu_2^2}{2g} $$
    \end{formulabox}
    
    To find the friction factor $f$, we need the relative roughness of the pipe. For commercial steel pipe, the average roughness height is $\epsilon \approx 0.00015$ ft. The relative roughness is:
    $$ \frac{\epsilon}{D} = \frac{0.00015 \, \text{ft}}{0.1667 \, \text{ft}} \approx 0.0009 $$
    
    Using the Moody Chart: Locate $Re = 1.35 \times 10^5$ on the horizontal axis, move vertically up until you intersect the curve corresponding to $\epsilon/D = 0.0009$, then move horizontally left to read the friction factor from the vertical axis. This gives $f \approx 0.021$.
    
    Now calculate the major head loss:
    $$ h_{L,major} = (0.021) \frac{1000 \, \text{ft}}{0.1667 \, \text{ft}} \frac{(11.46 \, \text{ft/s})^2}{2 \cdot (32.2 \, \text{ft/s}^2)} $$
    $$ h_{L,major} = (0.021) \cdot (6000) \cdot (2.035 \, \text{ft}) = 126 \cdot 2.035 = \textbf{256.4 ft} $$
    
\end{enumerate}
\end{stepbox}

\begin{stepbox}[title=Solution (Part 4 of 6): Calculating Head Losses]
\begin{enumerate}[label=\textbf{Step \arabic*:}, wide=0pt, leftmargin=*, itemsep=2pt, start = 5]

    \item \textbf{Calculate Minor Head Loss (Fittings and Components):}
    Minor losses occur due to flow disruption through fittings, valves, entrances, exits, and direction changes. Each component has a specific loss coefficient.
    
    \begin{formulabox}[title=Minor Loss Equation]
    $$ h_{L,minor} = \sum K_L \frac{\nu_2^2}{2g} $$
    \end{formulabox}
    
    The term $K_L$ is a dimensionless \textbf{loss coefficient} specific to each fitting, obtained from standard engineering tables:
    
    Sharp-edged entrance from reservoir to pipe: $K_L = 0.5$. This represents the energy lost as the fluid accelerates from essentially zero velocity in the large reservoir to the pipe velocity.
    
    Fully open gate valve: $K_L = 0.15$. Gate valves, when fully open, create minimal flow disruption.
    
    Two standard 90° elbows: $K_L = 0.9$ each. Elbows cause flow separation and secondary flows, leading to energy loss. Total for two elbows: $2 \times 0.9 = 1.8$.
    
    Total loss coefficient: $\sum K_L = 0.5 + 0.15 + 1.8 = 2.45$.
    
    Calculate the minor head loss:
    $$ h_{L,minor} = (2.45) \frac{(11.46 \, \text{ft/s})^2}{2 \cdot (32.2 \, \text{ft/s}^2)} = (2.45) \cdot (2.035 \, \text{ft}) = \textbf{5.0 ft} $$
    
    \item \textbf{Calculate Total Head Loss:}
    The total head loss is the sum of major and minor losses:
    $$ h_L = h_{L,major} + h_{L,minor} = 256.4 + 5.0 = \textbf{261.4 ft} $$
    
    Notice that the major head loss (pipe friction) dominates in this long pipeline system, representing over 98\% of the total head loss.

\end{enumerate}
\end{stepbox}



\newpage
\begin{stepbox}[title=Solution (Part 5 of 6): Final Pump Head and Power Calculations]
\begin{enumerate}[label=\textbf{Step \arabic*:}, wide=0pt, leftmargin=*, itemsep=2pt, start=7]
    \item \textbf{Calculate the Total Pump Head Required:}
    Now we have all components needed to solve our simplified energy balance equation from Step 2:
    $$ h_p = \frac{P_2}{\gamma} + \frac{\nu_2^2}{2g} + z_2 + h_L $$
    
    Let's calculate each term systematically:
    
    \textbf{Pressure Head at Point 2:} We need to convert the gauge pressure at the snow machine inlet to head units. The specific weight of water is $\gamma = 62.4$ lb$_f$/ft$^3$.
    
    First, convert pressure to consistent units: $P_2 = 20 \, \frac{\text{lb}_f}{\text{in}^2} \times \frac{144 \, \text{in}^2}{1 \, \text{ft}^2} = 2880$ lb$_f$/ft$^2$.
    
    Then calculate pressure head: $\frac{P_2}{\gamma} = \frac{2880 \, \text{lb}_f/\text{ft}^2}{62.4 \, \text{lb}_f/\text{ft}^3} = 46.2$ ft.
    
    \textbf{Velocity Head at Point 2:} This represents the kinetic energy of the fluid at the delivery point.
    $$ \frac{\nu_2^2}{2g} = \frac{(11.46 \, \text{ft/s})^2}{2 \cdot (32.2 \, \text{ft/s}^2)} = \frac{131.3}{64.4} = 2.0 \text{ ft} $$
    
    \textbf{Elevation Head:} This is the gravitational potential energy difference between the reservoir and snow machine: $z_2 = 800$ ft.
    
    \textbf{Total Head Loss:} From our previous calculation: $h_L = 261.4$ ft.
    
    Substituting all values:
    $$ h_p = 46.2 + 2.0 + 800 + 261.4 = \textbf{1109.6 ft} $$
    
    The pump must provide approximately \textbf{1110 ft} of head to the water.
    
    \textbf{Physical Interpretation:} This means the pump must add enough energy to the water to: (1) overcome the 800 ft elevation gain, (2) provide the required 20 psi pressure at the snow machine, (3) overcome 261.4 ft of frictional losses, and (4) provide the kinetic energy for 11.46 ft/s flow velocity.

\end{enumerate}
\end{stepbox}

\begin{stepbox}[title=Solution (Part 6 of 6): Final Pump Head and Power Calculations]
\begin{enumerate}[label=\textbf{Step \arabic*:}, wide=0pt, leftmargin=*, itemsep=2pt, start = 8]
        
    \item \textbf{Calculate the Required Pump Power:}
    The hydraulic power that must be delivered to the fluid (often called "water horsepower") is calculated using the fundamental relationship between flow rate, head, and power.
    
    \begin{formulabox}[title=Fluid Power Calculation]
    $$ \text{Power}_{\text{fluid}} = \gamma Q h_p $$
    \end{formulabox}
    
    Where $\gamma$ is the specific weight of the fluid, $Q$ is the volumetric flow rate, and $h_p$ is the pump head.
    
    Substituting our values:
    $$ \text{Power}_{\text{fluid}} = (62.4 \, \text{lb}_f/\text{ft}^3) \cdot (0.25 \, \text{ft}^3/\text{s}) \cdot (1110 \, \text{ft}) $$
    $$ \text{Power}_{\text{fluid}} = 17,316 \, \frac{\text{ft}\cdot\text{lb}_f}{\text{s}} $$
    
    Convert to horsepower using the standard conversion factor (1 hp = 550 ft·lb$_f$/s):
    $$ \text{Horsepower} = 17,316 \, \frac{\text{ft}\cdot\text{lb}_f}{\text{s}} \times \frac{1 \, \text{hp}}{550 \, \text{ft}\cdot\text{lb}_f/\text{s}} = \textbf{31.5 hp} $$
    
    \begin{keybox}[title=Important Note on Pump Efficiency and Motor Power]
    This 31.5 hp represents the \textbf{hydraulic power} that must be delivered to the water. However, real pumps are not 100\% efficient due to mechanical friction, fluid leakage, and other losses. 
    
    If the pump operates at 80\% efficiency (typical for centrifugal pumps), the required shaft power would be: $31.5 / 0.80 = 39.4$ hp.
    
    Additionally, the electric motor driving the pump also has efficiency losses (typically 85-95\%). If the motor efficiency is 90\%, the required electrical power would be: $39.4 / 0.90 = 43.8$ hp.
    
    Therefore, while the theoretical minimum power to move the water is 31.5 hp, the actual electrical power consumption would likely be 40-45 hp in a real system.
    \end{keybox}
    
\end{enumerate}
\end{stepbox}


\newpage
\section*{Air Flow Through a Constriction}

\begin{conceptbox}[title=The Venturi Effect: High Velocity Means Low Pressure]
A classic application of the Bernoulli equation is analyzing flow through a constriction, such as a Venturi meter. This setup demonstrates a fundamental and often counter-intuitive principle:
\begin{enumerate}
    \item \textbf{Continuity:} For a fluid of constant density, the mass flow rate ($\dot{m} = \rho A \nu$) must be constant. Where the pipe area $A$ gets smaller (in the "throat"), the velocity $\nu$ \textit{must increase} to maintain the same flow rate.
    \item \textbf{Energy Conservation (Bernoulli):} The Bernoulli equation ($P + \frac{1}{2}\rho\nu^2 = \text{const}$ for a horizontal pipe) is a statement of energy conservation. As the kinetic energy ($\frac{1}{2}\rho\nu^2$) increases in the high-velocity throat, the pressure energy ($P$) \textit{must decrease} to keep the total energy constant.
\end{enumerate}
This phenomenon, where the pressure in the constricted section is lower than in the wider sections, is known as the \textbf{Venturi effect}.
\end{conceptbox}

\begin{examplebox}{Velocity in a Constriction (Venturi Effect)}
\textbf{Question:} Air with a density of 1.0 kg/m$^3$ flows steadily and incompressibly through a horizontal constriction. The pressure at the wide entrance (point 1) is 10 kPa greater than the pressure at the narrow throat (point 2). The cross-sectional area at the entrance is five times the area at the throat ($A_1 = 5A_2$). What is the air speed at the throat ($\nu_2$)?
\end{examplebox}

\begin{stepbox}
\begin{enumerate}[label=\textbf{Step \arabic*:}, wide=0pt, leftmargin=*, itemsep=2pt]
    \item \textbf{Strategy: Two Equations, Two Unknowns}
    We have two unknown velocities, $\nu_1$ and $\nu_2$. We can solve for them by setting up a system of two equations: the Bernoulli equation and the continuity equation.
    
    \item \textbf{Apply the Bernoulli Equation:}
    For a horizontal pipe ($z_1 = z_2$) with no friction or shaft work:
    $$ P_1 + \frac{1}{2}\rho\nu_1^2 = P_2 + \frac{1}{2}\rho\nu_2^2 $$
    Rearranging for the known pressure difference:
    $$ P_1 - P_2 = \frac{1}{2}\rho(\nu_2^2 - \nu_1^2) $$
    
\end{enumerate}
\end{stepbox}

\begin{stepbox}
\begin{enumerate}[label=\textbf{Step \arabic*:}, wide=0pt, leftmargin=*, itemsep=2pt, start = 3]

    \item \textbf{Apply the Continuity Equation:}
    For an incompressible fluid, the volumetric flow rate is constant: $A_1\nu_1 = A_2\nu_2$. We can express the upstream velocity $\nu_1$ in terms of the throat velocity $\nu_2$:
    $$ \nu_1 = \nu_2 \left(\frac{A_2}{A_1}\right) $$
    We are given that $A_1 = 5A_2$, so the ratio $\frac{A_2}{A_1} = \frac{1}{5} = 0.2$.
    $$ \nu_1 = 0.2 \nu_2 $$
    
    \item \textbf{Solve the System of Equations:}
    Substitute the expression for $\nu_1$ from continuity into the Bernoulli equation:
    $$ P_1 - P_2 = \frac{1}{2}\rho \left(\nu_2^2 - (0.2\nu_2)^2\right) = \frac{1}{2}\rho \left(\nu_2^2 - 0.04\nu_2^2\right) $$
    $$ P_1 - P_2 = \frac{1}{2}\rho (0.96\nu_2^2) $$
    Now, plug in the known values to solve for the unknown velocity, $\nu_2$.
    \begin{itemize}[itemsep=2pt]
        \item $P_1 - P_2 = 10 \, \text{kPa} = 10,000$ Pa.
        \item $\rho = 1.0$ kg/m$^3$.
    \end{itemize}
    $$ 10,000 \, \text{Pa} = \frac{1}{2} (1.0 \, \text{kg/m}^3) (0.96\nu_2^2) $$
    $$ 10,000 = 0.48 \cdot \nu_2^2 $$
    $$ \nu_2^2 = \frac{10,000}{0.48} = 20,833 \, \text{m}^2/\text{s}^2 $$
    $$ \nu_2 = \sqrt{20,833} = 144.3 \, \text{m/s} $$
    The speed of the air at the throat is approximately \textbf{140 m/s}.
\end{enumerate}
\end{stepbox}

\newpage
\section*{Manometry}
Manometry is the science of measuring pressure using liquid columns in tubes. A manometer is a simple yet powerful device that measures pressure differences by balancing the weight of a column of fluid against another pressure. Understanding how to analyze a manometer is a foundational skill in fluid mechanics and engineering, as it provides a very direct and intuitive way to think about pressure.

\subsection*{The Principle of Hydrostatic Head}
The operation of every manometer is based on the principle of \textbf{hydrostatic pressure}, which is the pressure exerted by a fluid at rest due to the force of gravity.

\begin{conceptbox}[title=Origin of Pressure in a Fluid Column]
Imagine a stationary column of water in a glass. The water has mass, and due to gravity, it has weight. This weight exerts a downward force on the bottom of the glass. The pressure at the bottom is simply this total downward force distributed over the bottom area. The deeper you go, the more water is above you, the greater the weight, and therefore the higher the pressure.
\end{conceptbox}

We can derive a simple formula for this pressure with a few logical steps:
\begin{enumerate}[itemsep=2pt]
    \item Pressure is defined as force per unit area: $P = \frac{F}{A}$.
    \item The force exerted by the fluid column is its weight: $F = \text{mass} \cdot g$.
    \item We can express mass in terms of density ($\rho$) and volume ($V$): $\text{mass} = \rho \cdot V$.
    \item The volume of a uniform column is its cross-sectional area times its height: $V = A \cdot h$.
    \item Combining these, we substitute steps 4, 3, and 2 into step 1: $P = \frac{(\rho \cdot A \cdot h) \cdot g}{A}$.
\end{enumerate}
The area ($A$) of the column cancels out, leaving the fundamental hydrostatic pressure equation:
\begin{formulabox}[title=Hydrostatic Pressure Equation]
$$ P_{\text{hydrostatic}} = \rho g h $$
\end{formulabox}
This important result tells us that the pressure exerted by a fluid column depends only on its density ($\rho$), height ($h$), and gravity ($g$)—not on the shape of the container or the total amount of fluid. The term $\rho g$ is also known as the \textbf{specific weight}, $\gamma$. So, the equation is often written as $P = \gamma h$.

\begin{conceptbox}[title=The Fundamental Rule of Manometry]
The analysis of any manometer, no matter how complex, relies on one simple, crucial rule:
\begin{center}
    \textit{The pressure at any two points at the same horizontal level in a single, continuous fluid at rest is the same.}
\end{center}
Why is this true? Imagine if the pressure at point A was higher than at point B on the same horizontal level. This pressure difference would create a net force, causing the fluid to flow from A to B. But a manometer contains fluid \textit{at rest} (in static equilibrium). Therefore, there can be no net forces, and the pressures must be equal. We use this rule by drawing a horizontal "datum" line at an interface between fluids and equating the pressures on the left and right sides of that line.
\end{conceptbox}

\subsection*{The Simple U-Tube Manometer}
The most common type of manometer is a U-shaped tube containing a dense liquid (often mercury or colored water), known as the manometer fluid.

\begin{conceptbox}[title=Measuring the Gauge Pressure of a Gas]
Consider a U-tube manometer with one end connected to a tank of pressurized gas and the other end open to the atmosphere. The gas pushes the manometer fluid down on the left side and up on the right side. To find the pressure of the gas, we apply our fundamental rule at the lowest fluid interface (we'll call this our datum line).
\begin{center}
    $P_{\text{left at datum}} = P_{\text{right at datum}}$
\end{center}
\begin{itemize}[itemsep=2pt]
    \item \textbf{Pressure on the Left Side:} The pressure at the datum line is simply the pressure of the gas, $P_{gas}$. We can ignore the weight of the small column of gas above the datum because the density of gas is thousands of times smaller than the density of the liquid, making its contribution negligible.
    \item \textbf{Pressure on the Right Side:} The pressure at the datum line is the sum of two pressures: the atmospheric pressure ($P_{atm}$) pushing down on the open end, plus the hydrostatic pressure from the column of manometer fluid of height $h$ that is above the datum. So, $P_{\text{right}} = P_{atm} + \rho_{\text{fluid}} g h$.
\end{itemize}
\end{conceptbox}
Equating the two sides gives the absolute pressure of the gas: $P_{gas} = P_{atm} + \rho g h$. More often, we are interested in the \textbf{gauge pressure}.
\begin{formulabox}[title=U-Tube Manometer Gauge Pressure]
$$ P_{gas, \text{gauge}} = P_{gas} - P_{atm} = \rho g h $$
\end{formulabox}

\newpage
\subsection*{Applying Manometry Equations from First Principles}

\begin{conceptbox}[title=The "Walking" Method: A Universal Approach]
Instead of memorizing a different formula for every type of manometer, it is much more powerful to use a single, universal method. We call this the "walking" method:
\begin{enumerate}
    \item Start at a point of known (or desired) pressure.
    \item "Walk" through the continuous fluid paths to the other end.
    \item Whenever you move \textbf{down} a vertical distance $h$, you \textbf{add} the hydrostatic pressure of that fluid column ($+\rho g h$). Whenever you move \textbf{up} a vertical distance $h$, you \textbf{subtract} the hydrostatic pressure ($-\rho g h$).
\end{enumerate}
\end{conceptbox}

\begin{keybox}[title=Piezometer Tube]
This is the simplest manometer: a single vertical tube attached to a container of liquid.
\begin{itemize}[itemsep=2pt]
    \item \textbf{Equation:} $P_{A,gauge} = \gamma_1 h_1 = \rho_1 g h_1$.
    \item \textbf{Explanation:} Start at the surface of the liquid in the tube, which is at atmospheric pressure ($P_{atm}$). Walk \textit{down} a height $h_1$ through the fluid to point A. This means we add pressure: $P_A = P_{atm} + \gamma_1 h_1$. The gauge pressure is therefore just $\gamma_1 h_1$. This only works for measuring liquid pressures that are greater than atmospheric.
\end{itemize}
\end{keybox}


\begin{keybox}[title=Inclined Manometer]
This is a variation of the differential manometer designed for measuring very small pressure differences with high precision.
\begin{conceptbox}[title=The Principle of Amplification]
The analysis is identical to the differential manometer, with one key difference. The vertical height of the manometer fluid column, $h_2$, is not measured directly. Instead, we measure the much larger displacement, $l_2$, along the inclined tube. From trigonometry, the vertical height is simply:
$$ h_2 = l_2 \sin(\theta) $$
By making the angle $\theta$ small, a very small vertical change $h_2$ (due to a small pressure difference) results in a large, easily readable displacement $l_2$, effectively amplifying the measurement.
\end{conceptbox}
\begin{formulabox}
$$ P_A - P_B = \gamma_2 l_2 \sin(\theta) + \gamma_3 h_3 - \gamma_1 h_1 $$
\end{formulabox}
\end{keybox}

\begin{keybox}[title=Differential U-Tube Manometer]
This device is extremely useful as it measures the pressure \textbf{difference} between two points, A and B, which might both be at high pressure.
\begin{formulabox}
$$ P_A - P_B = \gamma_2 h_2 + \gamma_3 h_3 - \gamma_1 h_1 $$
\end{formulabox}
\begin{itemize}[itemsep=2pt]
    \item \textbf{Explanation using the "Walking" Method:}
    \begin{enumerate}
        \item Start at point A, where the pressure is $P_A$.
        \item Walk \textbf{down} through fluid 1 by a height $h_1$. The pressure increases: $P_A + \gamma_1 h_1$. You are now at the lowest point on the left.
        \item Move horizontally across the continuous manometer fluid (fluid 2) to the right side. The pressure does not change.
        \item Now walk \textbf{up} through the manometer fluid (fluid 2) by a height $h_2$. The pressure decreases: $P_A + \gamma_1 h_1 - \gamma_2 h_2$.
        \item Continue walking \textbf{up} through fluid 3 by a height $h_3$ to get to point B. The pressure decreases again: $P_A + \gamma_1 h_1 - \gamma_2 h_2 - \gamma_3 h_3$.
        \item The expression you are left with is the pressure at point B. So, $P_B = P_A + \gamma_1 h_1 - \gamma_2 h_2 - \gamma_3 h_3$.
        \item Rearranging this equation to solve for the pressure difference, $P_A - P_B$, gives the formula above.
    \end{enumerate}
\end{itemize}
\end{keybox}

\newpage
\section*{Hydrostatic Balance in an Inclined Manometer}
A manometer is a classic instrument used to measure pressure differences by balancing the weight of fluid columns. An \textbf{inclined manometer} is a special type designed to increase the sensitivity of the measurement, making it easier to read very small pressure differences.

\begin{conceptbox}[title=Principle of the Inclined Manometer]
The key advantage of an inclined manometer is that a small vertical change in fluid height, $\Delta h$, results in a much larger and more easily readable displacement, $L$, along the inclined tube. The relationship is based on simple trigonometry:
$$ \Delta h = L \sin\theta $$
By making the angle $\theta$ very small (e.g., 10$^\circ$), the reading $L$ can be made many times larger than the actual vertical displacement $\Delta h$, thus amplifying the measurement and improving its precision.
\end{conceptbox}

\begin{examplebox}{Pressure Change in a Multi-Fluid Inclined Manometer}
\textbf{Question:} An inclined manometer contains water, oil (SG=0.85), and mercury (SG=13.6). The left vertical tube has an inner diameter (ID) of 2 cm, and the inclined right tube has an ID of 1 cm. Initially, the system is at equilibrium. Then, the pressure $P_1$ is increased by 50 mm of mercury. What is the change in the reading along the inclined scale ($y$)? The angle of inclination is 30$^\circ$.
\end{examplebox}

\begin{stepbox}
\begin{enumerate}[label=\textbf{Step \arabic*:}, wide=0pt, leftmargin=*, itemsep=2pt]
    \item \textbf{Strategy: Hydrostatic and Volume Balances}
    This is a hydrostatic problem, but because the fluid levels change, we must also consider a volume (or mass) balance. The plan is:
    \begin{enumerate}
        \item Use a \textbf{volume balance} on the mercury to find a geometric relationship between the level drop in the left tube ($x$) and the displacement along the incline in the right tube ($y$).
        \item Use a \textbf{hydrostatic pressure balance} to relate the applied change in pressure ($\Delta P_1$) to the changes in the fluid column heights.
        \item Combine these two relationships to solve for the unknown displacement, $y$.
    \end{enumerate}

\end{enumerate}
\end{stepbox}

\begin{stepbox}
\begin{enumerate}[label=\textbf{Step \arabic*:}, wide=0pt, leftmargin=*, itemsep=2pt, start = 2]
    \item \textbf{Relate Displacements with a Volume Balance:}
    When the pressure $P_1$ increases, it pushes the mercury down in the left tube by a vertical distance $x$. This volume of mercury must move into the right tube, causing the level to rise a distance $y$ along the incline. Since mercury is incompressible, the volume is conserved.
    $$ (\text{Volume of mercury lost from left tube}) = (\text{Volume of mercury gained in right tube}) $$
    $$ A_{\text{left}} \cdot x = A_{\text{right}} \cdot y $$
    Since area $A = \pi d^2 / 4$, the $\pi/4$ terms cancel:
    $$ d_{\text{left}}^2 x = d_{\text{right}}^2 y \implies (2 \, \text{cm})^2 x = (1 \, \text{cm})^2 y \implies \textbf{y = 4x} $$
    This is a crucial geometric constraint: the displacement along the incline is four times the vertical drop in the left tube.
    \item \textbf{Relate Pressure Change to Displacements via Hydrostatics:}
    By writing the full hydrostatic balance equation from $P_1$ to $P_2$ for the initial and final states and then subtracting them, we can arrive at an equation for the \textit{change} in pressure. The change in pressure, $\Delta P_1 = P_1' - P_1$, is balanced by the net change in the weights of the fluid columns due to the displacements $x$ and $y$:
    $$ \Delta P_1 = -\gamma_{\text{oil}} (y \sin\theta) + \gamma_{\text{Hg}} (y \sin\theta + x) - \gamma_{\text{water}} (x) $$
    \item \textbf{Solve for the Displacements:}
    Substitute the geometric constraint $y=4x$ into the pressure change equation:
    $$ \Delta P_1 = x \cdot [-\gamma_{\text{oil}} (4 \sin\theta) + \gamma_{\text{Hg}} (4 \sin\theta + 1) - \gamma_{\text{water}}] $$
    Using SI units: $\Delta P_1 = 50 \, \text{mmHg} \times 133.32 \, \text{Pa/mmHg} = 6666$ Pa. The specific weights are $\gamma_{\text{water}} = 1000 \times 9.81 = 9810$ N/m$^3$, $\gamma_{\text{oil}} = 0.85 \times 9810 = 8338.5$ N/m$^3$, $\gamma_{\text{Hg}} = 13.6 \times 9810 = 133,416$ N/m$^3$, and $\sin(30^\circ) = 0.5$.
    $$ 6666 = x \cdot [-8338.5(4 \times 0.5) + 133416(4 \times 0.5 + 1) - 9810] $$
    $$ 6666 = x \cdot [-16677 + 133416(3) - 9810] = x \cdot [373,761] $$
    $$ x = \frac{6666}{373761} = 0.0178 \, \text{m} = 1.78 \, \text{cm} $$
    The change in the reading on the inclined scale is:
    $$ y = 4x = 4 \times 1.78 = 7.12 \, \text{cm} $$
    A pressure increase of 50 mmHg causes the reading on the incline to change by \textbf{7.1 cm}.
    
\end{enumerate}
\end{stepbox}


nd{document}
