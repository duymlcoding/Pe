\documentclass[12pt]{article}
\usepackage[paperwidth=8.5in, paperheight=11in, margin=1.0in, headheight=15pt]{geometry}
\usepackage{amsmath,amssymb,amsthm}
\usepackage[most]{tcolorbox}
\usepackage{enumitem}
\usepackage{xcolor}
\usepackage{hyperref}
\usepackage{fancyhdr}
\usepackage{titlesec}
\usepackage{graphicx}
% Define custom colors for chemical engineering theme
\definecolor{conceptcolor}{RGB}{52, 73, 94}      % Dark blue-gray
\definecolor{formulacolor}{RGB}{231, 76, 60}     % Red for formulas
\definecolor{examplecolor}{RGB}{39, 174, 96}     % Green for examples
\definecolor{stepcolor}{RGB}{142, 68, 173}       % Purple for solution steps
\definecolor{keycolor}{RGB}{243, 156, 18}        % Orange for key points
% Configure fancy headers
\pagestyle{fancy}
\fancyhf{}
\fancyhead[L]{PE Study Guide}
\fancyhead[R]{Process Fundamentals}
\fancyfoot[C]{\thepage}
\renewcommand{\baselinestretch}{1.1}
\setlength{\parindent}{0.25in}
\setlength{\parskip}{3pt}
% Configure section formatting
\titleformat{\section}
  {\normalfont\LARGE\bfseries\color{conceptcolor}}
  {\thesection}{1em}{}
\titleformat{\subsection}
  {\normalfont\Large\bfseries\color{conceptcolor}}
  {\thesubsection}{1em}{}
% Define custom environments
\newtcolorbox{conceptbox}[1][]{
  enhanced,
  colback=conceptcolor!10,
  colframe=conceptcolor,
  arc=3mm,
  title=Key Concept,
  fonttitle=\bfseries\sffamily\normalsize,
  fontupper=\small,
  #1
}
\newtcolorbox{formulabox}[1][]{
  enhanced,
  colback=formulacolor!10,
  colframe=formulacolor,
  arc=2mm,
  title=Important Formula,
  fonttitle=\bfseries\sffamily\normalsize,
  fontupper=\small,
  #1
}
\newtcolorbox{examplebox}[2][]{
  enhanced,
  colback=examplecolor!10,
  colframe=examplecolor,
  arc=3mm,
  title=Example Problem: #2,
  fonttitle=\bfseries\sffamily\normalsize,
  fontupper=\small,
  #1
}
\newtcolorbox{stepbox}[1][]{
  enhanced,
  colback=stepcolor!10,
  colframe=stepcolor,
  arc=2mm,
  title=Solution Steps,
  fonttitle=\bfseries\sffamily\normalsize,
  fontupper=\small,
  #1
}
\newtcolorbox{keybox}[1][]{
  enhanced,
  colback=keycolor!10,
  colframe=keycolor,
  arc=2mm,
  title=Key Variables \& Definitions,
  fonttitle=\bfseries\sffamily\normalsize,
  fontupper=\small,
  #1
}

\begin{document}

\begin{center}
    \Huge\textbf{\color{conceptcolor}Mass Transfer}
\end{center}
\hrule

\section*{Diffusion into a Solid}
Diffusion is the net movement of a substance from a region of higher concentration to a region of lower concentration, driven by the random motion of particles. When this process occurs over time, it is known as \textbf{unsteady-state diffusion}, and it is crucial for many materials science and chemical engineering processes, such as hardening steel or doping semiconductors.

\subsection*{The Governing Equation: Fick's Second Law}

\begin{conceptbox}
Unsteady-state diffusion is described by \textbf{Fick's Second Law}, a partial differential equation that relates the change in concentration over time to the second derivative of concentration with respect to spatial position. It describes how the concentration profile evolves over time.
\end{conceptbox}

\begin{formulabox}
For one-dimensional diffusion with a constant diffusion coefficient, the law is:
$$ \frac{\partial C_A}{\partial t} = D_{AB} \frac{\partial^2 C_A}{\partial x^2} \quad \text{(Equation 1)} $$
\end{formulabox}

\begin{keybox}
\begin{itemize}[itemsep=0pt]
    \item $C_A$: Concentration of the diffusing substance A [mol/m$^3$ or wt\%].
    \item $t$: Time [s].
    \item $x$: Position or distance into the solid [m].
    \item $D_{AB}$: The diffusion coefficient (or diffusivity) of A in B [m$^2$/s].
\end{itemize}
\end{keybox}

\newpage

\subsection*{The Error Function Solution for a Semi-Infinite Solid}
\begin{conceptbox}
A common scenario involves diffusion into a solid that is thick enough to be considered \textbf{semi-infinite}. The initial concentration is uniform, and at time $t=0$, the surface concentration is suddenly changed to a new, constant value. The analytical solution for this specific case uses the mathematical \textbf{error function (erf)}.
\end{conceptbox}

\begin{formulabox}
The concentration profile is given by:
$$ \frac{C_A(x,t) - C_{A0}}{C_{As} - C_{A0}} = 1 - \text{erf}\left(\frac{x}{2\sqrt{D_{AB}t}}\right) \quad \text{(Equation 2)} $$
\end{formulabox}

\begin{keybox}
\begin{itemize}[itemsep=0pt]
    \item $C_A(x,t)$: The concentration at position $x$ and time $t$.
    \item $C_{A0}$: The initial, uniform concentration in the solid at $t=0$.
    \item $C_{As}$: The constant concentration maintained at the surface ($x=0$) for all time $t>0$.
    \item $\text{erf}(z)$: The error function, whose values are found in tables or calculated by software.
\end{itemize}
\end{keybox}

\newpage

\subsection*{Example Problem: Carburization of a Steel Alloy}
\begin{examplebox}{Carburization of a Steel Alloy}
A steel alloy initially has a uniform carbon concentration of 0.10 wt\%. To harden the surface, it is placed in a high-temperature furnace at 1000$^\circ$C where the carbon concentration at the surface is maintained at 1.5 wt\%. The diffusion coefficient of carbon in steel at this temperature is $D_{AB} = 2.0 \times 10^{-11}$ m$^2$/s. How long (in hours) will it take for the carbon concentration to reach 1.0 wt\% at a depth of 1.0 mm below the surface?
\end{examplebox}
\begin{stepbox}
\begin{enumerate}[label=\textbf{Step \arabic*:}, wide=0pt, leftmargin=*, itemsep=2pt]
    \item \textbf{Identify Knowns and Strategy:} The problem describes diffusion into a semi-infinite solid with constant surface concentration, matching the conditions for the error function solution. We will solve Equation 2 for time, $t$.
    \begin{itemize}[itemsep=0pt]
        \item Target Concentration: $C_A(x,t) = 1.0$ wt\%
        \item Initial Concentration: $C_{A0} = 0.10$ wt\%
        \item Surface Concentration: $C_{As} = 1.5$ wt\%
        \item Depth: $x = 1.0 \, \text{mm} = 1.0 \times 10^{-3}$ m
        \item Diffusivity: $D_{AB} = 2.0 \times 10^{-11}$ m$^2$/s
    \end{itemize}

    \item \textbf{Calculate the Concentration Ratio:} Substitute the known concentrations into the left-hand side of Equation 2.
    $$ \frac{C_A(x,t) - C_{A0}}{C_{As} - C_{A0}} = \frac{1.0 - 0.10}{1.5 - 0.10} = \frac{0.9}{1.4} \approx 0.6429 $$

\end{enumerate}
\end{stepbox}


\begin{stepbox}
\begin{enumerate}[label=\textbf{Step \arabic*:}, wide=0pt, leftmargin=*, itemsep=2pt, start = 3]
    \item \textbf{Solve for the Error Function Term:} Set the result equal to the right-hand side of Equation 2 and solve for erf(z).
    $$ 0.6429 = 1 - \text{erf}\left(\frac{x}{2\sqrt{D_{AB}t}}\right) $$
    $$ \text{erf}\left(\frac{x}{2\sqrt{D_{AB}t}}\right) = 1 - 0.6429 = 0.3571 $$

    \item \textbf{Find the Argument of the Error Function:} Using an error function table or calculator, find the value of $z$ for which $\text{erf}(z) = 0.3571$.
    $$ z = \frac{x}{2\sqrt{D_{AB}t}} \approx 0.328 $$
    
    \item \textbf{Solve for Time ($t$):} Rearrange the equation for $z$ to solve for the unknown time, $t$.
    $$ \sqrt{t} = \frac{x}{2 \cdot (0.328) \cdot \sqrt{D_{AB}}} $$
    $$ t = \left( \frac{x}{2 \cdot (0.328) \cdot \sqrt{D_{AB}}} \right)^2 = \frac{x^2}{4 \cdot (0.328)^2 \cdot D_{AB}} $$
    $$ t = \frac{(1.0 \times 10^{-3} \, \text{m})^2}{4 \cdot (0.328)^2 \cdot (2.0 \times 10^{-11} \, \text{m}^2/\text{s})} = \frac{1.0 \times 10^{-6}}{8.60 \times 10^{-12}} \, \text{s} \approx 116,279 \, \text{s} $$

    \item \textbf{Convert to Hours and Final Answer:} Convert the time from seconds to hours.
    $$ t = 116,279 \, \text{s} \times \frac{1 \, \text{h}}{3600 \, \text{s}} \approx 32.3 \, \text{h} $$
    The process will take approximately \textbf{32.3 hours}.
\end{enumerate}
\end{stepbox}

\newpage
\section*{McCabe-Thiele Diagrams}
The McCabe-Thiele method is a graphical technique used to determine the number of theoretical equilibrium stages required to separate a binary (two-component) mixture via distillation. It is a powerful visual tool that combines thermodynamic equilibrium data with material balances to model the distillation process. The entire method is built upon a key simplifying assumption.

\begin{conceptbox}[title=Key Assumption: Constant Molar Overflow (CMO)]
The McCabe-Thiele method assumes \textbf{Constant Molar Overflow (CMO)}. This means that for every mole of vapor that condenses on a tray, one mole of liquid vaporizes. As a result, the molar flow rates of liquid ($L$) and vapor ($V$) are considered constant throughout each section of the column (i.e., above the feed and below the feed). This simplification is reasonably accurate for ideal mixtures where the molar heats of vaporization of the two components are nearly equal.
\end{conceptbox}

\subsection*{Important Equations}
The McCabe-Thiele diagram is constructed using several key lines, each derived from material balances around different sections of the distillation column.

\begin{formulabox}[title=Key Operating and Feed Lines]
\textbf{Rectifying Section Operating Line (TOL):}
This line describes the material balance in the section above the feed. In terms of the reflux ratio, $R$:
$$ y = \left(\frac{R}{R+1}\right)x + \left(\frac{1}{R+1}\right)x_D $$

\textbf{Stripping Section Operating Line (BOL):}
This line describes the material balance in the section below the feed.
$$ y = \left(\frac{\bar{L}}{\bar{V}}\right)x - \left(\frac{B}{\bar{V}}\right)x_B $$

\textbf{The q-Line (Feed Line):}
This line is determined by the thermal condition of the feed and represents the locus of intersection points for the TOL and BOL.
$$ y = \left(\frac{q}{q-1}\right)x - \left(\frac{z_F}{q-1}\right) $$
\end{formulabox}

\begin{keybox}
\begin{itemize}[itemsep=0pt]
    \item $x, y$: Mole fraction of the more volatile component in the liquid and vapor phase, respectively.
    \item $x_D, x_B, z_F$: Mole fractions in the distillate, bottoms, and feed streams.
    \item $L, V$: Molar flow rates of liquid and vapor in the rectifying (top) section [mol/time].
    \item $\bar{L}, \bar{V}$: Molar flow rates of liquid and vapor in the stripping (bottom) section [mol/time].
    \item $D, B, F$: Molar flow rates of the distillate, bottoms, and feed streams.
    \item $R$: Reflux ratio, the ratio of liquid returned to the column to liquid taken as distillate ($R=L/D$).
    \item $q$: The thermal quality of the feed.
\end{itemize}
\end{keybox}

\subsection*{The Conceptual Basis: Why "Stepping Off Stages" Works}
\begin{conceptbox}[title=Combining Balances and Equilibrium]
A distillation column achieves separation because of two distinct physical phenomena that are represented by the two main lines on the diagram:
\begin{enumerate}[itemsep=2pt]
    \item \textbf{On a stage (tray):} The vapor and liquid phases are in intimate contact, allowing them to reach thermodynamic equilibrium. Their compositions ($x_n, y_n$) for a given stage $n$ are related by the \textbf{Vapor-Liquid Equilibrium (VLE) curve}. This is a thermodynamic relationship.
    \item \textbf{Between stages:} A vapor stream ($y_{n+1}$) rising from one stage passes a liquid stream ($x_n$) descending from the stage above. Their compositions are related by a \textbf{material balance}. This relationship is described by the \textbf{operating line}.
\end{enumerate}
The McCabe-Thiele "staircase" is a graphical method of alternating between these two relationships to count the number of stages needed to get from the reboiler composition to the distillate composition.
\end{conceptbox}

\begin{conceptbox}[title=Visualizing the Stepping Process]
The graphical procedure for stepping off stages simulates the movement of components through the column:
\begin{enumerate}[label=\textbf{Step \arabic*:}, wide=0pt, leftmargin=*, itemsep=2pt]
    \item \textbf{Start at the Distillate:} Begin at the top of the column. The liquid reflux returning to the first stage has the distillate composition, $x_D$. This is represented by the point ($x_D, x_D$) on the $y=x$ line.
    \item \textbf{Find Vapor from Stage 1 (Material Balance):} To find the composition of the vapor ($y_1$) rising from stage 1, move vertically from the starting point to the \textbf{Top Operating Line (TOL)}. This represents the material balance between the condenser and stage 1.
    \item \textbf{Find Liquid on Stage 1 (Equilibrium):} This vapor ($y_1$) is in equilibrium with the liquid on stage 1 ($x_1$). To find this liquid's composition, move horizontally from the point on the TOL to the \textbf{VLE curve}. The x-coordinate of this new point is $x_1$. This completes one "step," which represents one theoretical stage.
    \item \textbf{Repeat Down the Column:} From the point on the VLE curve, move vertically down to the operating line to find the vapor composition from the next stage ($y_2$). Then move horizontally to the VLE curve to find the corresponding liquid composition ($x_2$). This process is repeated. When the steps cross the q-line, you must switch from using the TOL to using the BOL for the vertical movements. Continue until the liquid composition $x_n$ is less than or equal to the desired bottoms composition, $x_B$.
\end{enumerate}
\end{conceptbox}

\newpage

\subsection*{The Role of the Feed: The q-Line}
The condition of the feed stream when it enters the column has a significant impact on the internal liquid and vapor flow rates. The \textbf{q-line} is a graphical tool that accounts for the thermal condition and composition of the feed, and it correctly links the operating lines for the top and bottom sections of the column.

\begin{conceptbox}[title=Definition of Feed Quality (q)]
The value of \textbf{q}, the feed quality, is defined as the fraction of the feed that becomes liquid in the stripping (bottom) section of the column. It can be thought of as the heat required to vaporize one mole of feed divided by the molar latent heat of vaporization of the feed.
$$ q = \frac{\text{Increase in liquid flow rate below the feed tray}}{\text{Total molar feed rate}} = \frac{\bar{L} - L}{F} $$
\end{conceptbox}

\begin{keybox}[title=Feed Quality (q) Classification]
The value of $q$ precisely defines the thermal state of the feed, which in turn determines the slope of the q-line.
\begin{itemize}[itemsep=2pt]
    \item \textbf{Subcooled Liquid ($q > 1$):} The feed is a cold liquid. It condenses some vapor rising from below, increasing the liquid flow rate in the stripping section by more than the feed rate. The q-line slope is positive and greater than 1.
    \item \textbf{Saturated Liquid ($q = 1$):} The feed is a liquid at its boiling point. All of the feed joins the liquid stream flowing down. The q-line is vertical.
    \item \textbf{Two-Phase Mixture ($0 < q < 1$):} The feed is a liquid-vapor mixture. The liquid fraction ($q$) flows down, and the vapor fraction ($1-q$) flows up. The q-line has a negative slope.
    \item \textbf{Saturated Vapor ($q = 0$):} The feed is a vapor at its dew point. All of the feed joins the vapor stream flowing up. The q-line is horizontal.
    \item \textbf{Superheated Vapor ($q < 0$):} The feed is a hot vapor. It vaporizes some of the liquid flowing down from above, resulting in less liquid in the stripping section. The q-line has a positive slope less than 1.
\end{itemize}
\end{keybox}

\begin{formulabox}[title=q-Line Properties]
The q-line is derived by finding the mathematical intersection of the TOL and BOL. Its equation is:
$$ y = \left(\frac{q}{q-1}\right)x - \left(\frac{z_F}{q-1}\right) $$
This line has two critical properties that are used to draw it on the diagram:
\begin{itemize}[itemsep=2pt]
    \item Its \textbf{slope} is determined by the feed quality: $m_q = \frac{q}{q-1}$.
    \item It always intersects the \textbf{$y=x$ line} at the feed composition, $z_F$.
\end{itemize}
To construct the diagram, one draws the q-line starting from the point $(z_F, z_F)$ on the $y=x$ line with the slope $m_q$. The intersection of this q-line with the TOL defines the point through which the BOL must pass.
\end{formulabox}

\newpage

\subsection*{Comprehensive Example: Acetone-Ethanol Separation}

\begin{examplebox}{Acetone-Ethanol Separation}
Acetone and ethanol are to be separated in a distillation column. The column has a \textbf{partial condenser} and a \textbf{partial reboiler}, which each function as an equilibrium stage. An equimolar ($z_F=0.5$), sub-cooled liquid feed enters at 100 kmol/hr. The feed is cold enough that it condenses 1 mole of vapor inside the column for every 6 moles of feed that enters.

The desired separation is a distillate \textbf{vapor} product of 95 mol\% acetone ($y_D=0.95$) and a bottoms liquid product of 5 mol\% acetone ($x_B=0.05$). The reflux returned from the condenser is a saturated liquid. The column is operated with a liquid-to-vapor flow ratio in the rectifying section of $(L/V) = 1.4 \times (L/V)_{\text{min}}$. Assume Constant Molar Overflow.

\textbf{Tasks:}
\begin{enumerate}[itemsep=0pt]
    \item Plot the operating lines for the rectifying and stripping sections and the feed line.
    \item Determine the total number of equilibrium stages required and the number of trays needed.
    \item Determine the optimal feed tray location.
    \item Determine the molar flow rates of the distillate (D) and bottoms (B) products, as well as the internal flow rates in the rectifying (L, V) and stripping ($\bar{L}, \bar{V}$) sections.
\end{enumerate}
\end{examplebox}

\begin{stepbox}
\begin{enumerate}[label=\textbf{Step \arabic*:}, wide=0pt, leftmargin=*, itemsep=2pt]
    \item \textbf{Analyze the Problem and List Knowns}
    Before beginning, it's crucial to identify all the given information and the goals.
    \begin{itemize}[itemsep=2pt]
        \item \textbf{Feed (F):} Rate = 100 kmol/hr, Composition $z_F = 0.5$.
        \item \textbf{Distillate (D):} Vapor product, Composition $y_D = 0.95$.
        \item \textbf{Bottoms (B):} Liquid product, Composition $x_B = 0.05$.
        \item \textbf{Column Type:} Partial condenser and partial reboiler.
        \item \textbf{Operating Condition:} $(L/V) = 1.4 \times (L/V)_{\text{min}}$.
    \end{itemize}
    Our task is to fully define the column's design and operation using the McCabe-Thiele method. We will start by defining the three key lines on the diagram: the q-line, the Top Operating Line (TOL), and the Bottom Operating Line (BOL).
\end{enumerate}
\end{stepbox}



\begin{stepbox}[title=Solution Part 1: Constructing the McCabe-Thiele Diagram]
\begin{enumerate}[label=\textbf{Step \arabic*:}, wide=0pt, leftmargin=*, itemsep=2pt, start = 2]

    \item \textbf{Determine the Feed Line (q-Line)}
    The q-line's properties are determined by the thermal condition of the feed. The feed is a sub-cooled liquid, which means $q > 1$.
    \begin{itemize}[itemsep=2pt]
        \item The increase in liquid flow across the feed stage is the sum of the feed itself plus any vapor it condenses: $\bar{L} - L = F + \frac{1}{6}F = \frac{7}{6}F$.
        \item We can now calculate $q$:
        $$ q = \frac{\bar{L} - L}{F} = \frac{(7/6)F}{F} = \frac{7}{6} \approx 1.167 $$
    \end{itemize}
    The q-line is plotted by starting at the point $(z_F, z_F) = (0.5, 0.5)$ on the $y=x$ line and drawing a line with a slope of $m_q = q/(q-1) = (7/6)/(1/6) = 7$.
    
    \item \textbf{Clarify Distillate Composition and the Partial Condenser}
    \begin{conceptbox}[title=Handling a Partial Condenser]
    A partial condenser acts as one equilibrium stage. It takes vapor from the top tray and splits it into two streams that are in equilibrium: the final vapor product ($D$, with composition $y_D$) and the saturated liquid reflux ($L$, with composition $x_D$). To find $x_D$, we must use the VLE data to find the liquid composition in equilibrium with the vapor $y_D=0.95$. The Top Operating Line (TOL) must originate from the point $(x_D, x_D)$ on the $y=x$ line.
    \end{conceptbox}
    From VLE data for Acetone-Ethanol, a vapor of $y=0.95$ is in equilibrium with a liquid of approximately $x=0.91$. We will use $\mathbf{x_D = 0.91}$ for plotting our operating line.
    
    \item \textbf{Determine the Rectifying Operating Line (TOL)}
    The operating liquid-to-vapor ratio is $(L/V) = 1.4 \times (L/V)_{\text{min}}$. First, we find the minimum ratio.
    \begin{itemize}[itemsep=2pt]
        \item The minimum TOL is the line drawn from $(x_D, x_D) = (0.91, 0.91)$ to the intersection of the \textbf{q-line} and the \textbf{VLE curve}.
        \item By drawing this line on the graph, its slope is measured to be approximately 0.62. So, $(L/V)_{\text{min}} \approx 0.62$.
        \item The actual operating ratio is: $(L/V)_{\text{actual}} = 1.4 \times 0.62 = 0.868 \approx 0.87$.
    \end{itemize}
    The TOL is plotted starting from $(x_D, x_D) = (0.91, 0.91)$ with a slope of 0.87.

    \item \textbf{Determine the Stripping Operating Line (BOL)}
    The BOL is the straight line that connects two points: $(x_B, x_B) = (0.05, 0.05)$ and the intersection point of the TOL and the q-line. Drawing this line completes the diagram.
\end{enumerate}
\end{stepbox}

\newpage

\begin{stepbox}[title=Solution Part 2: Determining Stages and Feed Location]
\begin{enumerate}[label=\textbf{Step \arabic*:}, wide=0pt, leftmargin=*, itemsep=2pt, resume]
    \item \textbf{Step Off Stages and Determine Feed Location}
    Now we use the completed diagram to count the required equilibrium stages, starting from the top.
    \begin{itemize}[itemsep=2pt]
        \item \textbf{Stage 1 (Partial Condenser):} This is our first equilibrium stage, establishing the equilibrium between $y_D=0.95$ and $x_D=0.91$.
        \item \textbf{Stepping Procedure:} We begin counting trays at $(x_D, x_D)=(0.91, 0.91)$. We draw "stair steps" between the \textbf{TOL} and the VLE curve.
        \item \textbf{Feed Stage and Switch to BOL:} When a step crosses the q-line, we switch to using the \textbf{BOL}. The tray on which this crossover occurs is the optimal feed tray.
        \item \textbf{Final Stage (Partial Reboiler):} We continue stepping down on the BOL until we reach or pass $x_B=0.05$. The partial reboiler is the final equilibrium stage.
    \end{itemize}
    Following this procedure on the McCabe-Thiele graph yields the following results:
    \begin{itemize}[itemsep=2pt]
        \item Total equilibrium stages required $\approx \textbf{10.75}$.
        \item Number of trays = Total Stages - Condenser - Reboiler = $10.75 - 1 - 1 = 8.75$ trays. We round up to \textbf{9 trays}.
        \item The optimal feed tray is where the switch from TOL to BOL occurs, which is \textbf{Tray 7} (counting from the top).
    \end{itemize}
\end{enumerate}
\end{stepbox}

\newpage

\begin{stepbox}[title=Solution Part 3: Calculating Molar Flow Rates]
\begin{enumerate}[label=\textbf{Step \arabic*:}, wide=0pt, leftmargin=*, itemsep=2pt, resume]
    \item \textbf{Calculate Overall Product Flow Rates (D and B)}
    We solve the overall material balances for the column.
    \begin{itemize}[itemsep=2pt]
        \item \textbf{Overall Total Balance:} $F = D + B \implies 100 = D + B$
        \item \textbf{Overall Component Balance (Acetone):} $F z_F = D y_D + B x_B$
        $$ (100)(0.5) = D(0.95) + B(0.05) $$
    \end{itemize}
    Solving the system of two equations (by substituting $B = 100 - D$):
    $$ 50 = 0.95D + (100 - D)(0.05) \implies 45 = 0.90D \implies D = 50 \, \text{kmol/hr} $$
    $$ B = 100 - D = 50 \, \text{kmol/hr} $$

    \item \textbf{Calculate Internal Column Flow Rates}
    \begin{itemize}[itemsep=2pt]
        \item \textbf{Rectifying Section (L, V):} From a balance around the partial condenser, $V = L + D$. We also know $L = 0.87V$.
        $$ V = 0.87V + 50 \implies 0.13V = 50 \implies V \approx 385 \, \text{kmol/hr} $$
        $$ L = 0.87 \times V = 0.87 \times 385 \approx 335 \, \text{kmol/hr} $$
        
        \item \textbf{Stripping Section ($\bar{L}, \bar{V}$):} We use the q-value to find $\bar{L}$.
        $$ \bar{L} = L + qF = 335 + (7/6)(100) \approx 452 \, \text{kmol/hr} $$
        From a balance around the reboiler, $\bar{L} = \bar{V} + B$.
        $$ \bar{V} = \bar{L} - B = 452 - 50 = 402 \, \text{kmol/hr} $$
    \end{itemize}
\end{enumerate}

\begin{keybox}[title=Final Answer Summary]
    \begin{itemize}[itemsep=2pt]
        \item \textbf{Total Equilibrium Stages:} 10.75 (Requires 1 partial condenser, 1 partial reboiler, and 9 trays).
        \item \textbf{Optimal Feed Location:} Tray 7 (from the top).
        \item \textbf{Product Flow Rates:} $D = 50$ kmol/hr, $B = 50$ kmol/hr.
        \item \textbf{Internal Flow Rates (Rectifying):} $L = 335$ kmol/hr, $V = 385$ kmol/hr.
        \item \textbf{Internal Flow Rates (Stripping):} $\bar{L} = 452$ kmol/hr, $\bar{V} = 402$ kmol/hr.
    \end{itemize}
\end{keybox}
\end{stepbox}


\newpage
\section*{Batch Column: Constant Reflux}
Batch distillation is a process used to separate a finite quantity (a "batch") of a liquid mixture. In \textbf{constant reflux} operation, the reflux ratio ($R$) is held constant. As the distillation proceeds, the mixture in the reboiler becomes weaker in the more volatile component, and consequently, the distillate product also becomes weaker over time.

\subsection*{The Rayleigh Equation and Numerical Integration}
\begin{conceptbox}
The governing material balance for batch distillation is the \textbf{Rayleigh Equation}. It relates the amount of liquid remaining in the reboiler to the change in composition of the reboiler liquid. Because the distillate composition ($x_D$) changes as the reboiler composition ($x_W$) changes, the integral often requires numerical evaluation.
\end{conceptbox}

\begin{formulabox}
The Rayleigh Equation is:
$$ \ln\left(\frac{W_{f}}{W_{i}}\right) = \int_{x_{W,i}}^{x_{W,f}} \frac{dx_W}{x_D - x_W} \quad \text{(Equation 3)} $$
The relationship between $x_D$ and $x_W$ must be determined for the given number of stages and reflux ratio, typically using the McCabe-Thiele method.
\end{formulabox}

\begin{keybox}
\begin{itemize}[itemsep=0pt]
    \item $W_{i}$, $W_{f}$: Initial and final moles of liquid in the reboiler (still).
    \item $x_{W,i}$, $x_{W,f}$: Initial and final mole fraction of the more volatile component in the reboiler.
    \item $x_D$: Instantaneous mole fraction of the more volatile component in the distillate.
\end{itemize}
\end{keybox}

\newpage

\subsection*{Example Problem: Constant Reflux Distillation}
\begin{examplebox}{Constant Reflux Distillation}
A batch distillation column with 2 equilibrium stages (including the partial reboiler) is operated with a constant reflux ratio of $R=3$. At a point when the distillate composition is $x_D = 0.60$, there are 1000 moles of liquid in the reboiler. How much liquid is left in the reboiler when the distillate composition has dropped to $x_D = 0.50$? Use Simpson's rule for numerical integration.
\end{examplebox}

\begin{stepbox}
\begin{enumerate}[label=\textbf{Step \arabic*:}, wide=0pt, leftmargin=*, itemsep=2pt]
    \item \textbf{Strategy and Knowns:} We must solve the Rayleigh Equation (Eq. 3). First, we will use the McCabe-Thiele method to find the relationship between $x_W$ and $x_D$ at the start, middle, and end points of the process. Then, we use this data with Simpson's rule to evaluate the integral and solve for $W_f$.
    
    The system has $N=2$ stages (partial reboiler + 1 stage), reflux ratio $R=3$ (constant), and initial moles $W_i = 1000$ moles.
    
    \item \textbf{Relate $x_D$ and $x_W$ via McCabe-Thiele:} The operating line slope is constant: $m = R/(R+1) = 3/4 = 0.75$. By stepping down 2 stages from the line $y=x$ for several $x_D$ values, we generate the required data.
    
    For $x_D = 0.60$, stepping down 2 stages gives $x_W = 0.09$ (initial state). For $x_D = 0.55$, stepping down 2 stages gives $x_W = 0.06$ (midpoint). For $x_D = 0.50$, stepping down 2 stages gives $x_W = 0.04$ (final state).
    
    \item \textbf{Numerical Integration (Simpson's Rule):} We must evaluate $I = \int_{0.09}^{0.04} \frac{dx_W}{x_D - x_W}$. The function is $f(x_W) = \frac{1}{x_D - x_W}$.
    
    At the initial point: $f(x_{W,i}) = \frac{1}{0.60 - 0.09} = \frac{1}{0.51} \approx 1.961$. At the midpoint: $f(x_{W,m}) = \frac{1}{0.55 - 0.06} = \frac{1}{0.49} \approx 2.041$. At the final point: $f(x_{W,f}) = \frac{1}{0.50 - 0.04} = \frac{1}{0.46} \approx 2.174$.
    
    Using Simpson's rule: $I \approx \frac{x_{W,f} - x_{W,i}}{6} [f(x_{W,i}) + 4f(x_{W,m}) + f(x_{W,f})]$.
    $$ I \approx \frac{0.04 - 0.09}{6} [1.961 + 4(2.041) + 2.174] = \frac{-0.05}{6} [12.299] \approx -0.1025 $$
    
    \item \textbf{Solve the Rayleigh Equation:} Substitute the value of the integral into Equation 3.
    $$ \ln\left(\frac{W_f}{W_i}\right) = -0.1025 \quad \frac{W_f}{1000} = e^{-0.1025} \approx 0.9026 $$
    $$ W_f = 1000 \times 0.9026 = 902.6 \, \text{moles} $$
    
    \item \textbf{Final Answer:} Approximately \textbf{903 moles} of liquid remain in the reboiler.
\end{enumerate}
\end{stepbox}

\newpage
nd{document}
