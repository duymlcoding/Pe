\documentclass[12pt]{article}
\usepackage[paperwidth=8.5in, paperheight=11in, margin=1.0in, headheight=12pt]{geometry}
\usepackage{amsmath,amssymb,amsthm}
\usepackage[most]{tcolorbox}
\usepackage{enumitem}
\usepackage{xcolor}
\usepackage{hyperref}
\usepackage{fancyhdr}
\usepackage{titlesec}
\usepackage{graphicx}
% Define custom colors for chemical engineering theme
\definecolor{conceptcolor}{RGB}{52, 73, 94}      % Dark blue-gray
\definecolor{formulacolor}{RGB}{231, 76, 60}     % Red for formulas
\definecolor{examplecolor}{RGB}{39, 174, 96}     % Green for examples
\definecolor{stepcolor}{RGB}{142, 68, 173}       % Purple for solution steps
\definecolor{keycolor}{RGB}{243, 156, 18}        % Orange for key points
% Configure fancy headers
\pagestyle{fancy}
\fancyhf{}
\fancyhead[L]{PE Study Guide}
\fancyhead[R]{Process Fundamentals}
\fancyfoot[C]{\thepage}
\renewcommand{\baselinestretch}{1.1}
\setlength{\parindent}{0.25in}
\setlength{\parskip}{3pt}
% Configure section formatting
\titleformat{\section}
  {\normalfont\LARGE\bfseries\color{conceptcolor}}
  {\thesection}{1em}{}
\titleformat{\subsection}
  {\normalfont\Large\bfseries\color{conceptcolor}}
  {\thesubsection}{1em}{}
% Define custom environments
\newtcolorbox{conceptbox}[1][]{
  enhanced,
  colback=conceptcolor!10,
  colframe=conceptcolor,
  arc=3mm,
  title=Key Concept,
  fonttitle=\bfseries\sffamily\normalsize,
  fontupper=\small,
  #1
}
\newtcolorbox{formulabox}[1][]{
  enhanced,
  colback=formulacolor!10,
  colframe=formulacolor,
  arc=2mm,
  title=Important Formula,
  fonttitle=\bfseries\sffamily\normalsize,
  fontupper=\small,
  #1
}
\newtcolorbox{examplebox}[2][]{
  enhanced,
  colback=examplecolor!10,
  colframe=examplecolor,
  arc=3mm,
  title=Example Problem: #2,
  fonttitle=\bfseries\sffamily\normalsize,
  fontupper=\small,
  #1
}
\newtcolorbox{stepbox}[1][]{
  enhanced,
  colback=stepcolor!10,
  colframe=stepcolor,
  arc=2mm,
  title=Solution Steps,
  fonttitle=\bfseries\sffamily\normalsize,
  fontupper=\small,
  #1
}
\newtcolorbox{keybox}[1][]{
  enhanced,
  colback=keycolor!10,
  colframe=keycolor,
  arc=2mm,
  title=Key Variables \& Definitions,
  fonttitle=\bfseries\sffamily\normalsize,
  fontupper=\small,
  #1
}
\section*{Introduction to Energy Balances}

\begin{conceptbox}
While material balances track the flow of mass, \textbf{energy balances} track the flow of energy based on the \textbf{First Law of Thermodynamics}, which states that energy is conserved. For chemical engineers, this typically involves analyzing heat transfer and the energy changes associated with temperature changes or phase transitions.

The fundamental principle: Energy can neither be created nor destroyed, only converted from one form to another.
\end{conceptbox}

Energy balances are essential for:
\begin{itemize}[itemsep=0pt]
\item Sizing heating and cooling equipment
\item Determining utility requirements (steam, cooling water)
\item Analyzing heat exchanger performance
\item Optimizing energy efficiency in processes
\item Safety analysis and temperature control
\end{itemize}

\section*{The First Law for Open, Steady-State Systems}

For chemical engineering systems where material flows continuously in and out at steady rates, we apply the First Law of Thermodynamics in its most general form.

\begin{formulabox}
\textbf{General Energy Balance for Open, Steady-State Systems:}
$$\Delta \dot{H} + \Delta \dot{E}_k + \Delta \dot{E}_p = \dot{Q} - \dot{W}_s$$

This equation states that the rate of energy accumulation in all forms equals the net rate of energy input to the system.
\end{formulabox}

\begin{keybox}
\textbf{Energy Balance Terms:}
\begin{itemize}[itemsep=0pt]
\item $\Delta \dot{H}$ = Rate of enthalpy change $(\dot{H}_{out} - \dot{H}_{in})$ $\text{[kW or kJ/s]}$
\item $\Delta \dot{E}_k$ = Rate of kinetic energy change $\text{[kW or kJ/s]}$
\item $\Delta \dot{E}_p$ = Rate of potential energy change $\text{[kW or kJ/s]}$
\item $\dot{Q}$ = Rate of heat transfer \textbf{to} the system $\text{[kW or kJ/s]}$
\item $\dot{W}_s$ = Rate of shaft work done \textbf{by} the system $\text{[kW or kJ/s]}$
\item $\hat{h}$ = Specific enthalpy (enthalpy per unit mass) $\text{[kJ/kg]}$
\item $\dot{m}$ = Mass flow rate $\text{[kg/s]}$
\end{itemize}

\textbf{Sign Conventions:}
\begin{itemize}[itemsep=0pt]
\item $\dot{Q} > 0$: Heat added to system; $\dot{Q} < 0$: Heat removed from system
\item $\dot{W}_s > 0$: Work done by system (turbine); $\dot{W}_s < 0$: Work done on system (pump)
\end{itemize}
\end{keybox}

\subsection*{Simplified Energy Balance}

For many common chemical engineering applications (heat exchangers, reactors, pipes), the kinetic and potential energy changes are negligible compared to enthalpy changes, and no shaft work is involved.

\begin{formulabox}
\textbf{Simplified Energy Balance (Most Common Form):}
$$\dot{Q} = \Delta \dot{H} = \dot{m}(\hat{h}_{out} - \hat{h}_{in})$$

This form is applicable when:
\begin{itemize}[itemsep=0pt]
\item $\Delta \dot{E}_k \approx 0$ (low velocities)
\item $\Delta \dot{E}_p \approx 0$ (small elevation changes)
\item $\dot{W}_s = 0$ (no pumps, turbines, or compressors)
\end{itemize}
\end{formulabox}

\newpage

\section*{Worked Example: Steam Heat Exchanger Analysis}

\begin{examplebox}{Energy Balance on Steam Heat Exchanger}
\textbf{Problem:} Superheated steam at 10 bar and 200°C flows at 2.0 kg/s through a heat exchanger. Steam heats reactor feed, and 450 kJ/s (450 kW) of heat is removed from steam. Determine outlet temperature, phase, and final specific enthalpy.
\end{examplebox}

\begin{stepbox}
\begin{enumerate}[label=\textbf{Step \arabic*:}, wide=0pt, leftmargin=*, itemsep=2pt]
    \item \textbf{Set Up Energy Balance}
    
    System: Steam flowing through heat exchanger
    
    Heat removed: $\dot{Q} = -450$ kW (negative because removed)
    
    $\dot{Q} = \dot{m}(\hat{h}_{out} - \hat{h}_{in}) \Rightarrow -450 = (2.0)(\hat{h}_{out} - \hat{h}_{in})$
    
    \item \textbf{Find Inlet Specific Enthalpy}
    
    From superheated steam tables at 10 bar and 200°C:
    
    $\hat{h}_{in} = 2828$ kJ/kg
    
    \item \textbf{Calculate Outlet Specific Enthalpy}
    
    $\hat{h}_{out} = \hat{h}_{in} + \frac{\dot{Q}}{\dot{m}} = 2828 + \frac{-450}{2.0} = 2828 - 225 = 2603$ kJ/kg
    
    \item \textbf{Determine Outlet Phase}
    
    Compare $\hat{h}_{out}$ to saturation properties at 10 bar:
    
    $\hat{h}_f = 762.5$ kJ/kg (saturated liquid)
    
    $\hat{h}_g = 2777.1$ kJ/kg (saturated vapor)
    
    Since $762.5 < 2603 < 2777.1$: \textbf{Two-phase mixture (wet steam)}
    
    Temperature = saturation temperature at 10 bar = 179.9°C
    
    \item \textbf{Calculate Quality (Vapor Fraction)}
    
    For two-phase mixture: $\hat{h}_{out} = x\hat{h}_g + (1-x)\hat{h}_f$
    
    $2603 = x(2777.1) + (1-x)(762.5) = 2014.6x + 762.5$
    
    $x = \frac{2603 - 762.5}{2014.6} = 0.914 = 91.4\%$ vapor
\end{enumerate}
\end{stepbox}

\newpage

\section*{Sensible and Latent Heat}

Energy transfer in chemical processes can result in either temperature changes or phase changes, requiring different calculation approaches.

\begin{conceptbox}
\textbf{Two Types of Heat Transfer:}
\begin{itemize}[itemsep=0pt]
\item \textbf{Sensible Heat:} Causes temperature change with no phase change
\item \textbf{Latent Heat:} Causes phase change at constant temperature
\end{itemize}

Understanding this distinction is crucial for accurate energy balance calculations, especially for systems involving boiling, condensation, melting, or freezing.
\end{conceptbox}

\subsection*{Sensible Heat Calculations}

\begin{formulabox}
\textbf{Sensible Heat Transfer:}
$$Q_s = mc_p\Delta T$$

\textbf{For continuous processes:}
$$\dot{Q}_s = \dot{m}c_p\Delta T$$

where:
\begin{itemize}[itemsep=0pt]
\item $c_p$ = specific heat capacity at constant pressure $\text{[kJ/kg·K]}$
\item $\Delta T$ = temperature change $\text{[K or °C]}$
\item Note: $\Delta T$ in Kelvin equals $\Delta T$ in Celsius
\end{itemize}
\end{formulabox}

\subsection*{Latent Heat Calculations}

\begin{formulabox}
\textbf{Latent Heat Transfer (Phase Change):}
$$Q_L = n\Delta H_{\text{phase change}}$$

\textbf{For continuous processes:}
$$\dot{Q}_L = \dot{n}\Delta H_{\text{phase change}}$$

\textbf{Common Phase Changes:}
\begin{itemize}[itemsep=0pt]
\item Vaporization: $\Delta H_{vap} > 0$ (liquid → vapor)
\item Condensation: $\Delta H_{cond} = -\Delta H_{vap} < 0$ (vapor → liquid)
\item Fusion (melting): $\Delta H_{fus} > 0$ (solid → liquid)
\item Freezing: $\Delta H_{freeze} = -\Delta H_{fus} < 0$ (liquid → solid)
\end{itemize}
\end{formulabox}

\begin{keybox}
\textbf{Important Properties for Water:}
\begin{itemize}[itemsep=0pt]
\item $\Delta H_{vap}$ (water at 100°C) = 40.6 kJ/mol = 2257 kJ/kg
\item $\Delta H_{fus}$ (water at 0°C) = 6.01 kJ/mol = 334 kJ/kg
\item $c_{p,steam}$ = 2.01 kJ/kg·K
\item $c_{p,water}$ = 4.184 kJ/kg·K  
\item $c_{p,ice}$ = 2.09 kJ/kg·K
\item Molecular weight of water = 18.015 g/mol
\end{itemize}
\end{keybox}

\newpage

\section*{Multi-Step Energy Balance Example}

\begin{examplebox}{Cooling Steam to Ice}
\textbf{Problem:} Calculate heat required to convert 45 g steam at 150°C to ice at -80°C.

\textbf{Given:}
\begin{itemize}[itemsep=0pt]
    \item $\Delta H_{vap}$ (water) = 40.6 kJ/mol
    \item $\Delta H_{fus}$ (water) = 6.01 kJ/mol
    \item $c_{steam} = 2.01$ kJ/kg·K, $c_{water} = 4.184$ kJ/kg·K, $c_{ice} = 2.09$ kJ/kg·K
\end{itemize}
\end{examplebox}

\begin{keybox}
\textbf{Process Path Analysis:}
\begin{enumerate}[itemsep=0pt]
    \item Cool steam: 150°C → 100°C (sensible)
    \item Condense steam → liquid at 100°C (latent)
    \item Cool water: 100°C → 0°C (sensible)
    \item Freeze water → ice at 0°C (latent)
    \item Cool ice: 0°C → -80°C (sensible)
\end{enumerate}

\textbf{Unit Conversions:}
\begin{itemize}[itemsep=0pt]
    \item Mass: $m = 45$ g = 0.045 kg
    \item Moles: $n = 45/18.015 = 2.50$ mol
\end{itemize}
\end{keybox}

\begin{stepbox}
\begin{enumerate}[label=\textbf{Step \arabic*:}, wide=0pt, leftmargin=*, itemsep=2pt]
    \item \textbf{Calculate Each Heat Transfer Step}
    
    $Q_{total} = Q_1 + Q_2 + Q_3 + Q_4 + Q_5$
    
    \item \textbf{Step 1: Cool Steam (150°C → 100°C)}
    
    $Q_1 = mc_{steam}\Delta T = (0.045)(2.01)(100-150) = -4.52$ kJ
    
    \item \textbf{Step 2: Condense Steam at 100°C}
    
    $Q_2 = n(-\Delta H_{vap}) = (2.50)(-40.6) = -101.5$ kJ
    
    \item \textbf{Step 3: Cool Water (100°C → 0°C)}
    
    $Q_3 = mc_{water}\Delta T = (0.045)(4.184)(0-100) = -18.83$ kJ
    
    \item \textbf{Step 4: Freeze Water at 0°C}
    
    $Q_4 = n(-\Delta H_{fus}) = (2.50)(-6.01) = -15.03$ kJ
    
    \item \textbf{Step 5: Cool Ice (0°C → -80°C)}
    
    $Q_5 = mc_{ice}\Delta T = (0.045)(2.09)(-80-0) = -7.52$ kJ
    
    \item \textbf{Sum Total Heat Transfer}
    
    $Q_{total} = -4.52 - 101.5 - 18.83 - 15.03 - 7.52 = -147.4$ kJ
\end{enumerate}
\end{stepbox}

\begin{conceptbox}[title=Final Answer]
A total of \textbf{147.4 kJ} of heat must be removed to convert 45 g of steam at 150°C to ice at -80°C.

The negative sign indicates heat removal from the system.
\end{conceptbox}

\newpage

\section*{Energy Balances with Phase Change}

Many industrial processes, such as condensation, evaporation, and distillation, involve phase changes. Calculating energy requirements for these processes requires accounting for both sensible heat (temperature change) and latent heat (phase change).

\subsection*{State Functions and Hypothetical Paths}

\begin{conceptbox}[title=State Function Property]
Enthalpy ($\hat{H}$) is a \textbf{state function}, meaning the change in enthalpy ($\Delta \hat{H}$) between two states depends only on initial and final conditions, not on the path taken. This allows us to construct convenient, hypothetical paths to calculate $\Delta \hat{H}$ when direct calculation is difficult.
\end{conceptbox}

\begin{keybox}
\textbf{Hypothetical Path Strategy:}
Break complex processes into simpler steps:
\begin{enumerate}[itemsep=0pt]
    \item Sensible heat change to saturation point (boiling/condensation)
    \item Latent heat change for phase transition at constant temperature
    \item Sensible heat change to final temperature
\end{enumerate}

\textbf{Total Change:} $\Delta \hat{H}_{total} = \Delta \hat{H}_1 + \Delta \hat{H}_2 + \Delta \hat{H}_3 + \cdots$
\end{keybox}

\subsection*{Calculating Enthalpy Changes}

\begin{formulabox}[title=Enthalpy Change Calculations]
\textbf{Sensible Heat (Temperature Change):}
$$ \Delta \hat{H} = \int_{T_1}^{T_2} C_p(T) \,dT $$

where $C_p$ is often given as: $C_p = A + BT + CT^2 + DT^3$

\textbf{Latent Heat (Phase Change):}
$$ \Delta \hat{H}_{\text{vaporization}} = -\Delta \hat{H}_{\text{condensation}} $$
\end{formulabox}

\begin{examplebox}{Energy Balance on Acetone Condenser}
\textbf{Problem:} Calculate cooling duty ($\dot{Q}$) to condense and cool 100 mol/s acetone from vapor at 100°C to liquid at 25°C at 1 atm.

\textbf{Given:}
\begin{itemize}[itemsep=0pt]
    \item $\Delta \hat{H}_{vap}$ = 30.2 kJ/mol at normal boiling point (56°C)
    \item Acetone vapor: $C_p = 0.07196 + 20.10 \times 10^{-5}T - 12.78 \times 10^{-8}T^2 + 34.76 \times 10^{-12}T^3$
    \item Acetone liquid: $C_p = 0.123 + 18.6 \times 10^{-5}T$
    \item ($C_p$ in kJ/mol·°C, $T$ in °C)
\end{itemize}
\end{examplebox}

\begin{conceptbox}[title=Solution Strategy]
Construct three-step hypothetical path from initial state (vapor, 100°C) to final state (liquid, 25°C). Use energy balance: $\dot{Q} = \Delta \dot{H} = \dot{n} \Delta \hat{H}_{total}$
\end{conceptbox}

\begin{keybox}
\textbf{Hypothetical Path Definition:}
\begin{enumerate}[itemsep=0pt]
    \item $\Delta \hat{H}_1$: Cool acetone vapor 100°C → 56°C (sensible)
    \item $\Delta \hat{H}_2$: Condense vapor → liquid at 56°C (latent)
    \item $\Delta \hat{H}_3$: Cool liquid 56°C → 25°C (sensible)
\end{enumerate}
\end{keybox}

\begin{stepbox}
\begin{enumerate}[label=\textbf{Step \arabic*:}, wide=0pt, leftmargin=*, itemsep=2pt]
    \item \textbf{Calculate $\Delta \hat{H}_1$ (Vapor Cooling)}
    
    $\Delta \hat{H}_1 = \int_{100}^{56} C_{p,vapor}(T) \,dT$
    
    $= \int_{100}^{56} (0.07196 + 20.10 \times 10^{-5}T - 12.78 \times 10^{-8}T^2 + 34.76 \times 10^{-12}T^3) \,dT$
    
    Evaluating the integral: $\Delta \hat{H}_1 = -3.82$ kJ/mol
    
    \item \textbf{Calculate $\Delta \hat{H}_2$ (Condensation)}
    
    Condensation is opposite of vaporization:
    
    $\Delta \hat{H}_2 = -\Delta \hat{H}_{vap} = -30.2$ kJ/mol
    
    \item \textbf{Calculate $\Delta \hat{H}_3$ (Liquid Cooling)}
    
    $\Delta \hat{H}_3 = \int_{56}^{25} C_{p,liquid}(T) \,dT$
    
    $= \int_{56}^{25} (0.123 + 18.6 \times 10^{-5}T) \,dT$
    
    Evaluating the integral: $\Delta \hat{H}_3 = -4.06$ kJ/mol
    
    \item \textbf{Calculate Total Cooling Duty}
    
    $\Delta \hat{H}_{total} = (-3.82) + (-30.2) + (-4.06) = -38.08$ kJ/mol
    
    $\dot{Q} = \dot{n} \Delta \hat{H}_{total} = (100)(-38.08) = -3808$ kJ/s
\end{enumerate}
\end{stepbox}

\begin{formulabox}[title=Final Answer]
Required cooling duty: \textbf{3808 kW}

The negative sign confirms heat is being removed from the system.
\end{formulabox}

\begin{examplebox}{Steam Heat Exchanger Analysis}
\textbf{Problem:} Superheated steam at 10 bar and 200°C flows at 2 kg/s into heat exchanger. Steam transfers 450 kW heat to preheat reactor feed. Pressure drop is negligible. Determine enthalpy, temperature, and phase of exiting steam.
\end{examplebox}

\begin{conceptbox}[title=Solution Strategy]
Use steam tables for direct enthalpy lookup. This simplifies calculations significantly compared to integration methods since extensive thermodynamic data is readily available for water.
\end{conceptbox}

\begin{stepbox}
\begin{enumerate}[label=\textbf{Step \arabic*:}, wide=0pt, leftmargin=*, itemsep=2pt]
    \item \textbf{Set Up Energy Balance}
    
    For steady-state open system with no work and negligible kinetic/potential energy:
    
    $\dot{Q} = \dot{m}(\hat{h}_{out} - \hat{h}_{in})$
    
    Heat transferred \textit{from} steam: $\dot{Q} = -450$ kW
    
    $-450 = (2)(\hat{h}_{out} - \hat{h}_{in})$
    
    \item \textbf{Find Inlet Specific Enthalpy}
    
    From superheated steam tables at 10 bar and 200°C:
    
    $\hat{h}_{in} = 2828$ kJ/kg
    
    \item \textbf{Calculate Outlet Specific Enthalpy}
    
    $\hat{h}_{out} = \hat{h}_{in} + \frac{\dot{Q}}{\dot{m}} = 2828 + \frac{-450}{2} = 2828 - 225 = 2603$ kJ/kg
    
    \item \textbf{Determine Outlet Phase and Temperature}
    
    Compare $\hat{h}_{out}$ to saturation properties at 10 bar:
    
    $\hat{h}_f = 762.5$ kJ/kg (saturated liquid)
    
    $\hat{h}_g = 2777.1$ kJ/kg (saturated vapor)
    
    Since $762.5 < 2603 < 2777.1$: \textbf{Two-phase mixture}
    
    Temperature = saturation temperature at 10 bar = \textbf{179.9°C}
    
    \item \textbf{Calculate Quality (Vapor Mass Fraction)}
    
    For two-phase mixture: $\hat{h}_{out} = x\hat{h}_g + (1-x)\hat{h}_f$
    
    $2603 = x(2777.1) + (1-x)(762.5)$
    
    $2603 = 2777.1x + 762.5 - 762.5x = 2014.6x + 762.5$
    
    $x = \frac{2603 - 762.5}{2014.6} = 0.914 = 91.4\%$
\end{enumerate}
\end{stepbox}

\begin{formulabox}[title=Final Results]
\textbf{Exiting Steam Conditions:}
\begin{itemize}[itemsep=0pt]
    \item Specific enthalpy: \textbf{2603 kJ/kg}
    \item Phase: \textbf{Two-phase mixture (wet steam)}
    \item Temperature: \textbf{179.9°C}
    \item Quality: \textbf{91.4\% vapor}
\end{itemize}
\end{formulabox}

\begin{conceptbox}[title=Key Phase Change Energy Balance Principles]
\textbf{Problem-Solving Guidelines:}
\begin{itemize}[itemsep=0pt]
    \item Use state function property to construct convenient calculation paths
    \item Separate sensible heat (temperature change) from latent heat (phase change)
    \item For complex heat capacity functions, integrate carefully over temperature ranges
    \item Use steam tables when available for water/steam systems
    \item Always check phase boundaries using saturation properties
    \item Verify energy balance closure and sign conventions
\end{itemize}

\textbf{Sign Conventions:}
\begin{itemize}[itemsep=0pt]
    \item Heat added to system: positive (+)
    \item Heat removed from system: negative (-)
    \item Condensation: negative enthalpy change
    \item Vaporization: positive enthalpy change
\end{itemize}
\end{conceptbox}

\newpage

\section*{Energy Balances on Reactive Systems}

When a chemical reaction occurs, energy is either released (exothermic reaction) or consumed (endothermic reaction). To determine the overall heating or cooling requirement for a reactor ($\dot{Q}$), we must account for both enthalpy changes due to temperature differences (sensible heat) and the enthalpy change from the reaction itself.

\begin{conceptbox}[title=Reactive System Energy Balance Principle]
Reactive systems require accounting for two distinct energy effects:
\begin{enumerate}[itemsep=0pt]
    \item Energy change from the chemical reaction itself
    \item Sensible heat changes due to temperature differences between inlet and outlet streams
\end{enumerate}
The total energy requirement is the sum of these effects.
\end{conceptbox}

\subsection*{The Heat of Reaction Method}

\begin{conceptbox}[title=Heat of Reaction Method Strategy]
Separate the overall enthalpy change into two distinct parts:
\begin{enumerate}[itemsep=0pt]
    \item Enthalpy change from chemical reaction at standard reference state
    \item Sensible heat changes to bring reactants from inlet to reference state, and products from reference state to outlet conditions
\end{enumerate}
\end{conceptbox}

\begin{formulabox}[title=Energy Balance for Steady-State Reactor]
$$ \dot{Q} = \Delta \dot{H} = \dot{\xi} \Delta \hat{H}_{rxn}^\circ + \sum_{\text{out}} \dot{n}_i \hat{h}_i - \sum_{\text{in}} \dot{n}_i \hat{h}_i $$

\textbf{Reference State:} Molecular species at 25°C and 1 atm
\end{formulabox}

\begin{keybox}
\textbf{Variable Definitions:}
\begin{itemize}[itemsep=0pt]
    \item $\dot{Q}$: Rate of heat transfer to reactor (+) or from reactor (-)
    \item $\dot{\xi}$: Extent of reaction (mol/time) from material balance
    \item $\Delta \hat{H}_{rxn}^\circ$: Standard heat of reaction (kJ/mol) at reference state
    \item $\dot{n}_i$: Molar flow rate of species $i$ in inlet/outlet streams
    \item $\hat{h}_i$: Specific enthalpy of species $i$ relative to reference state
\end{itemize}
\end{keybox}

\subsection*{Calculating Standard Heat of Reaction}

\begin{formulabox}[title=Standard Heat of Reaction Calculation]
$$ \Delta \hat{H}_{rxn}^\circ = \sum_{\text{products}} |\nu_i| \Delta \hat{H}_{f,i}^\circ - \sum_{\text{reactants}} |\nu_i| \Delta \hat{H}_{f,i}^\circ $$

where:
\begin{itemize}[itemsep=0pt]
    \item $\nu_i$: Stoichiometric coefficient of species $i$
    \item $\Delta \hat{H}_{f,i}^\circ$: Standard heat of formation of species $i$
    \item $\Delta \hat{H}_{f}^\circ = 0$ for elemental species ($O_2$, $H_2$, $N_2$, C(graphite))
\end{itemize}
\end{formulabox}

\begin{conceptbox}[title=Standard Heat of Formation Definition]
The standard heat of formation ($\Delta \hat{H}_{f}^\circ$) is the enthalpy change when one mole of a substance is formed from its constituent elements in their standard state at 25°C and 1 atm.
\end{conceptbox}

\newpage

\begin{examplebox}{Water-Gas Shift Reactor Energy Balance}
\textbf{Problem:} In the water-gas shift reaction, CO reacts with steam to produce $CO_2$ and $H_2$. Feed enters at 300°C, product exits at 500°C. Feed: 1.0 mol CO with 50\% excess steam. CO conversion: 80\%. Determine reactor heat duty ($\dot{Q}$).

\textbf{Reaction:} $CO(g) + H_2O(g) \rightarrow CO_2(g) + H_2(g)$
\end{examplebox}

\begin{stepbox}
\begin{enumerate}[label=\textbf{Step \arabic*:}, wide=0pt, leftmargin=*, itemsep=2pt]
    \item \textbf{Material Balance - Find Extent of Reaction}
    
    \textbf{Feed composition:}
    - CO: 1.0 mol
    - $H_2O$: Stoichiometric (1.0 mol) + 50\% excess = 1.5 mol
    
    \textbf{CO conversion:} $1.0 \times 0.80 = 0.8$ mol reacted
    
    \textbf{Extent of reaction:} $\xi = 0.8$ mol (stoichiometric coefficient of CO = -1)
    
    \item \textbf{Calculate Outlet Molar Flow Rates}
    
    Using $\dot{n}_{out} = \dot{n}_{in} + \nu\xi$:
    
    $\dot{n}_{CO} = 1.0 - (1)(0.8) = 0.2$ mol
    
    $\dot{n}_{H_2O} = 1.5 - (1)(0.8) = 0.7$ mol
    
    $\dot{n}_{CO_2} = 0 + (1)(0.8) = 0.8$ mol
    
    $\dot{n}_{H_2} = 0 + (1)(0.8) = 0.8$ mol
    
    \item \textbf{Compile Enthalpy Data}
    
    Reference: Molecular species at 25°C, 1 atm. Specific enthalpies from ideal gas tables:
    
    \begin{center}
    \begin{tabular}{|l|c|c|c|c|}
    \hline
    \textbf{Species} & $\dot{n}_{in}$ (mol) & $\hat{h}_{in}$ (kJ/mol) & $\dot{n}_{out}$ (mol) & $\hat{h}_{out}$ (kJ/mol) \\
    & & at 300°C & & at 500°C \\
    \hline
    CO & 1.0 & 8.17 & 0.2 & 14.38 \\
    $H_2O$ & 1.5 & 9.57 & 0.7 & 17.01 \\
    $CO_2$ & 0 & - & 0.8 & 21.34 \\
    $H_2$ & 0 & - & 0.8 & 13.82 \\
    \hline
    \end{tabular}
    \end{center}
\end{enumerate}
\end{stepbox}

\begin{stepbox}
\begin{enumerate}[label=\textbf{Step \arabic*:}, wide=0pt, leftmargin=*, itemsep=2pt]
    \setcounter{enumi}{3}
    \item \textbf{Calculate Standard Heat of Reaction}
    
    \textbf{Standard heats of formation (kJ/mol):}
    - $\Delta \hat{H}_{f,CO}^\circ = -110.53$
    - $\Delta \hat{H}_{f,H_2O(g)}^\circ = -241.83$
    - $\Delta \hat{H}_{f,CO_2}^\circ = -393.5$
    - $\Delta \hat{H}_{f,H_2}^\circ = 0$
    
    $\Delta \hat{H}_{rxn}^\circ = [\Delta \hat{H}_{f,CO_2}^\circ + \Delta \hat{H}_{f,H_2}^\circ] - [\Delta \hat{H}_{f,CO}^\circ + \Delta \hat{H}_{f,H_2O}^\circ]$
    
    $= [-393.5 + 0] - [-110.53 + (-241.83)]$
    
    $= -393.5 - (-352.36) = -41.14$ kJ/mol
    
    \item \textbf{Calculate Sensible Heat Terms}
    
    $\sum_{\text{in}} \dot{n}_i \hat{h}_i = (1.0)(8.17) + (1.5)(9.57) = 8.17 + 14.36 = 22.53$ kJ
    
    $\sum_{\text{out}} \dot{n}_i \hat{h}_i = (0.2)(14.38) + (0.7)(17.01) + (0.8)(21.34) + (0.8)(13.82)$
    
    $= 2.88 + 11.91 + 17.07 + 11.06 = 42.92$ kJ
    
    \item \textbf{Calculate Total Heat Duty}
    
    $Q = \xi \Delta \hat{H}_{rxn}^\circ + \sum_{\text{out}} \dot{n}_i \hat{h}_i - \sum_{\text{in}} \dot{n}_i \hat{h}_i$
    
    $= (0.8)(-41.14) + (42.92) - (22.53)$
    
    $= -32.91 + 20.39 = -12.52$ kJ
\end{enumerate}
\end{stepbox}

\begin{formulabox}[title=Final Answer]
\textbf{Reactor Heat Duty:} -12.5 kJ

The negative sign indicates the reaction is exothermic - this amount of heat must be removed to maintain the specified outlet temperature.
\end{formulabox}

\begin{conceptbox}[title=Key Reactive System Energy Balance Principles]
\textbf{Problem-Solving Strategy:}
\begin{enumerate}[itemsep=0pt]
    \item Complete material balance first to find extent of reaction
    \item Use standard reference state (25°C, 1 atm) for all enthalpy calculations
    \item Calculate standard heat of reaction from heats of formation
    \item Account for sensible heat changes separately from reaction enthalpy
    \item Apply proper sign conventions: exothermic reactions have negative $\Delta \hat{H}_{rxn}^\circ$
\end{enumerate}

\textbf{Sign Conventions:}
\begin{itemize}[itemsep=0pt]
    \item Heat added to reactor: positive (+)
    \item Heat removed from reactor: negative (-)
    \item Exothermic reaction: negative $\Delta \hat{H}_{rxn}^\circ$
    \item Endothermic reaction: positive $\Delta \hat{H}_{rxn}^\circ$
\end{itemize}

\textbf{Critical Data Sources:}
\begin{itemize}[itemsep=0pt]
    \item Standard heats of formation from thermodynamic tables
    \item Specific enthalpies from ideal gas property tables
    \item Reference state consistency is essential for accuracy
\end{itemize}
\end{conceptbox}

\newpage

\section*{Energy Balances on Reactive Systems (Continued)}

In the previous section, we introduced the Heat of Reaction method. Here, we explore an alternative approach - the Heat of Formation method - which leads to identical results through a different conceptual path. We will also learn how to calculate heats of reaction at temperatures other than the standard 25°C.

\subsection*{The Heat of Formation Method}

\begin{conceptbox}[title=Heat of Formation Method Principle]
This method calculates total enthalpy change by determining the absolute enthalpy of each component in inlet and outlet streams relative to a common baseline. The energy balance becomes a simple subtraction of total outlet and inlet enthalpies.
\end{conceptbox}

\begin{keybox}
\textbf{Reference State Definition:}
Constituent elements of each species in their standard form at 25°C and 1 atm:
\begin{itemize}[itemsep=0pt]
    \item Carbon: C(s) - solid graphite
    \item Hydrogen: $H_2$(g) - gaseous hydrogen
    \item Oxygen: $O_2$(g) - gaseous oxygen
    \item Nitrogen: $N_2$(g) - gaseous nitrogen
\end{itemize}
\end{keybox}

\begin{formulabox}[title=Absolute Enthalpy Calculation]
$$ \hat{H}_i(T) = \Delta \hat{H}_{f,i}^\circ + \Delta \hat{h}_i = \Delta \hat{H}_{f,i}^\circ + \int_{25 C}^{T} C_p(T) \,dT $$

\textbf{Simplified Energy Balance:}
$$ \dot{Q} = \Delta \dot{H} = \sum_{\text{out}} \dot{n}_i \hat{H}_i - \sum_{\text{in}} \dot{n}_i \hat{H}_i $$

The reaction term is implicitly included in the absolute enthalpies.
\end{formulabox}

\begin{examplebox}{Water-Gas Shift Reactor (Heat of Formation Method)}
\textbf{Problem:} Re-solve the previous water-gas shift example using Heat of Formation method to demonstrate equivalency. CO and $H_2O$ enter at 300°C, products leave at 500°C. Feed: 1.0 mol CO, 1.5 mol $H_2O$. CO conversion: 80%.

\textbf{Reaction:} $CO(g) + H_2O(g) \rightarrow CO_2(g) + H_2(g)$
\end{examplebox}

\begin{stepbox}
\begin{enumerate}[label=\textbf{Step \arabic*:}, wide=0pt, leftmargin=*, itemsep=2pt]
    \item \textbf{Material Balance (Identical to Previous Example)}
    
    \textbf{Inlet streams:} 1.0 mol CO, 1.5 mol $H_2O$
    
    \textbf{Outlet streams:} 0.2 mol CO, 0.7 mol $H_2O$, 0.8 mol $CO_2$, 0.8 mol $H_2$
    
    \item \textbf{Calculate Absolute Enthalpies}
    
    Reference: C(s), $H_2$(g), $O_2$(g) at 25°C, 1 atm
    
    Formula: $\hat{H} = \Delta \hat{H}_{f}^\circ + \Delta \hat{h}$
    
    \begin{center}
    \begin{tabular}{|l|c|c|c|c|c|}
    \hline
    \textbf{Species} & $\Delta \hat{H}_{f}^\circ$ & $\Delta \hat{h}_{in}$ & $\hat{H}_{in}$ & $\Delta \hat{h}_{out}$ & $\hat{H}_{out}$ \\
     & (kJ/mol) & (300°C) & (total) & (500°C) & (total) \\
     & & (kJ/mol) & (kJ/mol) & (kJ/mol) & (kJ/mol) \\
    \hline
    CO & -110.53 & 8.17 & -102.36 & 14.38 & -96.15 \\
    $H_2O$ & -241.83 & 9.57 & -232.26 & 17.01 & -224.82 \\
    $CO_2$ & -393.5 & - & - & 21.34 & -372.16 \\
    $H_2$ & 0 & - & - & 13.82 & 13.82 \\
    \hline
    \end{tabular}
    \end{center}
    
    \item \textbf{Calculate Inlet Enthalpy Sum}
    
    $\sum_{\text{in}} \dot{n}_i \hat{H}_i = (1.0)(-102.36) + (1.5)(-232.26)$
    
    $= -102.36 - 348.39 = -450.75$ kJ
    
    \item \textbf{Calculate Outlet Enthalpy Sum}
    
    $\sum_{\text{out}} \dot{n}_i \hat{H}_i = (0.2)(-96.15) + (0.7)(-224.82) + (0.8)(-372.16) + (0.8)(13.82)$
    
    $= -19.23 - 157.37 - 297.73 + 11.06 = -463.27$ kJ
    
    \item \textbf{Calculate Heat Duty}
    
    $Q = \sum_{\text{out}} \dot{n}_i \hat{H}_i - \sum_{\text{in}} \dot{n}_i \hat{H}_i$
    
    $= (-463.27) - (-450.75) = -12.52$ kJ
\end{enumerate}
\end{stepbox}

\begin{formulabox}[title=Method Comparison Result]
\textbf{Heat Duty:} -12.5 kJ

This is identical to the Heat of Reaction method result, demonstrating that both approaches are mathematically equivalent.
\end{formulabox}

\subsection*{Effect of Temperature on Heat of Reaction}

\begin{conceptbox}[title=Temperature-Dependent Heat of Reaction]
The standard heat of reaction ($\Delta \hat{H}_{rxn}^\circ$) is defined at 25°C. To find the heat of reaction at different temperatures, we use a hypothetical path based on Hess's Law.
\end{conceptbox}

\begin{formulabox}[title=Temperature Correction for Heat of Reaction]
$$ \Delta \hat{H}_{rxn}(T) = \Delta \hat{H}_{rxn}^\circ (298K) + \int_{298K}^{T} \Delta C_p \,dT $$

where $\Delta C_p$ is the change in heat capacity for the reaction:
$$ \Delta C_p = \sum_{\text{products}} |\nu_i| C_{p,i} - \sum_{\text{reactants}} |\nu_i| C_{p,i} $$
\end{formulabox}

\begin{examplebox}{Heat of Reaction at Elevated Temperature}
\textbf{Problem:} Calculate the heat of reaction at 600 K for hydrazine decomposition.

\textbf{Reaction:} $N_2H_4(g) \rightarrow N_2(g) + 2H_2(g)$

\textbf{Given:} $\Delta \hat{H}_{f,N_2H_4}^\circ = +95.35$ kJ/mol

Heat capacity: $C_P = A + BT + CT^2 + DT^3$ (J/mol·K)
\end{examplebox}

\begin{stepbox}
\begin{enumerate}[label=\textbf{Step \arabic*:}, wide=0pt, leftmargin=*, itemsep=2pt]
    \item \textbf{Calculate Standard Heat of Reaction at 298 K}
    
    $\Delta \hat{H}_{rxn}^\circ = [\Delta \hat{H}_{f,N_2}^\circ + 2\Delta \hat{H}_{f,H_2}^\circ] - [\Delta \hat{H}_{f,N_2H_4}^\circ]$
    
    Since $N_2$ and $H_2$ are elements in standard state: $\Delta \hat{H}_{f}^\circ = 0$
    
    $\Delta \hat{H}_{rxn}^\circ = [0 + 2(0)] - [95.35] = -95.35$ kJ/mol
    
    \item \textbf{Calculate $\Delta C_p$ as Function of Temperature}
    
    $\Delta C_p = [C_{p,N_2} + 2C_{p,H_2}] - [C_{p,N_2H_4}]$
    
    Substitute heat capacity polynomials for each species to get $\Delta C_p(T)$
    
    \item \textbf{Integrate $\Delta C_p$ for Temperature Correction}
    
    $\int_{298K}^{600K} \Delta C_p \,dT$
    
    Using spreadsheet integration or numerical methods:
    
    $\int_{298K}^{600K} \Delta C_p \,dT \approx +6.65$ kJ/mol
    
    \item \textbf{Calculate Heat of Reaction at 600 K}
    
    $\Delta \hat{H}_{rxn}(600K) = \Delta \hat{H}_{rxn}^\circ (298K) + \int_{298K}^{600K} \Delta C_p \,dT$
    
    $= -95.35 + 6.65 = -88.7$ kJ/mol
\end{enumerate}
\end{stepbox}

\begin{formulabox}[title=Temperature Effect Result]
\textbf{Heat of Reaction at 600 K:} -88.7 kJ/mol

The reaction is slightly less exothermic at higher temperature due to the positive temperature correction.
\end{formulabox}

\begin{conceptbox}[title=Key Reactive System Energy Balance Methods]
\textbf{Method Comparison:}

\textbf{Heat of Reaction Method:}
\begin{itemize}[itemsep=0pt]
    \item Uses standard heat of reaction plus sensible heat corrections
    \item Reference: molecular species at 25°C, 1 atm
    \item Separates reaction and temperature effects explicitly
\end{itemize}

\textbf{Heat of Formation Method:}
\begin{itemize}[itemsep=0pt]
    \item Uses absolute enthalpies referenced to constituent elements
    \item Reference: elements in standard state at 25°C, 1 atm
    \item Reaction effects included implicitly in formation enthalpies
\end{itemize}

\textbf{Temperature Effects:}
\begin{itemize}[itemsep=0pt]
    \item Heat of reaction varies with temperature due to different heat capacities
    \item Use Kirchhoff's equation with $\Delta C_p$ integration
    \item Generally small corrections unless temperature differences are large
\end{itemize}

Both methods yield identical results when applied correctly.
\end{conceptbox}

