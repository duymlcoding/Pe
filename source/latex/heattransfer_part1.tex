\documentclass[12pt]{article}
\usepackage[paperwidth=8.5in, paperheight=11in, margin=1.0in, headheight=15pt]{geometry}
\usepackage{amsmath,amssymb,amsthm}
\usepackage[most]{tcolorbox}
\usepackage{enumitem}
\usepackage{xcolor}
\usepackage{hyperref}
\usepackage{fancyhdr}
\usepackage{titlesec}
\usepackage{graphicx}
% Define custom colors for chemical engineering theme
\definecolor{conceptcolor}{RGB}{52, 73, 94}      % Dark blue-gray
\definecolor{formulacolor}{RGB}{231, 76, 60}     % Red for formulas
\definecolor{examplecolor}{RGB}{39, 174, 96}     % Green for examples
\definecolor{stepcolor}{RGB}{142, 68, 173}       % Purple for solution steps
\definecolor{keycolor}{RGB}{243, 156, 18}        % Orange for key points
% Configure fancy headers
\pagestyle{fancy}
\fancyhf{}
\fancyhead[L]{PE Study Guide}
\fancyhead[R]{Process Fundamentals}
\fancyfoot[C]{\thepage}
\renewcommand{\baselinestretch}{1.1}
\setlength{\parindent}{0.25in}
\setlength{\parskip}{3pt}
% Configure section formatting
\titleformat{\section}
  {\normalfont\LARGE\bfseries\color{conceptcolor}}
  {\thesection}{1em}{}
\titleformat{\subsection}
  {\normalfont\Large\bfseries\color{conceptcolor}}
  {\thesubsection}{1em}{}
% Define custom environments
\newtcolorbox{conceptbox}[1][]{
  enhanced,
  colback=conceptcolor!10,
  colframe=conceptcolor,
  arc=3mm,
  title=Key Concept,
  fonttitle=\bfseries\sffamily\normalsize,
  fontupper=\small,
  #1
}
\newtcolorbox{formulabox}[1][]{
  enhanced,
  colback=formulacolor!10,
  colframe=formulacolor,
  arc=2mm,
  title=Important Formula,
  fonttitle=\bfseries\sffamily\normalsize,
  fontupper=\small,
  #1
}
\newtcolorbox{examplebox}[2][]{
  enhanced,
  colback=examplecolor!10,
  colframe=examplecolor,
  arc=3mm,
  title=Example Problem: #2,
  fonttitle=\bfseries\sffamily\normalsize,
  fontupper=\small,
  #1
}
\newtcolorbox{stepbox}[1][]{
  enhanced,
  colback=stepcolor!10,
  colframe=stepcolor,
  arc=2mm,
  title=Solution Steps,
  fonttitle=\bfseries\sffamily\normalsize,
  fontupper=\small,
  #1
}
\newtcolorbox{keybox}[1][]{
  enhanced,
  colback=keycolor!10,
  colframe=keycolor,
  arc=2mm,
  title=Key Variables \& Definitions,
  fonttitle=\bfseries\sffamily\normalsize,
  fontupper=\small,
  #1
}
\begin{document}

\begin{center}
    \Huge\textbf{\color{conceptcolor}Heat Transfer}
\end{center}
\hrule

\section*{Heat Transfer by Conduction: The Plane Wall}
This guide covers the principles of one-dimensional, steady-state heat conduction through a simple plane wall. We will derive the temperature profile and the resulting heat transfer rate from the fundamental heat diffusion equation.

\subsection*{Fundamental Concept: Fourier's Law of Heat Conduction}
\begin{conceptbox}
Conduction is the transfer of thermal energy through a medium by the direct interaction of its constituent particles. The rate of this transfer is governed by Fourier's Law, which states that the heat transfer rate is proportional to the area normal to the heat flow and the temperature gradient in that direction.
\end{conceptbox}

\begin{keybox}[title=Key Variables for Conduction]
\begin{itemize}[itemsep=0pt]
    \item \textbf{$q$}: The rate of heat transfer [W].
    \item \textbf{$q''$}: The heat flux, or heat transfer rate per unit area ($q/A$) [W/m$^2$].
    \item \textbf{$k$}: The thermal conductivity of the material, a material property [W/m$\cdot$K].
    \item \textbf{$A$}: The cross-sectional area perpendicular to the direction of heat flow [m$^2$].
    \item \textbf{$\frac{dT}{dx}$}: The temperature gradient in the direction of heat flow [K/m]. The negative sign in Fourier's Law signifies that heat flows down the temperature gradient (from hot to cold).
\end{itemize}
\end{keybox}

\begin{formulabox}
Fourier's Law of Heat Conduction is expressed as:
$$ q = -k A \frac{dT}{dx} $$
\end{formulabox}

\newpage

\subsection*{Derivation of the Temperature Profile for a Plane Wall}
The following steps show how to derive the specific equations for a plane wall by simplifying the general heat diffusion equation.

\begin{stepbox}[title=Step 1: Simplify the General Heat Diffusion Equation]
The general heat diffusion equation in Cartesian coordinates is:
$$ \frac{\partial}{\partial x}\left(k \frac{\partial T}{\partial x}\right) + \frac{\partial}{\partial y}\left(k \frac{\partial T}{\partial y}\right) + \frac{\partial}{\partial z}\left(k \frac{\partial T}{\partial z}\right) + \dot{q} = \rho c_p \frac{\partial T}{\partial t} $$
We apply a set of common simplifying assumptions:
\begin{itemize}[itemsep=2pt]
    \item \textbf{Steady-State}: Temperature does not change with time ($\frac{\partial T}{\partial t} = 0$).
    \item \textbf{One-Dimensional (1-D)}: Temperature only varies in the x-direction, so the y and z derivatives are zero.
    \item \textbf{No Heat Generation}: $\dot{q} = 0$.
    \item \textbf{Constant Thermal Conductivity}: $k$ is not a function of temperature.
\end{itemize}
Applying these assumptions, the general equation reduces to a simple ordinary differential equation:
$$ k \frac{d^2T}{dx^2} = 0 \quad \implies \quad \frac{d^2T}{dx^2} = 0 $$
\end{stepbox}

\begin{stepbox}[title=Step 2: Integrate to Find the Temperature Distribution]
We integrate the simplified equation twice with respect to position ($x$).
\begin{itemize}[itemsep=2pt]
    \item \textbf{First Integration}: This gives the temperature gradient.
    $$ \int \frac{d^2T}{dx^2} dx = \int 0 \,dx \quad \implies \quad \frac{dT}{dx} = C_1 $$
    \item \textbf{Second Integration}: This gives the general form of the temperature profile.
    $$ \int \frac{dT}{dx} dx = \int C_1 dx \quad \implies \quad T(x) = C_1 x + C_2 $$
\end{itemize}
This result shows that for these conditions, the temperature profile through the plane wall is linear. $C_1$ and $C_2$ are constants of integration.
\end{stepbox}

\begin{stepbox}[title=Step 3: Apply Boundary Conditions to Find the Constants]
To find the specific values of $C_1$ and $C_2$, we apply known boundary conditions. For a plane wall of thickness $L$, we assume the surface temperatures are known.
\begin{itemize}[itemsep=2pt]
    \item \textbf{Boundary Condition 1 (BC1)}: At $x = 0$, the temperature is $T(0) = T_{s1}$.
    $$ T(0) = C_1(0) + C_2 = T_{s1} \implies C_2 = T_{s1} $$
    \item \textbf{Boundary Condition 2 (BC2)}: At $x = L$, the temperature is $T(L) = T_{s2}$.
    $$ T(L) = C_1(L) + C_2 = T_{s2} $$
    Substitute $C_2 = T_{s1}$ and solve for $C_1$:
    $$ C_1 L + T_{s1} = T_{s2} \implies C_1 = \frac{T_{s2} - T_{s1}}{L} $$
\end{itemize}
\end{stepbox}

\begin{stepbox}[title=Step 4: Determine the Final Equations]
Substituting the expressions for $C_1$ and $C_2$ back into our general equations gives the final, usable forms.
\begin{formulabox}[title=Final Equations for a Plane Wall]
\begin{itemize}[itemsep=2pt]
    \item \textbf{Temperature Profile}:
    $$ T(x) = \left(\frac{T_{s2} - T_{s1}}{L}\right)x + T_{s1} $$
    \item \textbf{Heat Transfer Rate ($q$)}: Substitute the temperature gradient ($\frac{dT}{dx} = C_1$) into Fourier's Law.
    $$ q = -k A \left(\frac{T_{s2} - T_{s1}}{L}\right) = k A \frac{T_{s1} - T_{s2}}{L} $$
    \item \textbf{Heat Flux ($q''$)}: Heat rate per unit area.
    $$ q'' = \frac{q}{A} = k \frac{T_{s1} - T_{s2}}{L} $$
\end{itemize}
\end{formulabox}
\end{stepbox}

\newpage
\section*{Heat Transfer by Convection: Solving Problems}

\subsection*{Fundamental Concept: Newton's Law of Cooling}
\begin{conceptbox}
Convection is heat transfer between a solid surface and an adjacent moving fluid. The process is governed by Newton's Law of Cooling, which states that the rate of heat transfer is proportional to the surface area and the difference between the surface temperature and the fluid temperature.
\end{conceptbox}
\begin{keybox}[title=Key Variables for Convection]
\begin{itemize}[itemsep=0pt]
    \item \textbf{$q_{conv}$}: The rate of convective heat transfer [W].
    \item \textbf{$\bar{h}$}: The average convective heat transfer coefficient [W/m$^2\cdot$K]. This is an empirical parameter, not a fluid property.
    \item \textbf{$A_s$}: The surface area over which convection occurs [m$^2$].
    \item \textbf{$T_s$}: The temperature of the solid surface.
    \item \textbf{$T_{\infty}$}: The temperature of the fluid far away from the surface (the free-stream temperature).
\end{itemize}
\end{keybox}
\begin{formulabox}
Newton's Law of Cooling is expressed as:
$$ q_{conv} = \bar{h} A_s (T_s - T_{\infty}) $$
The primary challenge in convection problems is finding the correct value for the coefficient, $h$.
\end{formulabox}

\newpage

\subsection*{A Step-by-Step Approach to Finding the Convection Coefficient ($h$)}

\begin{stepbox}[title=Procedural Guide for Convection Calculations]
\begin{enumerate}[label=\textbf{Step \arabic*:}, wide=0pt, leftmargin=*, itemsep=2pt]
    \item \textbf{Identify the Geometry}: Is the fluid flowing over a flat plate, a cylinder, a sphere, or inside a pipe? Each geometry has its own set of correlations.
    
    \item \textbf{Determine Fluid Properties}: Fluid properties ($k_f, \mu, \rho, c_p$) must be evaluated at a specific reference temperature. Usually, this is the \textbf{film temperature}, $T_f$.
    $$ T_f = \frac{T_s + T_{\infty}}{2} $$
    Always verify the reference temperature specified by the correlation.
    
    \item \textbf{Calculate the Reynolds Number ($Re$)}: This is mandatory to determine the flow regime (laminar or turbulent).
    $$ Re_{L_c} = \frac{\rho v L_c}{\mu} = \frac{v L_c}{\nu} $$
    The characteristic length, $L_c$, depends on the geometry (e.g., diameter for a cylinder, length for a plate).
    
    \item \textbf{Select a Correlation for the Nusselt Number ($Nu$)}: Correlations solve for the dimensionless Nusselt number, not for $h$ directly. Based on the geometry (Step 1) and the Reynolds number (Step 3), select the appropriate correlation. These often take the form:
    $$ Nu = C \cdot Re^m \cdot Pr^n $$
    where $Pr$ is the Prandtl number ($Pr = c_p \mu / k_f$), another dimensionless fluid property. The constants $C, m,$ and $n$ are specific to the correlation.
    
    \item \textbf{Calculate the Convection Coefficient ($h$)}: Rearrange the definition of the Nusselt number to solve for $h$.
    $$ Nu = \frac{h L_c}{k_f} \implies h = \frac{Nu \cdot k_f}{L_c} $$
    Ensure the fluid thermal conductivity, $k_f$, is used.
    
    \item \textbf{Calculate the Heat Transfer Rate ($q_{conv}$)}: With the value of $h$ determined, use Newton's Law of Cooling to find the final answer.
    $$ q_{conv} = h A_s (T_s - T_{\infty}) $$
\end{enumerate}
\end{stepbox}

\newpage

\section*{Convective Heat Transfer over a Flat Plate}
This guide explores the fundamentals of convective heat transfer for fluid flow over a stationary flat plate. We will define the key concepts of momentum and thermal boundary layers and the dimensionless numbers that govern the process.

\subsection*{Introduction and Problem Setup}
\begin{conceptbox}
When a fluid flows over a surface with a different temperature, a complex interaction occurs. The surface's presence slows the fluid down due to friction, and heat is exchanged, altering the fluid's temperature near the surface. These effects are confined to thin regions known as boundary layers.
\end{conceptbox}
\begin{keybox}[title=Coordinate System and Key Parameters]
We analyze the flow using a 2D coordinate system and the following parameters:
\begin{itemize}[itemsep=2pt]
    \item \textbf{Coordinate System}:
    \begin{itemize}[itemsep=0pt]
        \item $x$: Direction parallel to the fluid flow, starting from the plate's leading edge ($x=0$).
        \item $y$: Direction perpendicular to the plate surface ($y=0$).
    \end{itemize}
    \item \textbf{Parameters}:
    \begin{itemize}[itemsep=0pt]
        \item $T_s$: The uniform surface temperature of the plate.
        \item $L$: The total length of the plate in the x-direction.
        \item $u_{\infty}$: The free-stream velocity of the fluid, far from the plate's influence.
        \item $T_{\infty}$: The free-stream temperature of the fluid, far from the plate's influence.
    \end{itemize}
\end{itemize}
\end{keybox}

\subsection*{Boundary Layer Concepts}
\begin{conceptbox}[title=The Momentum Boundary Layer ($\delta$)]
This is the region near the plate where the fluid's velocity is affected by viscous forces. Due to the no-slip condition, the fluid velocity is zero at the surface ($y=0$) and gradually increases to the free-stream velocity. The thickness of this layer, $\delta(x)$, grows as the fluid moves down the plate.
\end{conceptbox}
\begin{formulabox}
The formal definition of the momentum boundary layer thickness, $\delta$, is the distance $y$ from the surface where the fluid velocity, $u$, reaches 99\% of the free-stream velocity, $u_{\infty}$.
$$ u(y=\delta) = 0.99 \cdot u_{\infty} $$
\end{formulabox}

\begin{conceptbox}[title=The Thermal Boundary Layer ($\delta_t$)]
This is the region near the plate where the fluid's temperature is affected by heat transfer with the surface. The fluid temperature varies from $T_s$ at the surface to $T_{\infty}$ in the free stream. The thickness of this layer, $\delta_t(x)$, also grows along the plate.
\end{conceptbox}
\begin{formulabox}
The formal definition of the thermal boundary layer thickness, $\delta_t$, is the distance $y$ from the surface where the fluid's temperature change is 99\% of the total temperature difference between the surface and the free stream.
$$ \frac{T(y=\delta_t) - T_s}{T_{\infty} - T_s} = 0.99 $$
\end{formulabox}

\subsection*{Key Dimensionless Numbers for Flat Plate Flow}
\begin{conceptbox}[title=The Reynolds Number ($Re$)]
The Reynolds number is the ratio of inertial forces to viscous forces. It is the most important parameter for determining the flow regime. For flow over a flat plate, the \textit{local} Reynolds number, $Re_x$, increases with distance from the leading edge.
\end{conceptbox}
\begin{formulabox}
The local Reynolds number at a position $x$ is:
$$ Re_x = \frac{\rho u_{\infty} x}{\mu} = \frac{u_{\infty} x}{\nu} $$
where $\rho$ is density, $\mu$ is dynamic viscosity, and $\nu$ is kinematic viscosity ($\mu/\rho$).
\end{formulabox}
\begin{keybox}
The flow transitions from laminar (smooth) to turbulent (chaotic) at a critical Reynolds number.
\begin{itemize}[itemsep=0pt]
    \item \textbf{Laminar Flow}: Typically for $Re_x \lesssim 5 \times 10^5$.
    \item \textbf{Turbulent Flow}: Typically for $Re_x \gtrsim 5 \times 10^5$.
\end{itemize}
\end{keybox}

\begin{conceptbox}[title=The Prandtl Number ($Pr$)]
The Prandtl number is a dimensionless fluid property that relates the rate of momentum diffusion to the rate of thermal diffusion. It provides a direct comparison between the thickness of the momentum and thermal boundary layers.
\end{conceptbox}
\begin{formulabox}
The Prandtl number is defined as:
$$ Pr = \frac{\text{Momentum Diffusivity}}{\text{Thermal Diffusivity}} = \frac{\nu}{\alpha} $$
where $\alpha$ is the thermal diffusivity ($k / \rho c_p$).
\end{formulabox}
\begin{keybox}[title=Interpreting the Prandtl Number]
\begin{itemize}[itemsep=2pt]
    \item If $Pr < 1$ (e.g., liquid metals, air), heat diffuses faster than momentum. The thermal boundary layer is thicker than the momentum boundary layer ($\delta_t > \delta$).
    \item If $Pr > 1$ (e.g., water, oils), momentum diffuses faster than heat. The momentum boundary layer is thicker than the thermal boundary layer ($\delta > \delta_t$).
    \item If $Pr \approx 1$, the two boundary layers have roughly the same thickness ($\delta_t \approx \delta$).
\end{itemize}
\end{keybox}

\newpage
\subsection*{Problem Solving: Heat Generation in a Pipe}
\begin{examplebox}{Volumetric Heat Generation in a Pipe Wall}
Water is heated as it flows through a thick-walled pipe. We need to determine the uniform volumetric heat generation rate ($\dot{q}$) within the pipe wall required to achieve a specific temperature rise in the water.
\begin{itemize}[itemsep=0pt]
    \item Pipe Length, $L = 10$ m
    \item Pipe Inside Diameter, $D_i = 0.015$ m; Outside Diameter, $D_o = 0.030$ m
    \item Water Mass Flow Rate, $\dot{m} = 0.3$ kg/s
    \item Water Inlet Temperature, $T_{in} = 25^\circ$C; Outlet Temperature, $T_{out} = 45^\circ$C
    \item Specific Heat of Water, $c_p = 4178$ J/kg$\cdot$K
    \item The outer surface of the pipe is perfectly insulated.
\end{itemize}
\end{examplebox}

\begin{stepbox}[title=Step 1: Calculate the Total Heat Transfer Rate to the Water ($q$)]
\begin{conceptbox}
The total heat transfer rate required is equal to the rate of enthalpy gain by the water. This can be calculated using the thermodynamic energy balance equation. Since the outer surface is insulated, all heat generated in the pipe wall is transferred to the water.
\end{conceptbox}
\begin{formulabox}
The energy balance equation is:
$$ q = \dot{m} c_p (T_{out} - T_{in}) $$
Note that a temperature difference in $^\circ$C is equal to a temperature difference in K.
$$ q = (0.3 \, \text{kg/s}) \times (4178 \, \frac{\text{J}}{\text{kg}\cdot\text{K}}) \times (45^\circ\text{C} - 25^\circ\text{C}) $$
$$ q = (0.3) \times (4178) \times (20) = 25068 \, \text{W} $$
\end{formulabox}
\end{stepbox}

\begin{stepbox}[title=Step 2: Calculate the Volume of the Pipe Wall ($V_{wall}$)]
\begin{conceptbox}
The heat generation rate, $\dot{q}$, is defined per unit volume. Therefore, we must calculate the volume of the material in which the heat is being generated (the pipe wall).
\end{conceptbox}
\begin{formulabox}
The volume of a hollow cylinder is:
$$ V_{wall} = (\text{Cross-sectional Area}) \times (\text{Length}) = \frac{\pi}{4} (D_o^2 - D_i^2) \times L $$
$$ V_{wall} = \frac{\pi}{4} ((0.030 \, \text{m})^2 - (0.015 \, \text{m})^2) \times 10 \, \text{m} $$
$$ V_{wall} = \frac{\pi}{4} (0.0009 - 0.000225) \times 10 = \frac{\pi}{4} (0.000675) \times 10 \approx 0.0053 \, \text{m}^3 $$
\end{formulabox}
\end{stepbox}

\begin{stepbox}[title=Step 3: Calculate the Required Heat Generation Rate ($\dot{q}$)]
\begin{conceptbox}
The volumetric heat generation rate is the total heat transfer rate divided by the volume over which it is generated.
\end{conceptbox}
\begin{formulabox}
The defining equation is:
$$ \dot{q} = \frac{q}{V_{wall}} $$
$$ \dot{q} = \frac{25068 \, \text{W}}{0.0053 \, \text{m}^3} \approx 4.73 \times 10^6 \, \frac{\text{W}}{\text{m}^3} $$
This is the uniform heat generation rate required to heat the water as specified.
\end{formulabox}
\end{stepbox}

\newpage
\section*{Radiation: View Factors}
\subsection*{Fundamental Concept of the View Factor}
\begin{conceptbox}
In thermal radiation, the view factor (or shape factor, configuration factor) is a purely geometric quantity that describes how well two surfaces "see" each other. It is essential for calculating the rate of radiative heat exchange between surfaces.
\end{conceptbox}

\begin{keybox}[title=Definition of the View Factor]
The view factor, $F_{ij}$, is the \textbf{fraction of radiation leaving surface $i$} that directly strikes \textbf{surface $j$}.
\begin{itemize}[itemsep=0pt]
    \item The first subscript ($i$) denotes the emitting surface.
    \item The second subscript ($j$) denotes the receiving surface.
    \item The value of a view factor ranges from 0 to 1.
\end{itemize}
\end{keybox}

\subsection*{Rules for Determining View Factors}
While view factors for many complex geometries are found using charts or integration, most problems can be solved using three fundamental rules.

\begin{formulabox}[title=1. The Summation Rule]
For any enclosure of $n$ surfaces, all radiation leaving a surface $i$ must be intercepted by the surfaces of the enclosure. Therefore, the sum of all view factors from surface $i$ must equal one.
$$ \sum_{j=1}^{n} F_{ij} = F_{i1} + F_{i2} + \dots + F_{in} = 1 $$
This includes the view factor from a surface to itself, $F_{ii}$.
\begin{itemize}[itemsep=0pt]
    \item For a \textbf{flat} or \textbf{convex} surface, $F_{ii} = 0$.
    \item For a \textbf{concave} surface, $F_{ii} > 0$.
\end{itemize}
\end{formulabox}

\begin{formulabox}[title=2. The Reciprocity Rule]
This rule relates a view factor $F_{ij}$ to its inverse, $F_{ji}$. It allows you to find one view factor if the other is known.
$$ A_i F_{ij} = A_j F_{ji} $$
Where $A_i$ and $A_j$ are the surface areas of surfaces $i$ and $j$.
\end{formulabox}

\begin{examplebox}{View Factors for a Hemispherical Duct}
Consider a long duct whose cross-section is a hemisphere. Let Surface 1 be the flat floor and Surface 2 be the curved ceiling. Determine the view factors $F_{12}$, $F_{21}$, and $F_{22}$.
\end{examplebox}

\begin{stepbox}
\begin{enumerate}[label=\textbf{Step \arabic*:}, wide=0pt, leftmargin=*, itemsep=2pt]
    \item \textbf{Define Surfaces and Areas}
    
    Let the hemisphere have radius $r$ and length $L$.
    
    Surface 1 (Floor): A flat rectangle with area $A_1 = (2r) \times L$.
    
    Surface 2 (Ceiling): A curved semi-cylinder with area $A_2 = \frac{1}{2}(\text{Cylinder Circumference}) \times L = \frac{1}{2}(2\pi r) \times L = \pi r L$.
    
    \item \textbf{Find $F_{12}$ using the Summation Rule}
    
    Apply the summation rule for surface 1:
    $$ F_{11} + F_{12} = 1 $$
    Since surface 1 is flat, $F_{11} = 0$.
    $$ 0 + F_{12} = 1 \implies F_{12} = 1 $$
    
    \item \textbf{Find $F_{21}$ using the Reciprocity Rule}
    
    Using the reciprocity relationship:
    $$ A_2 F_{21} = A_1 F_{12} \implies F_{21} = F_{12} \left(\frac{A_1}{A_2}\right) $$
    $$ F_{21} = 1 \times \left(\frac{2rL}{\pi r L}\right) = \frac{2}{\pi} \approx 0.637 $$
    
    \item \textbf{Find $F_{22}$ using the Summation Rule}
    
    Apply the summation rule for surface 2:
    $$ F_{21} + F_{22} = 1 $$
    $$ F_{22} = 1 - F_{21} = 1 - \frac{2}{\pi} \approx 0.363 $$
\end{enumerate}
\end{stepbox}

\newpage

nd{document}
