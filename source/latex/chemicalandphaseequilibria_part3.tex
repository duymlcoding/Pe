\documentclass[12pt]{article}
\usepackage[paperwidth=8.5in, paperheight=11in, margin=1.0in, headheight=15pt]{geometry}
\usepackage{amsmath,amssymb,amsthm}
\usepackage[most]{tcolorbox}
\usepackage{enumitem}
\usepackage{xcolor}
\usepackage{hyperref}
\usepackage{fancyhdr}
\usepackage{titlesec}
\usepackage{graphicx}
% Define custom colors for chemical engineering theme
\definecolor{conceptcolor}{RGB}{52, 73, 94}      % Dark blue-gray
\definecolor{formulacolor}{RGB}{231, 76, 60}     % Red for formulas
\definecolor{examplecolor}{RGB}{39, 174, 96}     % Green for examples
\definecolor{stepcolor}{RGB}{142, 68, 173}       % Purple for solution steps
\definecolor{keycolor}{RGB}{243, 156, 18}        % Orange for key points
% Configure fancy headers
\pagestyle{fancy}
\fancyhf{}
\fancyhead[L]{PE Study Guide}
\fancyhead[R]{Process Fundamentals}
\fancyfoot[C]{\thepage}
\renewcommand{\baselinestretch}{1.1}
\setlength{\parindent}{0.25in}
\setlength{\parskip}{3pt}
% Configure section formatting
\titleformat{\section}
  {\normalfont\LARGE\bfseries\color{conceptcolor}}
  {\thesection}{1em}{}
\titleformat{\subsection}
  {\normalfont\Large\bfseries\color{conceptcolor}}
  {\thesubsection}{1em}{}
% Define custom environments
\newtcolorbox{conceptbox}[1][]{
  enhanced,
  colback=conceptcolor!10,
  colframe=conceptcolor,
  arc=3mm,
  title=Key Concept,
  fonttitle=\bfseries\sffamily\normalsize,
  fontupper=\small,
  #1
}
\newtcolorbox{formulabox}[1][]{
  enhanced,
  colback=formulacolor!10,
  colframe=formulacolor,
  arc=2mm,
  title=Important Formula,
  fonttitle=\bfseries\sffamily\normalsize,
  fontupper=\small,
  #1
}
\newtcolorbox{examplebox}[2][]{
  enhanced,
  colback=examplecolor!10,
  colframe=examplecolor,
  arc=3mm,
  title=Example Problem: #2,
  fonttitle=\bfseries\sffamily\normalsize,
  fontupper=\small,
  #1
}
\newtcolorbox{stepbox}[1][]{
  enhanced,
  colback=stepcolor!10,
  colframe=stepcolor,
  arc=2mm,
  title=Solution Steps,
  fonttitle=\bfseries\sffamily\normalsize,
  fontupper=\small,
  #1
}
\newtcolorbox{keybox}[1][]{
  enhanced,
  colback=keycolor!10,
  colframe=keycolor,
  arc=2mm,
  title=Key Variables \& Definitions,
  fonttitle=\bfseries\sffamily\normalsize,
  fontupper=\small,
  #1
}

This guide introduces fugacity, a thermodynamic property that represents the "escaping tendency" of a substance from a phase. It can be thought of as a corrected or effective pressure that accounts for non-ideal behavior. The most fundamental principle of phase and chemical equilibrium is based on this property.

\begin{conceptbox}
The condition for equilibrium in a multi-component, multi-phase system is that for any given component, its fugacity must be equal in all phases where it is present. This principle supersedes all other rules like Raoult's Law or Henry's Law, which are simply special cases of this universal criterion.
\end{conceptbox}

\begin{formulabox}
For a system with phases A, B, C, ..., the equilibrium criterion for component $i$ is:
$$ \hat{f}_i^A = \hat{f}_i^B = \hat{f}_i^C \dots $$
The hat symbol, $\hat{f}$, denotes the fugacity of a component within a mixture. Matter spontaneously moves from a region of higher fugacity to a region of lower fugacity until equilibrium (equal fugacity) is reached.
\end{formulabox}

\subsection*{Calculating Fugacity in a Liquid Mixture}
\begin{conceptbox}
The method for calculating the fugacity of a component in a liquid mixture depends on the ideality of the solution and the system pressure. The equations build upon each other, adding correction factors for increasing complexity.
\end{conceptbox}
\begin{keybox}
The key variables for calculating liquid-phase fugacity are:
\begin{itemize}[itemsep=0pt]
    \item \textbf{$\hat{f}_i$}: Fugacity of component $i$ in the liquid mixture [Pressure].
    \item \textbf{$x_i$}: Mole fraction of component $i$ in the liquid.
    \item \textbf{$\gamma_i$}: Activity coefficient of component $i$ (accounts for liquid-phase non-ideality).
    \item \textbf{$P_i^{sat}$}: Saturation (vapor) pressure of pure component $i$ at the system temperature.
    \item \textbf{$\phi_i^{sat}$}: Fugacity coefficient of pure saturated $i$ (accounts for vapor-phase non-ideality at $P_i^{sat}$).
    \item \textbf{$V_i^L$}: Molar volume of pure liquid component $i$.
    \item \textbf{$P$}: Total system pressure.
\end{itemize}
\end{keybox}
\begin{formulabox}[title=Fugacity Calculation Equations]
The following equations are used to calculate the fugacity of component $i$ in a liquid mixture. They are presented in order of increasing rigor.
\begin{itemize}[itemsep=2pt]
    \item \textbf{Ideal Liquid Solution (Low Pressure)}: This is equivalent to Raoult's Law.
    $$ \hat{f}_i = x_i P_i^{sat} $$
    \item \textbf{Non-Ideal Liquid Solution (Low Pressure)}: This is equivalent to Modified Raoult's Law.
    $$ \hat{f}_i = x_i \gamma_i P_i^{sat} $$
    \item \textbf{Rigorous Equation (High Pressure)}: This form includes corrections for vapor-phase non-ideality and the effect of pressure on the liquid phase.
    $$ \hat{f}_i = x_i \gamma_i \phi_i^{sat} P_i^{sat} \exp\left[\frac{V_i^L(P - P_i^{sat})}{RT}\right] $$
    The exponential term is known as the \textit{Poynting correction factor}. It is typically close to 1 unless the pressure $P$ is very high.
\end{itemize}
\end{formulabox}

\newpage
\subsection*{Conceptual Examples of Fugacity}
\begin{examplebox}{Fugacity in a Binary VLE System}
A binary liquid mixture containing 70 mol\% component A and 30 mol\% component B is in equilibrium with its vapor. The partial pressures in the vapor phase are measured to be $P_A = 0.2$ bar and $P_B = 0.8$ bar. Which component has the higher fugacity in the liquid phase?
\end{examplebox}
\begin{stepbox}
\begin{enumerate}[label=\textbf{Step \arabic*:}, wide=0pt, leftmargin=*, itemsep=2pt]
    \item \textbf{Apply the Fundamental Rule of Equilibrium}
    
    The system is at equilibrium, which means the fugacity of each component must be the same in the liquid (L) and vapor (V) phases.
    $$ \hat{f}_A^L = \hat{f}_A^V \quad \text{and} \quad \hat{f}_B^L = \hat{f}_B^V $$
    
    \item \textbf{Calculate Fugacity in the Vapor Phase}
    
    At the low pressures given (total P = 1.0 bar), the vapor phase can be assumed to behave as an ideal gas mixture. For an ideal gas, the fugacity of a component is equal to its partial pressure.
    $$ \hat{f}_A^V \approx P_A = 0.2 \, \text{bar} $$
    $$ \hat{f}_B^V \approx P_B = 0.8 \, \text{bar} $$
    
    \item \textbf{Determine Fugacity in the Liquid Phase and Conclude}
    
    Because of the equilibrium condition, the fugacities in the liquid phase must be equal to those in the vapor phase.
    $$ \hat{f}_A^L = 0.2 \, \text{bar} $$
    $$ \hat{f}_B^L = 0.8 \, \text{bar} $$
    \textbf{Conclusion}: Component B has the higher fugacity in the liquid ($\hat{f}_B^L > \hat{f}_A^L$), even though it is the minority component by mole fraction ($x_B < x_A$). This indicates component B is much more volatile than component A.
\end{enumerate}
\end{stepbox}

\newpage

\begin{examplebox}{Fugacity of Salt in Water}
Sufficient NaCl is added to water at 25$^\circ$C so that a saturated solution is formed, with solid salt crystals present at the bottom. Does the water or the salt have a higher fugacity in the liquid solution?
\end{examplebox}
\begin{stepbox}
\begin{enumerate}[label=\textbf{Step \arabic*:}, wide=0pt, leftmargin=*, itemsep=2pt]
    \item \textbf{Estimate the Fugacity of Water}
    
    The fugacity of water in the salt solution, $\hat{f}_{H_2O}^L$, is equal to the fugacity of water in the vapor phase it's in equilibrium with. We can estimate this using Modified Raoult's Law. At 25$^\circ$C, $P_{H_2O}^{sat} \approx 0.0317$ bar.
    $$ \hat{f}_{H_2O}^L \approx x_{H_2O} \gamma_{H_2O} P_{H_2O}^{sat} $$
    In a saturated salt solution, $x_{H_2O}$ is less than 1 but still high (e.g., $\sim$0.9). The activity coefficient will also be near 1. Therefore, the fugacity of water is on the order of 0.02-0.03 bar.
    
    \item \textbf{Estimate the Fugacity of Salt}
    
    The system contains solid salt in equilibrium with the dissolved salt. Therefore, the fugacity of the dissolved salt must equal the fugacity of the solid salt. Salt (NaCl) is a non-volatile solid, meaning its tendency to enter the vapor phase is extremely low. Its vapor pressure at 25$^\circ$C is infinitesimally small. Thus, its fugacity is practically zero.
    
    \item \textbf{Conclusion}
    
    The fugacity of water ($\sim$0.03 bar) is many orders of magnitude greater than the fugacity of the dissolved salt ($\sim$0 bar).
\end{enumerate}
\end{stepbox}

\newpage
\subsection*{Fugacity in Everyday Phenomena}
\begin{conceptbox}
The principle that matter spontaneously moves from a state of higher fugacity to a state of lower fugacity is a powerful tool for explaining a wide range of physical and chemical processes.
\end{conceptbox}
\begin{keybox}[title=Examples of Fugacity at Work]
\begin{itemize}[itemsep=4pt]
    \item \textbf{Opening a Soda Can}: The high pressure in the can creates a very high fugacity for the dissolved CO$_{2}$. When opened, this fugacity is much higher than that of CO$_{2}$ in the atmosphere. To reach equilibrium, CO$_{2}$ rapidly escapes the liquid.

    \item \textbf{Saturated Sugar in Iced Tea}: When solid sugar is present, the system is at equilibrium. This means the fugacity of sugar in the solid crystals is equal to the fugacity of the sugar dissolved in the tea. If you add more tea (pure water), the sugar fugacity in the liquid drops, so more solid sugar dissolves to restore equilibrium.

    \item \textbf{Reverse Osmosis}: High mechanical pressure is applied to a salt water solution. This pressure greatly increases the fugacity of the water in that solution (via the Poynting correction). When the fugacity of water in the salt solution exceeds the fugacity of pure water on the other side of a membrane, water is forced to move "backwards" against the concentration gradient.
    
    \item \textbf{A Carrot in Salt Water}: A carrot contains mostly fresh water, which has a high fugacity. The salt in the surrounding water lowers the fugacity of the water in the solution. Because the fugacity of water inside the carrot is now higher than that outside, water spontaneously moves out of the carrot, causing it to shrivel.
    
    \item \textbf{Melting Ice with Salt}: At 0$^\circ$C, pure solid ice and pure liquid water are in equilibrium, so $f_{ice} = f_{liquid}$. When salt is added, it dissolves in the liquid water and lowers the fugacity of the water in that phase ($f_{new\_liquid} < f_{liquid}$). Now, the fugacity of the solid ice is higher than the fugacity of the saltwater ($f_{ice} > f_{new\_liquid}$). To restore equilibrium, the ice melts, moving from a state of high fugacity to low fugacity.
    
    \item \textbf{Gas Mask Filter}: A gas mask filter contains an adsorbent material like activated carbon. The fugacity of a contaminant molecule is much lower when it is adsorbed onto the carbon surface than when it is in the air. This large fugacity difference drives the harmful molecules from the air onto the filter, purifying the air that is breathed.
\end{itemize}
\end{keybox}

\newpage

\section*{Phase Diagrams for Partially-Miscible Liquids}
This guide explores the behavior of liquid mixtures that exhibit partial miscibility, leading to the formation of two distinct liquid phases. This phenomenon, known as liquid-liquid equilibrium (LLE), occurs when positive deviations from Raoult's Law are large, indicating significant repulsive forces between unlike molecules.

\begin{conceptbox}
When the repulsive forces between unlike molecules (A-B) are much stronger than the attractive forces between like molecules (A-A and B-B), a single liquid phase becomes unstable over a certain composition range. The mixture spontaneously separates into two immiscible liquid phases, often denoted $\alpha$ and $\beta$, which are in equilibrium with each other.
\begin{itemize}[itemsep=0pt]
    \item The $\alpha$ phase is rich in one component (e.g., A) and contains a small amount of the other.
    \item The $\beta$ phase is rich in the other component (e.g., B) and contains a small amount of the first.
\end{itemize}
\end{conceptbox}

\subsection*{Key Equations for Multi-Phase Equilibrium}

\begin{conceptbox}
The fundamental principles of phase equilibrium still apply to partially-miscible systems, but the equations must now account for the existence of multiple liquid phases in addition to a vapor phase.
\end{conceptbox}

\begin{keybox}[title=Gibbs Phase Rule]
The Gibbs Phase Rule determines the number of independent intensive variables that can be changed while maintaining the number of phases at equilibrium.
\begin{itemize}[itemsep=0pt]
    \item \textbf{$F$}: Degrees of freedom.
    \item \textbf{$C$}: Number of chemical components.
    \item \textbf{$P$}: Number of phases in equilibrium.
\end{itemize}
\end{keybox}

\begin{formulabox}
For a non-reactive system, the Gibbs Phase Rule is:
$$ F = C - P + 2 $$
\end{formulabox}

\begin{formulabox}[title=Equilibrium Condition for LLE]
For two liquid phases, $\alpha$ and $\beta$, to be in equilibrium, the fugacity of each component $i$ must be the same in both phases.
$$ \hat{f}_i^\alpha = \hat{f}_i^\beta $$
Using the activity coefficient model for fugacity ($\hat{f}_i = x_i \gamma_i P_i^{sat}$), this becomes:
$$ x_i^\alpha \gamma_i^\alpha P_i^{sat} = x_i^\beta \gamma_i^\beta P_i^{sat} $$
The saturation pressure, $P_i^{sat}$, cancels, leading to the fundamental LLE condition that the \textit{activity} of each component must be equal in the two equilibrium liquid phases:
$$ x_i^\alpha \gamma_i^\alpha = x_i^\beta \gamma_i^\beta $$
\end{formulabox}

\newpage

\subsection*{Interpreting the Temperature-Composition (T-x-y) Diagram}
\begin{conceptbox}
A T-x-y diagram for a binary, partially-miscible system at constant pressure shows the phase behavior as a function of temperature and overall composition. It features a vapor phase at high temperatures, single-phase liquid regions, a two-phase liquid-liquid region (the "miscibility gap"), and a unique horizontal line where three phases coexist.
\end{conceptbox}
\begin{keybox}[title=Regions of the T-x-y Diagram]
\begin{itemize}[itemsep=2pt]
    \item \textbf{Single-Phase Regions (V, L$_\alpha$, L$_\beta$)}: At high T, a single vapor phase (V) exists. At lower T and compositions rich in one component, a single liquid phase exists, either rich in component A (L$_\alpha$) or rich in component B (L$_\beta$).
    
    \item \textbf{Two-Phase Regions (V+L, L+L)}: Any overall composition falling within these regions will separate into two phases. The compositions of the equilibrium phases are given by the endpoints of a horizontal tie line passing through the point. The lever rule can be used to find the relative amounts of each phase.
    
    \item \textbf{Three-Phase Equilibrium Line (V+L$_\alpha$+L$_\beta$)}: This is a horizontal line at a specific temperature, $T_{3\phi}$.
    \begin{itemize}[itemsep=0pt]
        \item \textbf{Degrees of Freedom}: For a binary system ($C=2$) with three phases ($P=3$) at a fixed external pressure, the Gibbs Phase Rule gives:
        $$ F = C-P+2 = 2-3+2=1 $$
        However, since the entire diagram is already constructed at a fixed pressure, this one degree of freedom is used. There are \textbf{zero} remaining degrees of freedom.
        \item \textbf{Implication}: The three phases can only coexist at a \textbf{single, unique temperature} for that given pressure. As heat is added or removed at this temperature, the amounts of the phases change, but the temperature and compositions of all three phases remain constant until one phase is completely consumed.
    \end{itemize}
\end{itemize}
\end{keybox}

\newpage
\subsection*{Example Problems}

\begin{examplebox}{Three-Phase Mass Balance}
One mole of a liquid mixture containing 25 mol\% component A and 75 mol\% component B is heated at constant pressure. When 0.05 moles of vapor have formed, the system is found to be at the three-phase equilibrium temperature. At this temperature, the compositions of the coexisting phases are:
\begin{itemize}[itemsep=0pt]
    \item Liquid $\alpha$ phase: $x_A = 0.73, x_B = 0.27$
    \item Liquid $\beta$ phase: $x_A = 0.17, x_B = 0.83$
    \item Vapor phase: $y_A = 0.40, y_B = 0.60$
\end{itemize}
Determine the number of moles of each phase present.
\end{examplebox}
\begin{stepbox}
\begin{enumerate}[label=\textbf{Step \arabic*:}, wide=0pt, leftmargin=*, itemsep=2pt]
    \item \textbf{Identify the State of the System}
    
    The problem states that vapor has begun to form from a liquid mixture and that the system is at the three-phase equilibrium temperature. This confirms that all three phases (Vapor, Liquid $\alpha$, Liquid $\beta$) must be present and coexisting in equilibrium.
    
    \item \textbf{List Phases, Compositions, and Amounts}
    
    We can summarize the knowns and unknowns for the system. Let $V$, $L_\alpha$, and $L_\beta$ be the mole amounts of each phase.
    \begin{itemize}[itemsep=2pt]
        \item \textbf{Phase 1: Vapor (V)}
            \begin{itemize}[itemsep=0pt]
                \item Amount: $V = 0.05$ mol (given)
                \item Composition: $y_B = 0.60$
            \end{itemize}
        \item \textbf{Phase 2: Liquid $\alpha$ (L$_\alpha$)}
            \begin{itemize}[itemsep=0pt]
                \item Amount: $L_\alpha = ?$
                \item Composition: $x_{B,\alpha} = 0.27$
            \end{itemize}
        \item \textbf{Phase 3: Liquid $\beta$ (L$_\beta$)}
            \begin{itemize}[itemsep=0pt]
                \item Amount: $L_\beta = ?$
                \item Composition: $x_{B,\beta} = 0.83$
            \end{itemize}
    \end{itemize}
\end{enumerate}
\end{stepbox}

\newpage
\begin{stepbox}
\begin{enumerate}[label=\textbf{Step \arabic*:}, wide=0pt, leftmargin=*, itemsep=2pt, start=3]
    \item \textbf{Set up and Solve Mole Balances}
    
    We can solve for the two unknown amounts, $L_\alpha$ and $L_\beta$, by writing two independent mole balances.
    
    \begin{itemize}[itemsep=2pt]
        \item \textbf{Overall Mole Balance}: The sum of the moles of the three phases must equal the total moles in the system, which is 1.0 mol.
        $$ V + L_\alpha + L_\beta = 1.0 $$
        $$ 0.05 + L_\alpha + L_\beta = 1.0 \implies L_\alpha + L_\beta = 0.95 $$
        
        \item \textbf{Component B Mole Balance}: The moles of component B distributed among the three phases must equal the total moles of B fed to the system (0.75 mol).
        $$ y_B V + x_{B,\alpha} L_\alpha + x_{B,\beta} L_\beta = 0.75 $$
        $$ (0.60)(0.05) + (0.27) L_\alpha + (0.83) L_\beta = 0.75 $$
        $$ 0.03 + 0.27 L_\alpha + 0.83 L_\beta = 0.75 \implies 0.27 L_\alpha + 0.83 L_\beta = 0.72 $$
    \end{itemize}
    
    Now we solve the system of two linear equations:
    \begin{enumerate}[label=\alph*)]
        \item $L_\alpha + L_\beta = 0.95$
        \item $0.27 L_\alpha + 0.83 L_\beta = 0.72$
    \end{enumerate}
    From equation (a), express $L_\alpha$ in terms of $L_\beta$: $L_\alpha = 0.95 - L_\beta$. Substitute this into equation (b):
    $$ 0.27(0.95 - L_\beta) + 0.83 L_\beta = 0.72 $$
    $$ 0.2565 - 0.27 L_\beta + 0.83 L_\beta = 0.72 $$
    $$ 0.56 L_\beta = 0.72 - 0.2565 = 0.4635 $$
    $$ L_\beta = \frac{0.4635}{0.56} \approx 0.828 \, \text{mol} $$
    Now solve for $L_\alpha$:
    $$ L_\alpha = 0.95 - 0.828 = 0.122 \, \text{mol} $$
    
    \item \textbf{Final Answer}
    
    The system contains three phases with the following amounts and compositions:
    \begin{itemize}[itemsep=2pt]
        \item \textbf{Vapor}: 0.05 mol, with composition $y_B=0.60$.
        \item \textbf{Liquid $\alpha$}: 0.12 mol, with composition $x_B=0.27$.
        \item \textbf{Liquid $\beta$}: 0.83 mol, with composition $x_B=0.83$.
    \end{itemize}
\end{enumerate}
\end{stepbox}

\newpage
\begin{examplebox}{Effect of Cooling from a Three-Phase State}
A cyclohexane-water system is at equilibrium at 50$^\circ$C and a fixed pressure. The system contains three phases: a vapor phase, a water-rich liquid phase, and a cyclohexane-rich liquid phase. If the temperature is now decreased to 45$^\circ$C while keeping the pressure constant, what change will occur in the system?
\end{examplebox}
\begin{stepbox}
\begin{enumerate}[label=\textbf{Step \arabic*:}, wide=0pt, leftmargin=*, itemsep=2pt]
    \item \textbf{Analyze the System using the Gibbs Phase Rule}
    
    The Gibbs Phase Rule provides a rigorous way to understand the constraints on the system.
    \begin{itemize}[itemsep=2pt]
        \item \textbf{Components}: $C = 2$ (cyclohexane, water).
        \item \textbf{Initial Phases}: $P = 3$ (vapor, liquid $\alpha$, liquid $\beta$).
        \item \textbf{Initial Degrees of Freedom}: $F = C - P + 2 = 2 - 3 + 2 = 1$.
        \item \textbf{Constraint}: The problem states the pressure is held constant. This uses up the single available degree of freedom.
        \item \textbf{Interpretation}: This means that for the given pressure, the three phases can only coexist at a single, unique temperature (50$^\circ$C). It is thermodynamically impossible for the system to remain in a three-phase state if the temperature is changed.
    \end{itemize}
    
    \item \textbf{Predict the Change}
    
    To exist at the new temperature of 45$^\circ$C, the system must change in a way that increases its degrees of freedom. This requires a reduction in the number of phases. Since the temperature is being lowered (energy is being removed), the highest-energy phase is the most likely to disappear. In a vapor-liquid system, the vapor phase is the highest-energy phase.
    
    \item \textbf{Visualize on a Phase Diagram}
    
    On a T-x-y diagram for this type of system, the three-phase equilibrium exists as a single horizontal line. The region directly below this line is the two-phase liquid-liquid region (L$_\alpha$ + L$_\beta$). Cooling the system from the three-phase line means moving vertically downward into this L+L region. This transition corresponds to the complete condensation of the vapor.
    
    \item \textbf{Conclusion}
    
    Decreasing the temperature from the three-phase equilibrium point at constant pressure will cause the \textbf{vapor phase to condense entirely}. The system will transition from having three phases (V + L$_\alpha$ + L$_\beta$) to having only two phases (L$_\alpha$ + L$_\beta$).
\end{enumerate}
\end{stepbox}

\newpage

\section*{Phase Diagrams for Immiscible Liquids}
This guide focuses on the vapor-liquid equilibrium (VLE) of immiscible liquids, such as oil and water. Immiscibility represents the extreme case of positive deviation from Raoult's Law, where repulsive forces between unlike molecules are so strong that the liquids do not mix. This leads to unique and simplified VLE behavior.

\subsection*{Core Principles of Immiscible VLE}
\begin{conceptbox}
When two immiscible liquids are mixed, they form separate, distinct liquid layers. Each pure liquid is unaware of the other's presence and behaves as if it were in the container alone. This has two major consequences for their phase equilibrium.
\end{conceptbox}

\begin{keybox}[title=Key Principles]
\begin{itemize}[itemsep=4pt]
    \item \textbf{Independent Vapor Pressures}: Each pure liquid in the mixture exerts its full saturation pressure ($P^{sat}$) at the given system temperature. The partial pressure of a component in the vapor phase is equal to its saturation pressure, as long as some of that component is present as a liquid.
    \begin{itemize}[itemsep=0pt]
        \item If liquid A is present, its partial pressure is $P_A = P_A^{sat}$.
        \item If liquid B is present, its partial pressure is $P_B = P_B^{sat}$.
    \end{itemize}
    \item \textbf{Boiling Point of the Mixture}: The immiscible mixture will boil when the \textit{sum} of the individual saturation pressures equals the total external pressure.
    \begin{itemize}[itemsep=0pt]
        \item A significant consequence is that an immiscible mixture always boils at a temperature \textit{lower} than the boiling point of either pure component. This is because the saturation pressures add together to reach the external pressure sooner. This principle is the basis for steam distillation.
    \end{itemize}
    \item \textbf{Vapor Phase Composition}: At the boiling point, the composition of the vapor phase is fixed and determined by the ratio of the individual saturation pressures.
\end{itemize}
\end{keybox}

\begin{formulabox}[title=Governing Equations for Immiscible VLE]
When both immiscible liquids (A and B) are present and in equilibrium with a vapor phase:
\begin{itemize}[itemsep=2pt]
    \item \textbf{Boiling Condition}:
    $$ P_{total} = P_A^{sat}(T) + P_B^{sat}(T) $$
    \item \textbf{Vapor Phase Composition}:
    $$ y_A = \frac{P_A}{P_{total}} = \frac{P_A^{sat}}{P_A^{sat} + P_B^{sat}} $$
    $$ y_B = \frac{P_B}{P_{total}} = \frac{P_B^{sat}}{P_A^{sat} + P_B^{sat}} $$
\end{itemize}
\end{formulabox}

\newpage
\subsection*{Example Problems}

\begin{examplebox}{Condensation of an Immiscible System}
A gas mixture containing 75 mol\% component A and 25 mol\% component B is compressed isothermally. At a total pressure of 1.6 bar, liquid A begins to condense. As the pressure is increased further, liquid B begins to condense at a total pressure of 2.4 bar. What are the saturation pressures of pure A and pure B at this temperature?
\end{examplebox}

\begin{stepbox}
\begin{enumerate}[label=\textbf{Step \arabic*:}, wide=0pt, leftmargin=*, itemsep=2pt]
    \item \textbf{Analyze the First Condensation Point (P = 1.6 bar)}
    
    At this pressure, the first drop of liquid A forms at the dew point. The vapor phase composition remains $y_A=0.75, y_B=0.25$.
    
    Calculate partial pressures using Dalton's Law:
    $$ P_A = y_A P_{total} = 0.75 \times 1.6 \, \text{bar} = 1.2 \, \text{bar} $$
    $$ P_B = y_B P_{total} = 0.25 \times 1.6 \, \text{bar} = 0.4 \, \text{bar} $$
    
    Since liquid A is just beginning to condense, its partial pressure equals its saturation pressure:
    $$ P_A = P_A^{sat} \implies P_A^{sat} = 1.2 \, \text{bar} $$
    
    \item \textbf{Analyze the Second Condensation Point (P = 2.4 bar)}
    
    At 2.4 bar, the system contains vapor and liquid A phases, and liquid B begins to condense.
    
    Since liquid A is present, the partial pressure of A remains fixed at its saturation pressure:
    $$ P_A = P_A^{sat} = 1.2 \, \text{bar} $$
    
    Find the partial pressure of B by subtraction:
    $$ P_B = P_{total} - P_A = 2.4 \, \text{bar} - 1.2 \, \text{bar} = 1.2 \, \text{bar} $$
    
    Since component B is just beginning to condense:
    $$ P_B = P_B^{sat} \implies P_B^{sat} = 1.2 \, \text{bar} $$
\end{enumerate}
\end{stepbox}

\newpage
\begin{examplebox}{Identifying Phases in an Immiscible System}
A closed system contains 6 mol of component A and 4 mol of component B. The system is at equilibrium at 100$^\circ$C and a total pressure of 2.0 atm. A and B are completely immiscible in the liquid phase. Their saturation pressures at 100$^\circ$C are known to be $P_A^{sat} = 2.0$ atm and $P_B^{sat} = 0.5$ atm. Determine the phases present at equilibrium.
\end{examplebox}
\begin{stepbox}
\begin{enumerate}[label=\textbf{Step \arabic*:}, wide=0pt, leftmargin=*, itemsep=2pt]
    \item \textbf{State the Strategy}
    
    We must determine which combination of phases (Vapor, Liquid A, Liquid B) is thermodynamically consistent with the given total pressure and saturation pressures. We will test each plausible hypothesis.
    
    \item \textbf{Hypothesis 1: Two Liquids (A+B) and Vapor are Present}
    
    \begin{itemize}[itemsep=2pt]
        \item \textbf{Condition}: If both liquid A and liquid B were present, the total pressure of the system would be fixed at the sum of their saturation pressures.
        \item \textbf{Check}:
        $$ P_{required} = P_A^{sat} + P_B^{sat} = 2.0 \, \text{atm} + 0.5 \, \text{atm} = 2.5 \, \text{atm} $$
        \item \textbf{Conclusion}: The actual system pressure is 2.0 atm. Since $P_{actual} \neq P_{required}$, it is impossible for both liquids to be present. This hypothesis is \textbf{incorrect}.
    \end{itemize}

    \item \textbf{Hypothesis 2: Only Liquid A and Vapor are Present}
    
    \begin{itemize}[itemsep=2pt]
        \item \textbf{Condition}: If liquid A is present, its partial pressure in the vapor phase must be equal to its saturation pressure: $P_A = P_A^{sat} = 2.0$ atm.
        \item \textbf{Check}: If $P_A = 2.0$ atm, then for the total pressure to be 2.0 atm, the partial pressure of B would have to be zero:
        $$ P_B = P_{total} - P_A = 2.0 \, \text{atm} - 2.0 \, \text{atm} = 0 \, \text{atm} $$
        \item \textbf{Conclusion}: This is impossible because we know the system contains 4 moles of component B. This hypothesis is \textbf{incorrect}.
    \end{itemize}
\end{enumerate}
\end{stepbox}

\newpage
\begin{stepbox}
\begin{enumerate}[label=\textbf{Step \arabic*:}, wide=0pt, leftmargin=*, itemsep=2pt, start=4]
    \item \textbf{Hypothesis 3: Only Vapor is Present}
    
    \begin{itemize}[itemsep=2pt]
        \item \textbf{Condition}: If the system is a single vapor phase, the partial pressures are determined by the overall mole fractions.
        \item \textbf{Check}: First, find the overall mole fractions.
        $$ y_A = \frac{6}{6+4} = 0.6 \quad \text{and} \quad y_B = \frac{4}{6+4} = 0.4 $$
        Next, calculate the corresponding partial pressures at $P_{total} = 2.0$ atm.
        $$ P_A = y_A P_{total} = 0.6 \times 2.0 \, \text{atm} = 1.2 \, \text{atm} $$
        $$ P_B = y_B P_{total} = 0.4 \times 2.0 \, \text{atm} = 0.8 \, \text{atm} $$
        Now, check for physical consistency. A component's partial pressure cannot exceed its saturation pressure. Here, the calculated $P_B$ (0.8 atm) is \textit{greater than} its saturation pressure ($P_B^{sat} = 0.5$ atm).
        \item \textbf{Conclusion}: This is physically impossible. If the partial pressure of B reached 0.5 atm, it would begin to condense, preventing the partial pressure from rising further. Therefore, the system cannot be all vapor. This hypothesis is \textbf{incorrect}.
    \end{itemize}

    \item \textbf{Hypothesis 4: Only Liquid B and Vapor are Present}
    
    \begin{itemize}[itemsep=2pt]
        \item \textbf{Condition}: This is the only remaining possibility. Let's test it. If liquid B is present, its partial pressure is fixed at its saturation pressure: $P_B = P_B^{sat} = 0.5$ atm.
        \item \textbf{Check}: The partial pressure of A would then be:
        $$ P_A = P_{total} - P_B = 2.0 \, \text{atm} - 0.5 \, \text{atm} = 1.5 \, \text{atm} $$
        Is this state possible? Yes, because this partial pressure ($P_A = 1.5$ atm) is \textit{less than} the saturation pressure of A ($P_A^{sat} = 2.0$ atm). Component A can exist as a vapor with this partial pressure without condensing.
        \item \textbf{Conclusion}: This state is thermodynamically consistent.
    \end{itemize}

    \item \textbf{Final Answer}
    
    The phases present at equilibrium are \textbf{liquid B coexisting with a vapor mixture} of A and B.
\end{enumerate}
\end{stepbox}

\newpage

\section*{Vapor-Liquid Equilibrium: Bubble Point Calculations}
This guide covers bubble point calculations, a fundamental process in chemical engineering used to determine the conditions at which a liquid mixture of known composition begins to boil. This is essential for designing separation equipment like distillation columns and flash drums.

\begin{conceptbox}
A \textit{bubble point} calculation determines the temperature (at a given pressure) or the pressure (at a given temperature) where the first infinitesimal bubble of vapor forms from a liquid mixture. The key principle is that at this point, the sum of the partial pressures exerted by the components in the liquid equals the total system pressure, and consequently, the mole fractions of the components in the first bubble of vapor must sum to one.
\end{conceptbox}

\begin{formulabox}[title=Governing Equations for Bubble Point Calculations]
The calculation is based on two fundamental principles for an ideal solution:
\begin{itemize}[itemsep=2pt]
    \item \textbf{Raoult's Law}: Defines the composition of the vapor bubble in equilibrium with the liquid.
    $$ y_i = \frac{x_i P_i^{sat}(T)}{P} $$
    \item \textbf{Summation Constraint}: The mole fractions in the vapor phase must sum to unity.
    $$ \sum_{i} y_i = 1 $$
\end{itemize}
Combining these gives the single governing equation that must be solved:
$$ \sum_{i} \frac{x_i P_i^{sat}(T)}{P} = 1 $$
Since the saturation pressure, $P_i^{sat}$, is a strong, non-linear function of temperature $T$ (often given by the Antoine equation), finding a bubble temperature requires an iterative solution method.
\end{formulabox}

\begin{stepbox}[title=Strategy for Iterative Bubble Temperature Calculation]
A numerical solver, such as Excel's Solver tool, is an efficient way to perform a bubble temperature calculation. The procedure is as follows:
\begin{enumerate}[label=\textbf{Step \arabic*:}, wide=0pt, leftmargin=*, itemsep=2pt]
    \item \textbf{Setup Spreadsheet}: Designate cells for the known variables: liquid mole fractions ($x_1, x_2, \dots$) and the total system pressure ($P$). Create another cell for the temperature, $T$, and enter an initial guess.

    \item \textbf{Enter Formulas}:
    \begin{itemize}[itemsep=2pt]
        \item In separate cells, calculate the saturation pressure, $P_i^{sat}$, for each component using its Antoine equation, referencing the cell with the guessed temperature $T$.
        \item In other cells, calculate the vapor mole fraction, $y_i$, for each component using Raoult's Law: $y_i = (x_i \times P_i^{sat}) / P$.
    \end{itemize}

    \item \textbf{Define Target Cell}: Create a cell that calculates the sum of the vapor mole fractions, $\sum y_i$. This cell's value will be used to check for convergence.
    
    \item \textbf{Configure Solver Tool}:
    \begin{itemize}[itemsep=2pt]
        \item \textbf{Set Objective}: The target cell containing $\sum y_i$.
        \item \textbf{To}: A value of \textbf{1}.
        \item \textbf{By Changing Variable Cells}: The cell containing the temperature guess, $T$.
    \end{itemize}
    
    \item \textbf{Run Solver}: The solver will iteratively adjust the temperature until it finds the value where the sum of the vapor mole fractions is exactly 1. For a specific system, this might yield a result like \textbf{$84.4^\circ$C}, which would be the bubble point temperature.
\end{enumerate}
\end{stepbox}

\newpage

\section*{Flash Separations}
Flash separation, also known as flash distillation, is a single-stage separation process widely used in the chemical industry. It is designed to separate components in a liquid mixture based on differences in their volatilities.

\begin{conceptbox}
The process involves three main steps:
\begin{enumerate}[itemsep=0pt]
    \item A pressurized liquid feed stream is heated.
    \item The stream is passed through a throttling device (like a valve) into a vessel at a lower pressure.
    \item The sudden pressure drop causes a fraction of the liquid to rapidly vaporize, or "flash". The resulting vapor and liquid phases are allowed to reach equilibrium inside the vessel (the "flash drum") and are then drawn off as separate product streams.
\end{enumerate}
The vapor product is enriched in the more volatile components, while the liquid product is enriched in the less volatile components.
\end{conceptbox}

\subsection*{Governing Equations for Flash Separations}
The analysis of a flash separation requires the simultaneous application of material balances, phase equilibrium relationships, and, for adiabatic cases, an energy balance.

\begin{keybox}[title=Key Variables]
\begin{itemize}[itemsep=0pt]
    \item \textbf{$F, V, L$}: Total molar flow rates of the Feed, Vapor product, and Liquid product.
    \item \textbf{$z_i, y_i, x_i$}: Mole fractions of component $i$ in the Feed, Vapor, and Liquid streams.
    \item \textbf{$K_i$}: The vapor-liquid equilibrium ratio (K-factor) for component $i$.
    \item \textbf{$H_F, H_V, H_L$}: Molar enthalpies of the Feed, Vapor, and Liquid streams.
\end{itemize}
\end{keybox}

\begin{formulabox}[title=Material Balance Equations]
\begin{itemize}[itemsep=2pt]
    \item \textbf{Overall Material Balance}:
    $$ F = V + L $$
    \item \textbf{Component Material Balance} (for each component $i$):
    $$ z_i F = y_i V + x_i L $$
\end{itemize}
\end{formulabox}

\begin{formulabox}[title=Phase Equilibrium Equations]
\begin{itemize}[itemsep=2pt]
    \item \textbf{K-factor Definition}: The K-factor relates the equilibrium compositions of the vapor and liquid phases.
    $$ K_i = \frac{y_i}{x_i} $$
    \item \textbf{K-factor for Ideal Systems}: For systems that follow Raoult's Law, the K-factor is a function of saturation pressure ($P_i^{sat}$) and total system pressure ($P$).
    $$ K_i = \frac{P_i^{sat}}{P} $$
\end{itemize}
\end{formulabox}

\begin{formulabox}[title=The Rachford-Rice Flash Equation]
By combining the material balance and equilibrium equations, a single powerful equation can be derived to solve for the fraction of the feed that is vaporized ($V/F$).
\begin{itemize}[itemsep=2pt]
    \item The liquid and vapor mole fractions can be expressed in terms of the feed composition and K-factors:
    $$ x_i = \frac{z_i}{1 + \frac{V}{F}(K_i - 1)} \quad \text{and} \quad y_i = \frac{z_i K_i}{1 + \frac{V}{F}(K_i - 1)} $$
    \item Since $\sum y_i - \sum x_i = 0$, we can subtract the expressions to get the Rachford-Rice equation, which is typically solved numerically for the term $V/F$:
    $$ \sum_{i} \frac{z_i (K_i - 1)}{1 + \frac{V}{F}(K_i - 1)} = 0 $$
\end{itemize}
\end{formulabox}

\begin{formulabox}[title=Adiabatic Energy Balance]
For an adiabatic flash process (no external heat exchange), the enthalpy is conserved:
$$ F H_F = V H_V + L H_L $$
\end{formulabox}

\newpage
\subsection*{Example Problems}

\begin{examplebox}{Flash with Known Vaporized Fraction}
A 50/50 molar liquid mixture of benzene(1) and toluene(2) is flashed to a drum operating at 1.4 bar. At the resulting equilibrium temperature, it is known that 25\% of the feed vaporizes. The saturation pressure of benzene at this temperature is $P_1^{sat} = 2.0$ bar. Assuming an ideal solution, what is the composition of the vapor leaving the drum?
\end{examplebox}

\begin{stepbox}
\begin{enumerate}[label=\textbf{Step \arabic*:}, wide=0pt, leftmargin=*, itemsep=2pt]
    \item \textbf{Identify Knowns and Goal}
    
    Feed composition: $z_1 = 0.50$ (benzene), $z_2 = 0.50$ (toluene). Drum pressure: $P = 1.4$ bar. Saturation pressure: $P_1^{sat} = 2.0$ bar. Vaporized fraction: $V/F = 0.25$. Goal: Find the vapor composition, $y_1$.
    
    \item \textbf{Calculate the K-factor for Benzene ($K_1$)}
    
    For an ideal solution, the K-factor is the ratio of the saturation pressure to the total pressure.
    $$ K_1 = \frac{P_1^{sat}}{P} = \frac{2.0 \, \text{bar}}{1.4 \, \text{bar}} \approx 1.4286 $$
    
    \item \textbf{Calculate the Liquid Mole Fraction of Benzene ($x_1$)}
    
    Use the rearranged form of the flash equation:
    $$ x_1 = \frac{z_1}{1 + \frac{V}{F}(K_1 - 1)} $$
    $$ x_1 = \frac{0.50}{1 + 0.25(1.4286 - 1)} = \frac{0.50}{1 + 0.25(0.4286)} = \frac{0.50}{1.10715} \approx 0.4516 $$
    
    \item \textbf{Calculate the Vapor Mole Fraction of Benzene ($y_1$)}
    
    Use the K-factor definition ($K_1 = y_1/x_1$):
    $$ y_1 = K_1 \times x_1 = 1.4286 \times 0.4516 \approx 0.6451 $$
    
    \item \textbf{Final Answer}
    
    The vapor composition is approximately \textbf{64.5\% benzene} ($y_1 = 0.645$) and 35.5\% toluene.
\end{enumerate}
\end{stepbox}

\newpage
\begin{examplebox}{Isothermal Flash Calculation}
A liquid feed containing 60 mol\% component 1 and 40 mol\% component 2 is flashed to an outlet condition of 150$^\circ$C and 1210 kPa. The system behaves as an ideal solution. The saturation pressures (in kPa) are given by the Antoine equations, where T is in $^\circ$C:
$$ \ln(P_1^{sat}) = 15 - \frac{3010}{T + 250} \quad \text{and} \quad \ln(P_2^{sat}) = 14 - \frac{2700}{T + 205} $$
Calculate the fraction of the feed that leaves as liquid ($L/F$) and the compositions of the equilibrium liquid ($x_1$) and vapor ($y_1$) phases.
\end{examplebox}
\begin{stepbox}
\begin{enumerate}[label=\textbf{Step \arabic*:}, wide=0pt, leftmargin=*, itemsep=2pt]
    \item \textbf{Strategy: Isothermal Flash}
    
    In an isothermal flash, the temperature and pressure in the drum are known. This means the K-factors are fixed. The first step is to determine the compositions of the liquid and vapor phases that can coexist at these conditions. Then, a material balance is used to find the relative amounts of the two phases.
    
    \item \textbf{Calculate Saturation Pressures and K-factors at 150$^\circ$C}
    
    Substitute $T=150^\circ$C into the given Antoine equations.
    $$ \ln(P_1^{sat}) = 15 - \frac{3010}{150 + 250} = 15 - 7.525 = 7.475 \implies P_1^{sat} = e^{7.475} \approx 1763.4 \, \text{kPa} $$
    $$ \ln(P_2^{sat}) = 14 - \frac{2700}{150 + 205} = 14 - 7.606 = 6.394 \implies P_2^{sat} = e^{6.394} \approx 598.2 \, \text{kPa} $$
    Now, calculate the K-factors at the drum pressure $P = 1210$ kPa.
    $$ K_1 = \frac{P_1^{sat}}{P} = \frac{1763.4}{1210} \approx 1.457 $$
    $$ K_2 = \frac{P_2^{sat}}{P} = \frac{598.2}{1210} \approx 0.494 $$
    
    \item \textbf{Solve the Rachford-Rice Equation (Iteratively)}
    
    We must now find the value of $V/F$ that solves the Rachford-Rice equation.
    $$ f(V/F) = \sum_{i} \frac{z_i (K_i - 1)}{1 + \frac{V}{F}(K_i - 1)} = \frac{0.6(1.457-1)}{1+\frac{V}{F}(1.457-1)} + \frac{0.4(0.494-1)}{1+\frac{V}{F}(0.494-1)} = 0 $$
    $$ \frac{0.2742}{1+0.457(V/F)} + \frac{-0.2024}{1-0.506(V/F)} = 0 $$
    Solving this non-linear equation for $V/F$ using a numerical solver (e.g., Excel Solver, Python's `fsolve`) yields:
    $$ \frac{V}{F} \approx 0.3125 $$
\end{enumerate}
\end{stepbox}

\newpage
\begin{stepbox}
\begin{enumerate}[label=\textbf{Step \arabic*:}, wide=0pt, leftmargin=*, itemsep=2pt, start=4]
    \item \textbf{Determine the Liquid Fraction ($L/F$)}
    
    The liquid fraction is simply one minus the vapor fraction.
    $$ \frac{L}{F} = 1 - \frac{V}{F} = 1 - 0.3125 = 0.6875 $$
    
    \item \textbf{Determine the Liquid and Vapor Compositions}
    
    Now that we know $V/F$, we can use the composition formulas from the Rachford-Rice derivation.
    \begin{itemize}[itemsep=2pt]
        \item \textbf{Liquid Composition ($x_i$)}:
        $$ x_1 = \frac{z_1}{1 + \frac{V}{F}(K_1 - 1)} = \frac{0.6}{1 + 0.3125(1.457 - 1)} = \frac{0.6}{1.1428} \approx 0.525 $$
        $$ x_2 = 1 - x_1 = 1 - 0.525 = 0.475 $$
        \item \textbf{Vapor Composition ($y_i$)}:
        $$ y_1 = K_1 x_1 = 1.457 \times 0.525 \approx 0.765 $$
        $$ y_2 = K_2 x_2 = 0.494 \times 0.475 \approx 0.235 $$
        (As a check, the vapor mole fractions sum to $0.765 + 0.235 = 1.0$).
    \end{itemize}
    
    \item \textbf{Final Answer}
    
    The results of the isothermal flash calculation are:
    \begin{itemize}[itemsep=2pt]
        \item The fraction of the effluent that is liquid is \textbf{68.8\%} ($L/F = 0.688$).
        \item \textbf{Liquid Phase Composition}: 52.5\% component 1 ($x_1 = 0.525$).
        \item \textbf{Vapor Phase Composition}: 76.5\% component 1 ($y_1 = 0.765$).
    \end{itemize}
\end{enumerate}
\end{stepbox}

\newpage
\section*{Solid-Liquid Equilibrium: Solubility}
This section focuses on the equilibrium between a solid solute and a liquid solvent, which is governed by the solute's solubility. This is the basis for designing crystallization and dissolution processes.

\begin{conceptbox}
Solubility describes the maximum amount of a solute that can dissolve in a solvent at a given temperature to form a stable solution.
\begin{itemize}[itemsep=2pt]
    \item \textbf{Saturated Solution}: A solution in equilibrium with undissolved solid solute. It contains the maximum possible amount of dissolved solute at that temperature.
    \item \textbf{Supersaturated Solution}: An unstable state where the solution contains more dissolved solute than its equilibrium solubility. Given time or a seed crystal, the excess solute will precipitate.
    \item \textbf{Kinetics vs. Equilibrium}: Solubility diagrams describe the final equilibrium state, not the rate (kinetics) at which dissolution or crystallization occurs.
\end{itemize}
\end{conceptbox}

\subsection*{Using Solubility Diagrams for Material Balances}
\begin{conceptbox}
Solubility diagrams, which plot solubility versus temperature, are essential tools for performing material balances on crystallizer systems. It is critical to interpret the units of solubility correctly.
\end{conceptbox}
\begin{formulabox}[title=Converting Solubility to Mass Fraction]
Solubility is often reported as a mass ratio. To use it in a mass balance, it must be converted to a mass fraction of the solute in the saturated solution.
\begin{itemize}[itemsep=2pt]
    \item \textbf{Common Unit}: Solubility = $\frac{\text{grams of solute}}{100 \text{ grams of solvent}}$
    \item \textbf{Conversion Formula}:
    $$ \text{Mass Fraction }(x) = \frac{\text{mass of solute}}{\text{mass of solute} + \text{mass of solvent}} $$
\end{itemize}
For example, if the solubility of NaCl is 40 g NaCl / 100 g H$_2$O, the mass fraction is:
$$ x_{NaCl} = \frac{40}{40 + 100} = \frac{40}{140} \approx 0.286 $$
\end{formulabox}

\begin{examplebox}{Crystallizer Material Balance}
A saturated aqueous solution of potassium dichromate (K$_2$Cr$_2$O$_7$) at 60$^\circ$C is fed to a crystallizer that operates at 20$^\circ$C. The outlet slurry is filtered, yielding 200 kg of solid K$_2$Cr$_2$O$_7$ crystals and 400 kg of saturated solution at 20$^\circ$C. How much water was evaporated in the crystallizer?
\end{examplebox}
\begin{stepbox}
\begin{enumerate}[label=\textbf{Step \arabic*:}, wide=0pt, leftmargin=*, itemsep=2pt]
    \item \textbf{Find Compositions from Solubility Data}
    
    We use a standard solubility chart for K$_2$Cr$_2$O$_7$ to find the compositions of the inlet and outlet solutions.
    \begin{itemize}[itemsep=2pt]
        \item \textbf{Inlet (60$^\circ$C)}: Solubility $\approx$ 38 g K$_2$Cr$_2$O$_7$ / 100 g H$_2$O.
        $$ x_{in} = \frac{38}{38+100} = \frac{38}{138} = 0.2754 $$
        \item \textbf{Outlet Solution (20$^\circ$C)}: Solubility $\approx$ 12 g K$_2$Cr$_2$O$_7$ / 100 g H$_2$O.
        $$ x_{out,sol} = \frac{12}{12+100} = \frac{12}{112} = 0.1071 $$
    \end{itemize}
    
    \item \textbf{Draw and Label a Process Flow Diagram}
    
    \begin{itemize}[itemsep=2pt]
        \item \textbf{Inlet}: One stream, the feed ($m_{feed}$), with a K$_2$Cr$_2$O$_7$ mass fraction of $x_{in}=0.2754$.
        \item \textbf{Outlets}: Three streams.
            \begin{enumerate}[label=\alph*)]
                \item Evaporated pure water ($m_{evap}$).
                \item Solid crystals ($m_{crys} = 200$ kg), which are pure K$_2$Cr$_2$O$_7$ (mass fraction = 1).
                \item Saturated solution ($m_{sol} = 400$ kg) with a K$_2$Cr$_2$O$_7$ mass fraction of $x_{out,sol}=0.1071$.
            \end{enumerate}
    \end{itemize}
\end{enumerate}
\end{stepbox}

\newpage
\begin{stepbox}
\begin{enumerate}[label=\textbf{Step \arabic*:}, wide=0pt, leftmargin=*, itemsep=2pt, start=3]
    \item \textbf{Set up Material Balances}
    
    We have two unknown flow rates ($m_{feed}$ and $m_{evap}$), so we need to set up two independent balance equations.
    
    \begin{itemize}[itemsep=2pt]
        \item \textbf{Overall Mass Balance} (Mass In = Mass Out):
        $$ m_{feed} = m_{evap} + m_{crys} + m_{sol} $$
        $$ m_{feed} = m_{evap} + 200 + 400 \implies m_{feed} = m_{evap} + 600 $$
        
        \item \textbf{Water Balance} (Water In = Water Out):
        $$ (\text{Water in feed}) = (\text{Water evaporated}) + (\text{Water in solution}) $$
        $$ m_{feed} \times (1 - x_{in}) = m_{evap} + m_{sol} \times (1 - x_{out,sol}) $$
        $$ m_{feed} \times (1 - 0.2754) = m_{evap} + 400 \times (1 - 0.1071) $$
        $$ 0.7246 \cdot m_{feed} = m_{evap} + (400 \times 0.8929) $$
        $$ 0.7246 \cdot m_{feed} = m_{evap} + 357.16 $$
    \end{itemize}
    
    \item \textbf{Solve the System of Equations}
    
    Substitute the expression for $m_{feed}$ from the overall balance into the water balance.
    $$ 0.7246 (m_{evap} + 600) = m_{evap} + 357.16 $$
    $$ 0.7246 \cdot m_{evap} + 434.76 = m_{evap} + 357.16 $$
    Rearrange to solve for $m_{evap}$:
    $$ 434.76 - 357.16 = m_{evap} - 0.7246 \cdot m_{evap} $$
    $$ 77.6 = 0.2754 \cdot m_{evap} $$
    $$ m_{evap} = \frac{77.6}{0.2754} \approx 281.8 \, \text{kg} $$
    
    \item \textbf{Final Answer}
    
    Approximately \textbf{282 kg} of water must be evaporated in the crystallizer.
\end{enumerate}
\end{stepbox}

\newpage
\section*{Phase Behavior Near the Critical Point}

\begin{conceptbox}
The critical point represents the terminus of the vapor-liquid equilibrium curve on a phase diagram. Beyond this point, the distinction between liquid and vapor phases ceases to exist, and the substance enters the supercritical fluid phase, which has unique properties combining those of liquids and gases.
\end{conceptbox}
\begin{keybox}[title=Key Definitions]
\begin{itemize}[itemsep=2pt]
    \item \textbf{Critical Point}: The specific temperature ($T_c$) and pressure ($P_c$) for a substance at which the liquid and vapor phases become identical and indistinguishable.
    \item \textbf{Supercritical Fluid}: A substance at a temperature and pressure above its critical point ($T > T_c$ and $P > P_c$). It has a density similar to a liquid but the viscosity and diffusivity of a gas.
\end{itemize}
\end{keybox}

\begin{conceptbox}[title=Phenomena at the Critical Point]
The transition into and out of the supercritical state involves unique and observable phenomena.
\begin{itemize}[itemsep=2pt]
    \item \textbf{Heating to the Critical Point}: Imagine a sealed container holding a liquid and its vapor, with the overall density being exactly the critical density. As the system is heated towards the critical temperature, the liquid expands (density decreases) and the vapor is compressed (density increases). The properties of the two phases converge. The meniscus—the visible boundary between the liquid and vapor—becomes progressively fainter and then vanishes completely at the critical point, as the system becomes one uniform, homogeneous supercritical fluid.
    
    \item \textbf{Cooling from the Supercritical State}: If a supercritical fluid is cooled back down through its critical point, it exhibits a phenomenon called \textbf{critical opalescence}. At the instant of passing through the critical point, large-scale density fluctuations form spontaneously throughout the fluid. These fluctuations have a size comparable to the wavelength of visible light, causing them to scatter light intensely. This makes the fluid, for a moment, appear milky, cloudy, or opaque. Immediately after this flash of opalescence, the fluid separates back into distinct liquid and vapor phases.
\end{itemize}
\end{conceptbox}

\end{document}
