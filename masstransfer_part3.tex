\documentclass[12pt]{article}
\usepackage[paperwidth=8.5in, paperheight=11in, margin=1.0in, headheight=15pt]{geometry}
\usepackage{amsmath,amssymb,amsthm}
\usepackage[most]{tcolorbox}
\usepackage{enumitem}
\usepackage{xcolor}
\usepackage{hyperref}
\usepackage{fancyhdr}
\usepackage{titlesec}
\usepackage{graphicx}
% Define custom colors for chemical engineering theme
\definecolor{conceptcolor}{RGB}{52, 73, 94}      % Dark blue-gray
\definecolor{formulacolor}{RGB}{231, 76, 60}     % Red for formulas
\definecolor{examplecolor}{RGB}{39, 174, 96}     % Green for examples
\definecolor{stepcolor}{RGB}{142, 68, 173}       % Purple for solution steps
\definecolor{keycolor}{RGB}{243, 156, 18}        % Orange for key points
% Configure fancy headers
\pagestyle{fancy}
\fancyhf{}
\fancyhead[L]{PE Study Guide}
\fancyhead[R]{Process Fundamentals}
\fancyfoot[C]{\thepage}
\renewcommand{\baselinestretch}{1.1}
\setlength{\parindent}{0.25in}
\setlength{\parskip}{3pt}
% Configure section formatting
\titleformat{\section}
  {\normalfont\LARGE\bfseries\color{conceptcolor}}
  {\thesection}{1em}{}
\titleformat{\subsection}
  {\normalfont\Large\bfseries\color{conceptcolor}}
  {\thesubsection}{1em}{}
% Define custom environments
\newtcolorbox{conceptbox}[1][]{
  enhanced,
  colback=conceptcolor!10,
  colframe=conceptcolor,
  arc=3mm,
  title=Key Concept,
  fonttitle=\bfseries\sffamily\normalsize,
  fontupper=\small,
  #1
}
\newtcolorbox{formulabox}[1][]{
  enhanced,
  colback=formulacolor!10,
  colframe=formulacolor,
  arc=2mm,
  title=Important Formula,
  fonttitle=\bfseries\sffamily\normalsize,
  fontupper=\small,
  #1
}
\newtcolorbox{examplebox}[2][]{
  enhanced,
  colback=examplecolor!10,
  colframe=examplecolor,
  arc=3mm,
  title=Example Problem: #2,
  fonttitle=\bfseries\sffamily\normalsize,
  fontupper=\small,
  #1
}
\newtcolorbox{stepbox}[1][]{
  enhanced,
  colback=stepcolor!10,
  colframe=stepcolor,
  arc=2mm,
  title=Solution Steps,
  fonttitle=\bfseries\sffamily\normalsize,
  fontupper=\small,
  #1
}
\newtcolorbox{keybox}[1][]{
  enhanced,
  colback=keycolor!10,
  colframe=keycolor,
  arc=2mm,
  title=Key Variables \& Definitions,
  fonttitle=\bfseries\sffamily\normalsize,
  fontupper=\small,
  #1
}

\section*{Gas Absorption Columns}
\textbf{Gas Absorption} is a mass transfer operation used to selectively remove one or more components (called \textbf{solutes}) from a gas stream by contacting it with a liquid \textbf{solvent}. The process works because the solute is more soluble in the liquid solvent than the other gases in the stream. A common industrial application is "sour gas sweetening," where acidic gases like H$_2$S are removed from natural gas using an amine solvent.

\begin{conceptbox}[title=The Principle of Gas Absorption]
The operation is typically carried out in a tower containing trays or packing where the gas stream flows upwards and the liquid solvent flows downwards. This \textbf{counter-current} flow maximizes the concentration difference (the driving force) between the gas and liquid at every point in the column, leading to the most efficient mass transfer of the solute from the gas phase to the liquid phase.
\end{conceptbox}

\subsection*{Fundamental Principles and Equations}
The design of an absorption column involves combining thermodynamic equilibrium relationships with overall material balances.

\begin{conceptbox}[title=Equilibrium Relationship]
On each theoretical stage of the column, the exiting gas and liquid streams are assumed to reach thermodynamic equilibrium. For dilute solutions, this relationship is described by \textbf{Henry's Law}, which states that the partial pressure of a solute in the gas phase is proportional to its concentration in the liquid phase.
\end{conceptbox}

\begin{formulabox}[title=Henry's Law and Equilibrium Constant]
The equilibrium relationship can be expressed in terms of mole fractions:
$$ y_n = \left(\frac{H}{P}\right) x_n \quad \text{or simply} \quad y_n = m \cdot x_n $$
The temperature dependence of Henry's constant is often described by:
$$ H(T) = H^0 \exp\left[-\frac{E}{R}\left(\frac{1}{T} - \frac{1}{T_0}\right)\right] $$
\end{formulabox}

\begin{keybox}
\begin{itemize}[itemsep=2pt]
    \item $y_n, x_n$: Mole fractions of the solute in the gas and liquid phases leaving stage $n$.
    \item $H$: Henry's Law constant [Pressure units, e.g., atm].
    \item $P$: Total pressure of the system [Pressure units].
    \item $m$: The equilibrium constant ($m = H/P$) [dimensionless].
    \item $L, V$: Molar flow rates of the liquid and gas (often on a solute-free basis) [mol/time].
\end{itemize}
\end{keybox}

\begin{conceptbox}[title=The Operating Line]
The relationship between a gas stream rising and a liquid stream descending \textit{between} stages is not governed by equilibrium, but by a \textbf{material balance}. The operating line equation relates the composition of the gas entering a stage from below to the liquid entering it from above.
\end{conceptbox}

\begin{formulabox}[title=The Operating Line Equation]
A material balance for the solute from the top of the column down to an arbitrary stage $n$ gives:
$$ L x_0 + V y_{n+1} = L x_n + V y_1 $$
Rearranging this into the form of a line ($y=mx+b$) gives the operating line equation:
$$ y_{n+1} = \left(\frac{L}{V}\right)x_n + \left(y_1 - \frac{L}{V}x_0\right) $$
\end{formulabox}

\subsection*{Graphical Method for Stage Calculation}
Similar to the McCabe-Thiele method, we can use a graphical method on an x-y diagram to determine the number of theoretical stages required.

\begin{keybox}[title=Key Lines on the Absorption Diagram]
\begin{itemize}[itemsep=2pt]
    \item \textbf{Equilibrium Line:} This is a plot of the thermodynamic equilibrium relationship. For dilute systems following Henry's Law, this is a straight line through the origin with a slope of $m$, i.e., $y=mx$.
    \item \textbf{Operating Line:} This is a plot of the material balance equation. It is a straight line with a slope of $L/V$ that connects the compositions at the top of the column (point $(x_0, y_1)$) with the compositions at the bottom (point $(x_N, y_{N+1})$).
\end{itemize}
\end{keybox}



\begin{stepbox}[title=Graphical Procedure for Stage Calculation]
The number of stages is found by constructing a staircase between the operating line and the equilibrium line, typically starting from the bottom of the column.
\begin{enumerate}[label=\textbf{Step \arabic*:}, wide=0pt, leftmargin=*, itemsep=2pt]
    \item \textbf{Start at the Bottom:} Begin at the point $(x_N, y_{N+1})$ on the operating line. This represents the rich entering gas ($y_{N+1}$) and the rich exiting liquid ($x_N$).
    \item \textbf{Move to Equilibrium:} The vapor that leaves stage N ($y_N$) is in equilibrium with the liquid on stage N ($x_N$). To find its composition, move \textbf{vertically down} from the operating line to the \textbf{equilibrium line}.
    \item \textbf{Move to Operating Line:} The vapor leaving stage N ($y_N$) passes the liquid entering from the stage above ($x_{N-1}$). To find this liquid's composition, move \textbf{horizontally to the left} from the equilibrium line to the \textbf{operating line}.
    \item \textbf{Repeat:} This completes one "step" (one theoretical stage). Repeat this process—vertically to equilibrium, horizontally to the operating line—counting each step until the gas composition ($y$) is less than or equal to the desired exit gas composition, $y_1$.
\end{enumerate}
\end{stepbox}

\subsection*{Effect of Operating Parameters}
The graphical method provides an excellent way to visualize how changing operating conditions will impact the required number of stages for a given separation.

\begin{conceptbox}[title=Effect of Temperature (T)]
\begin{itemize}[itemsep=2pt]
    \item \textbf{What it Affects:} The \textbf{Equilibrium Line}.
    \item \textbf{How:} Gas solubility in a liquid \textit{decreases} as temperature increases. This increases the Henry's constant, $H$, and thus increases the slope of the equilibrium line ($m=H/P$).
    \item \textbf{Result:} A higher temperature moves the equilibrium line \textbf{closer} to the operating line. This reduces the size of the "steps" in the graphical construction, meaning \textbf{more stages are required}.
    \item \textbf{Conclusion:} Gas absorption is favored at \textbf{lower temperatures}.
\end{itemize}
\end{conceptbox}

\begin{conceptbox}[title=Effect of Pressure (P)]
\begin{itemize}[itemsep=2pt]
    \item \textbf{What it Affects:} The \textbf{Equilibrium Line}.
    \item \textbf{How:} Increasing the total system pressure $P$ increases the partial pressure of the solute, driving more of it into the liquid phase. This \textit{decreases} the slope of the equilibrium line ($m=H/P$).
    \item \textbf{Result:} A higher pressure moves the equilibrium line \textbf{further away} from the operating line. This increases the size of the steps, meaning \textbf{fewer stages are required}.
    \item \textbf{Conclusion:} Gas absorption is favored at \textbf{higher pressures}.
\end{itemize}
\end{conceptbox}

\begin{conceptbox}[title=Effect of Solvent Flow Rate (L/V Ratio)]
\begin{itemize}[itemsep=2pt]
    \item \textbf{What it Affects:} The \textbf{Operating Line}.
    \item \textbf{How:} The slope of the operating line is equal to the ratio of the liquid to gas molar flow rates, $L/V$.
    \item \textbf{Result:} Increasing the liquid solvent flow rate $L$ increases the slope of the operating line. This moves the operating line \textbf{further away} from the equilibrium line, meaning \textbf{fewer stages are required}.
    \item \textbf{Conclusion and Trade-off:} A higher solvent flow rate requires a smaller, cheaper column (lower capital cost) but costs more to operate (pumping and regenerating more solvent, higher operating cost). Engineers must choose an optimal $L/V$ ratio that balances these costs. The theoretical \textbf{minimum L/V ratio} would require an infinite number of stages.
\end{itemize}
\end{conceptbox}

\begin{conceptbox}[title=Effect of Inlet and Outlet Compositions]
    \begin{itemize}[itemsep=2pt]
    \item \textbf{What they Affect:} The \textbf{anchor points} of the Operating Line.
    \item \textbf{How:} The operating line must connect the point representing the top of the column $(x_0, y_1)$ to the point representing the bottom $(x_N, y_{N+1})$.
    \item \textbf{Result:} If a recycled solvent with some initial solute is used ($x_0 > 0$), the top of the operating line shifts to the right. If a stricter outlet gas specification is required (a lower $y_1$), the top of the operating line shifts down. Both changes move the operating line closer to the equilibrium line and \textbf{require more stages}.
    \end{itemize}
\end{conceptbox}

\subsection*{Example Problem: Dilute System - Chloroform Scrubbing}
\begin{examplebox}{Dilute System: Chloroform Scrubbing}
A laboratory process releases an air stream of 1000 kmol/hr containing 200 ppm of chloroform. To meet environmental standards, this concentration must be reduced to 10 ppm before release. An absorption column using pure water as the solvent is proposed. The system operates at 25$^\circ$C and 1.5 atm. How many theoretical equilibrium stages are required for this separation?
\end{examplebox}

\begin{stepbox}
\begin{enumerate}[label=\textbf{Step \arabic*:}, wide=0pt, leftmargin=*, itemsep=2pt]
    \item \textbf{Strategy and Assumptions}
    The first step is to recognize that with solute concentrations in the parts-per-million (ppm) range, this is a \textbf{dilute system}.
    \begin{conceptbox}[title=Simplifications for Dilute Systems]
    For dilute systems (typically when solute concentrations are below 1-2\% in both phases), we can make several key simplifications that make the analysis much easier:
    \begin{itemize}[itemsep=2pt]
        \item The total liquid ($L$) and gas ($V$) molar flow rates can be assumed to be constant throughout the column because the amount of solute transferred is negligible compared to the total flow.
        \item We can work with mole fractions ($x, y$) directly. The operating and equilibrium lines are straight, allowing for an analytical solution.
    \end{itemize}
    \end{conceptbox}
    Our plan is to determine the minimum liquid flow rate ($L_{min}$), select a practical operating flow rate, and then solve for the number of stages using both the graphical method and the analytical Kremser equation.

    \item \textbf{Define Knowns and Find Equilibrium Data}

    \textbf{Gas Flow Rate ($V$):} $1000$ kmol/hr.
    
    \textbf{Inlet Gas ($y_{N+1}$):} $200 \, \text{ppm} = 200 \times 10^{-6}$.
    
    \textbf{Outlet Gas ($y_1$):} $10 \, \text{ppm} = 10 \times 10^{-6}$.
    
    \textbf{Inlet Liquid ($x_0$):} $0$ (pure water solvent).
    
    \textbf{Henry's Constant ($H$):} $H = 211$ atm/mole fraction.
    
    \textbf{Equilibrium Line Slope ($m$):} $m = H/P = 211 \, \text{atm} / 1.5 \, \text{atm} = 140.7$.
    
    The equilibrium line is therefore $\mathbf{y = 140.7 x}$.
\end{enumerate}
\end{stepbox}

\begin{stepbox}
\begin{enumerate}[label=\textbf{Step \arabic*:}, wide=0pt, leftmargin=*, itemsep=2pt, start = 3]

    \item \textbf{Determine the Minimum Liquid Flow Rate ($L_{min}$)}
    The minimum liquid flow rate corresponds to an infinite number of stages, which occurs when the operating line touches the equilibrium line at the bottom (high-concentration) end of the column. At this "pinch point," the exiting liquid ($x_{N,max}$) is in equilibrium with the entering gas ($y_{N+1}$).
    $$ x_{N,max} = \frac{y_{N+1}}{m} = \frac{200 \times 10^{-6}}{140.7} = 1.42 \times 10^{-6} $$
    The slope of this minimum operating line is $(L/V)_{min}$, found from a balance over the whole column:
    $$ \left(\frac{L}{V}\right)_{min} = \frac{y_{N+1} - y_1}{x_{N,max} - x_0} = \frac{(200 - 10) \times 10^{-6}}{(1.42 - 0) \times 10^{-6}} = 133.8 $$
    $$ L_{min} = V \cdot (L/V)_{min} = 1000 \frac{\text{kmol}}{\text{hr}} \cdot 133.8 = 133,800 \frac{\text{kmol}}{\text{hr}} $$

    \item \textbf{Determine the Actual Operating Line}
    A common heuristic is to use an actual liquid flow rate $1.1$ to $2.0$ times the minimum. We will use $L_{actual} = 1.4 \cdot L_{min}$.
    $$ (L/V)_{actual} = 1.4 \cdot (L/V)_{min} = 1.4 \cdot 133.8 = 187.3 $$

    \item \textbf{Method A: Graphical Solution}
    We plot the equilibrium line ($y=140.7x$) and the operating line. The operating line has a slope of 187.3 and connects the top point $(x_0, y_1) = (0, 10\text{ppm})$ and the bottom point $(x_N, y_{N+1}) = (x_N, 200\text{ppm})$. By stepping off the stages graphically, we find that approximately \textbf{7.25 theoretical stages} are required. We would design for \textbf{8 stages}.

\end{enumerate}

\end{stepbox}

\begin{stepbox}
\begin{enumerate}[label=\textbf{Step \arabic*:}, wide=0pt, leftmargin=*, itemsep=2pt, start = 6]
    
    \item \textbf{Method B: Analytical Solution (Kremser Equation)}
    For dilute systems with linear equilibrium and operating lines, the Kremser equation provides a direct analytical solution.
    \begin{formulabox}[title=The Kremser Equation]
    $$ N = \frac{\ln\left[ \frac{y_{N+1} - mx_0}{y_1 - mx_0} \left(1 - \frac{1}{A}\right) + \frac{1}{A} \right]}{\ln(A)} $$
    where $A$ is the dimensionless \textbf{absorption factor}, $A = \frac{L}{mV}$.
    \end{formulabox}
    First, calculate the absorption factor:
    $$ A = \frac{L/V}{m} = \frac{187.3}{140.7} = 1.33 $$
    Since we are using pure solvent ($x_0 = 0$), the equation simplifies:
    $$ N = \frac{\ln\left[ \frac{y_{N+1}}{y_1} \left(1 - \frac{1}{A}\right) + \frac{1}{A} \right]}{\ln(A)} = \frac{\ln\left[ \frac{200}{10} \left(1 - \frac{1}{1.33}\right) + \frac{1}{1.33} \right]}{\ln(1.33)} $$
    $$ N = \frac{\ln[20(0.248) + 0.752]}{\ln(1.33)} = \frac{\ln(5.712)}{0.285} \approx 6.1 $$
\end{enumerate}

    \begin{keybox}[title=Final Answer Summary for Example 1]
    The Kremser equation predicts that \textbf{6.1 theoretical stages} are required. We would round up and design the column for \textbf{7 stages}. The small difference between the graphical and analytical methods is expected and acceptable.
    \end{keybox}
\end{stepbox}


\newpage

\subsection*{Example Problem: Concentrated System - CO$_2$ Removal}
\begin{examplebox}{Concentrated System: CO$_2$ Removal}
We wish to remove 65\% of the CO$_2$ from a 100 mol/hr gas stream that is initially 8 mol\% CO$_2$ in nitrogen. The scrubbing is done with pure water at room temperature and atmospheric pressure, where the VLE relationship is approximately $y=1640x$. How many equilibrium stages are required if the solvent flow rate is 1.5 times the minimum?
\end{examplebox}

\begin{stepbox}
\begin{enumerate}[label=\textbf{Step \arabic*:}, wide=0pt, leftmargin=*, itemsep=2pt]
    \item \textbf{Strategy: Analysis for a Concentrated System}
    First, we must determine if the system is dilute. A significant change in the total molar flow rate means we cannot use the simplifying assumptions of the dilute case.
    \begin{itemize}[itemsep=2pt]
        \item Moles CO$_2$ entering: $100 \, \text{mol/hr} \times 0.08 = 8.0$ mol/hr.
        \item Moles CO$_2$ removed: $8.0 \, \text{mol/hr} \times 0.65 = 5.2$ mol/hr.
        \item Percent change in total gas flow rate: $\frac{5.2 \, \text{mol/hr removed}}{100 \, \text{mol/hr total in}} = 5.2\%$.
    \end{itemize}
    \begin{conceptbox}[title=Rule of Thumb: Concentrated Systems]
    When the amount of solute transferred causes the total molar flow rate of either the gas or liquid phase to change by more than 5-10\%, the system should be treated as \textbf{concentrated}. This requires using constant \textbf{solute-free flow rates} (G for carrier gas, S for solvent) and compositions expressed as \textbf{mole ratios} (Y, X).
    \end{conceptbox}

    \item \textbf{Define Solute-Free Flow Rates and Mole Ratios}
    \begin{itemize}[itemsep=2pt]
        \item \textbf{Carrier Gas Flow ($G$):} The inert N$_2$ flow is constant: $G = 100 \cdot (1-0.08) = 92$ mol/hr.
        \item \textbf{Solvent Flow ($S$):} The flow of pure water, to be determined.
        \item \textbf{Mole Ratios:} $Y = \frac{\text{moles CO}_2}{\text{moles N}_2}$, $X = \frac{\text{moles CO}_2}{\text{moles H}_2\text{O}}$.
        \item \textbf{Known Compositions in Mole Ratios:}
            \begin{itemize}
                \item Inlet gas ($Y_{N+1}$): $Y_{N+1} = \frac{y_{in}}{1-y_{in}} = \frac{0.08}{1-0.08} = 0.0870$.
                \item Outlet gas ($Y_1$): Moles CO$_2$ out = $8.0 - 5.2 = 2.8$ mol/hr. $Y_1 = \frac{2.8}{92} = 0.0304$.
                \item Inlet liquid ($X_0$): Pure water, so $X_0 = 0$.
            \end{itemize}
    \end{itemize}
\end{enumerate}
\end{stepbox}


\begin{stepbox}
\begin{enumerate}[label=\textbf{Step \arabic*:}, wide=0pt, leftmargin=*, itemsep=2pt, start = 3]
 
    \item \textbf{Convert Equilibrium Data to Mole Ratios}
    The VLE data ($y = 1640x$) must be converted to mole ratios using the relations $Y = y/(1-y)$ and $X = x/(1-x)$. This results in a \textit{curved} equilibrium line on a Y-X plot.
    
    \item \textbf{Determine the Operating Line}
    The operating line on a Y-X plot is a straight line with slope $S/G$. We first find the minimum slope by finding the pinch point at the bottom of the column.
    \begin{itemize}[itemsep=2pt]
        \item At the pinch, the exiting liquid $X_{N,max}$ is in equilibrium with the entering gas $Y_{N+1}=0.0870$.
        \item Convert to mole fractions to use VLE data: $y = Y/(1+Y) = 0.0870/(1.0870) = 0.0800$.
        \item Find equilibrium liquid mole fraction: $x = y/1640 = 0.0800/1640 = 4.88 \times 10^{-5}$.
        \item Convert back to mole ratio: $X_{N,max} = x/(1-x) \approx 4.88 \times 10^{-5}$.
        \item $(S/G)_{min} = \frac{Y_{N+1} - Y_1}{X_{N,max} - X_0} = \frac{0.0870 - 0.0304}{4.88 \times 10^{-5} - 0} = 1160$.
        \item $(S/G)_{actual} = 1.5 \cdot (S/G)_{min} = 1.5 \cdot 1160 = 1740$.
    \end{itemize}
    
    \item \textbf{Graphical Solution}
    We plot the curved equilibrium line and the straight operating line (slope=1740, passing through $(X_0, Y_1) = (0, 0.0304)$). We then step off the stages starting from the bottom point $(X_N, Y_{N+1})$. By constructing the staircase on the Y-X diagram, we find that we cross the target outlet gas composition of $Y_1 = 0.0304$ after approximately \textbf{1.9 equilibrium stages}.
\end{enumerate}
\begin{keybox}[title=Final Answer Summary for Example 2]
The graphical analysis shows that 1.9 equilibrium stages are needed. Therefore, we would design the column with \textbf{2 theoretical stages}.
\end{keybox}
\end{stepbox}

\newpage

\section*{Stripping Columns}
\textbf{Stripping}, also known as desorption, is a mass transfer operation that is the reverse of absorption. Its purpose is to selectively remove a dissolved solute from a liquid stream by contacting it with a gas, known as the \textbf{stripping agent}. 

\begin{conceptbox}[title=The Principle of Stripping]
Like absorption, stripping is typically carried out in a counter-current column. The solute-rich liquid enters at the top, and the clean stripping gas enters at the bottom. The counter-current flow maximizes the concentration difference at every point, providing the driving force for the solute to transfer \textbf{from the liquid phase to the gas phase}.
\end{conceptbox}

\subsection*{Fundamental Principles and Equations}
The design equations for stripping are identical in form to those for absorption; however, the relative positions of the operating and equilibrium lines are different, and the direction of mass transfer is from liquid to gas.

\begin{formulabox}[title=Key Stripping Equations]
\textbf{Equilibrium Relationship (Henry's Law):}
For dilute solutions, the equilibrium between the liquid and gas on a given stage is described by Henry's Law.
$$ y_n = m \cdot x_n \quad \text{where } m = H/P $$

\textbf{The Operating Line (Material Balance):}
A material balance on the solute relates the compositions of the passing streams between stages.
$$ y_{n+1} = \left(\frac{L}{V}\right)x_n + \left(y_1 - \frac{L}{V}x_0\right) $$
\end{formulabox}

\begin{keybox}
\begin{itemize}[itemsep=2pt]
    \item $x_0$: Mole fraction of solute in the \textbf{entering liquid feed} (at the top, stage 1).
    \item $x_N$: Mole fraction of solute in the \textbf{exiting stripped liquid} (at the bottom, stage N).
    \item $y_{N+1}$: Mole fraction of solute in the \textbf{entering stripping gas} (at the bottom). Often this is 0.
    \item $y_1$: Mole fraction of solute in the \textbf{exiting solute-rich gas} (at the top).
    \item $L, V$: Molar flow rates of the liquid and gas [mol/time].
\end{itemize}
\end{keybox}

\subsection*{Graphical Method for Stripping Columns}
\begin{conceptbox}[title=Key Feature of the Stripping Diagram]
The number of theoretical stages for a stripping column can be found graphically on an x-y diagram. The key difference from absorption is that the \textbf{operating line must lie below the equilibrium line}. This ensures that at every point in the column, the actual vapor-phase concentration ($y$) is lower than the equilibrium concentration, providing a driving force for the solute to move from the liquid to the gas phase.
\end{conceptbox}

\begin{stepbox}[title=Graphical Procedure for Stage Calculation]
The number of stages is found by constructing a staircase between the equilibrium line and the operating line. It is often intuitive to start from the top of the column (stage 1).
\begin{enumerate}[label=\textbf{Step \arabic*:}, wide=0pt, leftmargin=*, itemsep=2pt]
    \item \textbf{Plot the Lines:} Draw the equilibrium line ($y=mx$) and the operating line, which is the straight line connecting the known top-of-column point $(x_0, y_1)$ and bottom-of-column point $(x_N, y_{N+1})$.
    \item \textbf{Begin at Stage 1:} The liquid entering the column is $x_0$. This liquid enters Stage 1. Find the point $(x_0, y_1)$ on the diagram. This point is on the operating line. The liquid leaving stage 1 ($x_1$) is in equilibrium with the vapor leaving stage 1 ($y_1$). To find $x_1$, start at $y_1$ and move \textbf{horizontally to the right} to the \textbf{equilibrium line}. The x-coordinate of this point is $x_1$.
    \item \textbf{Move to Stage 2:} The liquid $x_1$ flows from stage 1 to stage 2. The vapor that it passes, which is rising from stage 2, is $y_2$. Their compositions are related by the operating line. To find $y_2$, start at $x_1$ and move \textbf{vertically up} to the \textbf{operating line}. The y-coordinate is $y_2$.
    \item \textbf{Repeat:} Now find the liquid on stage 2 ($x_2$) that is in equilibrium with $y_2$ by moving \textbf{horizontally to the right} to the \textbf{equilibrium line}. This completes the step for Stage 2. Continue this process—vertically to the operating line, horizontally to the equilibrium line—counting each "triangle" as one stage, until the liquid composition $x$ is less than or equal to the desired exit liquid composition, $x_N$.
\end{enumerate}
\end{stepbox}

\newpage

\subsection*{Effect of Operating Parameters}
The performance of a stripping column is highly sensitive to the operating conditions. Understanding these effects is key to efficient design.

\begin{conceptbox}[title=Effect of Temperature (T)]
\begin{itemize}[itemsep=2pt]
    \item \textbf{What it Affects:} The \textbf{Equilibrium Line}.
    \item \textbf{How:} Increasing temperature generally \textit{decreases} the solubility of a gas in a liquid, making it easier to strip out. This increases the Henry's constant, $H$, and thus the slope of the equilibrium line ($m=H/P$).
    \item \textbf{Result:} A higher temperature moves the equilibrium line \textbf{further away} from the operating line. This increases the size of the graphical steps, meaning \textbf{fewer stages are required}.
    \item \textbf{Conclusion:} Stripping is favored at \textbf{higher temperatures}.
\end{itemize}
\end{conceptbox}

\begin{conceptbox}[title=Effect of Pressure (P)]
\begin{itemize}[itemsep=2pt]
    \item \textbf{What it Affects:} The \textbf{Equilibrium Line}.
    \item \textbf{How:} The slope of the equilibrium line, $m=H/P$, is inversely proportional to the total pressure $P$.
    \item \textbf{Result:} Decreasing the system pressure increases the slope of the equilibrium line, moving it further away from the operating line. This means \textbf{fewer stages are required}.
    \item \textbf{Conclusion:} Stripping is favored at \textbf{lower pressures}.
\end{itemize}
\end{conceptbox}

\begin{conceptbox}[title=Effect of Stripping Gas Flow Rate (V)]
\begin{itemize}[itemsep=2pt]
    \item \textbf{What it Affects:} The \textbf{Operating Line}.
    \item \textbf{How:} The slope of the operating line is $L/V$. Increasing the stripping gas flow rate $V$ (for a fixed liquid rate $L$) \textit{decreases} the slope of the operating line.
    \item \textbf{Result:} A lower slope moves the operating line \textbf{further away} from the equilibrium line. This increases the size of the steps, meaning \textbf{fewer stages are required}.
    \item \textbf{Conclusion and Trade-off:} A higher stripping gas flow rate requires a smaller, cheaper column (lower capital cost) but has a higher operating cost (compressing and supplying more gas). The theoretical \textbf{minimum stripping gas rate} corresponds to an operating line that touches the equilibrium line (a "pinch point"), which would require an infinite number of stages. Engineers must choose an optimal V that balances these costs.
\end{itemize}
\end{conceptbox}

\newpage

\subsection*{Example Problem: Stripping an Organic from Water}
\begin{examplebox}{Single-Stage Stripping of an Organic from Water}
A single-stage stripping process is used to remove a dissolved organic compound from a water stream. The contaminated water enters the stage at 20$^\circ$C with an organic concentration of 0.05 mol\%. A stream of pure, dry air at 5 bar is contacted with the water to act as the stripping agent. The Henry's Law constant for the organic in water at this temperature is 2.5 bar. What flow rate of air (in moles of air per mole of water) is needed to reduce the amount of organic in the water by 95\%?
\end{examplebox}



\begin{stepbox}
\begin{enumerate}[label=\textbf{Step \arabic*:}, wide=0pt, leftmargin=*, itemsep=2pt]
    \item \textbf{Strategy and Process Basis}
    The core of this problem is a material balance on the organic solute around the single equilibrium stage. We will relate the compositions of the exiting liquid and vapor streams using Henry's Law. Since the air flow rate is the unknown, we will solve the material balance for this quantity. Let's first establish a basis for our calculation.
    \begin{keybox}[title=Calculation Basis and Streams]
    \begin{itemize}[itemsep=2pt]
        \item \textbf{Basis:} 1 mole of contaminated water entering the process ($L_{in} = 1$ mol).
        \item \textbf{Liquid In ($L_{in}$):} Contains water and the organic solute.
        \item \textbf{Gas In ($V_{in}$):} Contains $n_{air}$ moles of pure, dry air.
        \item \textbf{Liquid Out ($L_{out}$):} Contains the remaining water and organic. Assumed to be in equilibrium with $V_{out}$.
        \item \textbf{Gas Out ($V_{out}$):} Contains the inlet air plus the stripped organic. Assumed to be in equilibrium with $L_{out}$.
    \end{itemize}
    \end{keybox}
    Our goal is to find the ratio $n_{air} / L_{in}$. Since our basis is $L_{in}=1$ mol, we just need to find the value of $n_{air}$.


\end{enumerate}
\end{stepbox}

\begin{stepbox}
\begin{enumerate}[label=\textbf{Step \arabic*:}, wide=0pt, leftmargin=*, itemsep=2pt, start = 2]
    
    \item \textbf{Material Balance on the Organic Solute}
    First, let's quantify the moles of the organic contaminant (C) entering and leaving based on our 1-mole basis.
    \begin{itemize}[itemsep=2pt]
        \item \textbf{Moles of C IN:} The inlet water has a concentration of 0.05 mol\%, which is a mole fraction of $x_{C,in} = 0.0005$.
        $$ n_{C,in} = x_{C,in} \cdot L_{in} = (0.0005) \cdot (1 \, \text{mol}) = 5 \times 10^{-4} \, \text{mol} $$
        \item \textbf{Moles of C OUT (in Liquid):} The process removes 95\% of the organic, meaning 5\% remains in the liquid.
        $$ n_{C,L,out} = 0.05 \cdot n_{C,in} = 0.05 \cdot (5 \times 10^{-4}) = 2.5 \times 10^{-5} \, \text{mol} $$
        \item \textbf{Moles of C OUT (in Gas):} By difference, 95\% of the inlet organic must have been transferred to the air stream.
        $$ n_{C,V,out} = 0.95 \cdot n_{C,in} = 0.95 \cdot (5 \times 10^{-4}) = 4.75 \times 10^{-4} \, \text{mol} $$
    \end{itemize}
    
    \item \textbf{State Simplifying Assumptions}
    \begin{conceptbox}[title=Key Simplifying Assumptions]
    To simplify the calculation, we will make two initial assumptions, which are common for processes with very low solute concentrations. We will check their validity later.
    \begin{enumerate}[itemsep=2pt]
        \item The total molar flow rate of the liquid stream is approximately constant ($L_{out} \approx L_{in} = 1$ mol).
        \item The total molar flow rate of the gas stream is approximately constant ($V_{out} \approx V_{in} = n_{air}$). This implies that both the amount of solute transferred and the amount of water evaporated are negligible compared to the air flow.
    \end{enumerate}
    \end{conceptbox}


\end{enumerate}
\end{stepbox}

\begin{stepbox}
\begin{enumerate}[label=\textbf{Step \arabic*:}, wide=0pt, leftmargin=*, itemsep=2pt, start = 4]
        \item \textbf{Use Equilibrium Relationship (Henry's Law)}
    Because this is an equilibrium stage, the exiting liquid and gas streams are in thermodynamic equilibrium. We can relate their compositions using Henry's Law.
    \begin{formulabox}
    $$ y_C \cdot P_{total} = x_C \cdot H_C $$
    \end{formulabox}
    First, let's find the mole fraction in the exiting liquid, $x_{C,out}$, using our simplifying assumptions:
    $$ x_{C,out} = \frac{\text{moles of C in liquid out}}{\text{total moles of liquid out}} \approx \frac{n_{C,L,out}}{L_{in}} = \frac{2.5 \times 10^{-5} \, \text{mol}}{1 \, \text{mol}} = 2.5 \times 10^{-5} $$
    Now, use Henry's Law to find the corresponding equilibrium mole fraction in the exiting gas, $y_{C,out}$:
    $$ y_{C,out} = \frac{x_{C,out} \cdot H_C}{P_{total}} = \frac{(2.5 \times 10^{-5}) \cdot (2.5 \, \text{bar})}{5 \, \text{bar}} = 1.25 \times 10^{-5} $$

    \item \textbf{Calculate the Required Air Flow Rate}
    We know the total moles of contaminant leaving in the gas stream ($n_{C,V,out}$) and its mole fraction in that stream ($y_{C,out}$). We can use these to find the total moles of the exiting gas stream, $V_{out}$.
    $$ n_{C,V,out} = y_{C,out} \cdot V_{out} $$
    $$ 4.75 \times 10^{-4} \, \text{mol} = (1.25 \times 10^{-5}) \cdot V_{out} $$
    $$ V_{out} = \frac{4.75 \times 10^{-4}}{1.25 \times 10^{-5}} = 38 \, \text{moles} $$
    Based on our simplifying assumption that $V_{out} \approx V_{in} = n_{air}$, the required air flow rate is 38 moles.

\end{enumerate}
\end{stepbox}

\begin{stepbox}
\begin{enumerate}[label=\textbf{Step \arabic*:}, wide=0pt, leftmargin=*, itemsep=2pt, start = 6]

    \item \textbf{Evaluate the Assumptions}
    Is it valid to assume the total flow rates are constant? Let's check the two effects we ignored.
    \begin{itemize}[itemsep=2pt]
        \item \textbf{Solute Transfer:} The amount of organic transferred ($4.75 \times 10^{-4}$ mol) is tiny compared to the water (1 mol) and air (38 mol) flows. This part of the assumption is excellent.
        \item \textbf{Water Evaporation:} Dry air enters and will become saturated with water vapor. The exiting liquid is almost pure water, so $x_W \approx 1$. The saturation pressure of water at 20$^\circ$C is $P_{W,sat} \approx 0.0234$ bar. The mole fraction of water in the exit gas will be:
        $$ y_W = \frac{x_W P_{W,sat}}{P_{total}} \approx \frac{1 \cdot (0.0234 \, \text{bar})}{5 \, \text{bar}} = 0.00468 $$
        The moles of water that evaporate into the exiting gas stream is:
        $$ n_{W,evap} = \frac{y_W}{1-y_W-y_C} \cdot (n_{air}+n_{C,V,out}) \approx y_W \cdot V_{out} \approx (0.00468) \cdot (38) = 0.178 \, \text{mol} $$
        The total exiting gas flow is actually $V_{out} = n_{air} + n_{C,V,out} + n_{W,evap} \approx 38 + 0.000475 + 0.178 \approx 38.18$ moles. This is very close to our assumed 38 moles, so the assumption that $V_{out} \approx V_{in}$ is reasonable for this calculation. However, the water loss ($0.178$ mol) is nearly 18\% of the inlet liquid, which might be significant depending on the process objectives.
    \end{itemize}

\end{enumerate}
\end{stepbox}

\begin{keybox}[title=Final Answer Summary]
To achieve 95\% removal of the organic compound, approximately \textbf{38 moles of air are needed per mole of water}. The high ratio of air to water is required because the organic is not highly volatile out of water.
\end{keybox}

\end{document}
