\documentclass[12pt]{article}
\usepackage[paperwidth=8.5in, paperheight=11in, margin=1.0in, headheight=15pt]{geometry}
\usepackage{amsmath,amssymb,amsthm}
\usepackage[most]{tcolorbox}
\usepackage{enumitem}
\usepackage{xcolor}
\usepackage{hyperref}
\usepackage{fancyhdr}
\usepackage{titlesec}
\usepackage{graphicx}
% Define custom colors for chemical engineering theme
\definecolor{conceptcolor}{RGB}{52, 73, 94}      % Dark blue-gray
\definecolor{formulacolor}{RGB}{231, 76, 60}     % Red for formulas
\definecolor{examplecolor}{RGB}{39, 174, 96}     % Green for examples
\definecolor{stepcolor}{RGB}{142, 68, 173}       % Purple for solution steps
\definecolor{keycolor}{RGB}{243, 156, 18}        % Orange for key points
% Configure fancy headers
\pagestyle{fancy}
\fancyhf{}
\fancyhead[L]{PE Study Guide}
\fancyhead[R]{Process Fundamentals}
\fancyfoot[C]{\thepage}
\renewcommand{\baselinestretch}{1.1}
\setlength{\parindent}{0.25in}
\setlength{\parskip}{3pt}
% Configure section formatting
\titleformat{\section}
  {\normalfont\LARGE\bfseries\color{conceptcolor}}
  {\thesection}{1em}{}
\titleformat{\subsection}
  {\normalfont\Large\bfseries\color{conceptcolor}}
  {\thesubsection}{1em}{}
% Define custom environments
\newtcolorbox{conceptbox}[1][]{
  enhanced,
  colback=conceptcolor!10,
  colframe=conceptcolor,
  arc=3mm,
  title=Key Concept,
  fonttitle=\bfseries\sffamily\normalsize,
  fontupper=\small,
  #1
}
\newtcolorbox{formulabox}[1][]{
  enhanced,
  colback=formulacolor!10,
  colframe=formulacolor,
  arc=2mm,
  title=Important Formula,
  fonttitle=\bfseries\sffamily\normalsize,
  fontupper=\small,
  #1
}
\newtcolorbox{examplebox}[2][]{
  enhanced,
  colback=examplecolor!10,
  colframe=examplecolor,
  arc=3mm,
  title=Example Problem: #2,
  fonttitle=\bfseries\sffamily\normalsize,
  fontupper=\small,
  #1
}
\newtcolorbox{stepbox}[1][]{
  enhanced,
  colback=stepcolor!10,
  colframe=stepcolor,
  arc=2mm,
  title=Solution Steps,
  fonttitle=\bfseries\sffamily\normalsize,
  fontupper=\small,
  #1
}
\newtcolorbox{keybox}[1][]{
  enhanced,
  colback=keycolor!10,
  colframe=keycolor,
  arc=2mm,
  title=Key Variables \& Definitions,
  fonttitle=\bfseries\sffamily\normalsize,
  fontupper=\small,
  #1
}
\begin{document}
\section*{Log Mean Temperature Difference (LMTD)}
\subsection*{Fundamental Concept}
\begin{conceptbox}
In a heat exchanger, the temperature difference between the hot and cold fluids is not constant; it varies along the length of the exchanger. The Log Mean Temperature Difference ($\Delta T_{lm}$ or LMTD) is the appropriate \textit{average} temperature difference to use in the heat exchanger duty equation.
\end{conceptbox}
\begin{formulabox}
The total heat transfer rate in a heat exchanger is:
$$ q = U A \Delta T_{lm} $$
Where $U$ is the overall heat transfer coefficient and $A$ is the total heat transfer surface area.
\end{formulabox}

\subsection*{Calculating the LMTD}
\begin{keybox}[title=Key Variables]
\begin{itemize}[itemsep=0pt]
    \item $\Delta T_A$: The temperature difference between the hot and cold streams at one end of the exchanger (End A).
    \item $\Delta T_B$: The temperature difference between the hot and cold streams at the other end of the exchanger (End B).
\end{itemize}
\end{keybox}
\begin{formulabox}
The LMTD is calculated as the logarithmic average of the end-point temperature differences:
$$ \Delta T_{lm} = \frac{\Delta T_A - \Delta T_B}{\ln(\Delta T_A / \Delta T_B)} $$
\end{formulabox}

\subsection*{Application to Flow Configurations}
\begin{conceptbox}
The calculation of $\Delta T_A$ and $\Delta T_B$ depends on the flow configuration of the heat exchanger.
\begin{itemize}[itemsep=2pt]
    \item \textbf{Co-Current (Parallel) Flow}: Both fluids enter at the same end and flow in the same direction.
    \item \textbf{Counter-Current Flow}: The fluids enter at opposite ends and flow in opposite directions. This is the more efficient configuration as it yields a higher LMTD.
\end{itemize}
\end{conceptbox}
\begin{formulabox}[title=LMTD Calculations for Different Flows]
\begin{itemize}[itemsep=4pt]
    \item \textbf{For Co-Current Flow}:
    $$ \Delta T_A = T_{h,in} - T_{c,in} $$
    $$ \Delta T_B = T_{h,out} - T_{c,out} $$
    \item \textbf{For Counter-Current Flow}:
    $$ \Delta T_A = T_{h,in} - T_{c,out} $$
    $$ \Delta T_B = T_{h,out} - T_{c,in} $$
\end{itemize}
\end{formulabox}

\begin{keybox}[title=Important Design Considerations]
\begin{itemize}[itemsep=2pt]
    \item \textbf{Calculation Convention}: Always calculate the temperature differences at each end as $\Delta T = T_{hot} - T_{cold}$ to ensure positive values.
    \item \textbf{Verification}: It is good practice to manually verify LMTD values calculated by process simulation software.
    \item \textbf{Thermal Stress Heuristic}: An LMTD greater than $100^\circ$C can be a red flag in design. Large temperature differences can cause significant thermal stress on the equipment, potentially requiring more complex and expensive exchanger designs (e.g., Floating Head or U-tube) to accommodate the thermal expansion.
\end{itemize}
\end{keybox}

\newpage

\section*{Sizing a Heat Exchanger: Parallel Flow}
\subsection*{Objective}
\begin{conceptbox}
The goal of this calculation is to determine the required length ($L$) of a parallel-flow, concentric tube heat exchanger. This is a common design problem where the process requirements (temperatures, flow rates) are known, and a key physical dimension of the equipment must be calculated.
\end{conceptbox}

\subsection*{Problem Setup and Assumptions}
\begin{examplebox}{Concentric Tube Heat Exchanger Design}
Hot oil is cooled in a parallel-flow heat exchanger using water.
\begin{itemize}[itemsep=2pt]
    \item \textbf{Flow Configuration}: Parallel-flow.
    \item \textbf{Hot Fluid (Oil)}:
    \begin{itemize}[itemsep=0pt]
        \item Inlet Temperature, $T_{oil,in} = 100^\circ$C.
        \item Outlet Temperature, $T_{oil,out} = 60^\circ$C.
        \item Mass Flow Rate, $\dot{m}_{oil} = 0.15$ kg/s.
        \item Specific Heat, $c_{p,oil} = 2131$ J/kg$\cdot^\circ$C.
    \end{itemize}
    \item \textbf{Cold Fluid (Water)}:
    \begin{itemize}[itemsep=0pt]
        \item Inlet Temperature, $T_{water,in} = 25^\circ$C.
    \end{itemize}
    \item \textbf{Exchanger Properties}:
    \begin{itemize}[itemsep=0pt]
        \item Temperature Difference at Outlet, $\Delta T_2 = 10^\circ$C.
        \item Inner Tube Diameter, $D_i = 0.03$ m.
        \item Overall Heat Transfer Coefficient, $U = 38.1$ W/m$^2\cdot^\circ$C.
    \end{itemize}
\end{itemize}
\end{examplebox}
\begin{keybox}[title=Key Assumptions]
\begin{itemize}[itemsep=2pt]
    \item The heat exchanger is adiabatic (no heat loss to the surroundings).
    \item The conductive resistance of the inner tube wall is negligible (thin wall assumption).
\end{itemize}
\end{keybox}

\subsection*{Governing Equation and Solution Strategy}
\begin{conceptbox}
The fundamental design equation for a heat exchanger relates the heat transfer rate ($q$), the overall heat transfer coefficient ($U$), the surface area ($A$), and the log mean temperature difference ($\Delta T_{lm}$).
\end{conceptbox}
\begin{formulabox}
The design equation is:
$$ q = U A \Delta T_{lm} $$
For a concentric tube exchanger, the heat transfer area is $A = \pi D_i L$. We can rearrange the equation to solve for the required length, $L$.
$$ L = \frac{q}{U \pi D_i \Delta T_{lm}} $$
The strategy is to calculate each term on the right-hand side ($q$, $U$, $\Delta T_{lm}$) and then solve for $L$.
\end{formulabox}

\subsection*{Step-by-Step Calculation}
\begin{stepbox}[title=Step 1: Calculate the Heat Transfer Rate ($q$)]
\begin{conceptbox}
Under the adiabatic assumption, the heat lost by the hot fluid is equal to the heat gained by the cold fluid. We can calculate this heat duty using the information provided for the oil stream.
\end{conceptbox}
\begin{formulabox}
The energy balance on the oil stream is:
$$ q = \dot{m}_{oil} c_{p,oil} (T_{oil,in} - T_{oil,out}) $$
$$ q = (0.15 \, \text{kg/s}) \times (2131 \, \frac{\text{J}}{\text{kg}\cdot^\circ\text{C}}) \times (100^\circ\text{C} - 60^\circ\text{C}) $$
$$ q = (0.15) \times (2131) \times (40) = 12786 \, \text{W} $$
\end{formulabox}
\end{stepbox}

\begin{stepbox}[title=Step 2: Determine the Overall Heat Transfer Coefficient ($U$)]
\begin{conceptbox}
The overall heat transfer coefficient, $U$, accounts for all thermal resistances in series. Due to the thin wall assumption, we only consider the two convective resistances.
\end{conceptbox}
\begin{formulabox}
$$ \frac{1}{U} = R''_{total} = R''_{conv,oil} + R''_{conv,water} = \frac{1}{h_{oil}} + \frac{1}{h_{water}} $$
The problem provides the calculated value based on the fluid properties and flow conditions:
$$ U = 38.1 \, \frac{\text{W}}{\text{m}^2\cdot^\circ\text{C}} $$
\end{formulabox}
\end{stepbox}

\begin{stepbox}[title=Step 3: Calculate the Log Mean Temperature Difference ($\Delta T_{lm}$)]
\begin{conceptbox}
The LMTD is the appropriate average temperature difference for a heat exchanger. For parallel flow, the temperature differences are calculated at the inlet and outlet ends.
\end{conceptbox}
\begin{formulabox}
The temperature differences at the ends are:
$$ \Delta T_1 = T_{oil,in} - T_{water,in} = 100^\circ\text{C} - 25^\circ\text{C} = 75^\circ\text{C} $$
$$ \Delta T_2 = T_{oil,out} - T_{water,out} = 10^\circ\text{C} \quad (\text{Given}) $$
The LMTD is then:
$$ \Delta T_{lm} = \frac{\Delta T_1 - \Delta T_2}{\ln(\Delta T_1 / \Delta T_2)} = \frac{75 - 10}{\ln(75 / 10)} = \frac{65}{\ln(7.5)} \approx 32.2^\circ\text{C} $$
\end{formulabox}
\end{stepbox}

\begin{stepbox}[title=Step 4: Calculate the Required Tube Length ($L$)]
\begin{conceptbox}
With all other parameters known, we can now calculate the required length of the heat exchanger tube.
\end{conceptbox}
\begin{formulabox}
$$ L = \frac{q}{U \pi D_i \Delta T_{lm}} $$
$$ L = \frac{12786 \, \text{W}}{(38.1 \, \frac{\text{W}}{\text{m}^2\cdot^\circ\text{C}}) \times \pi \times (0.03 \, \text{m}) \times (32.2^\circ\text{C})} $$
$$ L = \frac{12786}{115.6} \approx 110.6 \, \text{m} $$
\end{formulabox}
\begin{conceptbox}[title=Final Answer]
The required length of the heat exchanger is approximately \textbf{110.6 meters}.
\end{conceptbox}
\end{stepbox}

\newpage
\section*{Heat Exchanger: Mass Flow Rate Calculation}
\subsection*{Objective}
\begin{conceptbox}
The goal of this calculation is to determine the required mass flow rate of a coolant (water) needed to cool a process fluid (oil) to a specified outlet temperature in a counter-current heat exchanger.
\end{conceptbox}

\subsection*{Problem Setup and Governing Principle}
\begin{examplebox}{Counter-Current Heat Exchanger}
Lubricating oil is cooled by water in a counter-current concentric tube heat exchanger.
\begin{itemize}[itemsep=2pt]
    \item \textbf{Hot Fluid (Oil)}:
    \begin{itemize}[itemsep=0pt]
        \item Mass Flow Rate, $\dot{m}_{oil} = 0.1$ kg/s.
        \item Temperatures: $T_{oil,in} = 100^\circ$C, $T_{oil,out} = 55^\circ$C.
        \item Specific Heat, $c_{p,oil} = 2131$ J/kg$\cdot^\circ$C.
    \end{itemize}
    \item \textbf{Cold Fluid (Water)}:
    \begin{itemize}[itemsep=0pt]
        \item Temperatures: $T_{water,in} = 30^\circ$C, $T_{water,out} = 40^\circ$C.
        \item Specific Heat, $c_{p,water} = 4178$ J/kg$\cdot^\circ$C.
    \end{itemize}
\end{itemize}
\end{examplebox}
\begin{conceptbox}[title=Governing Principle: Energy Balance]
Assuming the heat exchanger is adiabatic (no heat loss to the surroundings), the heat lost by the hot fluid must be equal to the heat gained by the cold fluid.
$$ q_{lost, hot} = q_{gained, cold} $$
\end{conceptbox}

\subsection*{Step-by-Step Calculation}
\begin{stepbox}[title=Step 1: Set up the Energy Balance Equation]
\begin{conceptbox}
The heat transfer for each fluid stream is given by the equation $q = \dot{m} c_p \Delta T$. We can equate the heat transfer expressions for the oil and water streams.
\end{conceptbox}
\begin{formulabox}
$$ \dot{m}_{oil} c_{p,oil} (T_{oil,in} - T_{oil,out}) = \dot{m}_{water} c_{p,water} (T_{water,out} - T_{water,in}) $$
The temperature differences are arranged to yield positive values on both sides of the equation.
\end{formulabox}
\end{stepbox}

\begin{stepbox}[title=Step 2: Solve for the Unknown Mass Flow Rate ($\dot{m}_{water}$)]
\begin{conceptbox}
We rearrange the energy balance equation to isolate the unknown mass flow rate of water.
\end{conceptbox}
\begin{formulabox}
$$ \dot{m}_{water} = \frac{\dot{m}_{oil} c_{p,oil} (T_{oil,in} - T_{oil,out})}{c_{p,water} (T_{water,out} - T_{water,in})} $$
\end{formulabox}
\end{stepbox}

\newpage
\begin{stepbox}[title=Step 3: Substitute Values and Calculate]
\begin{conceptbox}
Plug the known values into the rearranged equation to find the final answer.
\end{conceptbox}
\begin{formulabox}
$$ \dot{m}_{water} = \frac{(0.1 \, \text{kg/s}) \times (2131 \, \frac{\text{J}}{\text{kg}\cdot^\circ\text{C}}) \times (100^\circ\text{C} - 55^\circ\text{C})}{(4178 \, \frac{\text{J}}{\text{kg}\cdot^\circ\text{C}}) \times (40^\circ\text{C} - 30^\circ\text{C})} $$
$$ \dot{m}_{water} = \frac{0.1 \times 2131 \times 45}{4178 \times 10} = \frac{9589.5}{41780} \approx 0.2295 \, \text{kg/s} $$
\end{formulabox}
\begin{conceptbox}[title=Final Answer]
The required mass flow rate of water is approximately \textbf{0.23 kg/s}.
\end{conceptbox}
\end{stepbox}

\begin{stepbox}[title=Step 4: Verification (Optional)]
\begin{conceptbox}
We can verify our result by calculating the heat duty ($q$) independently for each stream. The values should be equal, confirming the energy balance is satisfied.
\end{conceptbox}
\begin{formulabox}
\begin{itemize}[itemsep=2pt]
    \item \textbf{Heat Lost by Oil}:
    $$ q_{oil} = (0.1) \times (2131) \times (100 - 55) = 9589.5 \, \text{W} $$
    \item \textbf{Heat Gained by Water}:
    $$ q_{water} = (0.2295) \times (4178) \times (40 - 30) = 9589.5 \, \text{W} $$
\end{itemize}
Since $q_{oil} = q_{water}$, our calculation is consistent.
\end{formulabox}
\end{stepbox}

\newpage

\section*{NTU-Effectiveness Method for Heat Exchangers}
\subsection*{Objective}
\begin{conceptbox}
The goal of this guide is to demonstrate the use of the Effectiveness-NTU method to analyze heat exchanger performance. Specifically, we will determine the overall heat transfer coefficient, $U$, for a given counter-flow heat exchanger where all inlet and outlet temperatures are known.
\end{conceptbox}

\subsection*{Problem Setup and Given Data}
\begin{examplebox}{Counter-Flow Heat Exchanger Analysis}
A thin-walled, counter-flow, concentric tube heat exchanger is used to heat pressurized water with hot gases.
\begin{itemize}[itemsep=2pt]
    \item \textbf{Cold Fluid (Water)}:
        \begin{itemize}[itemsep=0pt]
            \item Mass Flow Rate, $\dot{m}_c = 1.0$ kg/s
            \item Specific Heat, $c_{p,c} = 4197$ J/kg$\cdot$K
            \item Temperatures: $T_{c,in} = 40^\circ$C, $T_{c,out} = 140^\circ$C
        \end{itemize}
    \item \textbf{Hot Fluid (Gas)}:
        \begin{itemize}[itemsep=0pt]
            \item Mass Flow Rate, $\dot{m}_h = 1.9$ kg/s
            \item Specific Heat, $c_{p,h} = 1000$ J/kg$\cdot$K
            \item Temperatures: $T_{h,in} = 350^\circ$C, $T_{h,out} = 100^\circ$C
        \end{itemize}
    \item \textbf{Geometry}:
        \begin{itemize}[itemsep=0pt]
            \item Heat Transfer Surface Area, $A = 20$ m$^2$
        \end{itemize}
\end{itemize}
\end{examplebox}
\begin{keybox}[title=Key Assumption]
The term \textit{thin-walled} implies that the conductive thermal resistance of the tube wall is negligible. Therefore, the overall heat transfer coefficient, $U$, is determined solely by the convective resistances of the two fluids.
\end{keybox}

\subsection*{Theoretical Framework: The Effectiveness-NTU Method}
\begin{conceptbox}
The Effectiveness-NTU method is an alternative to the LMTD method for analyzing heat exchangers, and it is particularly useful when outlet temperatures are unknown. The method is based on three key dimensionless parameters.
\end{conceptbox}

\begin{keybox}[title=Key Parameters]
\begin{itemize}[itemsep=2pt]
    \item \textbf{Heat Capacity Rate ($C$)}: The rate at which a fluid stream can transport thermal energy. It is calculated for both the hot ($C_h$) and cold ($C_c$) fluids.
    $$ C = \dot{m} c_p \quad [\text{W/K}] $$
    We identify $C_{min}$ (the smaller of $C_c$ and $C_h$) and $C_{max}$ (the larger).
    \item \textbf{Effectiveness ($\epsilon$)}: A measure of the heat exchanger's thermal performance, defined as the ratio of the actual heat transfer rate to the maximum possible heat transfer rate.
    \item \textbf{Number of Transfer Units (NTU)}: A measure of the heat exchanger's thermal size, which incorporates the overall heat transfer coefficient and the surface area.
\end{itemize}
\end{keybox}

\begin{formulabox}[title=Governing Equations]
\begin{itemize}[itemsep=4pt]
    \item \textbf{Effectiveness}:
    $$ \epsilon = \frac{q}{q_{max}} $$
    \item \textbf{Maximum Possible Heat Transfer Rate}: This is the heat transfer that would occur in an infinitely long counter-flow exchanger. It is limited by the fluid with the minimum heat capacity rate ($C_{min}$) and the overall temperature difference available.
    $$ q_{max} = C_{min} (T_{h,in} - T_{c,in}) $$
    \item \textbf{Number of Transfer Units}:
    $$ NTU = \frac{UA}{C_{min}} $$
\end{itemize}
\end{formulabox}

\subsection*{Solution Strategy}
\begin{stepbox}
Our objective is to find $U$. We can rearrange the NTU definition to solve for it:
$$ U = \frac{NTU \cdot C_{min}}{A} $$
The procedure to find the terms on the right-hand side is as follows:
\begin{enumerate}[label=\textbf{Step \arabic*:}, wide=0pt, leftmargin=*, itemsep=2pt]
    \item Calculate the heat capacity rates ($C_c, C_h$) to determine $C_{min}$, $C_{max}$, and their ratio, $C_r = C_{min} / C_{max}$.
    \item Calculate the actual heat transfer rate, $q$, and the maximum possible rate, $q_{max}$, to find the effectiveness, $\epsilon$.
    \item Use the appropriate effectiveness-NTU correlation for a counter-flow heat exchanger to calculate the NTU value from $\epsilon$ and $C_r$.
    \item Substitute the known and calculated values into the rearranged NTU equation to solve for $U$.
\end{enumerate}
\end{stepbox}

\subsection*{Step-by-Step Calculation}

\begin{stepbox}[title=Step 1: Determine Heat Capacity Rates and Ratio]
\begin{formulabox}
Cold Fluid (Water):
$$ C_c = \dot{m}_c \times c_{p,c} = (1.0 \, \text{kg/s}) \times (4197 \, \frac{\text{J}}{\text{kg}\cdot\text{K}}) = 4197 \, \text{W/K} $$

Hot Fluid (Gas):
$$ C_h = \dot{m}_h \times c_{p,h} = (1.9 \, \text{kg/s}) \times (1000 \, \frac{\text{J}}{\text{kg}\cdot\text{K}}) = 1900 \, \text{W/K} $$

Comparing the two values:
$$ C_{min} = C_h = 1900 \, \text{W/K} \quad \text{and} \quad C_{max} = C_c = 4197 \, \text{W/K} $$

The heat capacity ratio:
$$ C_r = \frac{C_{min}}{C_{max}} = \frac{1900}{4197} \approx 0.453 $$
\end{formulabox}
\end{stepbox}

\begin{stepbox}[title=Step 2: Calculate Effectiveness ($\epsilon$)]
\begin{formulabox}
First, calculate the actual heat transfer rate, $q$. We can use either fluid; using the hot gas stream:
$$ q = C_h (T_{h,in} - T_{h,out}) = (1900 \, \text{W/K}) \times (350^\circ\text{C} - 100^\circ\text{C}) = 475,000 \, \text{W} $$
Next, calculate the maximum possible heat transfer rate, $q_{max}$:
$$ q_{max} = C_{min} (T_{h,in} - T_{c,in}) = (1900 \, \text{W/K}) \times (350^\circ\text{C} - 40^\circ\text{C}) = 589,000 \, \text{W} $$
Finally, calculate the effectiveness, $\epsilon$:
$$ \epsilon = \frac{q}{q_{max}} = \frac{475,000 \, \text{W}}{589,000 \, \text{W}} \approx 0.806 $$
\end{formulabox}
\begin{keybox}[title=Alternative Effectiveness Calculation]
Since the hot fluid is the fluid with $C_{min}$, the effectiveness can also be calculated directly from the temperatures of that fluid stream:
$$ \epsilon = \frac{\Delta T_{h}}{T_{h,in} - T_{c,in}} = \frac{T_{h,in} - T_{h,out}}{T_{h,in} - T_{c,in}} = \frac{350-100}{350-40} = \frac{250}{310} \approx 0.806 $$
\end{keybox}
\end{stepbox}

\newpage
\begin{stepbox}[title=Step 3: Calculate NTU from Effectiveness]
\begin{conceptbox}
For each heat exchanger geometry (e.g., counter-flow, parallel-flow, shell-and-tube), there is a specific correlation relating $\epsilon$, NTU, and $C_r$. We must use the correct one for our system.
\end{conceptbox}
\begin{formulabox}[title=Effectiveness-NTU Relation for Counter-Flow]
The correlation for a counter-flow heat exchanger is:
$$ NTU = \frac{1}{C_r - 1} \ln\left(\frac{\epsilon - 1}{\epsilon C_r - 1}\right) $$
Substituting our calculated values for $\epsilon$ and $C_r$:
$$ NTU = \frac{1}{0.453 - 1} \ln\left(\frac{0.806 - 1}{0.806 \cdot 0.453 - 1}\right) $$
$$ NTU = \frac{1}{-0.547} \ln\left(\frac{-0.194}{0.365 - 1}\right) = \frac{1}{-0.547} \ln\left(\frac{-0.194}{-0.635}\right) $$
$$ NTU = \frac{1}{-0.547} \ln(0.3055) \approx (-1.828) \times (-1.186) \approx 2.17 $$
\end{formulabox}
\end{stepbox}

\begin{stepbox}[title=Step 4: Calculate the Overall Heat Transfer Coefficient ($U$)]
\begin{conceptbox}
Finally, we rearrange the definition of NTU to solve for the overall heat transfer coefficient, $U$.
\end{conceptbox}
\begin{formulabox}
$$ U = \frac{NTU \cdot C_{min}}{A} $$
Substituting our calculated and given values:
$$ U = \frac{2.17 \times 1900 \, \text{W/K}}{20 \, \text{m}^2} = \frac{4123}{20} \approx 206.15 \, \frac{\text{W}}{\text{m}^2\cdot\text{K}} $$
\end{formulabox}
\begin{conceptbox}[title=Final Answer]
The overall heat transfer coefficient for this heat exchanger is approximately \textbf{206 W/m$^2\cdot$K}.
\end{conceptbox}
\end{stepbox}

\end{document}
