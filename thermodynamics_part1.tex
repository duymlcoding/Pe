\documentclass[12pt]{article}
\usepackage[paperwidth=8.5in, paperheight=11in, margin=1.0in, headheight=15pt]{geometry}
\usepackage{amsmath,amssymb,amsthm}
\usepackage[most]{tcolorbox}
\usepackage{enumitem}
\usepackage{xcolor}
\usepackage{hyperref}
\usepackage{fancyhdr}
\usepackage{titlesec}
\usepackage{graphicx}
% Define custom colors for chemical engineering theme
\definecolor{conceptcolor}{RGB}{52, 73, 94}      % Dark blue-gray
\definecolor{formulacolor}{RGB}{231, 76, 60}     % Red for formulas
\definecolor{examplecolor}{RGB}{39, 174, 96}     % Green for examples
\definecolor{stepcolor}{RGB}{142, 68, 173}       % Purple for solution steps
\definecolor{keycolor}{RGB}{243, 156, 18}        % Orange for key points
% Configure fancy headers
\pagestyle{fancy}
\fancyhf{}
\fancyhead[L]{PE Study Guide}
\fancyhead[R]{Process Fundamentals}
\fancyfoot[C]{\thepage}
\renewcommand{\baselinestretch}{1.1}
\setlength{\parindent}{0.25in}
\setlength{\parskip}{3pt}
% Configure section formatting
\titleformat{\section}
  {\normalfont\LARGE\bfseries\color{conceptcolor}}
  {\thesection}{1em}{}
\titleformat{\subsection}
  {\normalfont\Large\bfseries\color{conceptcolor}}
  {\thesubsection}{1em}{}
% Define custom environments
\newtcolorbox{conceptbox}[1][]{
  enhanced,
  colback=conceptcolor!10,
  colframe=conceptcolor,
  arc=3mm,
  title=Key Concept,
  fonttitle=\bfseries\sffamily\normalsize,
  fontupper=\small,
  #1
}
\newtcolorbox{formulabox}[1][]{
  enhanced,
  colback=formulacolor!10,
  colframe=formulacolor,
  arc=2mm,
  title=Important Formula,
  fonttitle=\bfseries\sffamily\normalsize,
  fontupper=\small,
  #1
}
\newtcolorbox{examplebox}[2][]{
  enhanced,
  colback=examplecolor!10,
  colframe=examplecolor,
  arc=3mm,
  title=Example Problem: #2,
  fonttitle=\bfseries\sffamily\normalsize,
  fontupper=\small,
  #1
}
\newtcolorbox{stepbox}[1][]{
  enhanced,
  colback=stepcolor!10,
  colframe=stepcolor,
  arc=2mm,
  title=Solution Steps,
  fonttitle=\bfseries\sffamily\normalsize,
  fontupper=\small,
  #1
}
\newtcolorbox{keybox}[1][]{
  enhanced,
  colback=keycolor!10,
  colframe=keycolor,
  arc=2mm,
  title=Key Variables \& Definitions,
  fonttitle=\bfseries\sffamily\normalsize,
  fontupper=\small,
  #1
}

\begin{document}

\begin{center}
\Huge\textbf{\color{conceptcolor}Basic Thermodynamics}\\
\end{center}
\hrule
\vspace{0.3in}

\section*{Ideal Gas Law}

\begin{conceptbox}
\textbf{Ideal Gas Behavior:} An ideal gas is a theoretical gas composed of randomly moving point particles with no intermolecular forces. Real gases approach ideal behavior at low pressures and high temperatures when molecular interactions become negligible compared to kinetic energy.

\textbf{State Functions:} Properties that depend only on the current state of the system, not on the path taken to reach that state. For ideal gases, enthalpy and internal energy are functions of temperature only.
\end{conceptbox}

\begin{keybox}
\textbf{Common Variables \& Notation:}
\begin{itemize}[itemsep=0pt]
    \item $P$ = Absolute pressure \text{(Pa, atm, MPa)}
    \item $V_t$ = Total volume \text{(m}$^3$\text{, L)}
    \item $V$ = Molar volume \text{(L/mol, m}$^3$\text{/mol)}
    \item $n$ = Number of moles \text{(mol)}
    \item $T$ = Absolute temperature \text{(K)}
    \item $R$ = Universal gas constant = 8.314 \text{ J/(mol·K)} = 0.08206 \text{ L·atm/(mol·K)}
    \item $C_p$ = Heat capacity at constant pressure \text{(J/(mol·K))}
    \item $C_v$ = Heat capacity at constant volume \text{(J/(mol·K))}
    \item $y_i$ = Mole fraction of component $i$ \text{(dimensionless)}
    \item $P_i$ = Partial pressure of component $i$
\end{itemize}
\end{keybox}

\subsection*{Essential Equations}

\begin{formulabox}
\textbf{Ideal Gas Law (Total Volume):}
$$PV_t = nRT \quad \text{(Equation 1)}$$

\textbf{Ideal Gas Law (Molar Volume):}
$$PV = RT \quad \text{(Equation 2)}$$
where $V = V_t/n$ is the molar volume.
\end{formulabox}

\begin{formulabox}
\textbf{Enthalpy Change for Ideal Gas:}
$$\Delta H = C_p \Delta T \quad \text{(Equation 3)}$$

\textbf{Internal Energy Change for Ideal Gas:}
$$\Delta U = C_v \Delta T \quad \text{(Equation 4)}$$

Note: These relationships hold for \textbf{any process} involving an ideal gas, not just constant pressure or constant volume processes.
\end{formulabox}

\begin{formulabox}
\textbf{Heat Capacity Relationship:}
$$C_p = C_v + R \quad \text{(Equation 5)}$$

\textbf{Dalton's Law (Binary Mixture):}
$$P = P_A + P_B \quad \text{(Equation 6)}$$

\textbf{Mole Fraction Definition:}
$$y_A = \frac{P_A}{P} \quad \text{(Equation 7)}$$
\end{formulabox}

\newpage

\subsection*{Worked Examples}

\begin{examplebox}{Water Vapor Volume Calculation \& Error Analysis}
Calculate the volume of water vapor using the ideal gas law and determine the error compared to steam table values:
\begin{itemize}[itemsep=0pt]
    \item \textbf{Condition 1:} $T = 500 \text{C}$, $P = 0.5$ MPa (Steam table: $v = 0.7109$ m$^3$/kg)
    \item \textbf{Condition 2:} $T = 500 \text{C}$, $P = 17.0$ MPa (Steam table: $v = 0.0482$ m$^3$/kg)
\end{itemize}
\end{examplebox}

\begin{stepbox}
\begin{enumerate}[label=\textbf{Step \arabic*:}, wide=0pt, leftmargin=*, itemsep=2pt]
    \item \textbf{Define Constants and Convert Units}
    \begin{itemize}[itemsep=0pt]
        \item $R = 8.314$ MPa·cm$^3$/(mol·K)
        \item $M_{H_2O} = 18.015$ g/mol
        \item $T = 500 \text{C} + 273.15 = 773.15$ K
    \end{itemize}
    
    \item \textbf{Calculate Molar Volume for Low Pressure (0.5 MPa)}
    $$V = \frac{RT}{P} = \frac{(8.314)(773.15)}{0.5} = 12859.5 \text{ cm}^3\text{/mol}$$
    
    \item \textbf{Convert to Specific Volume}
    $$v_{calc} = 12859.5 \times \frac{1}{18.015} \times \frac{1}{1000} \times \frac{1}{10^6} = 0.7141 \text{ m}^3\text{/kg}$$
    
    \item \textbf{Calculate Error for Low Pressure}
    $$\text{Error \%} = \left|\frac{0.7141 - 0.7109}{0.7109}\right| \times 100\% = 0.45\%$$
    
    \item \textbf{Calculate for High Pressure (17.0 MPa)}
    $$V = \frac{(8.314)(773.15)}{17.0} = 378.2 \text{ cm}^3\text{/mol}$$
    $$v_{calc} = 378.2 \times \frac{1}{18.015} \times \frac{1}{1000} \times \frac{1}{10^6} = 0.0210 \text{ m}^3\text{/kg}$$
    
    \item \textbf{Calculate Error for High Pressure}
    $$\text{Error \%} = \left|\frac{0.0210 - 0.0482}{0.0482}\right| \times 100\% = 56.4\%$$

\end{enumerate}
\end{stepbox}

\newpage

\begin{examplebox}{Gas Mixture Analysis}
An ideal gas mixture at 2 atm and 35°C has volume compositions: 15\% O$_2$, 65\% N$_2$, 12\% CO$_2$, 7\% CO, 1\% H$_2$O. Calculate:
\begin{itemize}[itemsep=0pt]
    \item a) Partial pressures of each species
    \item b) Mass fractions of O$_2$ and CO
    \item c) Average molecular weight
    \item d) Density of the gas mixture
\end{itemize}
\end{examplebox}

\begin{stepbox}
\begin{enumerate}[label=\textbf{Step \arabic*:}, wide=0pt, leftmargin=*, itemsep=2pt]

\item \textbf{Partial Pressures (Part a)}  
For ideal gases: $P_i = y_i \cdot P_{total}$, where $P_{total} = 2$ atm.  
\[
\begin{aligned}
P_{O_2} &= 0.15 \cdot 2 = 0.30\ \text{atm} \\
P_{N_2} &= 0.65 \cdot 2 = 1.30\ \text{atm} \\
P_{CO_2} &= 0.12 \cdot 2 = 0.24\ \text{atm} \\
P_{CO} &= 0.07 \cdot 2 = 0.14\ \text{atm} \\
P_{H_2O} &= 0.01 \cdot 2 = 0.02\ \text{atm}
\end{aligned}
\]

\item \textbf{Mass Fractions (Part b)}  
Assume 1 mol of mixture. Multiply mole fraction by molar mass:
\[
\begin{aligned}
m_{O_2} &= 0.15 \cdot 32.00 = 4.80\ \text{g} \\
m_{N_2} &= 0.65 \cdot 28.02 = 18.21\ \text{g} \\
m_{CO_2} &= 0.12 \cdot 44.01 = 5.28\ \text{g} \\
m_{CO} &= 0.07 \cdot 28.01 = 1.96\ \text{g} \\
m_{H_2O} &= 0.01 \cdot 18.02 = 0.18\ \text{g}
\end{aligned}
\]
Total mass $= 30.43$ g. Mass fractions:
\[
x_{O_2} = \frac{4.80}{30.43} = 0.158 \quad (15.8\%), \qquad
x_{CO} = \frac{1.96}{30.43} = 0.064 \quad (6.4\%)
\]

\item \textbf{Average Molecular Weight (Part c)}  
\[
M_{avg} = \frac{\text{total mass}}{\text{total moles}} = \frac{30.43\ \text{g}}{1\ \text{mol}} = 30.43\ \text{g/mol}
\]

\item \textbf{Density (Part d)}  
Ideal gas law: $\rho = \frac{PM}{RT}$, with $P = 2$ atm, $R = 0.08206$ L·atm/mol·K, $T = 308.15$ K.  
\[
\rho = \frac{(2)(30.43)}{(0.08206)(308.15)} = 2.41\ \text{g/L} = 2.41\ \text{kg/m}^3
\]

\end{enumerate}
\end{stepbox}

\newpage

\section*{State Functions}

\begin{conceptbox}
\textbf{State Function Definition:} A state function (or state variable) is a property of a system that depends only on its current thermodynamic state, not on the path taken to reach that state. This is one of the most powerful concepts in thermodynamics because it allows us to simplify complex calculations.

\textbf{Key Principle:} If a system goes from initial state A to final state B, the change in any state function will be the same regardless of the path taken (reversible, irreversible, fast, slow, etc.).
\end{conceptbox}

\begin{keybox}
\textbf{State Functions vs. Path Functions:}
\begin{itemize}[itemsep=0pt]
    \item \textbf{State Functions:} Enthalpy ($\Delta H$), Entropy ($\Delta S$), Internal Energy ($\Delta U$), Volume ($\Delta V$), Gibbs Free Energy ($\Delta G$), Helmholtz Free Energy ($\Delta A$)
    \item \textbf{Path Functions:} Heat ($Q$) and Work ($W$) - these depend on the specific process path
    \item \textbf{Calculation Strategy:} To find changes in state functions, ignore the actual complicated path and devise a simpler hypothetical path that is easier to calculate
    \item \textbf{Reference States:} Most thermodynamic properties are measured as changes ($\Delta$) relative to defined reference states
\end{itemize}
\end{keybox}

\begin{formulabox}
\textbf{State Function Property:}
For any state function $\phi$ going from state A to state B:
$$\Delta \phi = \phi_B - \phi_A = \text{constant (independent of path)}$$

\textbf{First Law for Closed Systems:}
$$\Delta U = Q - W \quad \text{(Equation 1)}$$

\textbf{Enthalpy Definition:}
$$H = U + PV \quad \text{(Equation 2)}$$
$$\Delta H = \Delta U + \Delta(PV) \quad \text{(Equation 3)}$$
\end{formulabox}

\subsection*{Worked Examples}

\begin{examplebox}{Adiabatic Water Evaporation and Freezing}
One kg of liquid water at $40^\circ\text{C}$ is in a flask connected through a valve to a vacuum pump. Water evaporates when the valve is opened. The water vapor is removed fast enough that the flask can be considered adiabatic ($Q=0$). As water evaporates, the remaining liquid cools, and when it reaches $0^\circ\text{C}$, it starts to freeze. What fraction of the water will have evaporated when the rest is frozen?
\end{examplebox}

\begin{stepbox}
\begin{enumerate}[label=\textbf{Step \arabic*:}, wide=0pt, leftmargin=*, itemsep=2pt]
    \item \textbf{Conceptual Analysis}
    The process is adiabatic ($Q = 0$) with no shaft work ($W = 0$). Energy for evaporation (latent heat of vaporization) must come from the liquid water itself, causing temperature drop. We need to find fraction $x$ that evaporates when remaining water completely freezes.
    
    \item \textbf{Apply First Law and State Function Principle}
    Since total energy is conserved and enthalpy is a state function:
    $$H_{initial} = H_{final}$$
    $$m_{initial} \cdot \hat{H}_{liquid, 40^\circ\text{C}} = m_{ice} \cdot \hat{H}_{ice, 0^\circ\text{C}} + m_{vapor} \cdot \hat{H}_{vapor}$$
    
    \item \textbf{Define Variables and Reference State}
    Using liquid water at $0^\circ\text{C}$ as reference ($\hat{H}_{liquid, 0^\circ\text{C}} = 0$):
    \begin{itemize}[itemsep=0pt]
        \item $x$ = fraction of water that evaporates
        \item $\hat{H}_{liquid, 40^\circ\text{C}} = 168$ kJ/kg (from steam tables)
        \item $\hat{H}_{ice, 0^\circ\text{C}} = -334$ kJ/kg (negative of latent heat of fusion)
        \item $\hat{H}_{vapor} \approx 2537$ kJ/kg (at average temperature $\approx 20^\circ\text{C}$)
    \end{itemize}
\end{enumerate}
\end{stepbox}

\begin{stepbox}
\begin{enumerate}[label=\textbf{Step \arabic*:}, wide=0pt, leftmargin=*, itemsep=2pt, resume]
    \setcounter{enumi}{3}
    \item \textbf{Set Up Energy Balance Equation}
    Substituting masses and enthalpy values:
    $$(1 \text{ kg}) \cdot (168 \text{ kJ/kg}) = (1-x) \text{ kg} \cdot (-334 \text{ kJ/kg}) + (x) \text{ kg} \cdot (2537 \text{ kJ/kg})$$
    
    \item \textbf{Solve for Evaporation Fraction}
    $$168 = -334 + 334x + 2537x$$
    $$168 + 334 = (334 + 2537)x$$
    $$502 = 2871x$$
    $$x = \frac{502}{2871} = 0.175$$
    
    \item \textbf{Final Answer}
    Approximately \textbf{17.5\%} of the water will have evaporated by the time the rest has frozen.
    
    \textbf{Physical Interpretation:} The energy stored as sensible heat in the $40^\circ\text{C}$ water (168 kJ/kg) provides the energy needed for both freezing the remaining liquid ($-334$ kJ/kg per kg frozen) and vaporizing the evaporated portion (2537 kJ/kg per kg evaporated).
\end{enumerate}
\end{stepbox}

\newpage

\begin{examplebox}{Heat of Reaction Temperature Dependence}
Calculate the heat of reaction ($\Delta H_{rxn}$) for ammonia formation at $175^\circ\text{C}$, given data at $25^\circ\text{C}$ and heat capacities.

$$N_2(g) + 3H_2(g) \rightarrow 2NH_3(g)$$

\textbf{Given Data:}
\begin{itemize}[itemsep=0pt]
    \item Heat of formation of NH$_3$ at $25^\circ\text{C}$: $\Delta H_{f, 25^\circ\text{C}} = -45.9$ kJ/mol
    \item $C_{P, N_2} = 29$ J/(mol$\cdot$K)
    \item $C_{P, H_2} = 29$ J/(mol$\cdot$K)  
    \item $C_{P, NH_3} = 35.7$ J/(mol$\cdot$K)
\end{itemize}
\end{examplebox}

\begin{stepbox}
\begin{enumerate}[label=\textbf{Step \arabic*:}, wide=0pt, leftmargin=*, itemsep=2pt]
    \item \textbf{State Function Strategy}
    Since enthalpy is a state function, we can construct a hypothetical three-step path from reactants at $175^\circ\text{C}$ to products at $175^\circ\text{C}$:
    \begin{itemize}[itemsep=0pt]
        \item Step 1 ($\Delta H_1$): Cool reactants from $175^\circ\text{C}$ to $25^\circ\text{C}$
        \item Step 2 ($\Delta H_{rxn, 25^\circ\text{C}}$): React at $25^\circ\text{C}$ (standard conditions)
        \item Step 3 ($\Delta H_2$): Heat products from $25^\circ\text{C}$ to $175^\circ\text{C}$
    \end{itemize}
    $$\Delta H_{rxn, 175^\circ\text{C}} = \Delta H_1 + \Delta H_{rxn, 25^\circ\text{C}} + \Delta H_2$$
    
    \item \textbf{Calculate Reaction Enthalpy at $25^\circ\text{C}$}
    Heat of formation for elemental species = 0:
    $$\Delta H_{rxn, 25^\circ\text{C}} = [2 \cdot \Delta H_{f, NH_3}] - [1 \cdot \Delta H_{f, N_2} + 3 \cdot \Delta H_{f, H_2}]$$
    $$\Delta H_{rxn, 25^\circ\text{C}} = [2 \cdot (-45.9)] - [0 + 0] = -91.8 \text{ kJ}$$
\end{enumerate}
\end{stepbox}

\newpage

\begin{stepbox}
\begin{enumerate}[label=\textbf{Step \arabic*:}, wide=0pt, leftmargin=*, itemsep=2pt, resume]
    \setcounter{enumi}{2}
    \item \textbf{Calculate $\Delta H_1$ (Cooling Reactants)}
    Using $\Delta H = nC_P\Delta T$ with $\Delta T = 25^\circ\text{C} - 175^\circ\text{C} = -150$ K:
    $$\Delta H_1 = [(1 \text{ mol} \cdot 29 \text{ J/(mol$\cdot$K)}) + (3 \text{ mol} \cdot 29 \text{ J/(mol$\cdot$K)})] \cdot (-150 \text{ K})$$
    $$\Delta H_1 = (116 \text{ J/K}) \cdot (-150 \text{ K}) = -17,400 \text{ J} = -17.4 \text{ kJ}$$
    
    \item \textbf{Calculate $\Delta H_2$ (Heating Products)}
    With $\Delta T = 175^\circ\text{C} - 25^\circ\text{C} = +150$ K:
    $$\Delta H_2 = (2 \text{ mol} \cdot 35.7 \text{ J/(mol$\cdot$K)}) \cdot (150 \text{ K})$$
    $$\Delta H_2 = (71.4 \text{ J/K}) \cdot (150 \text{ K}) = 10,710 \text{ J} = +10.71 \text{ kJ}$$
    
    \item \textbf{Sum All Enthalpy Changes}
    $$\Delta H_{rxn, 175^\circ\text{C}} = (-17.4) + (-91.8) + (10.71) = -98.49 \text{ kJ}$$
    
    \item \textbf{Final Answer and Analysis}
    The heat of reaction at $175^\circ\text{C}$ is \textbf{-98.49 kJ}, making the reaction slightly more exothermic at higher temperature compared to -91.8 kJ at $25^\circ\text{C}$.
    
    \textbf{Physical Insight:} The net effect shows that the heat capacity difference between products and reactants ($\Delta C_P = 71.4 - 116 = -44.6$ J/K) results in a more negative $\Delta H_{rxn}$ at higher temperatures.
\end{enumerate}
\end{stepbox}

\begin{conceptbox}
\textbf{Key State Function Applications:}
\begin{itemize}[itemsep=0pt]
    \item Use hypothetical paths for complex calculations
    \item Energy balances in adiabatic processes rely on state function properties
    \item Temperature dependence of reaction enthalpies can be calculated using heat capacities
    \item Always verify energy conservation in closed systems
    \item Reference states simplify enthalpy calculations significantly
\end{itemize}
\end{conceptbox}

\section*{First Law of Thermodynamics - Closed Systems}

\begin{conceptbox}
\textbf{Closed System Definition:} A closed system is one where \textbf{no mass crosses the system boundaries}. The total mass within the system remains constant throughout any process. Energy can still be transferred as heat and work, but material cannot enter or leave.

\textbf{First Law Principle:} The First Law is a statement of energy conservation. For a closed system, the change in internal energy equals the net energy transferred across boundaries as heat and work. For most chemical engineering applications, changes in kinetic and potential energy are negligible.
\end{conceptbox}

\begin{keybox}
\textbf{Key Variables \& Definitions:}
\begin{itemize}[itemsep=0pt]
    \item $\Delta U$ = Change in internal energy of the system \text{(J, kJ)}
    \item $Q$ = Heat transferred across system boundary \text{(J, kJ)}
    \item $W_{EC}$ = Expansion/compression work \text{(J, kJ)}
    \item $W_S$ = Shaft work (e.g., rotating impeller) \text{(J, kJ)}
    \item $P_{ext}$ = External pressure opposing volume change \text{(Pa, atm)}
    \item $P_{gas}$ = Internal gas pressure \text{(Pa, atm)}
    \item $V$ = System volume \text{(m}$^3$\text{, L)}
    \item $T_1, T_2$ = Initial and final absolute temperatures \text{(K)}
    \item $P_1, P_2$ = Initial and final pressures \text{(Pa, atm)}
\end{itemize}
\end{keybox}

\begin{keybox}
\textbf{Critical Sign Conventions:}
\begin{itemize}[itemsep=0pt]
    \item Heat \textbf{added} to system: $Q$ $>$ $0$ (positive)
    \item Heat \textbf{removed} from system: $Q$ $<$ $0$ (negative)
    \item Adiabatic process: $Q = 0$ (no heat transfer)
    \item Work done \textbf{on} system (compression): $W$ $>$ $0$ (positive)
    \item Work done \textbf{by} system (expansion): $W$ $<$ $0$ (negative)
    \item Reversible process: $P_{gas} = P_{ext}$ (system always in equilibrium)
\end{itemize}
\end{keybox}

\begin{formulabox}
\textbf{First Law for Closed Systems:}
$$\Delta U = Q + W_{EC} + W_S \quad \text{(Equation 1)}$$

\textbf{Expansion/Compression Work (General):}
$$W_{EC} = -\int P_{ext} \,dV \quad \text{(Equation 2)}$$

\textbf{Expansion/Compression Work (Reversible):}
$$W_{EC, rev} = -\int P_{gas} \,dV \quad \text{(Equation 3)}$$
\end{formulabox}

\begin{formulabox}
\textbf{Adiabatic Reversible Process (Ideal Gas):}
$$\frac{T_2}{T_1} = \left( \frac{P_2}{P_1} \right)^{\frac{R}{C_P}} \quad \text{(Equation 4)}$$

\textbf{Heat Capacity Relations:}
$$C_P = C_V + R \quad \text{(Equation 5)}$$
$$\Delta U = C_V \Delta T \quad \text{(Equation 6)}$$
$$\Delta H = C_P \Delta T \quad \text{(Equation 7)}$$
\end{formulabox}

\newpage

\subsection*{Special Process Cases}

\begin{conceptbox}
\textbf{Constant Volume Process (Isochoric):}
When volume is constant, $dV = 0$, so no expansion/compression work is done ($W_{EC} = 0$).

\textbf{Constant Pressure Process (Isobaric):}
When pressure is constant and the process is reversible, heat added equals the change in enthalpy.
\end{conceptbox}

\begin{formulabox}
\textbf{Constant Volume Process:}
$$W_{EC} = 0 \quad \text{(no volume change)}$$
$$\Delta U = Q_v \quad \text{(Equation 8)}$$
$$Q_v = C_V \Delta T \quad \text{(Equation 9)}$$

\textbf{Constant Pressure Process (Reversible):}
$$W_{EC} = -P\Delta V \quad \text{(Equation 10)}$$
$$Q_p = \Delta H = \Delta U + P\Delta V \quad \text{(Equation 11)}$$
$$Q_p = C_P \Delta T \quad \text{(Equation 12)}$$
\end{formulabox}

\newpage

\subsection*{Derivation of Adiabatic Reversible Process}

\begin{examplebox}{Temperature-Pressure Relationship for Adiabatic Reversible Process}
Derive the relationship between initial and final temperatures and pressures for an ideal gas undergoing an adiabatic, reversible process in a closed system.
\end{examplebox}

\begin{stepbox}
\begin{enumerate}[label=\textbf{Step \arabic*:}, wide=0pt, leftmargin=*, itemsep=2pt]
    \item \textbf{Apply First Law in Differential Form}
    For an adiabatic ($dQ = 0$), reversible process with no shaft work ($dW_S = 0$):
    $$dU = dW_{EC, rev}$$
    
    \item \textbf{Substitute Ideal Gas Relations}
    For an ideal gas: $dU = C_V dT$\\
    For reversible process: $dW_{EC, rev} = -P dV$
    $$C_V dT = -P dV$$
    
    \item \textbf{Eliminate Pressure Using Ideal Gas Law}
    From $PV = nRT$, we get $P = \frac{nRT}{V}$. Substituting:
    $$C_V dT = -\frac{nRT}{V} dV$$
    
    \item \textbf{Separate Variables for Integration}
    Dividing both sides appropriately:
    $$\frac{C_V}{T} dT = -\frac{nR}{V} dV$$

    \item \textbf{Integrate from Initial to Final State}
    Assuming constant heat capacities:
    $$\int_{T_1}^{T_2} \frac{C_V}{T} dT = -\int_{V_1}^{V_2} \frac{nR}{V} dV$$
    $$C_V \ln\left(\frac{T_2}{T_1}\right) = -nR \ln\left(\frac{V_2}{V_1}\right)$$
    
    \item \textbf{Relate Volume Ratio to Temperature and Pressure}
    From ideal gas law at two states: $\frac{P_1V_1}{T_1} = \frac{P_2V_2}{T_2}$
    $$\frac{V_2}{V_1} = \frac{T_2 P_1}{T_1 P_2}$$
\end{enumerate}
\end{stepbox}

\newpage

\begin{stepbox}
\begin{enumerate}[label=\textbf{Step \arabic*:}, wide=0pt, leftmargin=*, itemsep=2pt, resume]
    \setcounter{enumi}{6}

    
    \item \textbf{Substitute Volume Ratio}
    $$C_V \ln\left(\frac{T_2}{T_1}\right) = -nR \ln\left(\frac{T_2 P_1}{T_1 P_2}\right)$$
    
    \item \textbf{Expand Logarithm and Rearrange}
    Using $\ln(ab) = \ln(a) + \ln(b)$:
    $$C_V \ln\left(\frac{T_2}{T_1}\right) = -nR \ln\left(\frac{T_2}{T_1}\right) - nR \ln\left(\frac{P_1}{P_2}\right)$$
    $$(C_V + nR) \ln\left(\frac{T_2}{T_1}\right) = nR \ln\left(\frac{P_2}{P_1}\right)$$

    \item \textbf{Apply Heat Capacity Relationship}
    For ideal gas: $C_P = C_V + nR$, so:
    $$C_P \ln\left(\frac{T_2}{T_1}\right) = nR \ln\left(\frac{P_2}{P_1}\right)$$
    $$\ln\left(\frac{T_2}{T_1}\right) = \frac{nR}{C_P} \ln\left(\frac{P_2}{P_1}\right)$$
    
    \item \textbf{Final Result}
    Exponentiating both sides:
    $$\frac{T_2}{T_1} = \left( \frac{P_2}{P_1} \right)^{\frac{nR}{C_P}}$$
    
    For molar basis ($n = 1$ mol), this becomes:
    $$\frac{T_2}{T_1} = \left( \frac{P_2}{P_1} \right)^{\frac{R}{C_P}}$$
    
    \item \textbf{Physical Interpretation}
    This relationship shows that for adiabatic compression ($P_2$ $>$ $P_1$), the temperature must increase ($T_2$ $>$ $T_1$). The exponent $R/C_P$ is always positive and less than 1 for real gases, making this relationship physically meaningful.
\end{enumerate}
\end{stepbox}

\subsection*{Examples for Closed Systems}

\begin{examplebox}{Adiabatic Reversible Compression of Ideal Gas}
Calculate the final temperature and pressure for the adiabatic, reversible compression of an ideal gas compressed to a final volume of 1 cm$^3$.

\textbf{Given Initial Conditions:}
\begin{itemize}[itemsep=0pt]
    \item Initial Volume: $V_1 = 10$ cm$^3$
    \item Initial Pressure: $P_1 = 1$ bar
    \item Initial Temperature: $T_1 = 300$ K
\end{itemize}
\end{examplebox}

\begin{stepbox}
\begin{enumerate}[label=\textbf{Step \arabic*:}, wide=0pt, leftmargin=*, itemsep=2pt]
    \item \textbf{Derive Temperature-Volume Relationship}
    Starting with First Law for adiabatic ($Q = 0$), reversible, closed system with no shaft work:
    $$dU = dW_{EC, rev}$$
    
    For ideal gas: $dU = C_V dT$ and $dW_{EC, rev} = -P dV$
    $$C_V dT = -P dV$$
    
    \item \textbf{Substitute Ideal Gas Law}
    Using $P = \frac{RT}{V}$:
    $$C_V dT = -\frac{RT}{V} dV$$
    
    \item \textbf{Separate Variables and Integrate}
    $$\frac{C_V}{T} dT = -\frac{R}{V} dV$$
    
    Integrating from state 1 to state 2:
    $$\int_{T_1}^{T_2} \frac{C_V}{T} dT = -\int_{V_1}^{V_2} \frac{R}{V} dV$$
    $$C_V \ln\left(\frac{T_2}{T_1}\right) = -R \ln\left(\frac{V_2}{V_1}\right)$$
\end{enumerate}
\end{stepbox}

\newpage

\begin{stepbox}
\begin{enumerate}[label=\textbf{Step \arabic*:}, wide=0pt, leftmargin=*, itemsep=2pt, resume]
    \setcounter{enumi}{3}
    \item \textbf{Solve for Temperature Ratio}
    $$\ln\left(\frac{T_2}{T_1}\right) = -\frac{R}{C_V} \ln\left(\frac{V_2}{V_1}\right) = \ln\left[\left(\frac{V_1}{V_2}\right)^{\frac{R}{C_V}}\right]$$
    
    Therefore:
    $$\frac{T_2}{T_1} = \left(\frac{V_1}{V_2}\right)^{\frac{R}{C_V}}$$
    
    \item \textbf{Calculate Final Temperature}
    For diatomic gases: $\gamma = C_P/C_V \approx 1.4$
    Since $C_P - C_V = R$: $\frac{R}{C_V} = \gamma - 1 = 0.4$
    
    $$T_2 = T_1 \left(\frac{V_1}{V_2}\right)^{0.4} = 300 \text{ K} \cdot \left(\frac{10}{1}\right)^{0.4}$$
    $$T_2 = 300 \text{ K} \cdot (10)^{0.4} = 300 \text{ K} \cdot 2.51 = 753 \text{ K}$$
    
    Converting to Celsius: $753 - 273 = 480^\circ\text{C}$
    
    \item \textbf{Calculate Final Pressure}
    Using combined ideal gas law:
    $$\frac{P_1 V_1}{T_1} = \frac{P_2 V_2}{T_2}$$
    
    Solving for $P_2$:
    $$P_2 = P_1 \left(\frac{V_1}{V_2}\right) \left(\frac{T_2}{T_1}\right) = 1 \text{ bar} \cdot \left(\frac{10}{1}\right) \cdot \left(\frac{753}{300}\right)$$
    $$P_2 = 1 \cdot 10 \cdot 2.51 = 25.1 \text{ bar}$$
    
    \item \textbf{Final Answer}
    Final conditions: $T_2 = \textbf{753 K}$ and $P_2 = \textbf{25.1 bar}$
\end{enumerate}
\end{stepbox}

\begin{conceptbox}
\textbf{Note on Irreversibility:} If this compression were irreversible, more work would be required to reach the same final volume. Since $\Delta U = W$ for an adiabatic process, larger work input results in higher internal energy, leading to final temperature and pressure \textbf{higher} than the reversible case.
\end{conceptbox}

\newpage

\begin{examplebox}{Constant Pressure vs. Constant Volume Heating}
Two containers at 350 K and 0.5 MPa each contain 1.0 mol of ideal gas with $C_P = 31$ J/(mol$\cdot$K).

\textbf{Container A:} Constant pressure piston-cylinder\\
\textbf{Container B:} Fixed volume container

What are the final temperature and pressure in each container after 8.5 kJ of heat are added to each?
\end{examplebox}

\begin{stepbox}
\begin{enumerate}[label=\textbf{Step \arabic*:}, wide=0pt, leftmargin=*, itemsep=2pt]
    \item \textbf{Container A: Constant Pressure Analysis}
    For reversible constant pressure process in closed system:
    $$Q_p = \Delta H = nC_P\Delta T = nC_P(T_2 - T_1)$$
    
    Given: $Q_p = 8.5$ kJ = 8500 J, $n = 1.0$ mol, $C_P = 31$ J/(mol$\cdot$K), $T_1 = 350$ K
    
    \item \textbf{Solve for Final Temperature (Container A)}
    $$8500 \text{ J} = (1.0 \text{ mol}) \cdot (31 \text{ J/(mol$\cdot$K)}) \cdot (T_2 - 350 \text{ K})$$
    $$T_2 - 350 = \frac{8500}{31} = 274 \text{ K}$$
    $$T_2 = 350 + 274 = \textbf{624 K}$$
    
    Pressure remains constant: $P_2 = P_1 = \textbf{0.5 MPa}$

    \item \textbf{Container B: Constant Volume Analysis}
    For constant volume process, no expansion work ($W = 0$):
    $$Q_v = \Delta U = nC_V\Delta T = nC_V(T_2 - T_1)$$
    
    First find $C_V$ using ideal gas relationship:
    $$C_V = C_P - R = 31 - 8.314 = 22.686 \text{ J/(mol$\cdot$K)}$$

    \item \textbf{Solve for Final Temperature (Container B)}
    $$8500 \text{ J} = (1.0 \text{ mol}) \cdot (22.686 \text{ J/(mol$\cdot$K)}) \cdot (T_2 - 350 \text{ K})$$
    $$T_2 - 350 = \frac{8500}{22.686} = 375 \text{ K}$$
    $$T_2 = 350 + 375 = \textbf{725 K}$$
    
\end{enumerate}
\end{stepbox}

\newpage

\begin{stepbox}
\begin{enumerate}[label=\textbf{Step \arabic*:}, wide=0pt, leftmargin=*, itemsep=2pt, resume]
    \setcounter{enumi}{4}

    \item \textbf{Calculate Final Pressure (Container B)}
    For constant volume, using ideal gas law:
    $$\frac{P_1}{T_1} = \frac{P_2}{T_2}$$
    $$P_2 = P_1 \left(\frac{T_2}{T_1}\right) = 0.5 \text{ MPa} \cdot \left(\frac{725}{350}\right) = \textbf{1.04 MPa}$$
    
    \item \textbf{Summary of Results}
    \textbf{Container A (Constant Pressure):} $T_2 = 624$ K, $P_2 = 0.5$ MPa\\
    \textbf{Container B (Constant Volume):} $T_2 = 725$ K, $P_2 = 1.04$ MPa
\end{enumerate}
\end{stepbox}

\begin{conceptbox}
\textbf{Physical Interpretation:} The constant volume system reaches a higher temperature (725 K vs. 624 K) because all added heat increases internal energy directly. In the constant pressure system, some heat energy converts to expansion work pushing the piston, resulting in lower temperature rise.

\textbf{Energy Analysis:}
\begin{itemize}[itemsep=0pt]
    \item Constant pressure: Heat $\rightarrow$ Internal energy increase + Expansion work
    \item Constant volume: Heat $\rightarrow$ Internal energy increase only
    \item Same heat input produces different temperature changes depending on process constraints
\end{itemize}
\end{conceptbox}

\begin{keybox}
\textbf{Key Problem-Solving Steps for Closed Systems:}
\begin{itemize}[itemsep=0pt]
    \item Identify process type (adiabatic, constant pressure, constant volume, etc.)
    \item Apply appropriate First Law form based on process constraints
    \item For adiabatic processes: $\Delta U = W$ (no heat transfer)
    \item For constant volume: $Q = \Delta U$ (no expansion work)
    \item For constant pressure: $Q = \Delta H$ (reversible processes)
    \item Use ideal gas relationships to connect temperature, pressure, and volume changes
    \item Check final answer for physical reasonableness
\end{itemize}
\end{keybox}

nd{document}
