\documentclass[12pt]{article}
\usepackage[paperwidth=8.5in, paperheight=11in, margin=1.0in, headheight=15pt]{geometry}
\usepackage{amsmath,amssymb,amsthm}
\usepackage[most]{tcolorbox}
\usepackage{enumitem}
\usepackage{xcolor}
\usepackage{hyperref}
\usepackage{fancyhdr}
\usepackage{titlesec}
\usepackage{graphicx}
% Define custom colors for chemical engineering theme
\definecolor{conceptcolor}{RGB}{52, 73, 94}      % Dark blue-gray
\definecolor{formulacolor}{RGB}{231, 76, 60}     % Red for formulas
\definecolor{examplecolor}{RGB}{39, 174, 96}     % Green for examples
\definecolor{stepcolor}{RGB}{142, 68, 173}       % Purple for solution steps
\definecolor{keycolor}{RGB}{243, 156, 18}        % Orange for key points
% Configure fancy headers
\pagestyle{fancy}
\fancyhf{}
\fancyhead[L]{PE Study Guide}
\fancyhead[R]{Process Fundamentals}
\fancyfoot[C]{\thepage}
\renewcommand{\baselinestretch}{1.1}
\setlength{\parindent}{0.25in}
\setlength{\parskip}{3pt}
% Configure section formatting
\titleformat{\section}
  {\normalfont\LARGE\bfseries\color{conceptcolor}}
  {\thesection}{1em}{}
\titleformat{\subsection}
  {\normalfont\Large\bfseries\color{conceptcolor}}
  {\thesubsection}{1em}{}
% Define custom environments
\newtcolorbox{conceptbox}[1][]{
  enhanced,
  colback=conceptcolor!10,
  colframe=conceptcolor,
  arc=3mm,
  title=Key Concept,
  fonttitle=\bfseries\sffamily\normalsize,
  fontupper=\small,
  #1
}
\newtcolorbox{formulabox}[1][]{
  enhanced,
  colback=formulacolor!10,
  colframe=formulacolor,
  arc=2mm,
  title=Important Formula,
  fonttitle=\bfseries\sffamily\normalsize,
  fontupper=\small,
  #1
}
\newtcolorbox{examplebox}[2][]{
  enhanced,
  colback=examplecolor!10,
  colframe=examplecolor,
  arc=3mm,
  title=Example Problem: #2,
  fonttitle=\bfseries\sffamily\normalsize,
  fontupper=\small,
  #1
}
\newtcolorbox{stepbox}[1][]{
  enhanced,
  colback=stepcolor!10,
  colframe=stepcolor,
  arc=2mm,
  title=Solution Steps,
  fonttitle=\bfseries\sffamily\normalsize,
  fontupper=\small,
  #1
}
\newtcolorbox{keybox}[1][]{
  enhanced,
  colback=keycolor!10,
  colframe=keycolor,
  arc=2mm,
  title=Key Variables \& Definitions,
  fonttitle=\bfseries\sffamily\normalsize,
  fontupper=\small,
  #1
}

\section*{Isothermal Plug Flow Reactors (PFRs)}
The Plug Flow Reactor (PFR) model is used to describe chemical reactions in continuous, flowing systems, typically within a tube or pipe. It is a cornerstone of chemical reaction engineering, particularly for large-scale, continuous processes where reactants are converted as they flow along the length of the reactor.

\subsection*{Fundamental Equations}
The design of a PFR is based on a differential mole balance taken over a differential segment of the reactor volume. For an irreversible reaction $A \rightarrow 2B$ with a rate law of $-r_A = k C_A^n$:

\begin{keybox}[title=Variable Definitions]
\begin{itemize}[itemsep=2pt]
    \item $F_i$: Molar flow rate of component $i$ at a specific point in the reactor (mol/s).
    \item $V$: Cumulative reactor volume from the inlet to a specific point (L). $V_T$ is the total reactor volume.
    \item $z, L$: Distance down the reactor and total reactor length (m).
    \item $A_c$: The cross-sectional area of the tubular reactor (m$^2$).
    \item $k$: Reaction rate constant.
    \item $C_A$: Molar concentration of A (mol/L).
    \item $v, v_0$: Volumetric flow rate at a point in the reactor and at the inlet (L/s).
    \item $F_T$: Total molar flow rate (mol/s).
    \item $P, T, R$: Absolute pressure (bar), absolute temperature (K), and the ideal gas constant.
    \item $X_A$: Fractional conversion of component A.
\end{itemize}
\end{keybox}

\begin{formulabox}[title=PFR Differential Mole Balances (Volume Basis)]
The change in molar flow rate with respect to reactor volume is:
\begin{align*}
    \frac{dF_A}{dV} &= r_A = -kC_A^n \\
    \frac{dF_B}{dV} &= -2r_A = 2kC_A^n
\end{align*}
\end{formulabox}

\subsection*{Conceptual Model of a PFR}
The PFR model simplifies the complex fluid dynamics within a tubular reactor by making several key assumptions.

\begin{conceptbox}[title=The Ideal Plug Flow Model]
\begin{itemize}[itemsep=2pt]
    \item \textbf{Plug Flow:} The fluid is assumed to flow in discrete "plugs," as if separated by invisible pistons.
    \item \textbf{No Axial Mixing:} There is no mixing or diffusion in the direction of flow. Material in one plug does not mix with the material in the plugs ahead of or behind it.
    \item \textbf{Perfect Radial Mixing:} Within each plug (radially, across the tube's diameter), properties like concentration and temperature are assumed to be uniform. This is an idealization; real flows often have velocity and temperature profiles.
    \item \textbf{Batch Reactor Analogy:} Each plug moving through the PFR can be viewed as a tiny, isolated batch reactor that reacts for a specific amount of time equal to its residence time in that segment.
    \item \textbf{Uniform Residence Time:} A key consequence of the model is that all molecules spend the exact same amount of time in the reactor. This is fundamentally different from a CSTR, which exhibits a wide distribution of residence times.
\end{itemize}
\end{conceptbox}

\newpage
\subsection*{Mole Balance Derivation for a PFR}

The PFR design equation is derived by performing a mole balance on a differential slice of the reactor volume, $dV$, at steady state.

For a component A, the general mole balance on the differential volume $dV$ is:

\begin{formulabox}[title=General Mole Balance]
$$ (\text{Rate of Accumulation}) = (\text{Rate In}) - (\text{Rate Out}) + (\text{Rate of Generation}) $$
\end{formulabox}

At steady state, the accumulation term is zero. The balance becomes:

\begin{formulabox}[title=Steady-State Mole Balance]
$$ 0 = F_A\big|_V - F_A\big|_{V+dV} + r_A \cdot dV $$
\end{formulabox}

Here, $r_A$ is the rate of formation of A per unit volume, and $F_A\big|_{V+dV}$ can be expressed using a Taylor series as $F_A\big|_{V+dV} \approx F_A\big|_V + \frac{dF_A}{dV}dV$. The balance simplifies to:

\begin{formulabox}[title=Simplified PFR Design Equation]
$$ 0 = F_A\big|_V - \left(F_A\big|_V + \frac{dF_A}{dV}dV\right) + r_A dV \implies -\frac{dF_A}{dV}dV + r_A dV = 0 $$
\end{formulabox}

\begin{formulabox}[title=PFR Design Equations]
Rearranging gives the fundamental \textbf{differential form} of the PFR design equation:
$$ \frac{dF_A}{dV} = r_A $$
To find the total reactor volume $V$ required for a certain conversion, this equation is integrated. By relating molar flow to conversion ($F_A = F_{A0}(1-X_A) \implies dF_A = -F_{A0}dX_A$), we get the \textbf{integral form}:
$$ V = F_{A0} \int_0^{X_A} \frac{dX_A}{-r_A} $$
\end{formulabox}

\newpage

\subsection*{Accounting for Gas-Phase Reactions}
For liquid-phase reactions, density and volumetric flow rate ($v$) are typically assumed to be constant. For \textbf{gas-phase reactions}, this assumption is often invalid.

\begin{conceptbox}[title=Causes of Volumetric Flow Rate Change in Gas-Phase PFRs]
The volumetric flow rate, $v$, can change along the length of the reactor due to:
\begin{enumerate}
    \item \textbf{Change in Moles:} If a reaction changes the total number of moles (e.g., $A(g) \rightarrow 2B(g)$), the gas will expand or contract.
    \item \textbf{Change in Temperature (T):} In non-isothermal reactors, temperature changes affect gas density ($v \propto T$).
    \item \textbf{Change in Pressure (P):} Pressure drop through a packed bed or due to friction causes the gas to expand ($v \propto 1/P$).
\end{enumerate}
\end{conceptbox}

Because $v$ can change, we cannot simply use $F_A = vC_A$ with a constant $v$. Instead, we must express the concentration, $C_A$, in terms of the variables we are integrating ($F_i, P, T$).

\begin{formulabox}[title=Concentration in Gas-Phase Systems (Ideal Gas)]
Using the ideal gas law, concentration can be written in terms of molar flow rates:
$$ C_A = \frac{P_A}{RT} = \frac{y_A P}{RT} = \left(\frac{F_A}{F_T}\right) \frac{P}{RT} $$
Where $F_T = F_A + F_B + \dots + F_{\text{inerts}}$ is the total molar flow rate. This complete expression for $C_A$ is substituted into the rate law ($r_A$). This creates a system of coupled differential equations for the molar flow rates ($F_i$), temperature ($T$), and pressure ($P$) that must be solved simultaneously with numerical methods.
\end{formulabox}

\newpage

\subsection*{Applications and Design Considerations}

\begin{keybox}[title=Typical PFR Applications]
PFRs are generally preferred for:
\begin{itemize}[itemsep=2pt]
    \item \textbf{Large-scale production} in continuous, steady-state processes.
    \item \textbf{Fast reactions} that do not require long residence times.
    \item \textbf{Most heterogeneous catalytic reactions}, where the reactor is a tube packed with solid catalyst pellets.
    \item Situations where \textbf{high conversion} is desired, as a PFR is typically more volume-efficient than a CSTR for positive-order reactions.
\end{itemize}
\end{keybox}

\begin{conceptbox}[title=Key PFR Design Concerns]
\begin{itemize}[itemsep=2pt]
    \item \textbf{Poor Mixing:} PFRs have no moving parts for agitation. If reactants need to be mixed, static mixers can be installed within the tube.
    \item \textbf{Temperature Control and Hot Spots:} For highly exothermic reactions, the temperature can rise dangerously along the reactor, leading to thermal runaway or undesired side reactions. This is managed by using a \textbf{shell-and-tube reactor}. This design consists of many small-diameter tubes containing the catalyst, all housed within a larger shell. A coolant flows through the shell to remove heat effectively from the large surface area provided by the tubes.
\end{itemize}
\end{conceptbox}

\newpage
\subsection*{Example: Gas-Phase Reaction with Changing Mole Number}

\begin{examplebox}{Gas-Phase PFR with Changing Volumetric Flow}
\textbf{Question:} Develop the equations required to determine the plug flow reactor volume needed to achieve 50\% conversion of reactant A. The gas-phase reaction, $A + 2B \rightarrow 2D$, is carried out in an isothermal PFR at 5.0 atm and 55$^\circ$C. The feed volumetric flow rate is 50 L/min. The rate law is $r_A = -2.5 C_A^{0.5} C_B$ mol/(L$\cdot$min). The feed consists of mole fractions $y_A = 0.2$, $y_B = 0.5$, and $y_C = 0.3$, where C is an inert.
\end{examplebox}

\begin{stepbox}
\begin{enumerate}[label=\textbf{Step \arabic*:}, wide=0pt, leftmargin=*, itemsep=2pt]
    \item \textbf{Strategy:}
    
    This is a gas-phase reaction where the total number of moles changes (1 mole of A + 2 moles of B $\rightarrow$ 2 moles of D, a net change of -1 mole of gas per mole of A reacted). Even at constant temperature and pressure, this change in moles will cause the volumetric flow rate, $v$, to change along the reactor. Therefore, we cannot solve a simple integral in terms of concentration. We must solve a system of differential equations in terms of molar flow rates ($F_i$).

    \item \textbf{Calculate Initial Conditions (at V=0):}
    First, we need the initial molar flow rate of each species. This requires finding the total molar feed rate, $F_{T0}$, using the ideal gas law.
    \begin{itemize}[itemsep=2pt]
        \item Temperature: $T = 55 + 273.15 = 328.15$ K.
        \item Pressure: $P = 5.0$ atm.
        \item Gas Constant: $R = 0.08206$ L$\cdot$atm/(mol$\cdot$K).
    \end{itemize}
    \begin{formulabox}[title=Initial Molar Flow Rates]
    Total Molar Feed Rate:
    $$ F_{T0} = \frac{P v_0}{R T} = \frac{(5.0 \, \text{atm})(50 \, \text{L/min})}{(0.08206 \, \frac{\text{L}\cdot\text{atm}}{\text{mol}\cdot\text{K}})(328.15 \, \text{K})} = 9.29 \, \text{mol/min} $$
    Individual Molar Feed Rates:
    \begin{align*}
        F_{A0} &= y_{A0} F_{T0} = 0.2 \cdot (9.29) = \textbf{1.858 mol/min} \\
        F_{B0} &= y_{B0} F_{T0} = 0.5 \cdot (9.29) = \textbf{4.645 mol/min} \\
        F_{C0} &= y_{C0} F_{T0} = 0.3 \cdot (9.29) = \textbf{2.787 mol/min} \\
        F_{D0} &= 0
    \end{align*}
    \end{formulabox}

\end{enumerate}
\end{stepbox}

\begin{stepbox}
\begin{enumerate}[label=\textbf{Step \arabic*:}, wide=0pt, leftmargin=*, itemsep=2pt, start=3]

    \item \textbf{Formulate the System of Differential Equations:}
    We write the PFR design equation, $\frac{dF_i}{dV} = r_i$, for each species.
    \begin{itemize}[itemsep=2pt]
        \item \textbf{Reactant A:} $\frac{dF_A}{dV} = r_A = -2.5 C_A^{0.5} C_B$
        \item \textbf{Reactant B:} The stoichiometry is 2 moles of B per 1 mole of A, so $r_B = 2r_A$.
            $$ \frac{dF_B}{dV} = 2r_A = -5.0 C_A^{0.5} C_B $$
        \item \textbf{Product D:} The stoichiometry is 2 moles of D formed per 1 mole of A reacted, so $r_D = -2r_A$.
            $$ \frac{dF_D}{dV} = -2r_A = 5.0 C_A^{0.5} C_B $$
        \item \textbf{Inert C:} The inert does not react, so $\frac{dF_C}{dV} = 0$. This means $F_C = F_{C0}$ throughout the reactor.
    \end{itemize}

    \item \textbf{Define Algebraic Relationships and Stopping Condition:}
    The rate laws depend on concentrations, which must be related to the molar flow rates.
    \begin{itemize}[itemsep=2pt]
        \item \textbf{Algebraic Equations:}
        \begin{align*}
            F_T &= F_A + F_B + F_C + F_D \quad (\text{Total molar flow rate}) \\
            v &= v_0 \left(\frac{F_T}{F_{T0}}\right) \quad (\text{Volumetric flow rate, since P and T are constant}) \\
            C_A &= \frac{F_A}{v} \quad (\text{Concentration of A}) \\
            C_B &= \frac{F_B}{v} \quad (\text{Concentration of B})
        \end{align*}
        \item \textbf{Differential Equations:}
        \begin{align*}
             \frac{dF_A}{dV} &= -2.5 C_A^{0.5} C_B \\
             \frac{dF_B}{dV} &= -5.0 C_A^{0.5} C_B \\
             \frac{dF_D}{dV} &= 5.0 C_A^{0.5} C_B
        \end{align*}
        \item \textbf{Initial Conditions:} Use the $F_{i0}$ values from Step 2 at $V=0$.
        \item \textbf{Stopping Condition:} Integrate with respect to $V$ until conversion $X_A = \frac{F_{A0} - F_A}{F_{A0}}$ reaches 0.50.
    \end{itemize}

    \item \textbf{Final Result:}
    Solving this system with a numerical package provides the reactor volume $V$ at which the stopping condition is met. The result is:
    $$ V \approx \textbf{28.8 L} $$

\end{enumerate}
\end{stepbox}

\newpage
\subsection*{Example: Gas-to-Solid Reaction}

\begin{examplebox}{PFR with a Phase Change}
\textbf{Question:} An isothermal plug flow reactor is used for the reaction $10A(g) \rightarrow B(s)$. The rate law is first order in A, with a rate expression $r_A = -10kC_A$, where the intrinsic rate constant is $k=0.30$ L/(mol$\cdot$min). Small solid particles of product B are entrained in the gas flow. The solid is assumed to occupy a negligible volume compared to the gas. The feed of pure A is at a pressure of 10 bar, a temperature of 450 K, and a molar flow rate of 120 mol/min. The reactor volume is 100 L. Determine the exit conversion.
\end{examplebox}

\begin{stepbox}
\begin{enumerate}[label=\textbf{Step \arabic*:}, wide=0pt, leftmargin=*, itemsep=2pt]
    \item \textbf{Strategy and Key Insight:}
    \begin{conceptbox}[title=Constant Concentration due to Phase Change]
    The crucial point in this problem is that 10 moles of gaseous reactant A are consumed and removed from the gas phase, and replaced by a negligible volume of solid B. The feed is pure A, and since only A is removed from the gas phase, the remaining gas is also pure A. Because the temperature and pressure are constant, the concentration of the pure gas A must also be constant throughout the reactor.
    \end{conceptbox}
    
    \item \textbf{Calculate the Constant Gas-Phase Concentration ($C_{A0}$):}
    Using the ideal gas law for the inlet conditions (which apply everywhere in the reactor):
    \begin{itemize}[itemsep=2pt]
        \item $P = 10$ bar
        \item $T = 450$ K
        \item $R = 0.08314$ L$\cdot$bar/(mol$\cdot$K)
    \end{itemize}
    $$ C_A = C_{A0} = \frac{P}{RT} = \frac{10 \, \text{bar}}{(0.08314 \, \frac{\text{L}\cdot\text{bar}}{\text{mol}\cdot\text{K}})(450 \, \text{K})} = \textbf{0.267 mol/L} $$


\end{enumerate}
\end{stepbox}

\begin{stepbox}
\begin{enumerate}[label=\textbf{Step \arabic*:}, wide=0pt, leftmargin=*, itemsep=2pt, start = 3]
    
    \item \textbf{Set up and Integrate the Mole Balance:}
    The PFR design equation is $\frac{dF_A}{dV} = r_A$. The rate of reaction of A is given as $r_A = -10 k C_A$. Since we've established that $C_A$ is constant and equal to $C_{A0}$, the rate of reaction is also constant throughout the reactor.
    $$ \frac{dF_A}{dV} = -10 \cdot k \cdot C_{A0} = \text{constant} $$
    We can integrate this simple differential equation directly from the inlet ($V=0, F_A=F_{A0}$) to the outlet ($V=V_T, F_A=F_{A,exit}$):
    $$ \int_{F_{A0}}^{F_{A,exit}} dF_A = \int_0^{V_T} (-10 k C_{A0}) dV $$
    $$ F_{A,exit} - F_{A0} = -10 \cdot k \cdot C_{A0} \cdot V_T $$

    \item \textbf{Calculate Conversion:}
    The conversion $X_A$ is defined as $X_A = \frac{F_{A0} - F_{A,exit}}{F_{A0}}$. The term $F_{A0} - F_{A,exit}$ represents the total moles of A reacted per minute. From our integrated equation:
    \begin{formulabox}[title=Moles of A Reacted]
    $$ \text{Moles Reacted} = F_{A0} - F_{A,exit} = 10 \cdot k \cdot C_{A0} \cdot V_T $$
    \end{formulabox}
    Now, substitute the known values:
    \begin{itemize}[itemsep=2pt]
        \item $k = 0.30$ L/(mol$\cdot$min)
        \item $C_{A0} = 0.267$ mol/L
        \item $V_T = 100$ L
    \end{itemize}
    $$ \text{Moles Reacted} = 10 \cdot (0.30) \cdot (0.267) \cdot (100) = 80.1 \, \text{mol/min} $$
    The conversion is the ratio of moles reacted to moles fed ($F_{A0} = 120$ mol/min):
    $$ X_A = \frac{80.1 \, \text{mol/min}}{120 \, \text{mol/min}} = 0.6675 $$
    The exit conversion is \textbf{67\%}.

\end{enumerate}
\end{stepbox}

\newpage
\section*{Comparing Reactors in Series}
A fundamental task in chemical engineering is selecting the appropriate reactor type and configuration to achieve a desired conversion efficiently. This section uses a graphical method, the Levenspiel plot, to compare the performance and required volumes of Continuous Stirred-Tank Reactors (CSTRs) and Plug Flow Reactors (PFRs), both individually and in series.

\subsection*{The Levenspiel Plot: A Graphical Tool for Reactor Design}
For an isothermal reaction, the Levenspiel plot provides a powerful visual representation of reactor performance by plotting a function of the inverse reaction rate against conversion.

\begin{conceptbox}[title=Understanding the Levenspiel Plot]
A Levenspiel plot graphs the term $\frac{F_{A0}}{-r_A}$ on the y-axis versus the conversion, $X_A$, on the x-axis.
\begin{itemize}[itemsep=2pt]
    \item \textbf{Y-Axis ($\frac{F_{A0}}{-r_A}$):} This term has units of volume (e.g., L or m$^3$). A higher point on the y-axis signifies a slower reaction rate (since $-r_A$ is in the denominator).
    \item \textbf{X-Axis ($X_A$):} This axis represents the fractional conversion of reactant A, ranging from 0 to 1.
\end{itemize}
The area on this plot represents the reactor volume required to achieve a certain conversion, making it an excellent tool for visual comparison.
\end{conceptbox}

\subsection*{Visualizing Single Reactor Volume}
For a given conversion, the required volume for a CSTR and a PFR can be represented as distinct areas on the Levenspiel plot.

\begin{formulabox}[title=CSTR Design Equation and Graphical Area]
The design equation for a CSTR is an algebraic equation:
$$ V_{CSTR} = \frac{F_{A0} X_A}{(-r_A)_{\text{exit}}} \quad \text{or} \quad V_{CSTR} = \left(\frac{F_{A0}}{-r_A}\right)_{\text{at }X_A} \times X_A $$
Graphically, this represents the area of a \textbf{rectangle}. The height of the rectangle is the value of $\frac{F_{A0}}{-r_A}$ at the final exit conversion (the slowest rate in the process), and the width is the total conversion $X_A$.
\end{formulabox}

\begin{formulabox}[title=PFR Design Equation and Graphical Area]
The design equation for a PFR is an integral equation:
$$ V_{PFR} = F_{A0} \int_0^{X_A} \frac{dX_A}{-r_A} \quad \text{or} \quad V_{PFR} = \int_0^{X_A} \left(\frac{F_{A0}}{-r_A}\right) dX_A $$
Graphically, this represents the \textbf{area under the curve} of the Levenspiel plot, integrated from zero conversion to the final conversion $X_A$.
\end{formulabox}


\begin{keybox}[title=Single Reactor Comparison]
For any reaction where the rate decreases as conversion increases (a typical nth-order reaction), the Levenspiel plot curve will be upward-sloping. In this common case, the rectangle representing the CSTR volume will always be larger than the area under the curve representing the PFR volume for the same final conversion.

\textbf{Conclusion:} To achieve the same conversion for a typical reaction, a \textbf{PFR requires a smaller volume than a CSTR}. This is because the entire CSTR must operate at the lowest reaction rate (corresponding to the final exit conditions), whereas the PFR takes advantage of the higher reaction rates that exist at lower conversions near the reactor inlet.
\end{keybox}

\newpage
\subsection*{Comparing Reactors in Series}
The graphical method also provides clear insight into configuring multiple reactors in series to optimize total volume.

\begin{conceptbox}[title=PFRs in Series]
The total volume required to achieve a final conversion $X_f$ using two PFRs in series is the sum of their individual volumes.
$$ V_{total} = V_1 + V_2 = \int_0^{X_1} \left(\frac{F_{A0}}{-r_A}\right) dX_A + \int_{X_1}^{X_f} \left(\frac{F_{A0}}{-r_A}\right) dX_A = \int_0^{X_f} \left(\frac{F_{A0}}{-r_A}\right) dX_A $$
The total volume is exactly the same as the volume of a single PFR used to achieve the same final conversion $X_f$. Therefore, for an isothermal reaction, there is \textbf{no volumetric advantage or disadvantage} to using multiple PFRs in series instead of one larger PFR.
\end{conceptbox}

\begin{conceptbox}[title=CSTRs in Series]
Using two or more CSTRs in series offers a significant advantage in reducing the total required volume compared to a single CSTR for the same final conversion.
\begin{itemize}[itemsep=2pt]
    \item A single CSTR to reach high conversion must be very large, as its volume is defined by the large rectangle at the slowest reaction rate.
    \item If two CSTRs are used, the first operates to an intermediate conversion ($X_1$), and its volume is a smaller rectangle. The second reactor then operates from $X_1$ to the final conversion $X_f$. Its volume is also a rectangle, but its height is determined by the final, slowest rate.
\end{itemize}
The sum of the areas of the two smaller rectangles is significantly less than the area of the single large rectangle required for one CSTR.
\end{conceptbox}

\begin{keybox}[title=Series Reactor Comparison Conclusion]
\begin{itemize}[itemsep=2pt]
    \item For a given final conversion, the total volume required for \textbf{multiple CSTRs in series is less} than the volume of a single CSTR.
    \item The optimal arrangement for minimizing volume is an infinite number of infinitesimally small CSTRs in series, which is mathematically equivalent to a single PFR. This reinforces the conclusion that PFRs are more volume-efficient for typical reactions.
\end{itemize}
\end{keybox}

\newpage
\section*{Adiabatic Temperature and Energy Balances}
For an adiabatic process, there is no heat exchange with the surroundings ($\Delta H = 0$). The heat generated or consumed by the reaction directly changes the temperature of the reactor contents. This section demonstrates how to calculate the final temperature (the adiabatic temperature) for a reaction with a given conversion.

\subsection*{Example: Adiabatic Temperature Calculation}

\begin{examplebox}{Calculating Adiabatic Temperature Rise}
\textbf{Question:} The reaction $NO + 0.5 O_2 \rightarrow NO_2$ occurs in an adiabatic reactor. The feed enters at 100$^\circ$C and consists of 1 mole of NO and 0.5 moles of O$_2$. The heat of reaction at 25$^\circ$C is $\Delta H_{rxn, 298K} = -57.0$ kJ/mol. The heat capacities ($C_p/R$) are given as:
\begin{itemize}
    \item $C_p(\text{NO})/R = 3.5 + 0.001 T$
    \item $C_p(\text{O}_2)/R = 3.2 + 0.0018 T$
    \item $C_p(\text{NO}_2)/R = 4.2 + 0.0025 T$
\end{itemize}
where T is in Kelvin. Calculate the adiabatic temperature for 30\% conversion of NO.
\end{examplebox}

\begin{stepbox}[title=Solution (Part 1 of 4): Strategy and Hypothetical Path]
\begin{enumerate}[label=\textbf{Step \arabic*:}, wide=0pt, leftmargin=*, itemsep=2pt]
    \item \textbf{Strategy: Enthalpy as a State Function}
    \begin{conceptbox}[title=Hypothetical Enthalpy Path]
    Since enthalpy is a state function, the overall enthalpy change of a process is independent of the path taken. For an adiabatic process, the total enthalpy change is zero ($\Delta H_{total} = 0$). We can construct a convenient hypothetical three-step path from the initial state (reactants at 373 K) to the final state (product mixture at $T_f$) and set the sum of the enthalpy changes to zero.
    \begin{enumerate}[label=(\alph*), itemsep=2pt]
        \item \textbf{$\Delta H_1$:} Cool the reactants from the feed temperature (373 K) to a reference temperature (298 K).
        \item \textbf{$\Delta H_2$:} Carry out the reaction at 298 K to the specified conversion (30\%).
        \item \textbf{$\Delta H_3$:} Heat the final product mixture from 298 K to the unknown final adiabatic temperature, $T_f$.
    \end{enumerate}
    The energy balance is: $\Delta H_1 + \Delta H_2 + \Delta H_3 = \Delta H_{total} = 0$.
    \end{conceptbox}
\end{enumerate}
\end{stepbox}

\newpage
\begin{stepbox}[title=Solution (Part 2 of 4): Calculating Enthalpy Changes]
\begin{enumerate}[label=\textbf{Step \arabic*:}, wide=0pt, leftmargin=*, itemsep=2pt]\setcounter{enumi}{1}
    \item \textbf{Calculate $\Delta H_1$ (Cooling Reactants):}
    This is the enthalpy change from cooling the initial 1 mole of NO and 0.5 moles of O$_2$.
    $$ \Delta H_1 = \int_{373K}^{298K} \sum n_i C_{p,i} \, dT = \int_{373}^{298} R \left[ (1)(3.5 + 0.001T) + (0.5)(3.2 + 0.0018T) \right] dT $$
    $$ \Delta H_1 = R \int_{373}^{298} (5.1 + 0.0019T) dT = R \left[ 5.1T + \frac{0.0019}{2}T^2 \right]_{373}^{298} $$
    Plugging in the limits and $R = 8.314$ J/(mol$\cdot$K):
    $$ \Delta H_1 = 8.314 \left[ (5.1(298-373)) + (0.00095(298^2-373^2)) \right] $$
    $$ \Delta H_1 = 8.314 [-382.5 - 47.8] = -3578 \, \text{J} = \textbf{-3.58 kJ} $$
    The sign is negative because heat is removed during cooling.

    \item \textbf{Calculate $\Delta H_2$ (Heat of Reaction at 298 K):}
    The reaction proceeds to 30\% conversion, so 0.3 moles of the basis reactant (NO) react.
    $$ \Delta H_2 = (\text{moles NO reacted}) \times (\Delta H_{rxn, 298K}) = (0.3 \, \text{mol}) \times (-57.0 \, \text{kJ/mol}) $$
    $$ \Delta H_2 = \textbf{-17.1 kJ} $$
\end{enumerate}
\end{stepbox}

\newpage
\begin{stepbox}[title=Solution (Part 3 of 4): Setting up the Final Enthalpy Term]
\begin{enumerate}[label=\textbf{Step \arabic*:}, wide=0pt, leftmargin=*, itemsep=2pt]\setcounter{enumi}{3}
    \item \textbf{Set up $\Delta H_3$ (Heating the Final Mixture):}
    First, determine the composition of the final mixture based on 30\% conversion of 1 mole of NO.
    \begin{itemize}[itemsep=2pt]
        \item Moles NO remaining: $1.0 - 0.3 = 0.7$ mol
        \item Moles O$_2$ remaining: $0.5 - (0.5 \times 0.3) = 0.35$ mol
        \item Moles NO$_2$ formed: $1 \times 0.3 = 0.3$ mol
    \end{itemize}
    Now, set up the integral for heating this final mixture from 298 K to $T_f$.
    $$ \Delta H_3 = \int_{298K}^{T_f} \sum n_i C_{p,i} \, dT $$
    $$ \Delta H_3 = R \int_{298}^{T_f} \left[ (0.7)(3.5+0.001T) + (0.35)(3.2+0.0018T) + (0.3)(4.2+0.0025T) \right] dT $$
    $$ \Delta H_3 = R \int_{298}^{T_f} (2.45+0.0007T + 1.12+0.00063T + 1.26+0.00075T) dT $$
    $$ \Delta H_3 = R \int_{298}^{T_f} (4.83 + 0.00208T) dT $$
    Integrating this expression gives:
    \begin{formulabox}[title=Expression for $\Delta H_3$]
    $$ \Delta H_3 = R \left[ 4.83T + \frac{0.00208}{2}T^2 \right]_{298}^{T_f} $$
    $$ \Delta H_3 (\text{in kJ}) = \frac{8.314}{1000} \left[ 4.83(T_f-298) + 0.00104(T_f^2 - 298^2) \right] $$
    \end{formulabox}
\end{enumerate}
\end{stepbox}

\newpage
\begin{stepbox}[title=Solution (Part 4 of 4): Solving for the Final Temperature]
\begin{enumerate}[label=\textbf{Step \arabic*:}, wide=0pt, leftmargin=*, itemsep=2pt]\setcounter{enumi}{4}
    \item \textbf{Solve the Energy Balance for the Final Temperature ($T_f$):}
    Set the sum of the enthalpies to zero: $\Delta H_1 + \Delta H_2 + \Delta H_3 = 0$.
    $$ -3.58 \, \text{kJ} - 17.1 \, \text{kJ} + 0.008314 \left[ 4.83(T_f-298) + 0.00104(T_f^2 - 298^2) \right] = 0 $$
    $$ -20.68 + 0.008314 [4.83 T_f - 1439.34 + 0.00104 T_f^2 - 92.1] = 0 $$
    $$ -20.68 + 0.04017 T_f - 11.97 + 8.647 \times 10^{-6} T_f^2 - 0.766 = 0 $$
    This simplifies to a quadratic equation for $T_f$:
    \begin{formulabox}[title=Final Quadratic Equation for Temperature]
    $$ 8.647 \times 10^{-6} T_f^2 + 0.04017 T_f - 33.416 = 0 $$
    \end{formulabox}
    Solving this using the quadratic formula, $T_f = \frac{-b \pm \sqrt{b^2 - 4ac}}{2a}$, yields two roots. Only the positive root is physically meaningful.
    $$ T_f = \frac{-0.04017 + \sqrt{0.04017^2 - 4(8.647 \times 10^{-6})(-33.416)}}{2(8.647 \times 10^{-6})} \approx \textbf{720 K} $$
    The final adiabatic temperature is approximately 720 K (or 447$^\circ$C).
\end{enumerate}
\end{stepbox}

\newpage
\subsection*{Example: Endothermic Reaction in an Adiabatic PBR}

\begin{examplebox}{Adiabatic PBR with Constant Heat Capacities}
\textbf{Question:} An endothermic reaction occurs in an adiabatic packed bed reactor (PBR). The feed enters at 500 K, and the exit stream is at 487 K with 40\% conversion. The heat capacities and heat of reaction are assumed to be constant.
\begin{enumerate}[label=(\alph*)]
    \item Is the reaction endothermic or exothermic?
    \item If the space time is doubled, causing the outlet temperature to drop to 482 K, what is the new conversion?
\end{enumerate}
\end{examplebox}

\begin{stepbox}
\begin{enumerate}[label=\textbf{Step \arabic*:}, wide=0pt, leftmargin=*, itemsep=2pt]
    \item \textbf{Part (a): Endothermic or Exothermic?}
    
    In an adiabatic system, no heat is exchanged with the surroundings. Any temperature change is caused solely by the heat of reaction. If the temperature drops, the reaction must be consuming energy from the fluid (endothermic). If the temperature rises, the reaction must be releasing energy (exothermic).

    The reactor is adiabatic, and the temperature drops from 500 K to 487 K. Therefore, the reaction is \textbf{endothermic}.

    \item \textbf{Part (b): Calculating New Conversion via Proportionality}
    For an adiabatic reactor with constant heat capacities, the energy balance shows a direct linear relationship between conversion ($X$) and the change in temperature ($\Delta T = T_{out} - T_{in}$).
    $$ X_A = \frac{\sum \Theta_i C_{p,i}}{-\Delta H_{rxn}}(T - T_{in}) = (\text{Constant}) \cdot (T_{out} - T_{in}) $$
    We can solve for this constant using the initial set of conditions.
    \begin{itemize}[itemsep=2pt]
        \item Initial conversion: $X_1 = 0.40$
        \item Initial temperature change: $\Delta T_1 = 487 \, \text{K} - 500 \, \text{K} = -13 \, \text{K}$
    \end{itemize}
    $$ 0.40 = \text{Constant} \cdot (-13 \, \text{K}) \implies \text{Constant} = \frac{0.40}{-13} \approx -0.03077 \, \text{K}^{-1} $$
    Now we use this constant to find the new conversion ($X_2$) for the new temperature change. Doubling the space time allows the reaction to proceed further, consuming more heat and dropping the temperature more.
    \begin{itemize}[itemsep=2pt]
        \item New temperature change: $\Delta T_2 = 482 \, \text{K} - 500 \, \text{K} = -18 \, \text{K}$
    \end{itemize}
    $$ X_2 = (\text{Constant}) \cdot (\Delta T_2) = (-0.03077 \, \text{K}^{-1}) \cdot (-18 \, \text{K}) = 0.554 $$
    The new conversion is \textbf{55\%}.
\end{enumerate}
\end{stepbox}

\newpage
\subsection*{Example: Initial Temperature Profile in a Heated PFR}

\begin{examplebox}{Initial Temperature Profile in a Heated PFR}
\textbf{Question:} An endothermic, second-order reaction ($A \rightarrow \text{Products, } -r_A = k C_A^2$) takes place in a PFR with a diameter of 8 cm. The reactor is heated by a steam jacket that maintains a constant wall temperature of 300$^\circ$C. The feed enters at 250$^\circ$C. Other data includes: $\Delta H_{rxn} = 80$ kJ/mol, $k = 0.5$ L/(mol$\cdot$min) at 250$^\circ$C, and $U = 5000$ kJ/(m$^2\cdot$h$\cdot$K). Immediately downstream from the reactor inlet, will the fluid temperature increase, decrease, or stay the same?
\end{examplebox}

\begin{stepbox}
\begin{enumerate}[label=\textbf{Step \arabic*:}, wide=0pt, leftmargin=*, itemsep=2pt]
    \item \textbf{Strategy: Analyze the Differential Energy Balance}
    \begin{conceptbox}[title=Comparing Heat Effects]
    To determine the initial temperature trend, we must evaluate the sign of the temperature derivative with respect to reactor volume, $\frac{dT}{dV}$, at the inlet ($V=0$). This requires comparing the rate of heat consumption by the endothermic reaction with the rate of heat addition from the steam jacket at the reactor entrance.
    \end{conceptbox}
    The differential energy balance for a PFR with heat exchange is:
    $$ \frac{dT}{dV} = \frac{Ua(T_a - T) + r_A(-\Delta H_{rxn})}{\sum F_i C_{p,i}} $$
    The sign of $\frac{dT}{dV}$ is determined by the sign of the numerator: $Ua(T_a - T) + r_A(-\Delta H_{rxn})$.
    \item \textbf{Evaluate Heat Transfer Term at the Inlet:}
    This term represents heat addition from the steam jacket. The heat transfer area per unit volume is $a = 4/D$.
    \begin{itemize}[itemsep=2pt]
        \item $U = 5000 \, \frac{\text{kJ}}{\text{m}^2 \cdot \text{h} \cdot \text{K}} \times \frac{1 \text{ h}}{60 \text{ min}} = 83.33 \, \frac{\text{kJ}}{\text{m}^2 \cdot \text{min} \cdot \text{K}}$
        \item $D = 8 \, \text{cm} = 0.08 \, \text{m}$
        \item Temperature difference at inlet: $T_a - T = 300^\circ\text{C} - 250^\circ\text{C} = 50 \, \text{K}$
    \end{itemize}
    $$ \text{Heat Added} = U \left(\frac{4}{D}\right)(T_a - T) = (83.33) \left(\frac{4}{0.08}\right)(50) = 208,333 \, \frac{\text{kJ}}{\text{m}^3 \cdot \text{min}} $$
    $$ \text{Heat Added} = 208,333 \frac{\text{kJ}}{\text{m}^3\cdot\text{min}} \times \frac{1 \, \text{m}^3}{1000 \, \text{L}} = \textbf{208.3} \, \frac{\text{kJ}}{\text{L} \cdot \text{min}} $$
\end{enumerate}
\end{stepbox}

\newpage
\begin{stepbox}
\begin{enumerate}[label=\textbf{Step \arabic*:}, wide=0pt, leftmargin=*, itemsep=2pt, start=3]
    \item \textbf{Evaluate Heat Consumption Term at the Inlet:}
    This term represents heat consumed by the endothermic reaction ($\Delta H_{rxn} > 0$).
    $$ \text{Heat Consumed} = -r_A(\Delta H_{rxn}) = (k C_{A0}^2)(\Delta H_{rxn}) $$
    $$ \text{Heat Consumed} = (0.5 \frac{\text{L}}{\text{mol}\cdot\text{min}} \cdot C_{A0}^2)(80 \frac{\text{kJ}}{\text{mol}}) = \textbf{40} \cdot C_{A0}^2 \, \frac{\text{kJ}}{\text{L} \cdot \text{min}} $$
    \item \textbf{Compare Heat Effects and Conclude:}
    The initial temperature trend depends on the sign of $(208.3 - 40 C_{A0}^2)$.
    The problem implies a realistic operating condition where the initial concentration is not excessively high. For any reasonable initial concentration (e.g., even if $C_{A0}$ were as high as 2 mol/L, the heat consumption term would be $40 \cdot 2^2 = 160$ kJ/(L$\cdot$min)), the heat addition term dominates.
    $$ \text{Heat Added (208.3)} > \text{Heat Consumed (e.g., 160)} $$
    Since the rate of heat addition from the steam jacket is greater than the rate of heat consumption by the reaction at the inlet, the numerator in the energy balance is positive. Therefore, $\frac{dT}{dV}$ is positive, and the temperature will \textbf{increase} immediately downstream from the inlet.
\end{enumerate}
\end{stepbox}

\end{document}



