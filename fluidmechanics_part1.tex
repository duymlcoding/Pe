\documentclass[12pt]{article}
\usepackage[paperwidth=8.5in, paperheight=11in, margin=1.0in, headheight=15pt]{geometry}
\usepackage{amsmath,amssymb,amsthm}
\usepackage[most]{tcolorbox}
\usepackage{enumitem}
\usepackage{xcolor}
\usepackage{hyperref}
\usepackage{fancyhdr}
\usepackage{titlesec}
\usepackage{graphicx}
% Define custom colors for chemical engineering theme
\definecolor{conceptcolor}{RGB}{52, 73, 94}      % Dark blue-gray
\definecolor{formulacolor}{RGB}{231, 76, 60}     % Red for formulas
\definecolor{examplecolor}{RGB}{39, 174, 96}     % Green for examples
\definecolor{stepcolor}{RGB}{142, 68, 173}       % Purple for solution steps
\definecolor{keycolor}{RGB}{243, 156, 18}        % Orange for key points
% Configure fancy headers
\pagestyle{fancy}
\fancyhf{}
\fancyhead[L]{PE Study Guide}
\fancyhead[R]{Process Fundamentals}
\fancyfoot[C]{\thepage}
\renewcommand{\baselinestretch}{1.1}
\setlength{\parindent}{0.25in}
\setlength{\parskip}{3pt}
% Configure section formatting
\titleformat{\section}
  {\normalfont\LARGE\bfseries\color{conceptcolor}}
  {\thesection}{1em}{}
\titleformat{\subsection}
  {\normalfont\Large\bfseries\color{conceptcolor}}
  {\thesubsection}{1em}{}
% Define custom environments
\newtcolorbox{conceptbox}[1][]{
  enhanced,
  colback=conceptcolor!10,
  colframe=conceptcolor,
  arc=3mm,
  title=Key Concept,
  fonttitle=\bfseries\sffamily\normalsize,
  fontupper=\small,
  #1
}
\newtcolorbox{formulabox}[1][]{
  enhanced,
  colback=formulacolor!10,
  colframe=formulacolor,
  arc=2mm,
  title=Important Formula,
  fonttitle=\bfseries\sffamily\normalsize,
  fontupper=\small,
  #1
}
\newtcolorbox{examplebox}[2][]{
  enhanced,
  colback=examplecolor!10,
  colframe=examplecolor,
  arc=3mm,
  title=Example Problem: #2,
  fonttitle=\bfseries\sffamily\normalsize,
  fontupper=\small,
  #1
}
\newtcolorbox{stepbox}[1][]{
  enhanced,
  colback=stepcolor!10,
  colframe=stepcolor,
  arc=2mm,
  title=Solution Steps,
  fonttitle=\bfseries\sffamily\normalsize,
  fontupper=\small,
  #1
}
\newtcolorbox{keybox}[1][]{
  enhanced,
  colback=keycolor!10,
  colframe=keycolor,
  arc=2mm,
  title=Key Variables \& Definitions,
  fonttitle=\bfseries\sffamily\normalsize,
  fontupper=\small,
  #1
}

\begin{document}

\begin{center}
    \Huge\textbf{\color{conceptcolor}Fluid Mechanics}
\end{center}
\hrule

\section*{Mechanical Energy Balance}
The Mechanical Energy Balance is a fundamental tool in fluid dynamics, derived from the first law of thermodynamics. It is used to analyze systems where mechanical forms of energy are of primary interest, such as in pipe flow, pumps, and turbines.

\subsection*{Fundamental Equations}
\begin{keybox}[title=Variable Definitions]
\begin{itemize}[itemsep=2pt]
    \item $\Delta P$: Change in pressure ($P_{out} - P_{in}$) [Pa].
    \item $\rho$: Fluid density [kg/m$^3$].
    \item $\Delta u^2$: Change in the square of the average fluid velocity ($u_{out}^2 - u_{in}^2$) [m$^2$/s$^2$].
    \item $g$: Gravitational constant (9.81 m/s$^2$).
    \item $\Delta z$: Change in elevation ($z_{out} - z_{in}$) [m].
    \item $\dot{W}_s$: Rate of shaft work added to the fluid (by a pump) or removed from the fluid (by a turbine) [J/s or W].
    \item $\dot{m}$: Mass flow rate of the fluid [kg/s].
    \item $\hat{F}$: Frictional energy loss per unit mass [J/kg]. This term is always positive.
\end{itemize}
\end{keybox}

\begin{formulabox}[title=Mechanical Energy Balance (with Friction)]
This is the most common form used for real fluids. The term $\hat{F}$ accounts for the conversion of mechanical energy into thermal energy due to viscous dissipation (friction).
$$ \frac{\Delta P}{\rho} + \frac{\Delta u^2}{2} + g\Delta z + \hat{F} = \frac{\dot{W}_s}{\dot{m}} $$
\end{formulabox}

\begin{formulabox}[title=The Bernoulli Equation (Ideal Fluid)]
This is a simplified form for a system with no shaft work ($\dot{W}_s=0$) and no frictional losses ($\hat{F}=0$). It applies to ideal, inviscid fluid flow.
$$ \frac{\Delta P}{\rho} + \frac{\Delta u^2}{2} + g\Delta z = 0 $$
\end{formulabox}

\subsection*{Derivation from the First Law of Thermodynamics}
The Mechanical Energy Balance is a practical rearrangement of the First Law of Thermodynamics for a steady-state, open system.

\begin{stepbox}[title=Derivation of the Mechanical Energy Balance]
\begin{enumerate}[label=\textbf{Step \arabic*:}, wide=0pt, leftmargin=*, itemsep=2pt]
    \item \textbf{Start with the General Energy Balance:} The first law for a steady-state, open system is:
    $$ \dot{m} \left( \Delta h + \frac{\Delta u^2}{2} + g\Delta z \right) = \dot{Q} + \dot{W}_s $$
    where $\Delta h$ is the change in specific enthalpy ($h_{out} - h_{in}$).

    \item \textbf{Introduce the Definition of Enthalpy:} Substitute the definition of specific enthalpy, $h = \bar{U} + Pv$, where $\bar{U}$ is specific internal energy and $v$ is specific volume ($v=1/\rho$):
    $$ \dot{m} \left( \Delta \bar{U} + \Delta(Pv) + \frac{\Delta u^2}{2} + g\Delta z \right) = \dot{Q} + \dot{W}_s $$

    \item \textbf{Apply the Incompressible Fluid Assumption:} For liquids, we can assume the fluid is incompressible, meaning its density $\rho$ (and specific volume $v$) is constant. This simplifies the pressure-volume term:
    $$ \Delta(Pv) = P_{out}v - P_{in}v = v(P_{out} - P_{in}) = v\Delta P = \frac{\Delta P}{\rho} $$

    \item \textbf{Rearrange and Define Frictional Loss:} Substitute the simplified pressure term and divide the entire equation by the mass flow rate $\dot{m}$:
    $$ \Delta \bar{U} + \frac{\Delta P}{\rho} + \frac{\Delta u^2}{2} + g\Delta z = \frac{\dot{Q}}{\dot{m}} + \frac{\dot{W}_s}{\dot{m}} $$
    Moving the internal energy and heat transfer terms to the left side isolates the mechanical terms:
    $$ \frac{\Delta P}{\rho} + \frac{\Delta u^2}{2} + g\Delta z + \left( \Delta \bar{U} - \frac{\dot{Q}}{\dot{m}} \right) = \frac{\dot{W}_s}{\dot{m}} $$
    The term $(\Delta \bar{U} - \dot{Q}/\dot{m})$ represents the irreversible loss of mechanical energy to friction. We define this group as the frictional loss term, $\hat{F}$, which gives the final form of the balance.
\end{enumerate}
\end{stepbox}

\newpage
\subsection*{Example: Flow from a Coffee Urn}

\begin{examplebox}{Calculating Flow from a Coffee Urn}
\textbf{Question:} A 60-cm tall coffee urn is filled to the top. It dispenses coffee through a 0.7-cm diameter nozzle that is 12 cm above the table surface. How long does it take to pour a 200-mL cup of coffee?
\end{examplebox}

\begin{stepbox}
\begin{enumerate}[label=\textbf{Step \arabic*:}, wide=0pt, leftmargin=*, itemsep=2pt]
    \item \textbf{Strategy and Assumptions:}
    
    We can model this using the Bernoulli equation, which is appropriate for low-viscosity fluids with negligible friction over short distances. We apply the equation between the top surface of the coffee (point 1) at height $z_1 = 0.60$ m and the nozzle exit (point 2) at height $z_2 = 0.12$ m. We assume no friction and no shaft work.
    
    \item \textbf{Analyze the Bernoulli Equation Terms:}
    $$ \frac{P_2 - P_1}{\rho} + \frac{u_2^2 - u_1^2}{2} + g(z_2 - z_1) = 0 $$
    
    Pressure: Both the top surface and nozzle exit are open to atmosphere, so $P_1 = P_2$ and $\Delta P = 0$.
    
    Velocity: The urn diameter is much larger than the nozzle, so the liquid level drops slowly and $u_1 \approx 0$. We solve for the exit velocity $u_2$.
    
    Potential Energy: The height change is $z_2 - z_1 = 0.12 - 0.60 = -0.48$ m.
    
    \item \textbf{Solve for Exit Velocity ($u_2$):}
    The Bernoulli equation simplifies to Torricelli's Law:
    $$ \frac{u_2^2}{2} + g(-0.48 \, \text{m}) = 0 \implies u_2 = \sqrt{2 \cdot g \cdot (0.48 \, \text{m})} $$
    $$ u_2 = \sqrt{2 \cdot (9.81 \, \text{m/s}^2) \cdot (0.48 \, \text{m})} = \sqrt{9.4176 \, \text{m}^2/\text{s}^2} = 3.069 \, \text{m/s} $$
    
    \item \textbf{Calculate Time to Fill Cup:}
    Find the volumetric flow rate through the nozzle. The nozzle area is $A_2 = \frac{\pi d^2}{4} = \frac{\pi (0.007)^2}{4} = 3.848 \times 10^{-5} \, \text{m}^2$. The volumetric flow rate is $v = A_2 \cdot u_2 = (3.848 \times 10^{-5}) \cdot (3.069) = 1.181 \times 10^{-4} \, \text{m}^3/\text{s}$.
    
    For a 200 mL cup ($2 \times 10^{-4} \, \text{m}^3$):
    $$ \text{Time} = \frac{2 \times 10^{-4}}{1.181 \times 10^{-4}} = 1.69 \, \text{s} $$
    The time required is approximately \textbf{1.7 seconds}.
\end{enumerate}
\end{stepbox}

\newpage
\subsection*{Example: Vertical Water Jet}

\begin{examplebox}{Calculating the Height of a Vertical Water Jet}
\textbf{Question:} Water flows through a 2.5-cm ID pipe at 115 L/min and 0.15-bar gauge pressure. It vents vertically 1.0 m above the pipe. How high will the water shoot from the vent? Assume no frictional losses.
\end{examplebox}

\begin{stepbox}
\begin{enumerate}[label=\textbf{Step \arabic*:}, wide=0pt, leftmargin=*, itemsep=2pt]
    \item \textbf{Strategy and Coordinate System:}
    We will apply the Bernoulli equation between a point inside the pipe just before the vent (point 1) and the highest point of the water jet (point 2). We assume no friction and steady state. Let the centerline of the pipe be the reference height, so $z_1 = 0$. Point 2 is the peak of the jet's trajectory at height $z_2 = 1.0 \, \text{m} + H$, where $H$ is the height above the vent.
    
    \item \textbf{Analyze the Bernoulli Equation Terms:}
    The Bernoulli equation states that the total mechanical energy is conserved between points 1 and 2.
    $$ \frac{P_1}{\rho} + \frac{u_1^2}{2} + gz_1 = \frac{P_2}{\rho} + \frac{u_2^2}{2} + gz_2 $$
    
    Given: $P_1 = 0.15$ bar (gauge), $P_2 = 0$ bar (gauge) at atmospheric pressure, $u_2 = 0$ m/s at maximum height, $z_1 = 0$ m and $z_2 = 1.0 + H$.
    
    \item \textbf{Calculate Inlet Velocity ($u_1$):}
    Convert the volumetric flow rate to SI units:
    $$ v = 115 \frac{\text{L}}{\text{min}} \times \frac{1 \, \text{m}^3}{1000 \, \text{L}} \times \frac{1 \, \text{min}}{60 \, \text{s}} = 0.001917 \, \text{m}^3/\text{s} $$
    The pipe area is $A_1 = \frac{\pi d^2}{4} = \frac{\pi (0.025 \, \text{m})^2}{4} = 0.0004909 \, \text{m}^2$.
    $$ u_1 = \frac{v}{A_1} = \frac{0.001917 \, \text{m}^3/\text{s}}{0.0004909 \, \text{m}^2} = 3.905 \, \text{m/s} $$
    
    \item \textbf{Solve for Height ($H$):}
    Substitute the known values into the Bernoulli equation using SI units. Convert $P_1 = 0.15 \, \text{bar} = 15,000 \, \text{Pa}$ and use $\rho_{water} = 1000$ kg/m$^3$ and $g = 9.81$ m/s$^2$.
    $$ \frac{15000}{1000} + \frac{(3.905)^2}{2} + 0 = 0 + 0 + 9.81 \cdot (1.0 + H) $$
    $$ 15.0 + 7.625 = 9.81 \cdot (1.0 + H) $$
    $$ H = \frac{22.625}{9.81} - 1.0 = 2.306 - 1.0 = 1.306 \, \text{m} $$
    The water will shoot approximately \textbf{1.3 meters} above the vent.
\end{enumerate}
\end{stepbox}

\newpage
\section*{The Bernoulli Equation: A Force Balance Approach}
While the Mechanical Energy Balance is derived from thermodynamics (an energy balance), the famous Bernoulli equation can also be understood from first principles using Newton's second law ($\vec{F}=m\vec{a}$), which is a force balance. This approach provides a deeper physical intuition for how pressure, velocity, and elevation are related in a moving fluid.

\subsection*{Fundamental Equations}
The derivation begins with a force balance on a tiny fluid element and, after several key simplifying assumptions, results in the well-known algebraic form of the Bernoulli equation.

\begin{keybox}[title=Key Variables in this Section]
\begin{itemize}[itemsep=2pt]
    \item $\nu$: Fluid velocity along a streamline [m/s]. (Note: We use $\nu$ here to distinguish it from reactor volume $V$).
    \item $P$: Static pressure, the pressure you would feel if you were moving with the fluid [Pa or N/m$^2$].
    \item $\rho$: Fluid density [kg/m$^3$].
    \item $g$: Acceleration due to gravity (9.81 m/s$^2$).
    \item $\gamma$: Specific weight of the fluid, defined as $\gamma = \rho g$ [N/m$^3$].
    \item $z$: Elevation height, measured vertically from a reference point [m].
    \item $A$: Cross-sectional area of flow [m$^2$].
    \item $s$: A coordinate that follows the path of the fluid streamline.
\end{itemize}
\end{keybox}

\begin{formulabox}[title=Euler's Equation: The Equation of Motion]
This is the differential equation resulting directly from the force balance on a fluid element, before integration. It relates the change in velocity, elevation, and pressure along a streamline.
$$ \rho\nu \frac{d\nu}{ds} + \rho g \frac{dz}{ds} + \frac{dP}{ds} = 0 $$
\end{formulabox}

\begin{formulabox}[title=The Bernoulli Equation: The Integrated Form]
After integrating Euler's Equation and assuming the fluid is incompressible, we get the classic Bernoulli equation. It states that the sum of the pressure head, velocity head, and elevation head is constant along a single streamline for an ideal fluid.
$$ P + \frac{1}{2}\rho\nu^2 + \rho gz = \text{constant} $$
\end{formulabox}

\subsection*{Derivation of the Bernoulli Equation}
This derivation shows how applying a simple force balance to a small fluid element leads directly to the Bernoulli equation.

\begin{stepbox}[title=Derivation from Newton's Second Law]
\begin{enumerate}[label=\textbf{Step \arabic*:}, wide=0pt, leftmargin=*, itemsep=2pt]
    \item \textbf{Define a Fluid Element and its Acceleration:}
    Imagine a tiny, cylindrical "packet" of fluid moving along a path called a streamline. According to Newton's second law, the sum of the forces on this element ($\sum \delta F_s$) must equal its mass ($\delta m$) times its acceleration ($a_s$) in the direction of the streamline.
    $$ \sum \delta F_s = (\delta m) a_s $$
    The acceleration $a_s = \frac{d\nu}{dt}$ can be rewritten using the chain rule as $a_s = \frac{d\nu}{ds} \frac{ds}{dt}$. Since velocity is the rate of change of position ($\nu = \frac{ds}{dt}$), the acceleration becomes $a_s = \nu \frac{d\nu}{ds}$.

    \item \textbf{Analyze the Forces (with a Key Assumption):}
    \begin{conceptbox}[title=Crucial Assumption: Inviscid Flow]
    To simplify the force analysis, we make our first major assumption: the fluid is \textbf{inviscid}, meaning it has zero viscosity. This implies that there are no frictional (shear) forces between fluid layers or between the fluid and the pipe walls. The only forces we consider are pressure and gravity.
    \end{conceptbox}
    \begin{itemize}[itemsep=2pt]
        \item \textbf{Gravity Force:} The weight of the element is its specific weight times its volume, $\delta W = \gamma \cdot \delta V$. The component of this force acting along the streamline (opposing the motion) is $-\delta W \sin\theta = -(\gamma \sin\theta) \delta V$.
        \item \textbf{Pressure Force:} If the pressure changes along the streamline, there will be a net force. The net pressure force is caused by the pressure gradient and is given by $\delta F_{ps} = -\frac{dP}{ds} \delta V$.
    \end{itemize}
    
\end{enumerate}
\end{stepbox}

\begin{stepbox}
\begin{enumerate}[label=\textbf{Step \arabic*:}, wide=0pt, leftmargin=*, itemsep=2pt, start = 3]

    \item \textbf{Form the Equation of Motion:}
    Now we plug the forces and the acceleration term back into Newton's law:
    $$ \left(-\gamma \sin\theta - \frac{dP}{ds}\right)\delta V = (\rho \delta V) \left(\nu \frac{d\nu}{ds}\right) $$
    The differential volume element $\delta V$ cancels from all terms, giving Euler's equation of motion. We can also substitute $\gamma = \rho g$ and, from trigonometry, recognize that $\sin\theta = \frac{dz}{ds}$ (the change in height over the change in path length).
    $$ -\rho g \frac{dz}{ds} - \frac{dP}{ds} = \rho\nu \frac{d\nu}{ds} $$

    \item \textbf{Integrate Along the Streamline:}
    To get from this differential equation to an algebraic one, we make another key assumption: the flow is \textbf{steady} (it doesn't change with time). We can then multiply the entire equation by $ds$ and integrate each term along the streamline:
    $$ \int dP + \int \rho\nu d\nu + \int \rho g dz = \text{constant} $$

    \item \textbf{Apply the Incompressible Assumption:}
    To solve the integrals, we make our final key assumption: the fluid is \textbf{incompressible}, meaning its density $\rho$ is constant. The terms can now be integrated easily:
    $$ P + \frac{1}{2}\rho\nu^2 + \rho g z = \text{constant} $$
    This is the Bernoulli equation, which holds for any two points along the same streamline in a steady, inviscid, incompressible flow.
    
\end{enumerate}
\end{stepbox}

\newpage
\subsection*{Applications of the Bernoulli Equation}

\begin{examplebox}{Flow Through a Horizontal Nozzle}
\textbf{Question:} Air flows steadily through a horizontal pipe of 10 cm diameter. It exits into the atmosphere through an 8 cm diameter nozzle. The gauge pressure in the pipe is 800 Pa. Assuming air is incompressible with a density of 1.23 kg/m$^3$, what is the velocity of the air at the exit?
\end{examplebox}

\begin{stepbox}
\begin{enumerate}[label=\textbf{Step \arabic*:}, wide=0pt, leftmargin=*, itemsep=2pt]
    \item \textbf{Strategy: Using Two Equations for Two Unknowns}
    \begin{conceptbox}[title=Problem Solving Framework]
    This problem involves two unknown velocities: the velocity inside the pipe ($\nu_1$) and the velocity at the nozzle exit ($\nu_2$). To solve for two unknowns, we need two independent equations. We will use:
    \begin{enumerate}
        \item The \textbf{Bernoulli Equation}, which relates pressure and velocity changes (an energy relationship).
        \item The \textbf{Continuity Equation}, which relates velocities and areas (a mass balance).
    \end{enumerate}
    We apply these equations between point 1 (inside the pipe) and point 2 (at the nozzle exit).
    \end{conceptbox}

    \item \textbf{Apply the Bernoulli Equation:}
    $$ \frac{P_1}{\rho} + \frac{\nu_1^2}{2} + gz_1 = \frac{P_2}{\rho} + \frac{\nu_2^2}{2} + gz_2 $$
    We simplify this equation based on the problem statement:
    \begin{itemize}[itemsep=2pt]
        \item \textbf{Potential Energy:} The pipe is horizontal, so the elevation is constant ($z_1 = z_2$). The $gz$ terms cancel out.
        \item \textbf{Pressure:} We are given gauge pressures (pressure above atmospheric). $P_1 = 800$ Pa. The nozzle exits to the atmosphere, so its gauge pressure is $P_2 = 0$ Pa.
    \end{itemize}
    The simplified equation becomes: $\frac{P_1}{\rho} + \frac{\nu_1^2}{2} = \frac{\nu_2^2}{2}$. Rearranging gives our first main equation:
    $$ P_1 = \frac{1}{2}\rho (\nu_2^2 - \nu_1^2) $$


\end{enumerate}
\end{stepbox}

\begin{stepbox}
\begin{enumerate}[label=\textbf{Step \arabic*:}, wide=0pt, leftmargin=*, itemsep=2pt, start = 3]

    \item \textbf{Apply the Continuity Equation:}
    The continuity equation is a statement of mass conservation. For an incompressible fluid ($\rho = \text{constant}$), it simplifies to the volumetric flow rate being constant: $A_1 \nu_1 = A_2 \nu_2$.
    We can express $\nu_1$ in terms of $\nu_2$:
    $$ \nu_1 = \nu_2 \left( \frac{A_2}{A_1} \right) = \nu_2 \left( \frac{\pi d_2^2 / 4}{\pi d_1^2 / 4} \right) = \nu_2 \left( \frac{d_2}{d_1} \right)^2 $$
    $$ \nu_1 = \nu_2 \left( \frac{8 \, \text{cm}}{10 \, \text{cm}} \right)^2 = \nu_2 (0.8)^2 = 0.64 \nu_2 $$
    This is our second main equation.
    
    \item \textbf{Solve the System of Equations:}
    Now we substitute the expression for $\nu_1$ from Step 3 into the equation from Step 2:
    $$ P_1 = \frac{1}{2}\rho (\nu_2^2 - (0.64\nu_2)^2) = \frac{1}{2}\rho (\nu_2^2 - 0.4096\nu_2^2) = \frac{1}{2}\rho (0.5904 \nu_2^2) $$
    Finally, we plug in the known values for $P_1$ and $\rho$ and solve for our target variable, $\nu_2$:
    $$ 800 \, \text{Pa} = \frac{1}{2} (1.23 \, \text{kg/m}^3) (0.5904 \nu_2^2) $$
    $$ 800 = (0.615 \, \text{kg/m}^3) (0.5904 \nu_2^2) = 0.363 \cdot \nu_2^2 $$
    $$ \nu_2^2 = \frac{800}{0.363} = 2203.8 \implies \nu_2 = \sqrt{2203.8} = 46.94 \, \text{m/s} $$
    The velocity at the exit is approximately \textbf{47 m/s}.

    \item \textbf{Check the Incompressible Assumption:}
    Is it valid to treat air as incompressible? The rule of thumb is that this assumption is reasonable if the fluid's velocity is less than 30\% of the speed of sound (Mach number $< 0.3$). The speed of sound in air is about 343 m/s. Here, $M \approx 47/343 \approx 0.14$, which is well below 0.3. The assumption is justified.

\end{enumerate}
\end{stepbox}

\newpage
\begin{examplebox}{Torricelli's Law: Flow from a Tank}
\textbf{Question:} A 60-cm tall coffee urn is filled to the top. It dispenses coffee through a 0.7-cm diameter nozzle that is 12 cm above the table surface. How long does it take to pour a 200-mL cup of coffee?
\end{examplebox}

\begin{stepbox}
\begin{enumerate}[label=\textbf{Step \arabic*:}, wide=0pt, leftmargin=*, itemsep=2pt]
    \item \textbf{Strategy and Assumptions:}
    We apply the Bernoulli equation between the top surface of the coffee (point 1) at height $z_1 = 0.60$ m and the nozzle exit (point 2) at height $z_2 = 0.12$ m. We assume ideal flow: inviscid, steady, and incompressible.
    
    \item \textbf{Apply the Bernoulli Equation:}
    $$ \frac{P_1}{\rho} + \frac{\nu_1^2}{2} + gz_1 = \frac{P_2}{\rho} + \frac{\nu_2^2}{2} + gz_2 $$
    We simplify each term based on the physical situation. For pressure, both the top surface and nozzle exit are open to atmosphere, so $P_1 = P_2 = 0$ gauge and the pressure term cancels. For velocity, the urn's diameter is much larger than the nozzle's, so the surface level drops slowly and $\nu_1 \approx 0$. The height change is $\Delta z = z_2 - z_1 = 0.12 - 0.60 = -0.48$ m.
    
    \item \textbf{Solve for Exit Velocity ($\nu_2$):}
    The Bernoulli equation simplifies significantly:
    $$ 0 + 0 + gz_1 = 0 + \frac{\nu_2^2}{2} + gz_2 \implies \frac{\nu_2^2}{2} = g(z_1 - z_2) $$
    This result is known as \textbf{Torricelli's Law}. It shows that the exit velocity depends only on the height difference between the surface and the outlet.
    $$ \nu_2 = \sqrt{2g(z_1 - z_2)} = \sqrt{2 \cdot (9.81 \, \text{m/s}^2) \cdot (0.48 \, \text{m})} = 3.069 \, \text{m/s} $$
    
    \item \textbf{Calculate Time to Fill Cup:}
    Find the volumetric flow rate through the nozzle. Converting the exit velocity to cm/s gives $306.9 \, \text{cm/s}$.
    $$ v_{flow} = A_2 \cdot \nu_2 = \left(\frac{\pi (0.7 \, \text{cm})^2}{4}\right) \cdot (306.9 \, \text{cm/s}) = (0.3848 \, \text{cm}^2) \cdot (306.9 \, \text{cm/s}) = 118.1 \, \text{cm}^3/\text{s} $$
    For a 200 mL cup:
    $$ \text{Time} = \frac{200 \, \text{cm}^3}{118.1 \, \text{cm}^3/\text{s}} = 1.69 \, \text{s} $$
    The time required is approximately \textbf{1.7 seconds}.
\end{enumerate}
\end{stepbox}

\newpage
\begin{examplebox}{Measuring Velocity with a Pitot Tube}
\textbf{Question:} A pitot-static tube is placed on the nose of a plane traveling 100 m/s at an elevation of 4000 m. What is the pressure difference measured by the device? Assume the density of air is 0.8194 kg/m$^3$ at this elevation.
\end{examplebox}

\begin{stepbox}
\begin{enumerate}[label=\textbf{Step \arabic*:}, wide=0pt, leftmargin=*, itemsep=2pt]
    \item \textbf{Principle of the Pitot Tube:}
    
    A Pitot tube is a pressure-measurement instrument used to measure fluid flow velocity. It works by creating a \textbf{stagnation point}, a point in a fluid flow where the local velocity has been brought to zero. By converting the fluid's kinetic energy entirely into pressure energy at this single point, we can deduce the original velocity.
    \item \textbf{Strategy and Assumptions:}
    We apply the Bernoulli equation between a point in the undisturbed free stream far from the plane (point 1) and the stagnation point at the very tip of the Pitot tube (point 2). We assume steady, inviscid, incompressible flow relative to the moving plane.
    \item \textbf{Apply the Bernoulli Equation:}
    $$ \frac{P_1}{\rho} + \frac{\nu_1^2}{2} + gz_1 = \frac{P_2}{\rho} + \frac{\nu_2^2}{2} + gz_2 $$
    We simplify the terms for this application. For potential energy, the measurement is made over negligible elevation change, so $z_1 \approx z_2$ and the $gz$ terms cancel. For velocity, the free stream velocity is $\nu_1 = 100$ m/s and the stagnation point has $\nu_2 = 0$ m/s. For pressure, $P_1$ is the static pressure and $P_2$ is the stagnation pressure, with $\Delta P = P_2 - P_1$.
    
    \item \textbf{Solve for the Pressure Difference:}
    The Bernoulli equation simplifies to:
    $$ \frac{P_1}{\rho} + \frac{\nu_1^2}{2} = \frac{P_2}{\rho} $$
    Rearranging gives the dynamic pressure formula:
    \begin{formulabox}[title=Dynamic Pressure]
    $$ \Delta P = P_2 - P_1 = \frac{1}{2}\rho \nu_1^2 $$
    \end{formulabox}
    Substituting the given values:
    $$ \Delta P = \frac{1}{2} (0.8194 \, \text{kg/m}^3) (100 \, \text{m/s})^2 = \frac{1}{2} (0.8194) (10000) = 4097 \, \text{Pa} $$
    The pressure difference measured by the device is approximately \textbf{4.1 kPa}.
\end{enumerate}
\end{stepbox}

\newpage
\section*{Viscosity and Shear Stress}
Viscosity is arguably the most important property of a fluid when it comes to analyzing its motion. It describes the fluid's inherent "thickness" or resistance to flowing. Understanding the relationship between viscosity and the forces that cause flow (shear stress) is fundamental to nearly all problems in fluid dynamics, from pumping oil through a pipe to designing aerodynamic vehicles.

\subsection*{Fundamental Equations and Definitions}

\begin{formulabox}[title=Newton's Law of Viscosity]
For many common fluids, the relationship between the internal "smearing" force (shear stress) and the rate of fluid deformation is linear. This relationship is known as Newton's Law of Viscosity:
$$ \tau = \mu \frac{du}{dy} $$
\end{formulabox}

\begin{keybox}[title=Variable Definitions]
\begin{itemize}[itemsep=2pt]
    \item $\tau$: \textbf{Shear Stress} - This is the force per unit area that acts \textit{parallel} (tangential) to a surface. Imagine spreading honey on toast; the force you apply with the knife parallel to the bread creates a shear stress in the honey. Its units are Pascals (Pa) or Newtons per square meter (N/m$^2$).
    \item $\mu$: \textbf{Dynamic Viscosity} - This is a fluid property that measures its intrinsic resistance to flow. Honey has a high viscosity; water has a low viscosity. Its units are Pascal-seconds (Pa$\cdot$s).
    \item $\frac{du}{dy}$: \textbf{Velocity Gradient} (or Rate of Shearing Strain) - This term describes how quickly the fluid velocity changes as you move away from a surface. A steep gradient means the velocity changes rapidly over a short distance. Its units are inverse seconds (s$^{-1}$).
\end{itemize}
\end{keybox}

\subsection*{Conceptual Framework}

\begin{conceptbox}[title=What is Shear Stress?]
A fluid, by definition, is a substance that deforms continuously when a shear stress is applied, no matter how small. A shear stress is created when a force acts tangentially on a surface. The classic way to visualize this is to consider a fluid contained between two parallel plates. If we pull the top plate sideways, we are applying a shear stress to the fluid.
\end{conceptbox}


\begin{conceptbox}[title=The No-Slip Condition and the Velocity Gradient]
The scenario in Figure 1 illustrates two of the most important concepts in fluid mechanics:
\begin{enumerate}[itemsep=2pt]
    \item \textbf{The No-Slip Condition:} This is an empirical observation that a fluid in direct contact with a solid surface will "stick" to it and have the exact same velocity as that surface. In Figure 1, the fluid touching the bottom plate ($y=0$) has zero velocity, and the fluid touching the top plate ($y=H$) moves with the plate's velocity, $U$.
    \item \textbf{The Velocity Gradient:} Because of the no-slip condition, a velocity profile must develop within the fluid, transitioning from zero at the bottom to $U$ at the top. The rate at which the velocity changes from one fluid layer to the next, $\frac{du}{dy}$, is the velocity gradient. It is this gradient that, when multiplied by the fluid's viscosity, gives the shear stress.
\end{enumerate}
\end{conceptbox}

\subsection*{Newtonian and Non-Newtonian Fluids}
The relationship between shear stress ($\tau$) and the rate of strain ($\frac{du}{dy}$) defines the type of fluid.

\begin{conceptbox}[title=Newtonian Fluids]
A fluid is \textbf{Newtonian} if the shear stress is linearly proportional to the rate of shearing strain. For these fluids, viscosity ($\mu$) is a constant property that only depends on temperature and pressure, not on how fast the fluid is being sheared. This simple relationship, $\tau = \mu \frac{du}{dy}$, holds true for many common fluids like water, oil, gasoline, and air.
\end{conceptbox}

\begin{conceptbox}[title=Non-Newtonian Fluids]
Many important fluids do not follow this simple linear relationship. Their "apparent viscosity" changes depending on the applied shear rate.
\begin{itemize}[itemsep=2pt]
    \item \textbf{Shear Thinning (Pseudoplastic):} Apparent viscosity \textit{decreases} as the shear rate increases. They get "thinner" the faster you stir them. This is the most common non-Newtonian behavior. Examples: paint, blood, latex.
    \item \textbf{Shear Thickening (Dilatant):} Apparent viscosity \textit{increases} with increasing shear rate. They get "thicker" the faster you stir them. Example: a cornstarch and water mixture (oobleck).
    \item \textbf{Bingham Plastics:} These are materials that require a minimum stress, called a \textbf{yield stress} ($\tau_0$), to be applied before they begin to flow at all. Below this stress, they behave like a solid. Example: ketchup, toothpaste, mayonnaise.
\end{itemize}
\end{conceptbox}

\newpage
\subsection*{Example: Film Flow Down an Inclined Surface}

\begin{examplebox}{Shear Stress in Film Flow}
\textbf{Question:} Crude oil with a specific gravity of 0.85 flows steadily in a thin film down a wide surface inclined 30$^\circ$ below the horizontal. The film has a thickness of 0.125 inches. The velocity profile is given by the equation:
$$ u(y) = \frac{\rho g \sin\theta}{\mu} \left(Hy - \frac{y^2}{2}\right) $$
where $y$ is the distance perpendicular to the surface. Determine the magnitude of the shear stress that the fluid exerts on the inclined surface.
\end{examplebox}

\begin{stepbox}
\begin{enumerate}[label=\textbf{Step \arabic*:}, wide=0pt, leftmargin=*, itemsep=2pt]
    \item \textbf{Strategy: Apply Newton's Law of Viscosity}
    Our goal is to find the shear stress, $\tau$, at the solid surface ($y=0$). The governing equation is $\tau = \mu \frac{du}{dy}$. The plan is to:
    \begin{enumerate}
        \item Differentiate the given velocity profile $u(y)$ to find the velocity gradient, $\frac{du}{dy}$.
        \item Substitute this gradient into Newton's law of viscosity to get an expression for shear stress, $\tau(y)$.
        \item Evaluate this expression at the surface ($y=0$) to find the specific shear stress we need.
    \end{enumerate}

    \item \textbf{Differentiate the Velocity Profile:}
    We are given $u(y)$. The terms $\rho, g, \sin\theta, \mu,$ and $H$ are all constants with respect to the variable $y$.
    $$ \frac{du}{dy} = \frac{d}{dy} \left[ \frac{\rho g \sin\theta}{\mu} \left(Hy - \frac{y^2}{2}\right) \right] = \frac{\rho g \sin\theta}{\mu} \cdot \frac{d}{dy}\left(Hy - \frac{y^2}{2}\right) $$
    $$ \frac{du}{dy} = \frac{\rho g \sin\theta}{\mu} (H - y) $$

   
\end{enumerate}
\end{stepbox}

\begin{stepbox}
\begin{enumerate}[label=\textbf{Step \arabic*:}, wide=0pt, leftmargin=*, itemsep=2pt, start = 3]

     \item \textbf{Calculate the Shear Stress Expression, $\tau(y)$:}
    Now, substitute the expression for $\frac{du}{dy}$ into the shear stress equation, $\tau(y) = \mu \frac{du}{dy}$:
    $$ \tau(y) = \mu \left[ \frac{\rho g \sin\theta}{\mu} (H - y) \right] $$
    A very interesting thing happens: the viscosity term $\mu$ cancels out. This means for this specific gravity-driven film flow, the shear stress within the fluid does not depend on the fluid's viscosity, but rather on the weight of the fluid above it.
    $$ \tau(y) = \rho g (H - y) \sin\theta $$

    \item \textbf{Evaluate Shear Stress at the Surface ($y=0$):}
    The question asks for the shear stress acting \textit{on the surface}. This corresponds to the position $y=0$.
    $$ \tau_{\text{surface}} = \tau(0) = \rho g (H-0) \sin\theta = \rho g H \sin\theta $$

    \item \textbf{Perform the Numerical Calculation:}
    We must use a consistent set of units. The source problem uses English Engineering units, so we will use that system.
    \begin{itemize}[itemsep=2pt]
        \item \textbf{Density ($\rho$):} $\rho = (\text{Specific Gravity}) \times \rho_{\text{water}}$. In this system, $\rho_{\text{water}} = 1.94 \, \text{slugs/ft}^3$.
            $$ \rho = 0.85 \cdot (1.94 \, \text{slugs/ft}^3) = 1.649 \, \text{slugs/ft}^3 $$
        \item \textbf{Gravity ($g$):} $g = 32.2 \, \text{ft/s}^2$.
        \item \textbf{Film Thickness ($H$):} $H = 0.125 \, \text{in} \times \frac{1 \, \text{ft}}{12 \, \text{in}} = 0.01042 \, \text{ft}$.
        \item \textbf{Angle ($\theta$):} $\sin(30^\circ) = 0.5$.
    \end{itemize}
    Now, substitute these values. The unit `slug` is defined such that 1 lb$_\text{f}$ = 1 slug $\cdot$ ft/s$^2$.
    $$ \tau_{\text{surface}} = (1.649 \, \frac{\text{slug}}{\text{ft}^3}) \cdot (32.2 \, \frac{\text{ft}}{\text{s}^2}) \cdot (0.01042 \, \text{ft}) \cdot (0.5) $$
    $$ \tau_{\text{surface}} = 0.2766 \, \frac{\text{slug} \cdot \text{ft}}{\text{ft}^2 \cdot \text{s}^2} = 0.2766 \, \frac{\text{lb}_\text{f}}{\text{ft}^2} \approx \textbf{0.277 lb\textsubscript{f}/ft\textsuperscript{2}} $$

\end{enumerate}
\end{stepbox}

\newpage
\subsection*{Example: Determining Viscosity Experimentally}

\begin{examplebox}{Viscosity Measurement in an Annulus}
\textbf{Question:} A toothpick is placed concentrically inside a 5 mm diameter straw full of a mysterious fluid. There is a 1 mm gap between the toothpick and the straw on all sides. A length of 50 mm of the toothpick is submerged in the fluid. When a constant force of 0.1 N is used to pull the toothpick out, it moves at a constant velocity of 0.1 m/s. What is the viscosity of the fluid? Assume the fluid is Newtonian and the velocity profile in the narrow gap is linear.
\end{examplebox}

\begin{stepbox}
\begin{enumerate}[label=\textbf{Step \arabic*:}, wide=0pt, leftmargin=*, itemsep=2pt]
    \item \textbf{Strategy: Rearrange Newton's Law of Viscosity}
    Our governing equation is $\tau = \mu \frac{du}{dy}$. We want to find the viscosity, $\mu$. We can rearrange the equation to solve for it:
    $$ \mu = \frac{\tau}{du/dy} $$
    Our plan is to calculate the shear stress ($\tau$) and the velocity gradient ($\frac{du}{dy}$) from the information given in the problem.

    \item \textbf{Calculate the Shear Stress ($\tau$):}
    Shear stress is the tangential force applied divided by the surface area over which it acts.
    $$ \tau = \frac{F}{A} $$
    \begin{itemize}[itemsep=2pt]
        \item \textbf{Force ($F$):} The problem states a constant pulling force of $F = 0.1$ N.
        \item \textbf{Area ($A$):} This is the cylindrical surface area of the toothpick that is in contact with the fluid.
            \begin{itemize}
                \item The straw has a 5 mm diameter. The annular gap is 1 mm on each side.
                \item Toothpick diameter $d_{\text{toothpick}} = d_{\text{straw}} - 2 \cdot (\text{gap}) = 5 \, \text{mm} - 2(1 \, \text{mm}) = 3 \, \text{mm} = 0.003 \, \text{m}$.
                \item Length in fluid $L = 50$ mm = 0.050 m.
            \end{itemize}
            $$ A = (\text{circumference}) \times (\text{length}) = \pi d_{\text{toothpick}} L $$  
            $$ = \pi \cdot (0.003 \, \text{m}) \cdot (0.050 \, \text{m}) = 0.0004712 \, \text{m}^2 $$

    \end{itemize}
    Now we can calculate the shear stress:
    $$ \tau = \frac{0.1 \, \text{N}}{0.0004712 \, \text{m}^2} = 212.2 \, \text{N/m}^2 = \textbf{212.2 Pa} $$

\end{enumerate}
\end{stepbox}

\begin{stepbox}
\begin{enumerate}[label=\textbf{Step \arabic*:}, wide=0pt, leftmargin=*, itemsep=2pt, start = 3]
    
    \item \textbf{Calculate the Velocity Gradient ($\frac{du}{dy}$):}
    \begin{conceptbox}[title=The Linear Velocity Profile Assumption]
    For fluid flow in a very narrow gap (an annulus), it is often a very good approximation to assume the velocity changes linearly across the gap. This means we can approximate the differential term $\frac{du}{dy}$ with the difference term $\frac{\Delta u}{\Delta y}$.
    \end{conceptbox}
    \begin{itemize}[itemsep=2pt]
        \item The change in velocity, $\Delta u$, is the difference between the toothpick's velocity (0.1 m/s) and the stationary straw's velocity (0 m/s). So, $\Delta u = 0.1 \, \text{m/s}$.
        \item The distance over which this change occurs, $\Delta y$, is the size of the gap: $1 \, \text{mm} = 0.001 \, \text{m}$.
    \end{itemize}
    $$ \frac{du}{dy} \approx \frac{\Delta u}{\Delta y} = \frac{0.1 \, \text{m/s}}{0.001 \, \text{m}} = \textbf{100 s\textsuperscript{-1}} $$

    \item \textbf{Solve for Viscosity ($\mu$):}
    Now we can substitute our calculated values for $\tau$ and $\frac{du}{dy}$ into our rearranged equation from Step 1.
    \begin{formulabox}[title=Calculating Viscosity]
    $$ \mu = \frac{\tau}{du/dy} = \frac{212.2 \, \text{Pa}}{100 \, \text{s}^{-1}} = 2.122 \, \text{Pa}\cdot\text{s} $$
    \end{formulabox}

    \item \textbf{Interpret the Result:}
    The standard SI unit for viscosity is the Pascal-second (Pa$\cdot$s). A more common unit in practice is the centipoise (cP), where 1 Pa$\cdot$s = 1000 cP.
    $$ \mu = 2.122 \, \text{Pa}\cdot\text{s} \times \frac{1000 \, \text{cP}}{1 \, \text{Pa}\cdot\text{s}} = 2122 \, \text{cP} $$
    For context, the viscosity of water at room temperature is about 1 cP, olive oil is about 80 cP, and honey is around 2,000-10,000 cP. The mysterious fluid has a viscosity similar to that of honey or a thick syrup.
\end{enumerate}
\end{stepbox}

\newpage
nd{document}
