\documentclass[12pt]{article}
\usepackage[paperwidth=8.5in, paperheight=11in, margin=1.0in, headheight=15pt]{geometry}
\usepackage{amsmath,amssymb,amsthm}
\usepackage[most]{tcolorbox}
\usepackage{enumitem}
\usepackage{xcolor}
\usepackage{hyperref}
\usepackage{fancyhdr}
\usepackage{titlesec}
\usepackage{graphicx}
% Define custom colors for chemical engineering theme
\definecolor{conceptcolor}{RGB}{52, 73, 94}      % Dark blue-gray
\definecolor{formulacolor}{RGB}{231, 76, 60}     % Red for formulas
\definecolor{examplecolor}{RGB}{39, 174, 96}     % Green for examples
\definecolor{stepcolor}{RGB}{142, 68, 173}       % Purple for solution steps
\definecolor{keycolor}{RGB}{243, 156, 18}        % Orange for key points
% Configure fancy headers
\pagestyle{fancy}
\fancyhf{}
\fancyhead[L]{PE Study Guide}
\fancyhead[R]{Process Fundamentals}
\fancyfoot[C]{\thepage}
\renewcommand{\baselinestretch}{1.1}
\setlength{\parindent}{0.25in}
\setlength{\parskip}{3pt}
% Configure section formatting
\titleformat{\section}
  {\normalfont\LARGE\bfseries\color{conceptcolor}}
  {\thesection}{1em}{}
\titleformat{\subsection}
  {\normalfont\Large\bfseries\color{conceptcolor}}
  {\thesubsection}{1em}{}
% Define custom environments
\newtcolorbox{conceptbox}[1][]{
  enhanced,
  colback=conceptcolor!10,
  colframe=conceptcolor,
  arc=3mm,
  title=Key Concept,
  fonttitle=\bfseries\sffamily\normalsize,
  fontupper=\small,
  #1
}
\newtcolorbox{formulabox}[1][]{
  enhanced,
  colback=formulacolor!10,
  colframe=formulacolor,
  arc=2mm,
  title=Important Formula,
  fonttitle=\bfseries\sffamily\normalsize,
  fontupper=\small,
  #1
}
\newtcolorbox{examplebox}[2][]{
  enhanced,
  colback=examplecolor!10,
  colframe=examplecolor,
  arc=3mm,
  title=Example Problem: #2,
  fonttitle=\bfseries\sffamily\normalsize,
  fontupper=\small,
  #1
}
\newtcolorbox{stepbox}[1][]{
  enhanced,
  colback=stepcolor!10,
  colframe=stepcolor,
  arc=2mm,
  title=Solution Steps,
  fonttitle=\bfseries\sffamily\normalsize,
  fontupper=\small,
  #1
}
\newtcolorbox{keybox}[1][]{
  enhanced,
  colback=keycolor!10,
  colframe=keycolor,
  arc=2mm,
  title=Key Variables \& Definitions,
  fonttitle=\bfseries\sffamily\normalsize,
  fontupper=\small,
  #1
}

\begin{document}

\begin{center}
    \Huge\textbf{\color{conceptcolor}Chemical Equilibria}
\end{center}
\hrule

\vspace{0.5cm}

This study guide covers the principles of chemical equilibrium for gas-phase reactions. These calculations determine the final state of a system, representing the maximum possible conversion. It is important to note that equilibrium does not predict the reaction kinetics, which is the speed at which the reaction reaches equilibrium.

\section*{Gas-Phase Chemical Equilibrium}
This section introduces the key thermodynamic equations used to describe and calculate chemical equilibrium for gas-phase reactions.

\subsection*{Gibbs Free Energy of Reaction}
The spontaneity of a reaction and its equilibrium position are fundamentally linked to the change in Gibbs Free Energy.

\begin{conceptbox}
The Standard Gibbs Free Energy of Reaction ($\Delta G^\circ_T$) is the change in Gibbs energy for a process in which reactants in their standard states are converted to products in their standard states. For gases, the standard state is the ideal gas state at a pressure of 1 bar. It is calculated by summing the standard Gibbs free energies of formation for all species, weighted by their stoichiometric coefficients.
\end{conceptbox}

\begin{keybox}
\begin{itemize}[itemsep=0pt]
    \item \textbf{$\Delta G^\circ_T$}: Standard Gibbs free energy change of reaction at absolute temperature $T$ [\text{J/mol} or \text{kJ/mol}].
    \item \textbf{$\nu_i$}: Stoichiometric coefficient for component $i$. It is positive for products, negative for reactants, and dimensionless.
    \item \textbf{$\Delta G^\circ_{f,i}$}: Standard Gibbs free energy of formation for component $i$ at temperature $T$.
\end{itemize}
\end{keybox}

\begin{formulabox}
The standard Gibbs free energy change of reaction is calculated as:
$$ \Delta G^\circ_T = \sum_i \nu_i \Delta G^\circ_{f,i} \quad \text{(Equation 1)} $$
\end{formulabox}

\subsection*{The Thermodynamic Equilibrium Constant ($K_a$)}
The standard Gibbs free energy change is directly related to the thermodynamic equilibrium constant, $K_a$.

\begin{conceptbox}
The thermodynamic equilibrium constant ($K_a$) is a dimensionless quantity that relates the activities of products and reactants at equilibrium. A value of $K_a > 1$ indicates that the formation of products is favored at equilibrium. A value of $K_a < 1$ indicates that reactants are favored.
\end{conceptbox}

\begin{keybox}
\begin{itemize}[itemsep=0pt]
    \item \textbf{$K_a$}: The dimensionless thermodynamic equilibrium constant.
    \item \textbf{$R$}: The ideal gas constant, $8.314 \, \text{J/(mol}\cdot\text{K)}$.
    \item \textbf{$T$}: The absolute temperature of the system in Kelvin [\text{K}].
\end{itemize}
\end{keybox}

\begin{formulabox}
The relationship between the equilibrium constant and Gibbs free energy is:
$$ K_a = \exp\left(-\frac{\Delta G^\circ_T}{RT}\right) \quad \text{(Equation 2)} $$
\end{formulabox}

\newpage

\subsection*{Equilibrium for Ideal Gas Mixtures}
For reactions involving ideal gases, the equilibrium constant can be expressed in terms of mole fractions and total pressure.

\begin{conceptbox}
For an ideal gas mixture, the activity of a component is its partial pressure expressed in bar. The equilibrium constant $K_a$ can therefore be written as a function of the mole fractions of the components and the total system pressure. The term $\sum \nu_i$ determines how the total pressure affects the equilibrium position.
\end{conceptbox}

\begin{keybox}
\begin{itemize}[itemsep=0pt]
    \item \textbf{$y_i$}: Mole fraction of component $i$ in the gas mixture at equilibrium.
    \item \textbf{$P$}: Total pressure of the system [\text{bar}].
    \item \textbf{$P^\circ$}: Standard state pressure, defined as $1$ bar.
    \item \textbf{$\nu_i$}: Stoichiometric coefficient for component $i$.
\end{itemize}
\end{keybox}

\begin{formulabox}
The general form for the equilibrium constant in terms of activities is $K_a = \prod_i \hat{a}_i^{\nu_i}$. For ideal gases, the activity $\hat{a}_i = y_i P / P^\circ$. This leads to:
$$ K_a = \prod_i (y_i)^{\nu_i} \left(\frac{P}{P^\circ}\right)^{\sum_i \nu_i} \quad \text{(Equation 3)} $$
Since $P^\circ = 1$ bar, this is often simplified to $K_a = \left( \prod_i y_i^{\nu_i} \right) P^{\sum_i \nu_i}$.
\end{formulabox}

\newpage

\subsection*{Effect of Temperature (Van't Hoff Equation)}
The Van't Hoff equation describes how the equilibrium constant changes with temperature.

\begin{conceptbox}
The \textbf{Van't Hoff equation} shows that for an \textbf{endothermic reaction} ($\Delta H^\circ_{rxn} > 0$), the equilibrium constant increases with increasing temperature. For an \textbf{exothermic reaction }($\Delta H^\circ_{rxn} < 0$), the equilibrium constant decreases with increasing temperature. This is a quantitative expression of Le Châtelier's Principle.
\end{conceptbox}

\begin{keybox}
\begin{itemize}[itemsep=0pt]
    \item \textbf{$\Delta H^\circ_{rxn}$}: The standard enthalpy change (heat of reaction) at temperature $T$ [\text{J/mol}].
    \item \textbf{$K_1, K_2$}: Equilibrium constants at absolute temperatures $T_1$ and $T_2$.
\end{itemize}
\end{keybox}

\begin{formulabox}
The differential and integrated forms of the Van't Hoff equation are given below. The integrated form assumes that the standard heat of reaction, $\Delta H^\circ_{rxn}$, is constant over the temperature range from $T_1$ to $T_2$.
$$ \frac{d(\ln K_a)}{dT} = \frac{\Delta H^\circ_{rxn}}{RT^2} \quad \text{(Equation 4a - Differential Form)} $$
$$ \ln\left(\frac{K_2}{K_1}\right) = -\frac{\Delta H^\circ_{rxn}}{R} \left(\frac{1}{T_2} - \frac{1}{T_1}\right) \quad \text{(Equation 4b - Integrated Form)} $$
\end{formulabox}

\newpage
\subsection*{Example Problems}

\begin{examplebox}{Methanol Reforming for Hydrogen Production}
Hydrogen for a fuel cell can be formed by the reaction:
$$ \text{CH}_3\text{OH(g)} + \text{H}_2\text{O(g)} \rightleftharpoons \text{CO}_2\text{(g)} + 3\text{H}_2\text{(g)} $$
The reactor feed is at 2.0 bar and the water/methanol ratio is 1.5. What is the equilibrium conversion of methanol at $80^\circ\text{C}$ (353.15 K) and 2.0 bar? Assume ideal gas behavior.
\end{examplebox}

\begin{stepbox}
\begin{enumerate}[label=\textbf{Step \arabic*:}, wide=0pt, leftmargin=*, itemsep=2pt]
\item \textbf{Gather Data and Calculate $\Delta G^\circ_{298}$ and $\Delta H^\circ_{298}$}
    
    First, list the standard formation data at $25^\circ\text{C}$ (298.15 K).
    \begin{center}
    \begin{tabular}{|l|c|c|}
    \hline
    \textbf{Component} & \textbf{$\Delta H^\circ_f$ (kJ/mol)} & \textbf{$\Delta G^\circ_f$ (kJ/mol)} \\
    \hline
    $\text{CH}_3\text{OH(g)}$ & -201.2 & -162.3 \\
    $\text{H}_2\text{O(g)}$ & -241.8 & -228.6 \\
    $\text{CO}_2\text{(g)}$ & -393.5 & -394.4 \\
    $\text{H}_2\text{(g)}$ & 0 & 0 \\
    \hline
    \end{tabular}
    \end{center}
    The stoichiometric coefficients ($\nu_i$) are: $\nu_{\text{CO}_2} = 1$, $\nu_{\text{H}_2} = 3$, $\nu_{\text{CH}_3\text{OH}} = -1$, $\nu_{\text{H}_2\text{O}} = -1$.
    $$ \Delta G^\circ_{298} = [1(-394.4) + 3(0)] - [1(-162.3) + 1(-228.6)] = -3.5 \, \text{kJ/mol} $$
    $$ \Delta H^\circ_{298} = [1(-393.5) + 3(0)] - [1(-201.2) + 1(-241.8)] = +49.5 \, \text{kJ/mol} $$
    The reaction is endothermic since $\Delta H^\circ > 0$.

\item \textbf{Calculate Equilibrium Constant $K_{298}$ at $25^\circ\text{C}$}
    
    Using Equation 2 with $T_1 = 298.15$ K and $R = 8.314$ J/(mol$\cdot$K).
    $$ K_{298} = \exp\left(-\frac{\Delta G^\circ_{298}}{RT_1}\right) = \exp\left(-\frac{-3500 \, \text{J/mol}}{8.314 \, \frac{\text{J}}{\text{mol}\cdot\text{K}} \times 298.15 \, \text{K}}\right) = \exp(1.412) \approx 4.10 $$

\end{enumerate}
\end{stepbox}

\newpage

\begin{stepbox}
\begin{enumerate}[label=\textbf{Step \arabic*:}, wide=0pt, leftmargin=*, itemsep=2pt, start=3]


\item \textbf{Calculate Equilibrium Constant $K_{353}$ at $80^\circ\text{C}$}

    Using the integrated Van't Hoff equation (4b) with $T_1 = 298.15$ K and $T_2 = 353.15$ K.
    $$ \ln\left(\frac{K_{353}}{K_{298}}\right) = -\frac{\Delta H^\circ_{rxn}}{R} \left(\frac{1}{T_2} - \frac{1}{T_1}\right) $$
    $$ \ln\left(\frac{K_{353}}{4.10}\right) = -\frac{49500 \, \text{J/mol}}{8.314 \, \frac{\text{J}}{\text{mol}\cdot\text{K}}} \left(\frac{1}{353.15 \, \text{K}} - \frac{1}{298.15 \, \text{K}}\right) $$
    $$ \ln\left(\frac{K_{353}}{4.10}\right) = -5954.8 \times (0.0028316 - 0.0033540) = -5954.8 \times (-0.0005224) = 3.111 $$
    $$ K_{353} = 4.10 \times e^{3.111} = 4.10 \times 22.44 \approx 92.0 $$
    
\item \textbf{Set up Stoichiometry Table and Solve for Conversion}
    
    Let $\xi$ be the extent of reaction. Basis: 1 mole of $\text{CH}_3\text{OH}$ and 1.5 moles of $\text{H}_2\text{O}$ are fed.
    \begin{center}
    \begin{tabular}{|l|c|c|c|}
    \hline
    \textbf{Species} & \textbf{Initial (mol)} & \textbf{Equilibrium (mol)} & \textbf{Mole Fraction ($y_i$)} \\
    \hline
    $\text{CH}_3\text{OH}$ & 1 & $1 - \xi$ & $(1 - \xi) / n_{total}$ \\
    $\text{H}_2\text{O}$ & 1.5 & $1.5 - \xi$ & $(1.5 - \xi) / n_{total}$ \\
    $\text{CO}_2$ & 0 & $\xi$ & $\xi / n_{total}$ \\
    $\text{H}_2$ & 0 & $3\xi$ & $3\xi / n_{total}$ \\
    \hline
    \textbf{Total} & 2.5 & $n_{total} = 2.5 + 2\xi$ & 1.0 \\
    \hline
    \end{tabular}
    \end{center}
    The change in moles is $\sum \nu_i = (1+3) - (1+1) = 2$. Using Equation 3:
    $$ K_a = \frac{y_{\text{CO}_2} y_{\text{H}_2}^3}{y_{\text{CH}_3\text{OH}} y_{\text{H}_2\text{O}}} P^2 = \frac{(\frac{\xi}{n_{total}})(\frac{3\xi}{n_{total}})^3}{(\frac{1-\xi}{n_{total}})(\frac{1.5-\xi}{n_{total}})} P^2 = \frac{27\xi^4}{(1-\xi)(1.5-\xi)} \frac{P^2}{n_{total}^2} $$
    Substitute $K_a = 92.0$, $P=2.0$ bar, and $n_{total} = 2.5 + 2\xi$:
    $$ 92.0 = \frac{27\xi^4}{(1-\xi)(1.5-\xi)} \frac{(2.0)^2}{(2.5 + 2\xi)^2} $$
    This non-linear equation can be solved numerically. Solving yields $\xi \approx 0.923$ mol.
    
\item \textbf{Final Answer}
    
    The conversion of methanol (the limiting reactant) is defined as moles reacted per mole fed.
    $$ \text{Conversion} = X = \frac{\xi}{\text{initial moles of } \text{CH}_3\text{OH}} = \frac{0.923}{1.0} = 0.923 $$
    The equilibrium conversion is \textbf{92.3\%}.
\end{enumerate}
\end{stepbox}

\newpage
\begin{examplebox}{Dehydrogenation of Butene}
The catalytic reaction $\text{C}_4\text{H}_8\text{(g)} \rightleftharpoons \text{C}_4\text{H}_6\text{(g)} + \text{H}_2\text{(g)}$ takes place at $900^\circ\text{C}$ and 1.0 bar. The feed contains steam and $\text{C}_4\text{H}_8$ in a 10:1 ratio. At this temperature, $K_a = 0.329$.  What is the equilibrium conversion with steam present? Does equilibrium conversion increase or decrease if steam is not fed to the system? Explain why.
\end{examplebox}

\begin{stepbox}
\begin{enumerate}[label=\textbf{Step \arabic*:}, wide=0pt, leftmargin=*, itemsep=2pt]
\item \textbf{Calculate Conversion with Steam}

    The change in moles is $\sum \nu_i = (1+1) - 1 = 1$. The equilibrium expression is:
    $$ K_a = \frac{y_{\text{C}_4\text{H}_6} y_{\text{H}_2}}{y_{\text{C}_4\text{H}_8}} \left(\frac{P}{P^\circ}\right)^1 = \frac{y_{\text{C}_4\text{H}_6} y_{\text{H}_2}}{y_{\text{C}_4\text{H}_8}} $$
    Let $x$ be the extent of reaction. Basis: 1 mole of $\text{C}_4\text{H}_8$ and 10 moles of inert steam.
    \begin{center}
    \begin{tabular}{|l|c|c|}
    \hline
    \textbf{Species} & \textbf{Initial (mol)} & \textbf{Equilibrium (mol)} \\
    \hline
    $\text{C}_4\text{H}_8$ & 1 & $1 - x$ \\
    $\text{C}_4\text{H}_6$ & 0 & $x$ \\
    $\text{H}_2$ & 0 & $x$ \\
    $\text{H}_2\text{O}$ (inert) & 10 & 10 \\
    \hline
    \textbf{Total} & 11 & $n_{total} = 11 + x$ \\
    \hline
    \end{tabular}
    \end{center}
    Substitute mole fractions $y_i = n_i / n_{total}$ into the equilibrium expression:
    $$ 0.329 = \frac{(\frac{x}{11+x})(\frac{x}{11+x})}{(\frac{1-x}{11+x})} = \frac{x^2}{(1-x)(11+x)} $$
    This simplifies to the quadratic equation $1.329x^2 + 3.29x - 3.619 = 0$. Solving for the positive root:
    $$ x = \frac{-3.29 + \sqrt{(3.29)^2 - 4(1.329)(-3.619)}}{2(1.329)} \approx 0.825 $$
    The conversion is $x/1.0 = 0.825$, or \textbf{82.5\%}.

\item \textbf{Calculate Conversion without Steam}

    The stoichiometry table simplifies with $n_{total} = 1+x$.
    $$ K_a = 0.329 = \frac{(\frac{x}{1+x})(\frac{x}{1+x})}{(\frac{1-x}{1+x})} = \frac{x^2}{1-x^2} $$
    Solving for $x$: $0.329(1-x^2) = x^2 \implies 1.329x^2 = 0.329 \implies x = \sqrt{0.329/1.329} \approx 0.498$.
    Without steam, the equilibrium conversion is \textbf{49.8\%}.

\end{enumerate}
\end{stepbox}

\newpage

\section*{Thermal Effects in Chemical Reactions}
This section introduces the fundamental equations that govern thermal effects in reactors, including the temperature dependence of equilibrium and energy balances for adiabatic and isothermal systems.

\subsection*{Temperature Dependence of the Equilibrium Constant}
\begin{conceptbox}
The \textbf{Van't Hoff Equation} quantitatively describes how the equilibrium constant ($K_e$) changes with temperature. For an exothermic reaction ($\Delta H < 0$), $K_e$ decreases as temperature increases. For an endothermic reaction ($\Delta H > 0$), $K_e$ increases as temperature increases. This relationship is crucial for determining the optimal operating temperature for reversible reactions.
\end{conceptbox}

\begin{keybox}
\begin{itemize}[itemsep=0pt]
    \item \textbf{$K_e, K_m$}: Dimensionless equilibrium constants at temperatures $T$ and $T_m$, respectively.
    \item \textbf{$\Delta H$}: The heat of reaction, assumed constant [\text{J/mol} or \text{cal/mol}].
    \item \textbf{$R$}: The ideal gas constant [$8.314 \, \text{J/(mol}\cdot\text{K)}$ or $1.987 \, \text{cal/(mol}\cdot\text{K)}$].
    \item \textbf{$T, T_m$}: Absolute temperatures in Kelvin [\text{K}].
\end{itemize}
\end{keybox}

\begin{formulabox}
The integrated Van't Hoff equation is:
$$ K_e = K_m \exp\left[ \left(\frac{\Delta H}{R}\right) \left(\frac{1}{T_m} - \frac{1}{T}\right) \right] \quad \text{(Equation 1)} $$
\end{formulabox}

\subsection*{Equilibrium Conversion for A $\rightleftharpoons$ B}
\begin{conceptbox}
For a simple reversible reaction, \textbf{the equilibrium conversion ($X_e$)} is the maximum possible conversion that can be achieved at a given temperature and pressure. It is determined directly by the value of the equilibrium constant, $K_e$. At equilibrium, the forward and reverse reaction rates are equal.
\end{conceptbox}

\begin{formulabox}
For the elementary reversible reaction A $\rightleftharpoons$ B, the relationship is:
$$ X_e = \frac{K_e}{1 + K_e} \quad \text{(Equation 2)} $$
This is derived from $K_e = \frac{C_B}{C_A} = \frac{C_{A0}X_e}{C_{A0}(1-X_e)}$.
\end{formulabox}

\subsection*{Energy Balance: Adiabatic vs. Isothermal Reactors}
\begin{conceptbox}
The reactor energy balance relates the conversion achieved to the thermal conditions.
\begin{itemize}[itemsep=0pt]
    \item \textbf{Adiabatic Reactor}: No heat is exchanged with the surroundings ($\dot{Q}=0$). The temperature changes as a function of conversion due to the heat of reaction.
    \item \textbf{Isothermal Reactor}: The temperature is held constant. This requires active heat transfer ($\dot{Q} \neq 0$) to compensate for the heat of reaction.
\end{itemize}
\end{conceptbox}

\begin{keybox}
\begin{itemize}[itemsep=0pt]
    \item \textbf{$X, X_e$}: Fractional conversion and equilibrium fractional conversion.
    \item \textbf{$C_{PA}, C_{PI}$}: Molar heat capacities of reactant A and inert I.
    \item \textbf{$\alpha$}: Molar ratio of inerts to reactant A in the feed ($F_{I0} / F_{A0}$).
    \item \textbf{$T, T_{feed}$}: Reactor temperature and feed temperature.
    \item \textbf{$\dot{Q}$}: Rate of heat transfer [\text{W} or \text{J/s}].
    \item \textbf{$F_{A0}$}: Molar feed rate of reactant A [\text{mol/s}].
\end{itemize}
\end{keybox}

\begin{formulabox}
For a simple reaction A $\rightleftharpoons$ B with constant heat capacities ($C_{PA} = C_{PB}$):
$$ \text{Adiabatic:} \quad X_{energy} = \frac{(C_{PA} + \alpha C_{PI})(T - T_{feed})}{-\Delta H_{rxn}} \quad \text{(Equation 3a)} $$
$$ \text{Isothermal:} \quad \dot{Q} = X F_{A0} (-\Delta H_{rxn}) \quad \text{(Equation 3b)} $$
\end{formulabox}

\newpage
\subsection*{Example Problems}

\begin{examplebox}{Adiabatic Equilibrium Temperature and Conversion}
For the reversible, liquid-phase reaction, A $\leftrightarrow$ B, determine the adiabatic equilibrium temperature and conversion when pure A is fed to the reactor at 300 K.
\begin{itemize}[itemsep=0pt]
    \item $C_{PA} = C_{PB} = 50$ cal/(mol K)
    \item $K_e = 1.0 \times 10^5$ at 298 K
    \item $\Delta H_{rxn} = -2.0 \times 10^4$ cal/mol
\end{itemize}
\end{examplebox}


\begin{stepbox}
\begin{enumerate}[label=\textbf{Step \arabic*:}, wide=0pt, leftmargin=*, itemsep=2pt]
    \item \textbf{Formulate the Equilibrium Conversion ($X_e$) Equation}

    The equilibrium conversion $X_e$ depends on the equilibrium constant $K_e$, which in turn depends on temperature via the Van't Hoff equation (Eq. 1). Combining with Eq. 2:
    $$ X_e(T) = \frac{K_e(T)}{1 + K_e(T)} $$
    where
    $$ K_e(T) = (1.0 \times 10^5) \exp\left[ \frac{-20000 \, \text{cal/mol}}{1.987 \, \text{cal/(mol K)}} \left(\frac{1}{298 \text{ K}} - \frac{1}{T}\right) \right] $$
    This equation defines the S-shaped equilibrium curve. Since the reaction is exothermic ($\Delta H_{rxn} < 0$), $X_e$ decreases as $T$ increases.

    \item \textbf{Formulate the Energy Balance ($X_{energy}$) Equation}
    
    For an adiabatic reactor with pure A feed ($\alpha = 0$), the energy balance (Eq. 3a) gives a linear relationship between conversion and temperature:
    $$ X_{energy} = \frac{C_{PA}(T - T_{feed})}{-\Delta H_{rxn}} = \frac{50 \, \text{cal/(mol K)} \times (T - 300 \text{ K})}{20000 \, \text{cal/mol}} = \frac{T - 300}{400} $$
    This is the straight energy balance line, which starts at $X=0$ for $T=300$ K.
    
    \item \textbf{Solve for the Intersection}
    
    We set $X_e = X_{energy}$ and solve for the common point $(T, X)$. This system of equations is typically solved graphically or with a numerical solver.
    The intersection of the equilibrium curve and the energy balance line yields the unique adiabatic operating point.

    The simultaneous solution of the two equations gives the final state of the reactor:
    \begin{itemize}[itemsep=0pt]
        \item \textbf{Adiabatic Equilibrium Temperature: $T \approx 460$ K}
        \item \textbf{Adiabatic Equilibrium Conversion: $X \approx 0.40$ (or 40\%)}
    \end{itemize}
\end{enumerate}
\end{stepbox}

\newpage

\newpage
\begin{examplebox}{Heat Duty for an Isothermal Reactor}
100 mol/h of $\text{N}_2$ and 300 mol/h of $\text{H}_2$ are fed to an isothermal reactor at $350^\circ\text{C}$ to carry out the ammonia synthesis reaction: 
\[
\text{N}_2(g) + 3\text{H}_2(g) \rightarrow 2\text{NH}_3(g)
\]
How much heat must be removed to maintain the reactor temperature if the conversion of $\text{N}_2$ is 75\%?
\begin{itemize}[itemsep=0pt]
    \item $C_P(\text{N}_2) = C_P(\text{H}_2) = 29\ \text{J/(mol}\cdot\text{K)}$
    \item $C_P(\text{NH}_3) = 36\ \text{J/(mol}\cdot\text{K)}$
    \item $\Delta H^\circ_{f, \text{NH}_3} = -46\ \text{kJ/mol at } 25^\circ\text{C}$
\end{itemize}
\end{examplebox}

\begin{stepbox}
We calculate the total enthalpy change ($\Delta H_{\text{total}}$) from inlet to outlet using a hypothetical path with a reference temperature of $25^\circ\text{C}$ (298.15 K). The heat to be removed is $\dot{Q} = -\Delta H_{\text{total}}$.
\begin{enumerate}[label=\textbf{Step \arabic*:}, wide=0pt, leftmargin=*, itemsep=2pt]
    \item \textbf{Determine Outlet Molar Flow Rates}

    The feed is stoichiometric. For 75\% conversion of $\text{N}_2$ (the limiting reactant):
    \begin{itemize}[itemsep=0pt]
        \item $\text{N}_2$ reacted: $100\ \text{mol/h} \times 0.75 = 75\ \text{mol/h}$
        \item $\text{H}_2$ reacted: $75 \times 3 = 225\ \text{mol/h}$
        \item $\text{NH}_3$ produced: $75 \times 2 = 150\ \text{mol/h}$
    \end{itemize}
    The resulting outlet molar flow rates ($F_{\text{out},i}$) are:
    \begin{itemize}[itemsep=0pt]
        \item $F_{\text{N}_2,\ \text{out}} = 100 - 75 = 25\ \text{mol/h}$
        \item $F_{\text{H}_2,\ \text{out}} = 300 - 225 = 75\ \text{mol/h}$
        \item $F_{\text{NH}_3,\ \text{out}} = 0 + 150 = 150\ \text{mol/h}$
    \end{itemize}

    \item \textbf{Path Segment 1: Cool Reactants to $25^\circ\text{C}$}

    Calculate the enthalpy change ($\Delta H_1$) for cooling the feed stream from $350^\circ\text{C}$ (623.15 K) to $25^\circ\text{C}$ (298.15 K). Here, $\Delta T = -325$ K.
    \[
    \Delta H_1 = \sum F_{\text{in},i} C_{P,i} \Delta T = \left(100 \times 29 + 300 \times 29\right)\ \text{J/(h$\cdot$K)} \times (-325\ \text{K})
    \]
    \[
    \Delta H_1 = 11600 \times (-325) = -3,770,000\ \text{J/h} = -3770\ \text{kJ/h}
    \]
\end{enumerate}
\end{stepbox}

\newpage
\begin{stepbox}
\begin{enumerate}[label=\textbf{Step \arabic*:}, wide=0pt, leftmargin=*, itemsep=2pt, start=3]
    \item \textbf{Path Segment 2: Reaction at $25^\circ\text{C}$}

    Calculate the enthalpy change from the reaction ($\Delta H_2$) at the reference temperature. First, find the standard heat of reaction per mole of $\text{N}_2$.
    \[
    \Delta H_{\text{rxn},298} = \sum \nu_i \Delta H^\circ_{f,i} = \left[2(-46)\right] - \left[1(0) + 3(0)\right] = -92\ \text{kJ/mol of }\text{N}_2
    \]
    The total enthalpy change for reacting 75 mol/h of $\text{N}_2$ is:
    \[
    \Delta H_2 = 75\ \text{mol/h} \times (-92\ \text{kJ/mol}) = -6900\ \text{kJ/h}
    \]

    \item \textbf{Path Segment 3: Heat Products to $350^\circ\text{C}$}

    Calculate the enthalpy change ($\Delta H_3$) for heating the outlet mixture from $25^\circ\text{C}$ to $350^\circ\text{C}$. Here, $\Delta T = +325$ K.
    \[
    \Delta H_3 = \sum F_{\text{out},i} C_{P,i} \Delta T
    \]
    \[
    \Delta H_3 = \left(25 \times 29 + 75 \times 29 + 150 \times 36\right)\ \text{J/(h$\cdot$K)} \times 325\ \text{K}
    \]
    \[
    \Delta H_3 = \left(725 + 2175 + 5400\right) \times 325 = 8300 \times 325 = 2,697,500\ \text{J/h} = 2697.5\ \text{kJ/h}
    \]

    \item \textbf{Calculate Total Heat Duty ($\dot{Q}$)}

    Sum the enthalpy changes along the hypothetical path to find the total enthalpy change:
    \[
    \Delta H_{\text{total}} = \Delta H_1 + \Delta H_2 + \Delta H_3 = -3770 + (-6900) + 2697.5 = -7972.5\ \text{kJ/h}
    \]
    The heat that must be removed from the reactor is $\dot{Q} = -\Delta H_{\text{total}}$:
    \[
    \dot{Q} = -(-7972.5\ \text{kJ/h}) = 7972.5\ \text{kJ/h}
    \]
    Converting to kW (1 kW = 3600 kJ/h):
    \[
    \dot{Q} = 7972.5\ \text{kJ/h} \times \frac{1\ \text{kW}}{3600\ \text{kJ/h}} \approx 2.21\ \text{kW}
    \]

    \item \textbf{Final Answer}

    Approximately \textbf{2.2 kW} of heat must be removed to maintain the reactor at an isothermal temperature of $350^\circ\text{C}$.
\end{enumerate}
\end{stepbox}

\newpage

\section*{Chemical Equilibrium for Multiple Reactions}
When multiple chemical reactions occur simultaneously, the system's final equilibrium composition cannot be determined by a single equilibrium constant. This guide covers the method of Gibbs free energy minimization, a robust technique for solving complex reaction equilibria.

\subsection*{Gibbs Free Energy and Chemical Potential}
\begin{conceptbox}
At constant temperature and pressure, a chemical system will spontaneously evolve towards the state with the minimum possible total Gibbs free energy. The equilibrium composition corresponds to this global minimum. The key quantity for this calculation is the **chemical potential** of each species $i$, which is equivalent to its partial molar Gibbs free energy ($\bar{G}_i$) in the mixture.
\end{conceptbox}

\begin{keybox}
\begin{itemize}[itemsep=0pt]
    \item \textbf{$\bar{G}_i$}: Partial molar Gibbs free energy (chemical potential) of component $i$ in the mixture [\text{J/mol}].
    \item \textbf{$\Delta G^\circ_{f,i}$}: Standard Gibbs free energy of formation of component $i$ at temperature $T$ [\text{J/mol}].
    \item \textbf{$y_i$}: Mole fraction of component $i$ in the mixture.
    \item \textbf{$n_i$}: Number of moles of component $i$.
    \item \textbf{$n$}: Total number of moles in the mixture.
    \item \textbf{$G$}: Gibbs free energy per mole of the mixture.
    \item \textbf{$P$}: Total pressure of the system [\text{bar}].
    \item \textbf{$P^\circ$}: Standard-state pressure, 1 bar.
    \item \textbf{$R$}: Ideal gas constant [$8.314 \, \text{J/(mol}\cdot\text{K)}$].
    \item \textbf{$T$}: Absolute temperature [\text{K}].
\end{itemize}
\end{keybox}

\begin{formulabox}
For an ideal gas, the partial molar Gibbs free energy of component $i$ is:
$$ \bar{G}_i = \Delta G^\circ_{f,i} + RT \ln\left(\frac{P}{P^\circ}\right) + RT \ln(y_i) \quad \text{(Equation 1)} $$
The total Gibbs free energy of the mixture is the sum over all species:
$$ nG = \sum_i n_i \bar{G}_i = \sum_i n_i \left[ \Delta G^\circ_{f,i} + RT \ln\left(\frac{y_i P}{P^\circ}\right) \right] \quad \text{(Equation 2)} $$
This is the objective function to be minimized.
\end{formulabox}

\subsection*{Constraints: Atomic Balances}
\begin{conceptbox}
The minimization of Gibbs free energy is not unconstrained. It is subject to the fundamental law of conservation of mass, applied as **atomic balances**. The total number of atoms of each element (e.g., C, H, O) in the system must remain constant from the initial feed to the final equilibrium composition.
\end{conceptbox}

\begin{formulabox}
For each element $j$ present in the system, a constraint equation must be satisfied:
$$ (\text{Total moles of atoms of element } j)_{\text{in}} = (\text{Total moles of atoms of element } j)_{\text{out}} $$
$$ \sum_i n_{i,\text{in}} a_{ij} = \sum_i n_{i,\text{out}} a_{ij} \quad \text{(Equation 3)} $$
where $a_{ij}$ is the number of atoms of element $j$ in one molecule of species $i$.
\end{formulabox}

\newpage

\subsection*{Temperature-Dependent Thermodynamics}
\begin{conceptbox}
To perform Gibbs minimization at a specific temperature $T$, the standard Gibbs free energies of formation ($\Delta G^\circ_{f,i}$) for every species must be known at that temperature. These values are calculated by integrating heat capacity ($C_P$) data from a known reference state (usually 298.15 K).
\end{conceptbox}

\begin{formulabox}
The heat capacity is often expressed as a polynomial in temperature:
$$ C_{P,i} = A_i + B_i T + C_i T^2 + D_i T^3 $$
For a reaction, the change in heat capacity is $\Delta C_P = \sum_i \nu_i C_{P,i}$. This allows us to find the heat of reaction ($\Delta H^\circ_T$) at any temperature $T$ by integration from a reference temperature $T_R$:
$$ \Delta H^\circ_T = \Delta H^\circ_R + \int_{T_R}^T \Delta C_P \,dT $$
This integrates to a compact form using a constant of integration, $J$:
$$ \Delta H^\circ_T = J + \Delta A T + \frac{\Delta B}{2} T^2 + \frac{\Delta C}{3} T^3 + \frac{\Delta D}{4} T^4 $$
where $J = \Delta H^\circ_R - \Delta A T_R - \frac{\Delta B}{2} T_R^2 - \frac{\Delta C}{3} T_R^3 - \frac{\Delta D}{4} T_R^4$.
The Gibbs free energy of reaction is then found by integrating the Van't Hoff equation:
$$ \frac{\Delta G^\circ_T}{RT} - \frac{\Delta G^\circ_R}{RT_R} = \int_{T_R}^T -\frac{\Delta H^\circ_T}{RT^2} \,dT $$
This also integrates to a compact form using a second constant of integration, $I$:
$$ \frac{\Delta G^\circ_T}{RT} = I - \frac{J}{RT} + \frac{\Delta A}{R}\ln T + \frac{\Delta B}{2R} T + \frac{\Delta C}{6R} T^2 + \frac{\Delta D}{12R} T^3 $$
where $I = \frac{1}{R} [ \frac{\Delta G^\circ_R}{T_R} - \frac{J}{T_R} + \Delta A \ln T_R - \frac{\Delta B}{2} T_R - \frac{\Delta C}{6} T_R^2 - \frac{\Delta D}{12} T_R^3 ]$. These equations are applied to formation reactions to find each $\Delta G^\circ_{f,i}$ at the desired temperature.
\end{formulabox}

\newpage
\subsection*{Example Problems}

\begin{examplebox}{Water-Gas Shift Reaction}
Calculate the equilibrium composition at 750 K and 1 bar for the water-gas shift reaction for a feed of 2 mol/s CO and 2 mol/s H$_{2}$O. Use both the extent of reaction method and Gibbs minimization with temperature-dependent data.
$$ \text{CO(g)} + \text{H}_2\text{O(g)} \rightleftharpoons \text{H}_2\text{(g)} + \text{CO}_2\text{(g)} $$
\end{examplebox}

\begin{stepbox}
This method uses the reaction stoichiometry and a calculated equilibrium constant.
\begin{enumerate}[label=\textbf{Step \arabic*:}, wide=0pt, leftmargin=*, itemsep=2pt]
    \item \textbf{Set up Stoichiometry Table}

    Let $\xi$ be the extent of reaction.
    
    \begin{tabular}{l c c}
    \hline
    \textbf{Species} & \textbf{Initial Moles} & \textbf{Equilibrium Moles} \\
    \hline
    CO & 2 & $2 - \xi$ \\
    H$_{2}$O & 2 & $2 - \xi$ \\
    H$_{2}$ & 0 & $\xi$ \\
    CO$_{2}$ & 0 & $\xi$ \\
    \hline
    \textbf{Total} & 4 & $n_{T} = (2-\xi) + (2-\xi) + \xi + \xi = 4$ \\
    \hline
    \end{tabular}

    \item \textbf{Calculate Equilibrium Constant ($K_{eq}$)}

    First, calculate the standard Gibbs free energy of reaction at 750 K using the temperature-dependent equations. This yields $\Delta G^\circ_{rxn, 750K} = -11,210 \, \text{J/mol}$.
    $$ K_{eq} = \exp\left(-\frac{\Delta G^\circ_{rxn}}{RT}\right) = \exp\left(-\frac{-11210}{8.314 \times 750}\right) = \exp(1.799) \approx 6.04 $$

    \item \textbf{Solve for Extent of Reaction ($\xi$)}

    Since the total moles and pressure terms cancel ($\Delta \nu = 0$), $K_{eq}$ is a ratio of mole numbers.
    $$ K_{eq} = \frac{n_{H_2} n_{CO_2}}{n_{CO} n_{H_2O}} \implies 6.04 = \frac{(\xi)(\xi)}{(2-\xi)(2-\xi)} = \left(\frac{\xi}{2-\xi}\right)^2 $$
    Taking the square root:
    $$ \sqrt{6.04} = 2.458 = \frac{\xi}{2-\xi} \implies 4.916 - 2.458\xi = \xi \implies \xi = \frac{4.916}{3.458} \approx 1.422 \, \text{mol} $$
    
    \item \textbf{Determine Final Composition}

    Moles CO = Moles H$_{2}$O = $2 - 1.422 = \mathbf{0.578}$ mol. Moles H$_{2}$ = Moles CO$_{2}$ = $\xi = \mathbf{1.422}$ mol.
\end{enumerate}

\end{stepbox}

\newpage
\begin{examplebox}{Methane Steam Reforming (Multiple Reactions)}
Calculate the equilibrium composition at 950 K and 3.0 bar for a feed of 1.0 mol/s CH$_{4}$ and 1.0 mol/s H$_{2}$O. The possible products are CO, CO$_{2}$, and H$_{2}$. Use the Gibbs minimization method.
\end{examplebox}
\begin{stepbox}
\begin{enumerate}[label=\textbf{Step \arabic*:}, wide=0pt, leftmargin=*, itemsep=2pt]
    \item \textbf{Define the System}
    
    \textbf{Species:} CH$_4$, H$_2$O, CO, CO$_2$, H$_2$ \\
    \textbf{Conditions:} $T = 950\ \text{K}$, $P = 3.0\ \text{bar}$

    \item \textbf{Set up the Minimization Problem}

    \textbf{Variables:} Final mole numbers to be determined: $n_{CH_4}, n_{H_2O}, n_{CO}, n_{CO_2}, n_{H_2}$.
    \textbf{Objective Function:} Minimize total Gibbs free energy of the mixture:
    \[
    nG = \sum_{i=1}^{5} n_i \left[ \Delta G^\circ_{f,i,\ 950K} + RT \ln\left(y_i \frac{P}{P^\circ}\right) \right]
    \]
    \textbf{Constraints (Atomic Balances):}
    \begin{align*}
        &\text{Carbon:} && n_{CH_4} + n_{CO} + n_{CO_2} = 1.0 \\
        &\text{Hydrogen:} && 4n_{CH_4} + 2n_{H_2O} + 2n_{H_2} = 6.0 \\
        &\text{Oxygen:} && n_{H_2O} + n_{CO} + 2n_{CO_2} = 1.0 \\
        &\text{Non-negativity:} && n_i \ge 0\quad \text{for all species}
        \end{align*}
    \item \textbf{Solve and State Results}
    
    This constrained minimization is solved numerically. The calculation yields the following equilibrium composition:
    
    \begin{tabular}{l c c}
    \hline
    \textbf{Species} & \textbf{Equilibrium Moles} & \textbf{Mole Fraction ($y_i$)} \\
    \hline
    CH$_{4}$ & 0.052 & 0.017 \\
    H$_{2}$O & 0.170 & 0.056 \\
    CO & 0.730 & 0.240 \\
    CO$_{2}$ & 0.218 & 0.072 \\
    H$_{2}$ & 1.839 & 0.605 \\
    \hline
    \textbf{Total Moles} & \textbf{3.009} & \textbf{1.000} \\
    \hline
    \end{tabular}

    \item \textbf{Conclusion}

    At 950 K and 3.0 bar, the methane is almost completely converted. The results show that the Gibbs minimization method robustly handles complex systems without needing to define or solve for multiple reaction extents and equilibrium constants.
\end{enumerate}
\end{stepbox}

\newpage

\section*{Chemical Equilibrium and Le Châtelier's Principle}
This guide explores Le Châtelier's Principle, a qualitative tool for predicting how a system at equilibrium responds to external changes. We will use the industrially significant Haber-Bosch process for ammonia synthesis as a running example to illustrate these concepts.

\begin{conceptbox}
Le Châtelier's Principle states that when a system at equilibrium is subjected to a change in conditions (a "stress"), the system will shift its equilibrium position to partially counteract the effect of the change. The primary stresses include changes in:
\begin{itemize}[itemsep=0pt]
    \item Temperature
    \item Pressure
    \item Concentration (or moles) of a reacting species
\end{itemize}
\end{conceptbox}

\subsection*{The Ammonia Synthesis Reaction (Haber-Bosch Process)}
\begin{conceptbox}
The Haber-Bosch process is a reversible, gas-phase reaction that synthesizes ammonia from nitrogen and hydrogen.
\begin{itemize}[itemsep=0pt]
    \item It is an \textit{exothermic} reaction, meaning it releases heat as it proceeds in the forward direction.
    \item The change in the number of moles of gas is negative, which is a critical feature for analyzing the effect of pressure.
\end{itemize}
\end{conceptbox}

\begin{formulabox}
The balanced chemical equation is:
$$ \text{N}_2\text{(g)} + 3\text{H}_2\text{(g)} \rightleftharpoons 2\text{NH}_3\text{(g)} \quad\quad \Delta H < 0 $$
The change in moles of gas is:
$$ \Delta \nu = (\text{moles of gas products}) - (\text{moles of gas reactants}) = 2 - (1 + 3) = -2 $$
\end{formulabox}

\subsection*{The Equilibrium Constant Expression}
\begin{keybox}
The following variables are used to quantitatively describe the equilibrium state:
\begin{itemize}[itemsep=0pt]
    \item \textbf{$K_a$}: The thermodynamic equilibrium constant. It is dimensionless and depends only on temperature.
    \item \textbf{$y_i$}: The mole fraction of component $i$ (e.g., $y_{NH_3}$) at equilibrium.
    \item \textbf{$P$}: The total pressure of the system (e.g., in bar).
    \item \textbf{$P^\circ$}: The standard-state pressure (defined as 1 bar).
    \item \textbf{$K_y$}: A parameter representing the ratio of mole fractions at equilibrium. For a given temperature, $K_y$ will change if the pressure changes.
\end{itemize}
\end{keybox}

\begin{formulabox}
The thermodynamic equilibrium constant, $K_a$, is defined using the activities ($a_i$) of the species. For an ideal gas, $a_i = y_i P / P^\circ$.
$$ K_a = \frac{(a_{NH_3})^2}{(a_{N_2})(a_{H_2})^3} = \frac{(y_{NH_3}P/P^\circ)^2}{(y_{N_2}P/P^\circ)(y_{H_2}P/P^\circ)^3} $$
This expression can be separated into a mole fraction term ($K_y$) and a pressure term:
$$ K_a = \left( \frac{y_{NH_3}^2}{y_{N_2} y_{H_2}^3} \right) \left( \frac{P}{P^\circ} \right)^{2 - (1+3)} = K_y \left( \frac{P}{P^\circ} \right)^{-2} $$
This equation quantitatively governs the equilibrium position and is the basis for understanding the effects of pressure changes.
\end{formulabox}

\newpage
\subsection*{Applying Le Châtelier's Principle to Ammonia Synthesis}

\subsubsection*{1. Effect of Temperature}
\begin{conceptbox}
According to Le Châtelier's principle, changing the temperature of a system at equilibrium will cause a shift in the direction that counteracts the temperature change.
\begin{itemize}[itemsep=0pt]
    \item \textit{Increasing Temperature}: Favors the endothermic (heat-absorbing) direction.
    \item \textit{Decreasing Temperature}: Favors the exothermic (heat-releasing) direction.
\end{itemize}
\end{conceptbox}
\begin{keybox}
Application to the Haber-Bosch process ($\Delta H < 0$, exothermic):
\begin{itemize}[itemsep=0pt]
    \item \textbf{Prediction}: If temperature is increased, the system will try to "cool down" by shifting in the endothermic (reverse) direction. This means the equilibrium shifts to the left, consuming NH$_{3}$ and producing more N$_{2}$ and H$_{2}$.
    \item \textbf{Result}: A higher temperature leads to a lower equilibrium conversion and a smaller yield of ammonia.
    \item \textbf{Conclusion}: To maximize the equilibrium yield, a low temperature is thermodynamically favorable. However, reaction rates are slow at low temperatures, so a compromise temperature (e.g., 400-450$^\circ$C) is used in practice.
\end{itemize}
\end{keybox}
\begin{formulabox}
This effect is quantified by the Van't Hoff Equation. Since $\Delta H^\circ_{rxn}$ is negative for this reaction:
$$ \frac{d(\ln K_a)}{dT} = \frac{\Delta H^\circ_{rxn}}{RT^2} < 0 $$
This shows that as temperature ($T$) increases, $\ln(K_a)$ must decrease, meaning the equilibrium constant $K_a$ itself decreases. A smaller $K_a$ signifies a lower equilibrium yield of products.
\end{formulabox}

\subsubsection*{2. Effect of Pressure}
\begin{conceptbox}
Changing the total pressure on a gaseous system at equilibrium will cause a shift in the direction that reduces the number of gas molecules.
\begin{itemize}[itemsep=0pt]
    \item \textit{Increasing Pressure}: Favors the side of the reaction with fewer moles of gas.
    \item \textit{Decreasing Pressure}: Favors the side of the reaction with more moles of gas.
\end{itemize}
\end{conceptbox}
\begin{keybox}
Application to the Haber-Bosch process (4 moles of gas $\rightleftharpoons$ 2 moles of gas):
\begin{itemize}[itemsep=0pt]
    \item \textbf{Prediction}: The forward reaction reduces the number of gas molecules from 4 to 2. Therefore, increasing the total pressure will shift the equilibrium to the right to counteract the pressure increase.
    \item \textbf{Result}: Higher pressure leads to a higher equilibrium conversion and a greater yield of ammonia.
    \item \textbf{Conclusion}: Industrial reactors are operated at very high pressures (e.g., 150-250 bar) to maximize the yield.
\end{itemize}
\end{keybox}
\begin{formulabox}
The quantitative reason is found in the equilibrium expression:
$$ K_a = K_y \left( \frac{P}{P^\circ} \right)^{-2} \quad \implies \quad K_y = K_a \left( \frac{P}{P^\circ} \right)^{2} $$
At a constant temperature, $K_a$ is a constant. The equation shows that if the total pressure $P$ is increased, the mole fraction term $K_y$ must also increase to maintain the equality. A larger $K_y$ means the ratio of product mole fractions to reactant mole fractions has increased, which corresponds to a shift to the right.
\end{formulabox}

\subsubsection*{3. Effect of Changing Moles (Concentration)}
\begin{conceptbox}
Adding or removing a chemical species involved in the equilibrium will cause a shift that consumes the added species or replenishes the removed species. This is often analyzed using the reaction quotient, $Q$.
\begin{itemize}[itemsep=0pt]
    \item If $Q < K_a$: The ratio of products to reactants is too low. The reaction shifts to the right (products).
    \item If $Q > K_a$: The ratio of products to reactants is too high. The reaction shifts to the left (reactants).
    \item If $Q = K_a$: The system is at equilibrium.
\end{itemize}
\end{conceptbox}
\begin{keybox}
Application to the Haber-Bosch process:
\begin{itemize}[itemsep=0pt]
    \item \textbf{Adding Reactants (N$_{2}$ or H$_{2}$)}: This decreases the denominator of the reaction quotient $Q$, making $Q < K_a$. The equilibrium shifts to the right to consume the added reactants and form more ammonia.
    \item \textbf{Removing Product (NH$_{3}$)}: This decreases the numerator of $Q$, making $Q < K_a$. The equilibrium shifts to the right to replenish the ammonia that was removed. This is a common industrial strategy to drive a reaction to completion.
    \item \textbf{Adding an Inert Gas}: At constant volume, adding an inert gas increases total pressure but does not change partial pressures, so there is no shift. At constant total pressure, adding an inert gas decreases the partial pressures of all reacting species, which is equivalent to a pressure decrease, shifting the equilibrium to the side with more moles (left).
\end{itemize}
\end{keybox}

\newpage

\begin{center}
    \Huge\textbf{\color{conceptcolor}Phase Equilibria}
\end{center}
\hrule

nd{document}
