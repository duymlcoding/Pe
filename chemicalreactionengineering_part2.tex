\documentclass[12pt]{article}
\usepackage[paperwidth=8.5in, paperheight=11in, margin=1.0in, headheight=15pt]{geometry}
\usepackage{amsmath,amssymb,amsthm}
\usepackage[most]{tcolorbox}
\usepackage{enumitem}
\usepackage{xcolor}
\usepackage{hyperref}
\usepackage{fancyhdr}
\usepackage{titlesec}
\usepackage{graphicx}
% Define custom colors for chemical engineering theme
\definecolor{conceptcolor}{RGB}{52, 73, 94}      % Dark blue-gray
\definecolor{formulacolor}{RGB}{231, 76, 60}     % Red for formulas
\definecolor{examplecolor}{RGB}{39, 174, 96}     % Green for examples
\definecolor{stepcolor}{RGB}{142, 68, 173}       % Purple for solution steps
\definecolor{keycolor}{RGB}{243, 156, 18}        % Orange for key points
% Configure fancy headers
\pagestyle{fancy}
\fancyhf{}
\fancyhead[L]{PE Study Guide}
\fancyhead[R]{Process Fundamentals}
\fancyfoot[C]{\thepage}
\renewcommand{\baselinestretch}{1.1}
\setlength{\parindent}{0.25in}
\setlength{\parskip}{3pt}
% Configure section formatting
\titleformat{\section}
  {\normalfont\LARGE\bfseries\color{conceptcolor}}
  {\thesection}{1em}{}
\titleformat{\subsection}
  {\normalfont\Large\bfseries\color{conceptcolor}}
  {\thesubsection}{1em}{}
% Define custom environments
\newtcolorbox{conceptbox}[1][]{
  enhanced,
  colback=conceptcolor!10,
  colframe=conceptcolor,
  arc=3mm,
  title=Key Concept,
  fonttitle=\bfseries\sffamily\normalsize,
  fontupper=\small,
  #1
}
\newtcolorbox{formulabox}[1][]{
  enhanced,
  colback=formulacolor!10,
  colframe=formulacolor,
  arc=2mm,
  title=Important Formula,
  fonttitle=\bfseries\sffamily\normalsize,
  fontupper=\small,
  #1
}
\newtcolorbox{examplebox}[2][]{
  enhanced,
  colback=examplecolor!10,
  colframe=examplecolor,
  arc=3mm,
  title=Example Problem: #2,
  fonttitle=\bfseries\sffamily\normalsize,
  fontupper=\small,
  #1
}
\newtcolorbox{stepbox}[1][]{
  enhanced,
  colback=stepcolor!10,
  colframe=stepcolor,
  arc=2mm,
  title=Solution Steps,
  fonttitle=\bfseries\sffamily\normalsize,
  fontupper=\small,
  #1
}
\newtcolorbox{keybox}[1][]{
  enhanced,
  colback=keycolor!10,
  colframe=keycolor,
  arc=2mm,
  title=Key Variables \& Definitions,
  fonttitle=\bfseries\sffamily\normalsize,
  fontupper=\small,
  #1
}

\section*{Isothermal Batch Reactors}

Batch reactors are simple, closed systems used extensively in the chemical industry, especially for small-scale production, high-value products, and processes requiring flexibility. This section covers their fundamental principles, design considerations, and the material balances required for their analysis.

\subsection*{Introduction and Key Concepts}
A batch reactor is essentially a stirred tank that operates as a closed system. Reactants are charged into the vessel at the beginning of the process, and the products are removed only after the reaction has run for a specific amount of time.

\begin{conceptbox}[title=Key Characteristics of a Batch Reactor]
\begin{itemize}[itemsep=2pt]
    \item \textbf{Closed System:} No mass is added or removed during the reaction phase. The total mass inside the reactor remains constant.
    \item \textbf{Well-Mixed:} A critical assumption is that the reactor contents are perfectly mixed due to efficient stirring. This means there are no spatial gradients within the reactor volume at any given moment. Concentrations and temperature are uniform throughout.
    \item \textbf{Unsteady-State Operation:} While spatially uniform, the properties of the system (e.g., concentrations of each species) change with time as the reaction proceeds.
\end{itemize}
\end{conceptbox}

\begin{keybox}[title=Common Applications and Use Cases]
Batch reactors are favored in situations where continuous flow processes are impractical or uneconomical.
\begin{itemize}[itemsep=2pt]
    \item Small-scale production runs.
    \item Manufacturing of high-value products like pharmaceuticals or specialty chemicals.
    \item Processes with very long reaction times (e.g., hours or days).
    \item Fermentation processes, where preventing contamination is critical.
    \item Multi-product plants where the same reactor can be used to make different products sequentially.
\end{itemize}
\end{keybox}

\subsection*{Heat Transfer and Scale-Up Considerations}
Managing the heat of reaction is a critical design challenge, especially when scaling a process from the lab to production.

\begin{conceptbox}[title=The Challenge of Scale-Up]
The primary difficulty in scaling up a reactor arises from the relationship between reactor volume (which determines production capacity) and the available heat transfer area.
\begin{itemize}[itemsep=2pt]
    \item \textbf{Volume} scales with the cube of a characteristic dimension (e.g., diameter, $D$).
    \item \textbf{Heat Transfer Area} (the vessel surface) scales with the square of that dimension.
\end{itemize}
This means that as a reactor gets larger, its volume increases much faster than its surface area. The ratio of area to volume decreases, making it progressively harder to add or remove heat.
\end{conceptbox}

\begin{formulabox}[title=Geometric Scaling Relationships]
$$ \text{Volume: } V \propto D^3 $$
$$ \text{Area: } A \propto D^2 $$
$$ \frac{\text{Heat Transfer Area}}{\text{Volume}} = \frac{A}{V} \propto \frac{D^2}{D^3} = \frac{1}{D} $$
\end{formulabox}

\begin{keybox}[title=Other Scale-Up Issues]
\begin{itemize}[itemsep=2pt]
    \item \textbf{Mixing:} Achieving uniform mixing is more difficult and energy-intensive in a large vessel. Dead zones with poor mixing can lead to non-uniform temperatures and concentrations.
    \item \textbf{Startup Time:} Heating, cooling, and charging a large reactor takes significantly longer, reducing overall productivity.
    \item \textbf{Materials of Construction:} Lab reactors are often glass, while industrial reactors are metal. The metal surface can have unintended catalytic or inhibitory effects on the reaction chemistry.
\end{itemize}
\end{keybox}

\newpage
\subsection*{Material Balances on Tank Reactors}
To model the behavior of a reactor, we perform a mole balance on each chemical species.

\begin{conceptbox}[title=The General Mole Balance Principle]
The fundamental principle for any species in any control volume is:
$$ \text{Rate of Accumulation} = \text{Rate of Inflow} - \text{Rate of Outflow} + \text{Rate of Generation by Reaction} $$
\end{conceptbox}

\begin{keybox}
\begin{itemize}[itemsep=2pt]
    \item $N_i$: The number of moles of component $i$ within the reactor volume.
    \item $F_{i0}, F_i$: Molar flow rates of component $i$ \textit{in} and \textit{out} of the reactor (moles/time).
    \item $r_i$: The rate of formation of component $i$ per unit volume (moles/(volume$\cdot$time)). A negative value indicates consumption.
    \item $V$: The volume of the reacting mixture in the reactor.
    \item $C_i$: The concentration of component $i$ ($N_i/V$).
    \item $X$: The fractional conversion of a reactant.
\end{itemize}
\end{keybox}

\begin{formulabox}[title=General Mole Balance Equation]
For any component A, the mole balance is written as a differential equation:
$$ \frac{dN_A}{dt} = F_{A0} - F_A + r_A V \quad \text{(Equation 1)}$$
This general equation is simplified for different reactor types based on their flow characteristics.
\end{formulabox}

\begin{formulabox}[title=Batch Reactor Mole Balance]
In a batch reactor, there is no inflow or outflow during the reaction, so $F_{A0} = 0$ and $F_A = 0$. The general mole balance (Equation 1) simplifies to:
$$ \frac{dN_A}{dt} = r_A V $$
For liquid-phase reactions with constant volume, we can write this in terms of concentration ($N_A = C_A V$):
$$ \frac{dC_A}{dt} = r_A $$
\end{formulabox}

\newpage
\subsection*{Example: Batch Reactor with Multiple Reactions}

\begin{examplebox}{Mole Balances for a Multi-Reaction System}
\textbf{Question:} Consider an isothermal batch reactor where the following three liquid-phase reactions occur. The rates of reaction ($r_1, r_2, r_3$) are given in moles per liter per second. Write the mole balance equation for each species (A, B, C, D, E).
\begin{align*}
    \text{Reaction 1:} \quad & A + B \xrightarrow{k_1} C & r_1 &= k_1 C_A C_B \\
    \text{Reaction 2:} \quad & C \xrightarrow{k_2} 2E & r_2 &= k_2 C_C \\
    \text{Reaction 3:} \quad & 2A \xrightarrow{k_3} D & r_3 &= k_3 C_A^2
\end{align*}
\end{examplebox}

\begin{stepbox}
We apply the batch reactor mole balance, $\frac{dN_i}{dt} = r_i V$, to each component. The net rate of formation for a species, $r_i$, is the sum of its rates of formation in every reaction, accounting for stoichiometry ($\nu_{ij}$, which is negative for reactants and positive for products). The general form is $r_i = \sum_{j} \nu_{ij} r_j$.
\begin{enumerate}[label=\textbf{Step \arabic*:}, wide=0pt, leftmargin=*, itemsep=2pt]
    \item \textbf{Mole Balance on A:}
    Component A is consumed in Reaction 1 ($\nu_{A1} = -1$) and Reaction 3 ($\nu_{A3} = -2$).
    $$ r_A = (-1)r_1 + (-2)r_3 = -k_1 C_A C_B - 2k_3 C_A^2 $$
    $$ \frac{dN_A}{dt} = r_A V = (-k_1 C_A C_B - 2k_3 C_A^2) V $$
    \item \textbf{Mole Balance on B:}
    Component B is consumed only in Reaction 1 ($\nu_{B1} = -1$).
    $$ r_B = (-1)r_1 = -k_1 C_A C_B $$
    $$ \frac{dN_B}{dt} = r_B V = (-k_1 C_A C_B) V $$
    \item \textbf{Mole Balance on C:}
    Component C is formed in Reaction 1 ($\nu_{C1} = +1$) and consumed in Reaction 2 ($\nu_{C2} = -1$).
    $$ r_C = (+1)r_1 + (-1)r_2 = k_1 C_A C_B - k_2 C_C $$
    $$ \frac{dN_C}{dt} = r_C V = (k_1 C_A C_B - k_2 C_C) V $$
\end{enumerate}
\end{stepbox}

\newpage

\begin{stepbox}
\begin{enumerate}[label=\textbf{Step \arabic*:}, wide=0pt, leftmargin=*, itemsep=2pt, start=4]
    \item \textbf{Mole Balance on D:}
    Component D is formed only in Reaction 3 ($\nu_{D3} = +1$).
    $$ r_D = (+1)r_3 = k_3 C_A^2 $$
    $$ \frac{dN_D}{dt} = r_D V = (k_3 C_A^2) V $$
    \item \textbf{Mole Balance on E:}
    Component E is formed only in Reaction 2 ($\nu_{E2} = +2$).
    $$ r_E = (+2)r_2 = 2k_2 C_C $$
    $$ \frac{dN_E}{dt} = r_E V = (2k_2 C_C) V $$
    \item \textbf{Solving the System:}
    This gives a system of five coupled ordinary differential equations (ODEs). To find the number of moles of each species over time, this system must be solved numerically using software. The required inputs are the initial number of moles of each species ($N_{A0}, N_{B0}, N_{C0}, \dots$), the constant reactor volume ($V$), and the values of the rate constants ($k_1, k_2, k_3$) at the isothermal operating temperature.
\end{enumerate}
\end{stepbox}

\newpage
\subsection*{Example: Copolymerization in a Batch Reactor}

\begin{examplebox}{Copolymerization of Styrene and Butadiene}
\textbf{Question:} Styrene (S) and butadiene (B) are copolymerized in an isothermal batch reactor. The $27 \, \text{m}^3$ reactor was charged with $2,200 \, \text{kg}$ of styrene and $5,000 \, \text{kg}$ of butadiene. The polymerization is first order in S and first order in B, and the rate constant is $k = 0.036 \, \text{m}^3\text{/(kmol}\cdot\text{h)}$. The reaction is $S + 3.2 B \rightarrow \text{polymer}$, and constant density is assumed. What are the concentrations of S and B after 10 h?
\end{examplebox}

\begin{stepbox}
\begin{enumerate}[label=\textbf{Step \arabic*:}, wide=0pt, leftmargin=*, itemsep=2pt]
    \item \textbf{Analyze the Problem and Define the Goal:}
    The core task is to model a second-order reaction ($r = -kC_SC_B$) in a constant-volume batch reactor. This requires solving a system of coupled ordinary differential equations (ODEs) that describe the change in concentration of each reactant over time. The goal is to find the concentrations $C_S$ and $C_B$ at $t = 10$ h.
    \item \textbf{Calculate Initial Moles and Concentrations:}
    To solve the ODEs, we first need the initial conditions ($C_{S0}$, $C_{B0}$) at $t=0$.
    
    Molecular Weight of Styrene (S, $C_8H_8$): $104.15 \, \text{kg/kmol}$. Molecular Weight of Butadiene (B, $C_4H_6$): $54.09 \, \text{kg/kmol}$.
    \begin{formulabox}[title=Initial Molar Amounts]
    $$ N_{S0} = \frac{2200 \, \text{kg}}{104.15 \, \text{kg/kmol}} = 21.12 \, \text{kmol} $$
    $$ N_{B0} = \frac{5000 \, \text{kg}}{54.09 \, \text{kg/kmol}} = 92.44 \, \text{kmol} $$
    \end{formulabox}
    \begin{formulabox}[title=Initial Concentrations]
    $$ C_{S0} = \frac{N_{S0}}{V} = \frac{21.12 \, \text{kmol}}{27 \, \text{m}^3} = \textbf{0.782 kmol/m\textsuperscript{3}} $$
    $$ C_{B0} = \frac{N_{B0}}{V} = \frac{92.44 \, \text{kmol}}{27 \, \text{m}^3} = \textbf{3.424 kmol/m\textsuperscript{3}} $$
    \end{formulabox}
\end{enumerate}
\end{stepbox}

\newpage

\begin{stepbox}
\begin{enumerate}[label=\textbf{Step \arabic*:}, wide=0pt, leftmargin=*, itemsep=2pt, start=3]
    \item \textbf{Formulate the Mole Balance Equations:}
    For a constant-volume batch reactor, the mole balance for any species $i$ simplifies to $\frac{dC_i}{dt} = r_i$.
    
    The rate of formation of Styrene is $r_S = -k C_S C_B$. From the stoichiometry, 3.2 moles of B are consumed for every mole of S. Therefore, the rate of formation of Butadiene is $r_B = -3.2 (k C_S C_B)$.
    
    This gives us the system of ODEs to solve:
    \begin{formulabox}[title=System of Differential Equations]
    $$ \frac{dC_S}{dt} = -k C_S C_B $$
    $$ \frac{dC_B}{dt} = -3.2 k C_S C_B $$
    \end{formulabox}
    \item \textbf{Define the Solution Approach:}
    The problem is now fully defined as an initial value problem. We must solve the system of ODEs with the known rate constant and initial concentrations over the specified time interval.
    \begin{keybox}[title=Summary for Numerical Solver]
    Equations to Solve:
    \begin{align*}
        \frac{dC_S}{dt} &= -0.036 \cdot C_S \cdot C_B \\
        \frac{dC_B}{dt} &= -3.2 \cdot (0.036) \cdot C_S \cdot C_B = -0.1152 \cdot C_S \cdot C_B
    \end{align*}
    Initial Conditions (at $t=0$):
    \begin{align*}
        C_S(0) &= 0.782 \, \text{kmol/m}^3 \\
        C_B(0) &= 3.424 \, \text{kmol/m}^3
    \end{align*}
    Integration Interval: from $t=0$ to $t=10$ h.
    
    This system would be integrated using a numerical solver (e.g., in MATLAB, Python, or Polymath) to find the final concentrations $C_S(10)$ and $C_B(10)$.
    \end{keybox}
\end{enumerate}
\end{stepbox}

\newpage
\subsection*{Example: Pseudo-First-Order Reaction}

\begin{examplebox}{Hydration with a Component in Large Excess}
\textbf{Question:} The elementary, irreversible liquid-phase hydration of butylene oxide (A) produces butylene glycol (C): $C_4H_8O \text{ (A)} + H_2O \text{ (B)} \rightarrow C_4H_{10}O_2 \text{ (C)}$. The reaction is conducted using water (B) as the solvent, so it is in large excess. The initial concentration of butylene oxide is $C_{A0} = 0.25$ mol/L. The rate constant is $k' = 8.3 \times 10^{-4}$ L/(mol$\cdot$min) at 323 K. A batch reactor is used.
\begin{enumerate}[label=(\alph*)]
    \item Determine the final concentration of butylene oxide after 45 min.
    \item A consultant suggests enhancing the rate by continuously feeding water at 25 L/min into the initial 500 L volume. Will this proposal increase, decrease, or have no effect on the time required to reach a given conversion?
\end{enumerate}
\end{examplebox}

\begin{stepbox}[title=Solution (Part a): Final Concentration Calculation]
\begin{enumerate}[label=\textbf{Step \arabic*:}, wide=0pt, leftmargin=*, itemsep=2pt]
    \item \textbf{Simplify the Rate Law via Pseudo-First-Order Approximation:}
    
    When a second-order reaction ($r = -k'C_AC_B$) involves one reactant (B) in large excess (e.g., a solvent), its concentration $C_B$ remains essentially constant throughout the reaction. We can combine the true rate constant $k'$ and the constant concentration $C_B$ into a new, "pseudo-first-order" rate constant, $k$.
    
    Molar density of water: $C_B \approx \frac{1000 \, \text{g/L}}{18.015 \, \text{g/mol}} \approx 55.5$ mol/L. Since $C_B \gg C_{A0}$ ($55.5 \gg 0.25$), the assumption is valid.
    $$ r_A = -k' C_A C_B \approx -(k' C_B) C_A = -k C_A $$
    $$ k = k' C_B = (8.3 \times 10^{-4} \, \frac{\text{L}}{\text{mol}\cdot\text{min}}) \cdot (55.5 \, \frac{\text{mol}}{\text{L}}) = 0.046065 \, \text{min}^{-1} $$
    
    \item \textbf{Apply the First-Order Integrated Rate Law:}
    For a first-order reaction in a constant-volume batch reactor:
    $$ \ln\left(\frac{C_{A0}}{C_A}\right) = k t $$
    Solving for the final concentration, $C_A$, after $t = 45$ min:
    $$ \ln\left(\frac{0.25 \, \text{mol/L}}{C_A}\right) = (0.046065 \, \text{min}^{-1}) \cdot (45 \, \text{min}) = 2.073 $$
    $$ \frac{0.25}{C_A} = e^{2.073} \approx 7.948 $$
    $$ C_A = \frac{0.25}{7.948} \approx \textbf{0.0315 mol/L} $$
\end{enumerate}
\end{stepbox}

\begin{stepbox}[title=Solution (Part b): Effect of Adding More Solvent]
\begin{enumerate}[label=\textbf{Step \arabic*:}, wide=0pt, leftmargin=*, itemsep=2pt]
    \item \textbf{Analyze the Proposal using a Mole Basis:}
    The proposal involves changing the volume, so analyzing concentrations can be misleading. The fundamental measure of reaction progress is the number of \textbf{moles} converted. Let's write the mole balance for the product, C. This is now a semi-batch reactor, but the core principle holds.
    $$ \frac{dN_C}{dt} = r_C V $$

    \item \textbf{Examine the Rate of Molar Production:}
    The rate of formation of product C per unit volume is $r_C = k' C_A C_B$. Substituting this into the mole balance gives:
    $$ \frac{dN_C}{dt} = (k' C_A C_B) V $$
    Now, express the concentration of A in terms of moles and the changing volume, $C_A = N_A/V$.
    \begin{formulabox}[title=Molar Rate of Production]
    $$ \frac{dN_C}{dt} = \left(k' \frac{N_A}{V} C_B\right) V = k' C_B N_A $$
    Using the pseudo-first-order constant $k = k'C_B$:
    $$ \frac{dN_C}{dt} = k N_A $$
    \end{formulabox}

    \item \textbf{Conclusion:}
    The final equation is the key insight. The rate at which \textbf{moles of product C are formed} depends only on the number of moles of reactant A present ($N_A$) and the pseudo-rate constant ($k$). It is \textbf{independent of the total reactor volume $V$}.
    \begin{itemize}[itemsep=2pt]
        \item When adding more water, the concentration of A ($C_A = N_A/V$) decreases because the volume $V$ increases. This \textit{lowers} the volumetric reaction rate ($r_C$).
        \item At the same time, the total reactor volume $V$ is increasing.
    \end{itemize}
    As the mole balance explicitly shows, these two effects exactly cancel each other out. The rate of conversion of moles of A is unchanged. Therefore, the consultant's proposal will have \textbf{no effect} on the time required to reach a specific conversion (i.e., to convert a specific number of moles of A).
\end{enumerate}
\end{stepbox}

\newpage
\section*{Isothermal Semibatch Reactors}

A semibatch reactor is a variation of a batch reactor where one or more reactants are added to the reactor, or products are removed from the reactor, during the course of the reaction. This mode of operation provides additional control over concentration and temperature, making it a versatile tool for chemical synthesis.

\subsection*{Fundamental Equations}
The dynamic behavior of a semibatch reactor is described by a set of simultaneous ordinary differential equations derived from mole and volume balances.

\begin{keybox}[title=Variable Definitions]
\begin{itemize}[itemsep=2pt]
    \item $N_i$: Moles of component $i$ in the reactor at time $t$.
    \item $F_{i0}, F_i$: Molar flow rates of $i$ entering and leaving the reactor (mol/s).
    \item $v_0, v$: Volumetric flow rates into and out of the reactor (L/s).
    \item $V$: Volume of the reactor contents at time $t$ (L).
    \item $k$: Reaction rate constant.
    \item $C_i$: Molar concentration of component $i$ in the reactor, $C_i = N_i/V$ (mol/L).
\end{itemize}
\end{keybox}

\begin{formulabox}[title=General Mole and Volume Balances]
For an irreversible reaction $A + B \rightarrow 2C$ with a rate law of $r = k C_A^n C_B^m$:
\begin{itemize}[itemsep=2pt]
    \item \textbf{Mole Balances (moles):}
    \begin{align*}
        \frac{dN_A}{dt} &= F_{A0} - F_A - kC_A^n C_B^m V \\
        \frac{dN_B}{dt} &= F_{B0} - F_B - kC_A^n C_B^m V \\
        \frac{dN_C}{dt} &= F_{C0} - F_C + 2kC_A^n C_B^m V
    \end{align*}
    \item \textbf{Volume Balance (liquid phase):}
    $$ \frac{dV}{dt} = v_0 - v $$
\end{itemize}
To solve this system, initial conditions ($N_{A0}, N_{B0}, N_{C0}, V_0$) at $t=0$ are required.
\end{formulabox}

\subsection*{Conceptual Overview and Applications}
Semibatch reactors are typically stirred tanks operating in a non-steady-state, open-system mode, primarily used for liquid-phase reactions.

\begin{conceptbox}[title=Key Applications of Semibatch Reactors]
This reactor type is chosen to exert specific control over the reaction environment.
\begin{enumerate}[itemsep=2pt]
    \item \textbf{Selectivity Control:} By adding one reactant (A) slowly to another (B), the concentration of A can be kept low. This is advantageous if an undesired side reaction has a higher reaction order with respect to A than the desired reaction.
    \item \textbf{Sequential Reactions:} To prevent side reactions between reactants of different steps (e.g., B and D in $A+B\rightarrow C; C+D\rightarrow E$), one can react A and B to completion first, then add D.
    \item \textbf{Heat Transfer and Safety:} For highly exothermic reactions, adding a reactant slowly controls the reaction rate, and thus the rate of heat generation. This prevents dangerous temperature spikes (thermal runaway).
    \item \textbf{Shifting Equilibrium:} If a reaction is reversible ($A+B \leftrightarrow C+D$) and a product (D) is a gas, continuously removing D from the liquid phase will shift the equilibrium to the right, increasing yield.
    \item \textbf{Sparingly Soluble Reactants:} Gaseous reactants like H$_2$ or O$_2$ have low solubility in liquids. They must be continuously bubbled (fed) through the reactor to maintain their concentration in the liquid phase.
\end{enumerate}
\end{conceptbox}

\newpage
\subsection*{Example: Gas-Liquid Hydrogenation}

\begin{examplebox}{Hydrogenation in a Semibatch Reactor}
\textbf{Question:} In an isothermal semibatch reactor, a liquid-phase reactant A is hydrogenated to produce a liquid product P: $A(\text{liq}) + H_2(\text{g}) \rightarrow P(\text{liq})$. The reactor initially contains $N_{A0} = 150$ mol of A. Hydrogen gas is fed from a cylinder to maintain a constant reactor pressure of 3 bar. The rate law is $-r_A = k C_A P_{H_2}$, where $k = 0.0074$ bar$^{-1}$min$^{-1}$. Assume the volume of the liquid phase is constant.
\begin{enumerate}[label=(\alph*)]
    \item Determine the time required to reach 80\% conversion of A.
    \item Derive an expression for the molar flow rate of hydrogen into the reactor as a function of time.
\end{enumerate}
\end{examplebox}

\begin{stepbox}[title=Solution (Part a): Time to Reach 80\% Conversion]
\begin{enumerate}[label=\textbf{Step \arabic*:}, wide=0pt, leftmargin=*, itemsep=2pt]
    \item \textbf{Mole Balance on A:}
    Reactant A is only present in the liquid phase and is not fed or removed. Therefore, its mole balance is identical to that of a batch reactor. Let $V_L$ be the constant liquid volume.
    $$ \frac{dN_A}{dt} = r_A V_L = (-k C_A P_{H_2}) V_L $$
    Since $V_L$ is constant, we can use $N_A = C_A V_L$ and simplify to a concentration basis:
    $$ \frac{dC_A}{dt} = -k C_A P_{H_2} $$

    \item \textbf{Integrate the Rate Law:}

    The key simplification is that hydrogen is supplied to maintain a constant pressure, so $P_{H_2} = 3$ bar is constant throughout the reaction. The rate law simplifies to a pseudo-first-order form, $r_A = -k'C_A$, where the new constant is $k' = k \cdot P_{H_2}$.

    We can now separate variables and integrate the simplified rate law:
    $$ \int_{C_{A0}}^{C_A} \frac{dC_A}{C_A} = \int_0^t -k P_{H_2} dt \implies \ln\left(\frac{C_A}{C_{A0}}\right) = -k P_{H_2} t $$
    
    \item \textbf{Solve for Time:}
    The target is 80\% conversion ($X_A=0.8$). At this point, the final concentration is $C_A = C_{A0}(1-X_A) = 0.2 C_{A0}$. The concentration ratio is $C_A/C_{A0} = 0.2$.
    $$ \ln(0.2) = -(0.0074 \, \text{bar}^{-1}\text{min}^{-1}) \cdot (3 \, \text{bar}) \cdot t $$
    $$ -1.6094 = -0.0222 \cdot t $$
    $$ t = \frac{1.6094}{0.0222} = \textbf{72.5 min} $$
\end{enumerate}
\end{stepbox}

\newpage
\begin{stepbox}[title=Solution (Part b): Hydrogen Flow Rate vs. Time]
\begin{enumerate}[label=\textbf{Step \arabic*:}, wide=0pt, leftmargin=*, itemsep=2pt]
    \item \textbf{Mole Balance on Gaseous H$_2$:}
    Hydrogen is fed into the reactor ($F_{H_2,in}$) and consumed by the reaction. From the 1:1 stoichiometry, its rate of formation is $r_{H_2} = r_A = -k C_A P_{H_2}$.
    $$ \frac{dN_{H_2}}{dt} = F_{H_2,in} + r_{H_2} V_L = F_{H_2,in} - k C_A P_{H_2} V_L $$

    \item \textbf{Apply Simplification for Constant Gas Moles:}
    The problem states that the reactor pressure, temperature, and volume are constant. According to the ideal gas law ($PV=nRT$), if P, V, and T are constant for the gas phase, then the number of moles of hydrogen gas in the reactor headspace, $N_{H_2}$, must also be constant. Therefore, the accumulation term is zero:
    $$ \frac{dN_{H_2}}{dt} = 0 $$
    
    \item \textbf{Derive the Flow Rate Expression:}
    The mole balance simplifies to an algebraic equation for the inlet flow rate:
    $$ 0 = F_{H_2,in} - k C_A P_{H_2} V_L \implies F_{H_2,in} = k P_{H_2} V_L C_A $$
    To get the flow rate as a function of time, we need the expression for $C_A(t)$ from Part (a):
    $$ C_A(t) = C_{A0} e^{-k P_{H_2} t} $$
    Substituting this in and recognizing that $C_{A0} V_L = N_{A0}$ (the initial moles of A):
    \begin{formulabox}[title=Hydrogen Feed Rate as a Function of Time]
    $$ F_{H_2,in}(t) = k P_{H_2} N_{A0} e^{-k P_{H_2} t} $$
    \end{formulabox}
    This result shows that the hydrogen flow rate required is highest at $t=0$ when the concentration of A is highest, and it decays exponentially as reactant A is consumed.
\end{enumerate}
\end{stepbox}

\newpage
\subsection*{Example: Catalytic Semibatch Reaction}

\begin{examplebox}{Semibatch Reaction with Changing Volume}
\textbf{Question:} A catalytic reaction, $A \rightarrow C$, takes place in the liquid phase of an isothermal semibatch reactor. The rate law is $r_A = -k C_A C_B$, with $k = 0.25$ L/(mol$\cdot$min), where B is the catalyst. Initially, the reactor contains 2700 L of a solution with $C_{A0} = 20.0$ mol/L. The initial concentration of catalyst B is zero. Starting at $t=0$, a solution containing 0.05 mol/L of B is fed to the reactor at a flow rate of 12.5 L/min. How many moles of product C are in the reactor after 200 min?
\end{examplebox}

\begin{stepbox}
\begin{enumerate}[label=\textbf{Step \arabic*:}, wide=0pt, leftmargin=*, itemsep=2pt]
    \item \textbf{Strategy and Initial Conditions:}
    This system involves changing volume and coupled concentrations, requiring a numerical solution. We first define all initial conditions and constant feed rates.
    
    Initial Conditions ($t=0$): $V(0) = 2700$ L, $N_A(0) = C_{A0} V(0) = (20.0 \, \text{mol/L}) \cdot (2700 \, \text{L}) = 54,000$ mol, $N_B(0) = 0$ mol (Catalyst is not present initially), $N_C(0) = 0$ mol (Product is not present initially).
    
    Feed Stream ($t > 0$): Volumetric flow rate in: $v_0 = 12.5$ L/min. 
    
    Molar feed rate of B: $F_{B0} = v_0 C_{B,in} = (12.5 \, \text{L/min}) (0.05 \, \text{mol/L}) = 0.625$ mol/min. 
    
    Molar feed rates of A and C are zero: $F_{A0} = 0, F_{C0} = 0$.
    
    \item \textbf{Formulate the System of Differential Equations:}
    We need mole balances for A, B, and C, and a volume balance. The concentrations $C_A=N_A/V$ and $C_B=N_B/V$ must be calculated at each time step.
    
    Volume Balance: Flow in only. 
    
    $\frac{dV}{dt} = v_0 = 12.5$
    
    Mole Balance on A (Reactant): No flow, only consumption. 
    
    $\frac{dN_A}{dt} = r_A V = -k C_A C_B V$
    
    Mole Balance on B (Catalyst): Flow in, no consumption. 
    
    $\frac{dN_B}{dt} = F_{B0} = 0.625$
    
    Mole Balance on C (Product): No flow, only formation. 
    
    $\frac{dN_C}{dt} = r_C V = -r_A V = k C_A C_B V$
\end{enumerate}
\end{stepbox}

\newpage

\begin{stepbox}
\begin{enumerate}[label=\textbf{Step \arabic*:}, wide=0pt, leftmargin=*, itemsep=2pt, start=3]
    \item \textbf{Summarize for Numerical Solution:}
    The complete system of ODEs and initial conditions is prepared for a numerical solver.
    \begin{keybox}[title=System for Numerical Solver]
    Differential Equations:
    \begin{align*}
        \frac{dV}{dt} &= 12.5 \\
        \frac{dN_A}{dt} &= -0.25 \left(\frac{N_A}{V}\right) \left(\frac{N_B}{V}\right) V = -0.25 \frac{N_A N_B}{V} \\
        \frac{dN_B}{dt} &= 0.625 \\
        \frac{dN_C}{dt} &= 0.25 \left(\frac{N_A}{V}\right) \left(\frac{N_B}{V}\right) V = 0.25 \frac{N_A N_B}{V}
    \end{align*}
    Initial Conditions (at $t=0$): $V=2700$, $N_A=54000$, $N_B=0$, $N_C=0$.
    
    Integration Interval: from $t=0$ to $t=200$ min.
    \end{keybox}
    \item \textbf{Final Result:}
    Solving this system with a numerical package (such as POLYMATH, MATLAB, or Python) yields the number of moles of each component at $t=200$ min. The result for product C is:
    $$ N_C(200 \text{ min}) \approx \textbf{28,000 moles} $$
\end{enumerate}
\end{stepbox}

\newpage
\section*{Isothermal Continuous Stirred Tank Reactors (CSTRs)}
The Continuous Stirred-Tank Reactor (CSTR) is a common type of reactor used in industrial processing. In its ideal form, it operates at steady state with perfect mixing. This key assumption implies that the concentration and temperature of the effluent stream are the same as the conditions everywhere inside the reactor. This section focuses on the analysis of isothermal CSTRs.

\subsection*{Fundamental Equations}
The design of a CSTR is based on a steady-state mole balance, which results in an algebraic equation rather than a differential equation.

\begin{keybox}[title=Variable Definitions]
\begin{itemize}[itemsep=2pt]
    \item $F_{i0}, F_i$: Molar flow rates of component $i$ entering and leaving the reactor (mol/s).
    \item $V$: Volume of the reactor contents (L).
    \item $v_0$ or $v$: Volumetric flow rate (L/s). For liquid-phase systems, this is often assumed constant ($v_0 = v$).
    \item $k$: Reaction rate constant.
    \item $C_i$: Molar concentration of component $i$ in the reactor and in the exit stream (mol/L).
    \item $X_A$: Fractional conversion of reactant A.
\end{itemize}
\end{keybox}

\begin{formulabox}[title=CSTR Steady-State Mole Balance]
The general form is: $\text{Inflow} - \text{Outflow} + \text{Generation by Reaction} = 0$.
For a reaction $A \rightarrow \text{Products}$ with rate law $-r_A = k C_A^n$:
$$ F_{A0} - F_A + r_A V = 0 $$
Substituting $r_A = -kC_A^n$ and $F_A = vC_A$ (for constant liquid density):
$$ v C_{A0} - v C_A - kC_A^n V = 0 $$
This algebraic equation must be solved for the unknown outlet concentration, $C_A$.
\end{formulabox}

\begin{formulabox}[title=Key CSTR Relationships (Constant Density)]
\begin{itemize}[itemsep=2pt]
    \item \textbf{Space Time ($\tau$):} The average time a fluid element spends in the reactor.
        $$ \tau = \frac{V}{v} $$
    \item \textbf{Conversion ($X_A$):} The fraction of reactant A that has been converted.
        $$ X_A = \frac{F_{A0} - F_A}{F_{A0}} = \frac{v C_{A0} - v C_A}{v C_{A0}} = \frac{C_{A0} - C_A}{C_{A0}} $$
    \item \textbf{Concentration from Conversion:}
        $$ C_A = C_{A0}(1-X_A) $$
\end{itemize}
\end{formulabox}

\newpage
\subsection*{Example: Second-Order Reaction in a CSTR}

\begin{examplebox}{Second-Order Hydrolysis in a CSTR}
\textbf{Question:} The hydrolysis of acetic anhydride (A) to form acetic acid is carried out in a 1250-L CSTR. The feed contains 2.5 mol/L acetic anhydride and 50.0 mol/L of water (W). The reaction, $(CH_3CO)_2O + H_2O \rightarrow 2CH_3COOH$, is first order in acetic anhydride and first order in water. At the reactor temperature, the rate constant is $k = 0.075$ L/(mol$\cdot$s). The feed flow rate is 15 L/s. What is the conversion of acetic anhydride?
\end{examplebox}

\begin{stepbox}
\begin{enumerate}[label=\textbf{Step \arabic*:}, wide=0pt, leftmargin=*, itemsep=2pt]
    \item \textbf{Formulate the Steady-State Mole Balances:}
    We must solve the steady-state mole balances for both reactants, acetic anhydride (A) and water (W).
    
    Rate Law: $-r_A = k C_A C_W$. Stoichiometry: A and W react in a 1:1 ratio, so their rates of formation are equal: $r_A = r_W$.
    
    The general mole balance for species $i$ is $F_{i0} - F_i + r_i V = 0$. Assuming constant volumetric flow rate $v$:
    \begin{align*}
        v C_{A0} - v C_A + r_A V = 0 &\implies v(C_{A0} - C_A) - k C_A C_W V = 0 \\
        v C_{W0} - v C_W + r_W V = 0 &\implies v(C_{W0} - C_W) - k C_A C_W V = 0
    \end{align*}
    \item \textbf{Substitute Known Values to Form a System of Equations:}
    We have a system of two algebraic equations with two unknowns: the outlet concentrations $C_A$ and $C_W$.
    
    Given: $v = 15$ L/s, $V = 1250$ L, $C_{A0} = 2.5$ mol/L, $C_{W0} = 50.0$ mol/L, $k = 0.075$ L/(mol$\cdot$s).
    
    \begin{formulabox}[title=System of Algebraic Equations]
    Substituting the values:
    \begin{align*}
        15(2.5 - C_A) - (0.075) C_A C_W (1250) &= 0 \\
        15(50.0 - C_W) - (0.075) C_A C_W (1250) &= 0
    \end{align*}
    Simplifying:
    \begin{align*}
        37.5 - 15 C_A - 93.75 C_A C_W &= 0 \\
        750 - 15 C_W - 93.75 C_A C_W &= 0
    \end{align*}
    \end{formulabox}
\end{enumerate}
\end{stepbox}
\newpage
\begin{stepbox}
\begin{enumerate}[label=\textbf{Step \arabic*:}, wide=0pt, leftmargin=*, itemsep=2pt, start=3]
    \item \textbf{Solve the System of Equations:}
    This system of non-linear equations must be solved simultaneously. While analytical substitution is possible, a numerical solver is a more robust approach for complex systems. Based on the provided solution from such a solver:
    $$ C_A = 0.081 \, \text{mol/L} \quad \text{and} \quad C_W = 47.58 \, \text{mol/L} $$
    \item \textbf{Calculate the Final Conversion:}
    The conversion of acetic anhydride (A) is calculated from the inlet and outlet concentrations.
    $$ X_A = \frac{C_{A0} - C_A}{C_{A0}} = \frac{2.5 - 0.081}{2.5} = \frac{2.419}{2.5} = 0.9676 $$
    The conversion is \textbf{97\%}.
\end{enumerate}
\end{stepbox}

\newpage
\subsection*{Example: Reversible Reaction in a CSTR}

\begin{examplebox}{Reversible Reaction with a Change in Flow Rate}
\textbf{Question:} The liquid-phase, reversible reaction $A + B \rightleftharpoons 2C$ occurs in an isothermal CSTR. The feed contains equimolar amounts of A and B. The conversion of A is measured to be 60.0\%. The equilibrium conversion under the same feed conditions is 80.0\%. If the volumetric flow rate is increased by 50.0\%, what is the new conversion of A? The forward reaction is first order in A and B; the reverse reaction is second order in C.
\end{examplebox}

\begin{stepbox}[title=Solution (Part 1 of 3): Use Equilibrium Data]
This problem is solved by finding a key dimensionless group that characterizes the reactor system and then using it to predict performance under new conditions.
\begin{enumerate}[label=\textbf{Step \arabic*:}, wide=0pt, leftmargin=*, itemsep=2pt]
    \item \textbf{Relate Rate Constants using Equilibrium Conversion ($X_e$):}
    The net rate of reaction for A is $-r_A = k_1 C_A C_B - k_2 C_C^2$. At equilibrium, the net rate is zero ($-r_A = 0$) and the conversion is $X_e = 0.80$.
    $$ k_1 C_{A,e} C_{B,e} = k_2 C_{C,e}^2 $$
    We express the equilibrium concentrations in terms of the initial concentration $C_{A0}$ and $X_e$. Since the feed is equimolar, $C_{B0} = C_{A0}$.
    \begin{itemize}[itemsep=2pt]
        \item $C_{A,e} = C_{A0}(1-X_e) = C_{A0}(1-0.80) = 0.2 C_{A0}$
        \item $C_{B,e} = C_{A0}(1-X_e) = 0.2 C_{A0}$
        \item $C_{C,e} = 2 C_{A0} X_e = 2 C_{A0}(0.80) = 1.6 C_{A0}$
    \end{itemize}
    Substituting these into the equilibrium expression allows us to find the ratio of rate constants:
    $$ k_1 (0.2 C_{A0})(0.2 C_{A0}) = k_2 (1.6 C_{A0})^2 $$
    $$ k_1 (0.04 C_{A0}^2) = k_2 (2.56 C_{A0}^2) $$
    \begin{formulabox}[title=Relationship Between Rate Constants]
    $$ k_1 = \frac{2.56}{0.04} k_2 \implies k_1 = 64 k_2 $$
    \end{formulabox}
\end{enumerate}
\end{stepbox}

\newpage
\begin{stepbox}[title=Solution (Part 2 of 3): Characterize the Initial Reactor State]
\begin{enumerate}[label=\textbf{Step \arabic*:}, wide=0pt, leftmargin=*, itemsep=2pt]\setcounter{enumi}{1}
    \item \textbf{Use Initial Operating Data to Find a Reactor Constant Group:}
    For the initial case (Case 1), the conversion is $X_1 = 0.60$ at a flow rate $v_1$. We use the CSTR mole balance:
    $$ v_1(C_{A0} - C_{A1}) + r_{A1} V = 0 \implies v_1 C_{A0} X_1 = -r_{A1} V $$
    First, find the concentrations at $X_1 = 0.60$:
    \begin{itemize}[itemsep=2pt]
        \item $C_{A1} = C_{A0}(1-X_1) = 0.4 C_{A0}$
        \item $C_{B1} = C_{A0}(1-X_1) = 0.4 C_{A0}$
        \item $C_{C1} = 2 C_{A0} X_1 = 1.2 C_{A0}$
    \end{itemize}
    Now, substitute these into the rate law, using $k_1=64k_2$:
    $$ -r_{A1} = k_1 C_{A1} C_{B1} - k_2 C_{C1}^2 = 64k_2(0.4C_{A0})^2 - k_2(1.2C_{A0})^2 $$
    $$ -r_{A1} = k_2 C_{A0}^2 [64(0.16) - 1.44] = k_2 C_{A0}^2 [10.24 - 1.44] = 8.8 k_2 C_{A0}^2 $$
    Substitute this rate back into the mole balance:
    $$ v_1 C_{A0} (0.60) = (8.8 k_2 C_{A0}^2) V $$
    Rearranging gives us a key dimensionless group that combines all the unknown system parameters:
    \begin{formulabox}[title=Dimensionless Reactor Group]
    $$ \frac{k_2 C_{A0} V}{v_1} = \frac{0.60}{8.8} \approx 0.06818 $$
    \end{formulabox}
\end{enumerate}
\end{stepbox}

\newpage
\begin{stepbox}[title=Solution (Part 3 of 3): Solve for the New Conversion]
\begin{enumerate}[label=\textbf{Step \arabic*:}, wide=0pt, leftmargin=*, itemsep=2pt]\setcounter{enumi}{2}
    \item \textbf{Set up the Mole Balance for the New Condition:}
    The flow rate is increased by 50\%, so the new flow rate is $v_2 = 1.5 v_1$. We seek the new conversion, $X_2$. The CSTR design equation for this new case (Case 2) is:
    $$ v_2 C_{A0} X_2 = -r_{A2} V $$
    The rate law $-r_{A2}$ can be expressed in terms of the new conversion $X_2$:
    $$ -r_{A2} = k_1 C_{A2}C_{B2} - k_2 C_{C2}^2 = k_2 C_{A0}^2 [64(1-X_2)^2 - 4X_2^2] $$
    Substitute this into the design equation:
    $$ (1.5 v_1) C_{A0} X_2 = k_2 C_{A0}^2 [64(1-X_2)^2 - 4X_2^2] V $$
    Rearrange the equation to isolate the dimensionless group we found in the previous step:
    $$ X_2 = \left(\frac{k_2 C_{A0} V}{v_1}\right) \frac{[64(1-X_2)^2 - 4X_2^2]}{1.5} $$
    
    \item \textbf{Solve for the New Conversion $X_2$:}
    Substitute the value of the dimensionless group (0.06818) into the equation:
    $$ X_2 = (0.06818) \frac{64(1-2X_2+X_2^2) - 4X_2^2}{1.5} $$
    $$ 1.5 X_2 = 0.06818 [64 - 128X_2 + 64X_2^2 - 4X_2^2] $$
    $$ 1.5 X_2 = 0.06818 [60X_2^2 - 128X_2 + 64] $$
    Multiplying and rearranging gives a quadratic equation:
    $$ 22.0 X_2 = 60X_2^2 - 128X_2 + 64 $$
    \begin{formulabox}[title=Final Quadratic Equation for Conversion]
    $$ 60X_2^2 - 150X_2 + 64 = 0 $$
    \end{formulabox}
    Solving this using the quadratic formula yields two roots: $X_2 \approx 1.95$ and $X_2 \approx 0.546$. Since conversion cannot be greater than 1, the only physically meaningful answer is $X_2=0.546$.
    \newline
    \newline
    The new conversion is \textbf{55\%}. This result is logical, as increasing the flow rate decreases the reactor's space time, leading to a lower conversion.
\end{enumerate}
\end{stepbox}

\newpage
nd{document}
